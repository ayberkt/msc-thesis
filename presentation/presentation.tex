\documentclass[xcolor={dvipsnames}]{beamer}
\usepackage{amsmath}
\usetheme{metropolis}

\title{Formal Topology in Univalent Foundations}
\date{\today}
\author{Ayberk Tosun}
\institute{Chalmers University of Technology}

\setmonofont[Scale=0.85]{PragmataPro Mono Liga}

\definecolor{AgdaFunction}{HTML}{0000CD}
\definecolor{AgdaString}{HTML}{B22222}
\definecolor{codecolour}{RGB}{27, 134, 236}
\definecolor{gerbesred}{RGB}{164, 30, 50}

\newcommand{\fnname}[1]{{\color{codecolour} {\tt #1}}}
\newcommand{\prgoutput}[1]{{\color{codecolour} {\tt #1}}}

\newcommand{\pity}[3]{\prod_{(#1~:~#2)} #3}
\newcommand{\univ}{\mathsf{Type}}

%% Color customisation.

\begin{document}

\maketitle

\begin{frame}{What is topology?}
  \Large
  A topological space is a set $X$ together with a collection $\Omega(X)$ of its subsets
  such that
  \begin{itemize}
    \item $\emptyset, X \in \Omega(X)$,
    \item $\Omega(X)$ is closed under \alert{finite} intersections, and
    \item $\Omega(x)$ is closed under \alert{arbitrary} unions.
  \end{itemize}
\end{frame}

\begin{frame}{What is topology?}
  \Large
  \fnname{P}: a program that prints out a sequence of integers.

  \begin{center}
  \prgoutput{7~~11~~2~~2~~8~~42~~}~$\cdots$
  \end{center}

  We can consider certain properties \fnname{P}:
  \begin{center}
    ``\fnname{P} eventually prints \fnname{17}''\\
    \vspace{1em}
    ``\fnname{P} prints no more than two \fnname{2}s''
  \end{center}
\end{frame}

\begin{frame}{What is topology?}
  \Large
  \begin{center}
    ``$P$ is an \alert{observable property}''

    \vspace{2em}
    $\leftrightarrow$
    \vspace{2em}

    There exists a prefix $i$ of the output $\sigma$ at which $P$ is \alert{verified} to
    satisfy $P$: all extensions of $\sigma_i$ satisfy $P$.
  \end{center}
\end{frame}

\begin{frame}{What is topology?}
  Let $\phi_0, \ldots \phi_n$ be a \alert{finite} number of observable properties.

  \vspace{1em}

  Suppose $\phi_0 \wedge \ldots \wedge \phi_n$ holds.

  \vspace{1em}

  There must be \alert{stages} $i_0, \cdots, i_n$ such that $\phi_k$ is verified at stage $i_k$.

  \vspace{1em}

  $\phi_0 \wedge \cdots \wedge \phi_n$ must then be verified at $\max(i_0, \cdots, i_n)$.

  \vspace{1em}

  $\phi_0 \wedge \cdots \wedge \phi_n$ is \alert{observable}.
\end{frame}

\begin{frame}{What is topology?}
    Let $\{~\psi_i ~|~ i \in I~\}$ be an \alert{arbitrary} number of observable properties.

    \vspace{1em}

    Suppose $\bigvee_i \psi_i$ holds.

    \vspace{1em}

    Some $\psi_k$ must hold so there exists a stage $i$ at which $\psi_k$ is verified.

    \vspace{1em}

    $\bigvee_i \psi_i$ is hence verified at stage $i$.

    \vspace{1em}

    This means that $\bigvee_i \psi_i$ is \alert{observable}.
\end{frame}

\begin{frame}{What is topology?}
  \Huge
  \begin{center}
    Topology is a mathematical theory of \alert{observable} properties.
  \end{center}
\end{frame}

\begin{frame}{Frames}
  A \alert{frame} is a poset $\mathcal{O}$ such that
  \begin{itemize}
    \item \alert{finite subsets} of $\mathcal{O}$ have \alert{meets},
    \item \alert{all subsets} of $\mathcal{O}$ have \alert{joins}, and
    \item binary meets distribute over arbitrary joins:
      \begin{equation*}
        a \wedge \left( \bigvee_{i~\in~I} b_i \right) = \bigvee_{i~\in~I} \left( a \wedge b_i \right),
      \end{equation*}
      for any $a \in \mathcal{O}$ and $I$-indexed family $b$ over $\mathcal{O}$.
  \end{itemize}

  \uncover<2>{
    {\large
      In \alert{type theory}, the quantification over subsets is problematic.
    }
  }
\end{frame}

\begin{frame}{Formal Topologies}
  \large
  \begin{quote}
    ``What is formal topology? A good approximation to the correct answer is: formal
    topology is topology as developed in (Martin-Löf's) type theory.''
  \end{quote}

  \vspace{2em}

  \uncover<2>{
    One way to interpret frames \alert{predicatively} is to restrict attention to their
    presentations as \alert{formal topologies}.
  }
\end{frame}

\begin{frame}{Formal Topologies --- as Interaction Systems}
  \large

  An \alert{interaction system} on some type $A$ comprises three functions.

  \begin{align*}
    B  &\quad:\quad A \rightarrow \univ{}                              \\
    C  &\quad:\quad \pity{a}{A}{B(a) \rightarrow \univ{}}              \\
    d  &\quad:\quad \pity{a}{A}{\pity{b}{B(a)}{C(a, b) \rightarrow A}}
  \end{align*}
\end{frame}

\begin{frame}{Formal Topologies --- as Interaction Systems}
  \large

  A \alert{formal topology} is an interaction system $(B, C, d)$ on some poset $P$ that
  satisfies the following two conditions.

  \begin{enumerate}
    \item \textbf{Monotonicity}: $\forall a~b~c.~d(a, b, c) \sqsubseteq a$.
    \item \textbf{Simulation}:
  \end{enumerate}
\end{frame}

\begin{frame}{Formal Topologies --- as Interaction Systems}
\end{frame}

\end{document}
