\documentclass[11pt]{article}
\usepackage[margin=1in]{geometry}
%% \usepackage{placeins}
%% \usepackage{amsmath}
%% \usepackage{amsthm}
%% \usepackage{unicode-math}
%% \usepackage{fontspec}
%% \usepackage{todonotes}

%%  LocalWords: LocalWords Pagella Asana Gyre PragmataPro

%% \setmainfont{TeX Gyre Pagella}
%% \setmathfont{Asana Math}
%% \setmonofont[Scale=0.85]{PragmataPro}

\title{Master's Thesis Proposal}
\author{Ayberk Tosun}

\begin{document}
\maketitle

\section{Introduction}

Construction of a semantic model for the untyped $\lambda$-calculus had eluded computer
scientists for quite some time when the first solution was given by Dana Scott in
1969~\cite{scott:1969}; he termed the mathematical structures which can be appropriately
regarded as the model of $\lambda$-calculus \emph{domains} hence giving rise to what is now
known as \emph{domain theory}. The key insight in Scott's solution was the formulation of
an \emph{information ordering} between terms of the $\lambda$-calculus that orders them with
respect to ``how defined'' they are and then focus on functions that are \emph{continuous}
with respect to this ordering.

Scott initially formulated this story in the language of order theory. Even though this
approach is elegant and enables a smooth development of the required constructions (for
purposes of semantics), it doesn't express the underlying ideas in their full generality.
To put it in Scott's words~\cite[pg.~577]{scott:1982}:

\begin{quote}
  I feel I made a mistake in 1969 in using Lattice Theory as a mode of
  presentation---a mistake as far as Computer Science is concerned.
\end{quote}

It is for this reason that Scott tried another presentation in 1981 (???) formulating the
notion of \emph{neighborhood systems}, making a direct link with the underlying
topological insights. Even though this approach turned out to simplify the presentation,
Scott thought they left ``too much implicit'' and hence attempted a yet another
formulation of the idea of domains, formulating this time the notion of information
systems in 1982~\cite{scott:1982}.

\bibliography{references}
\bibliographystyle{alpha}

\end{document}
