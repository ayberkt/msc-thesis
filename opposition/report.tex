\documentclass[a4paper, 11pt]{article}

\usepackage[margin=1in]{geometry}

\usepackage{fontspec}
\setmainfont{Liberation Serif}
\setsansfont{Liberation Sans}
\usepackage{newtxtext,newtxmath}

\begin{document}
\pagenumbering{gobble}

\begin{center}
\Large Opposition report\\ 

\vspace{5pt}

\large Ayberk Tosun (MPALG)\\
\texttt{ayberk@student.chalmers.se}
\end{center}

\normalsize

\noindent Thesis: \textbf{A Reflexive Graph Model of Sized Types}.

\noindent Author of Thesis: \textbf{Jannis Limperg}.

\section{Summary}

%% One small paragraph summary of the thesis.
%% What is the topic?
%% What is the type of research done?
%% What is the main result?
The author develops a \emph{rational reconstruction} of the sized types technique used in
the Agda proof assistant for ensuring that no unbounded recursion takes place. Upon
pointing out that the complexity of sized types has resulted in bugs and inconsistencies
in Agda, the author goes on to study sized types by introducing a small calculus of sized
types that he calls $\lambda$ST and focuses on the question of \emph{size irrelevance} for it:
the requirement that sizes in types should not affect the runtime behaviour of programs.
The main result is the construction of a parametric denotational model for $\lambda$ST which
stands as evidence for size irrelevance.

%% One small paragraph on what you liked about the thesis.
%% What was well done in your point of view?
The thesis, in general, provides a clear exposition of the subject matter that is precise
yet easy to follow. Taking into account that it is not an easy task to ``informalise''
mathematics developed in the \emph{extremely} rigorous setting of a formal proof
assistant, the author accomplishes this task of presenting his denotational model
concisely yet precisely. Furthermore, the focus on the denotational study of size
irrelevance is a noteworthy contribution that will certainly be useful for showing size
irrelevance (with respect to operational semantics) in more realistic languages.

\section{Points of Critique}\label{sec:critique}

Our main point of critique in the thesis is the lack of proofs about syntactically
formulated properties of $\lambda$ST which is acknowledged by the author in Section 7.2,
``future work''. Of course, it is a significant contribution to construct such a model of
sized types but it was among our expectations to see the model connected back to syntactic
notions. For any denotational model, we believe that questions of \emph{computational
  adequacy}, for instance, should be considered to at least clarify the nature of the
model that helps one investigate other properties (in this case: size irrelevance). As the
author notes, the ideal accomplishment would be to connect the model back to an
operational notion of size irrelevance but even without this, some connections to the
operational behaviour could have been worked out.

One may also expect to see a better account of the the inconsistencies sized types have
led to in Agda. For instance, the author writes (in pg. 6)
\begin{quote}
  The culprit seems to be the highly dubious rule $n < \infty$, in particular $\infty < \infty$.
  This rule makes the $<$ relation obviously non-well-founded (\ldots) but Agda assumes that
  $<$ is well-founded.
\end{quote}
To the reader not familiar with the practical implementations of sized types, this remark
sounds bizarre: how can Agda assume something that is clearly false? What is even more
confusing is the phrasing of ``the culprit seems to be'': is it really not well-understood
what these problems were caused by?

Again in pg. 10, the author writes:
\begin{quote}
  This deviation is probably necessary considering that $\infty < \infty$ seems to be the root cause
  of multiple inconsistencies in Agda (see Sec. 2.2).
\end{quote}
Again, it is striking to us that there is no certainty regarding what exactly \emph{is}
the root cause of the problem and what exactly it is that needs to be done (rather than
``probably'') to prevent them.

If it was really caused by $\infty < \infty$ then why is it still in Agda? How was it permitted in
the first place? Is it the case that it is harmless in a different setting and does not
play well with something in Agda? If yes, what is that thing? We think all these questions
are confusing to readers not familiar with the history of sized types in Agda and they
should be answered a bit more explicitly.

On a related note: it would have been a nice experiment to remove $\star$ from $\lambda$ST and add
the rule $∞ < ∞$ instead and then show precisely what goes wrong by exhibiting a $\lambda$ST
program that does not terminate. In this way, $\lambda$ST would also serve the task of
clarifying what exactly goes wrong after we add the rule $\infty < \infty$ and how it is remedied by
making the $<$ relation well-founded.

\section{List of Questions}

The points of critique mentioned in Sec. \ref{sec:critique} will be brought up as
questions. We enumerate below some additional questions.

\begin{enumerate}
  \item The fact that $\uparrow \infty$ and $\uparrow \star$ are not less than any other size (pg. 10) seems a
    bit strange. The author remarks (again, in pg. 10): ``we could alternatively disallow
    them via the size typing rules, but would add much tedious code to the
    formalisation''. Does the author think that these might be problematic in other
    contexts? Might there be a system that naturally rules these out?
  \item Following the previous question, the author writes: ``our definition of $<$
    renders sizes of the form $\uparrow \ldots \uparrow \infty$ essentially useless: they are not less than any
    other size and therefore cannot be used in most situations'' (pg. 10). It seems to us
    that they cannot be used not in ``most situations'' but really in all situations. Can
    the author provide an example of a situation where such a size is useful?
  \item The author notes in pg. 13 that ``size quantification works essentially like
    quantification in System F''. Is this an analogy that goes deep? For instance, would
    it have been possible to define $\lambda$ST as a special case of System F.
  \item It might also be interesting to prove that $\lambda$ST is equivalent to
    $\lambda$-calculus (\emph{without} any kind of fixed point operator) augmented with the
    corresponding induction/coinduction principles. Strong normalisation, which the author
    mentions he did not have the time for, could be a corollary of this.
  \item In pg. 25, the author notes that he can prove isomorphic PRGraph families to be
    propositionally equal using univalence. Is univalence used elsewhere? In general, in
    what other ways does the development benefit from being carried out in a univalent
    setting.
\end{enumerate}

\end{document}
