\chapter{Frames}\label{chap:frames}

In this chapter, we develop the notion of a frame in \UF{}. As we explained in
Chapter~\ref{chap:intro}, a frame is like a system of finite observations.

\paragraphsummary{Topology from a CS perspective (1).}
Let us now motivate the notion of a frame \emph{directly} from a computer science
perspective. This is a story told by many others including Vickers~\cite{vickers} and
Escardó~\cite{synthetic-topology}. We will merely relay this story. Consider a black box
computing device that produces an infinite stream of bits. We can collect all possible
outputs of this device in the type $\mathsf{Bool}^{\mathbb{N}}$. We can talk about the
type of properties on these bitstreams i.e., the power set:
\begin{equation*}
  \pow{\mathsf{Bool}^{\mathbb{N}}} \quad\equiv\quad \mathsf{Bool}^{\mathbb{N}} \rightarrow \mathsf{Bool}.
\end{equation*}
Certain ones among these properties are distinguished in that they are \emph{finitely
observable} i.e., \emph{if} they hold, the fact that they hold is ascertained after
observing a finite number of digits. A property that is not observable is:
\begin{quote}
  ``the stream contains no $0$s'',
\end{quote}
because we can wait for an arbitrarily long period of time without seeing a $0$ but the
possibility that the stream contains a $0$ \emph{will not be ruled out}. A property that
is observable is:
\begin{quote}
  ``the stream contains at least one $0$ and one $1$''.
\end{quote}
Given a stream $s$ satisfying this property, there must be a finite prefix of $s$ that
satisfies this property i.e., a point in time at which an observer witness the emission of
one $0$ and one $1$. At this point in time, the fact that this sequence satisfies this
property will be ascertained in the sense that, in any possible way that things might
unfold after this stage (i.e., whatever bits might come after), this sequence will
continue to satisfy the property; no ramification thereafter can make this sequence not
satisfy the property.

Let us summarise this in the form of an informal definition:
\begin{quote}
  An observable property is a property such that if it holds, the fact that it holds can
  be ascertained by a finite examination of the data.
\end{quote}
Something important to note here is: we are not saying anything about properties that do
not hold. In other words, we are interested only in positive outcomes and not negative
outcomes.

\paragraphsummary{Topology from a CS perspective (2).}
Continuing the example of bitstreams, suppose that we have $n$ observable properties:
$\phi_0, \ldots, \phi_{n-1}$. Consider the conjunction $\phi_0 \wedge \ldots \wedge \phi_{n-1}$. Suppose that it holds.
Since all of $\phi_0, \ldots, \phi_{n-1}$ are observable, there must be corresponding indices $i_0,
\ldots, i_{n-1}$ such that property $\phi_j$ is observed to hold at stage $i_j$. If we wait until
$\max \{ i_0, \ldots, i_{n-1} \}$, we will then be able to witness the satisfaction of each of
$\phi_0, \ldots, \phi_{n-1}$ meaning we will be able to witness the satisfaction of
$\phi_0 \wedge \ldots \wedge \phi_{n-1}$; this means that it is observable.

\paragraphsummary{TODO.}
Conversely, if we have an arbitrary family $\{ \phi_i \}_{i \in I}$ of observable properties,
\begin{equation*}
  \bigvee_i \phi_i
\end{equation*}
will hold if there is some $k \in I$ such that $\phi_k$ holds. This means that if it holds,
some property $\phi_j$ will hold at a point $n \in \nats{}$. Notice that the truth of $\bigvee_i \phi_i$
will then be satisfied at $n$ as well since we will know that at least one of its
disjuncts is satisfied. In short, arbitrary disjunctions of observable properties are
observable.

\paragraphsummary{TODO.}
This is to say that the class of observable properties forms a topology, being closed
under finite intersection and arbitrary union. A frame is a system of finite observations
in this sense: we want to study things that behave like finitely justified observations
exactly in this way, so we construct an algebra of finitely justified observations with no
apparent need to talk about ``sets of points'' of any sort. In our case, the points are
the programs producing these infinite bitstreams whose internal structures are not
accessible to us due to which we are obligated understand them only through finite
approximations. We construct, \emph{directly}, an algebra of how these observations behave
rather than trying to define the observations as sets of such points.

There is one more law that needs justification. Suppose that we have the
observable property $\phi$ and the family of observable properties $\psi_0, \psi_1, \ldots$. Consider
the property:
\begin{equation*}
  \phi \wedge (\bigvee_i \psi_i).
\end{equation*}
If this property holds, $\phi$ holds and some $\psi_k$ holds meaning then the following holds.
\begin{equation*}
  \bigvee_i (\phi \wedge \psi_i).
\end{equation*}
The converse can also be immediately seen to be the case as well. This is a law we need
for our notion of a frame as well.

\paragraphsummary{Structure of chapter.}
We now start presenting our development of frames. We start with partially ordered sets in
Section~\ref{sec:poset}, which underlie frames. In Section~\ref{sec:frame}, we present the
definition of a frame. In Section~\ref{sec:frame-univ}, we present an important theorem
unique to \UF{}: isomorphic frames are equal. In Sections \ref{sec:down-set-frame} and
\ref{sec:nuclei}, we prove two important theorems in preparation for the succeeding
Chapter~\ref{chap:formal-topo} on formal topology: (1) the set of downward-closed subsets
of a poset forms a frame and (2) given a nucleus on a frame, its set of fixed points is
itself a frame.

\section{Partially ordered sets}\label{sec:poset}

\begin{defn}[Poset]
  Let $\order{A} \is A \rightarrow A \rightarrow \Omega$. A poset is then defined as
  \begin{equation*}
    \mathsf{Poset} \quad\is\quad \sigmaty{A}{\univ}{\posetstr{A}}
  \end{equation*}
  where
  \begin{align*}
    \posetstr{A} &\quad\is\quad \sigmaty{R}{Order(A)}{\posetax{A, R}}\\
    \posetaxnm \quad:&\quad \pity{A}{\univ}{Order(A) \rightarrow \univ}\\
    \posetax{A, R} \quad\is&\quad ~\pity{x}{A}{R(x, x)}\\
                      \times&~\pity{x~y~z}{A}{R(x, y) \rightarrow R(y, z) \rightarrow R(x, z)}\\
                      \times&~\pity{x~y}{A}{R(x, y) \rightarrow R(y, x) \rightarrow x =_A y}\\
                      \times&~\isaset{A}
  \end{align*}
\end{defn}

\paragraphsummary{Clarify notation.}
Given a poset $P$, we will refer to its relation as $\_\sqsubseteq_P\_$ (in cases there might be
ambiguity) and the underlying set of $P$ as $\abs{P}$. Notice that the fourth component of
$PosetAx(A, R)$ requires the carrier set to be an h-set.

Given a poset $P$ we will talk about its \emph{downward-closed subsets}: sets
that include all elements below their elements. As mentioned, we view the
elements of our poset as \emph{finite} observations. By this perspective, we
will view a downward-closed subset as \emph{general observation}. TODO: explain
better and give examples

This notion of a downward-closed subset is expressed formally in the following
definition.

\begin{defn}[Downward-closed sets]
  We first define a predicate expression that a given subset of $P$ is downward closed:
  \begin{align*}
    DownwardClosed    &:  \pow{\abs{P}} \rightarrow \Omega\\
    DownwardClosed(U) &\is{} \pity{x~y}{\abs{P}}{x \in U \rightarrow y \sqsubseteq x \rightarrow y \in U}.
  \end{align*}
  Propositionality follows directly from the propositionality of $y \in U$.

  We then define the type of downward-closed subsets of a poset as:
  \begin{align*}
    \dcsubsetnm{} &: Poset \rightarrow  \univ\\
    \dcsubset{P}  &\is{} \sigmaty{U}{\pow{\abs{P}}}{IsDownwardClosed(P, U)}
  \end{align*}
\end{defn}

The codomain of $DownwardClosedSubset$ is actually $\Omega$. However, this merits an explicit
mention.

\begin{prop}\label{isSetDCSubset}
  $\dcsubset{P}$ is an h-set.
\end{prop}
\begin{proof}
  TODO.
  Follows from Proposition \ref{isOfHLevelSigma}.
\end{proof}

In fact, the type of downward-closed subsets of a given poset is itself a poset
when ordered under the set inclusion relation.

\begin{thm}(Poset of downward-closed subsets)
  Let $P$ be a poset. The type $\dcsubset{P}$ forms a poset under the
  inclusion relation.
\end{thm}
\begin{proof}
  The fact that $\dcsubset{P}$ is a set is given by
  Proposition \ref{isSetDCSubset} so it suffices to show that the poset axioms are
  satisfied. Reflexivity and transitivity are immediate. For antisymmetry, let $U,
  V \in \pow{\abs{P}}$ and assume $U \subseteq V$, $V \subseteq U$. By function extensionality, it suffices
  to show that for every $x : \abs{P}$, $U(x) = V(x)$. Since $U(x), V(x) : \Omega$, it is
  sufficient to show $U(x) \leftrightarrow V(x)$ which is immediate by assumptions.
\end{proof}

\subsection{Monotonic functions}

The morphisms between two partially ordered sets are monotonic functions.

\begin{defn}[Monotonic function]
  Let $P, Q$ be posets. A function $f : \abs{P} \rightarrow \abs{Q}$ is monotonic if
  \begin{equation*}
    \ismonotonic{f} \quad\is\quad \pity{x}{\abs{P}}{x \sqsubseteq_P y \rightarrow f(x) \sqsubseteq_Q f(y)}.
  \end{equation*}
  We collect the type of monotonic function between $P$ and $Q$ in the following type:
  \begin{equation*}
    \monotonicmap{P}{Q} \quad\is\quad \sigmaty{f}{\abs{P} \rightarrow \abs{Q}}{\ismonotonic{f}}
  \end{equation*}
\end{defn}

\begin{defn}[Poset isomorphism]
  An isomorphism between two posets is a monotonic function with a monotonic inverse.
\end{defn}

\section{Least upper bounds and greatest lower bounds}

\section{Definition of a frame}\label{sec:frame}

We now proceed to define frames. As discussed previously, a frame is the algebraic
embodiment of the lattice of open sets. Accordingly, it has all finite meets and all
joins. Furthermore, there is one other axiom that we must require, which states the binary
meets distribute over arbitrary joins.

Notice that in a topological space, the following law holds:
\begin{equation*}
  U \cap \left( \bigcup V_0, V_1, \ldots \right) = \bigcup\left\{ (U \cap V_0), (U \cap V_1), (U \cap V_2) , \ldots \right\}.
\end{equation*}
due to set-theoretical properties. In the pointless situation, however, we must explicitly
require this.

\begin{defn}[Frame]\label{defn:frame}
  A frame structure on some type $A$ consists of (1) a poset structure, (2) a top element
  (3) a binary meet operation, and (4) a join operation of arbitrary arity, which we
  define using families:
  \begin{equation*}
    \rawframestr{A} \quad\is\quad \posetstr{A} \times A \times (A \rightarrow A \rightarrow A) \times (\sub{A} \rightarrow A).
  \end{equation*}
  This raw structure must be subject to the following axioms
  \begin{align*}
    \frameax{\sqsubseteq, \top, \wedge, \bigvee} \quad&\is\quad
      \mathsf{IsTop}(\top) \times \mathsf{IsGLB}(\wedge) \times \mathsf{IsLUB}\left(\bigvee\right)\\
    \mathsf{IsTop}(\top) \quad&\is\quad \pity{x~~}{A}{x \sqsubseteq \top}\\
    \mathsf{ISGLB}(\wedge) \quad&\is\quad \pity{x~y}{A}{(x \wedge y \sqsubseteq x) \times (x \wedge y \sqsubseteq y)}\\
                       &\hspace{0.7em}\times\quad \pity{z~~}{A}{(z \sqsubseteq x) \times (z \sqsubseteq y) \rightarrow z \sqsubseteq x \wedge y}\\
    \mathsf{IsLUB}\left(\bigvee\right) \quad&\is\quad \text{\todo{Complete.}}
  \end{align*}
\end{defn}

\section{Isomorphic frames are equal}\label{sec:frame-univ}

\todo{
  Prove that isomorphic frames are equal using the structure identity principle developed
  in Section~\ref{sec:sip}. This will consist in showing that definition of a frame with
  frame isomorphism forms a standard notion of structure and that frame axioms are
  propositions.
}

\section{Frame of downward-closed subsets}\label{sec:down-set-frame}

We have already shown how to construct the poset of downward-closed subsets from a given
poset. We will now show that this in fact forms a frame. We mentioned several times
throughout the previous discussion that a frame is a like system of finite observations.
We will interpret this frame of downward-closed subsets as generalising a given set of
observable properties.

\begin{thm}
  Given a poset $P$, its poset of downward-closed subsets forms a frame.
\end{thm}
\begin{proof}
  We start by defining the following $\top, \wedge, and \bigvee$ operations:
  \begin{align*}
    \top       \quad&\is\quad \top_A   && \text{(as in Definition~\ref{defn:full-set})}               \\
    U \wedge V   \quad&\is\quad U \cap V && \text{(as in Definition~\ref{defn:set-intersection})}       \\
    \bigvee \bF{} \quad&\is\quad \lambda x.~ \trunc{\sigmaty{i}{\indexset{\bF{}}}{x \epsilon \bF{}_i}}
  \end{align*}
  Notice that $\top$ and $\cap$ can immediately be seen to be propositional where as $\bigvee$
  requires a truncation to be forced to be propositional.

  We now proceed to show that these subsets we have defined are indeed downward-closed.
  \begin{itemize}
    \item $\top_A$ is trivially downward-closed.
    \item Downward-closure of $U \cap V$ follows immediately from the downward closure of $U$
      and $V$.
    \item This is also immediate: if $x$ is in $\bigvee \bF{}$ where $\bF{}$ is a family of
      downward-closed subsets, $x$ must belong to a particular $\bF{}_i$ which will contain
      any $y \sqsubseteq x$ meaning $\bigvee \bF{}$ itself will contain any $y \sqsubseteq x$.
  \end{itemize}
  The LUB and GLB properties follow from the fact that subsets of a type form a lattice
  (\todo{should this be explicitly proven?}).

  It remains to be shown that the distributivity law is satisfied. Let $U$ be a downward
  -closed subsets nd $\bF{}$ be a family of downward-closed subsets. We must show
  \begin{align*}
    U \cap \left(\lambda x.~ \trunc{\sigmaty{i}{\indexset{\bF{}}}{x \epsilon \bF{}_i}}\right) &=
      \bigvee \left( \left( \lambda x.~ \trunc{\sigmaty{i}{\indexset{\bF{}}}{x \epsilon \bF{}_i}}\right) \cap U \right)\\
      &= \bigvee \left( \lambda x.~ \trunc{\sigmaty{i}{\indexset{\bF{}}}{x \epsilon \bF{}_i}} \cap x \epsilon U \right)
  \end{align*}
  which follows from antisymmetry.
\end{proof}

\section{Nuclei and their fixed-points}\label{sec:nuclei}

To prepare for formal topology, we will now define a technical notion called a
\emph{nucleus}.

\begin{defn}[Nucleus]\label{defn:nucleus}
  Let $F : Frame$ and $j : \abs{F} \rightarrow \abs{F}$ and endofunction on it. We say that $F$ is
  nuclear if the following condition holds:
  \begin{align*}
    \isnuclear{F} \quad&\is\quad \pity{x~y}{\abs{F}}{j(\meet{x}{y}) = \meet{j(x)}{j(y)}}       \\
                   &\hspace{0.6em}\times\quad \pity{x~~}{\abs{F}}{x \sqsubseteq j(x)}                     \\
                   &\hspace{0.6em}\times\quad \pity{x~~}{\abs{F}}{j(j(x)) \sqsubseteq j(x)}.
  \end{align*}

  A nucleus is then just a $\sum$ type collecting nuclear endofunctions on a frame:
  \begin{equation*}
    \mathsf{Nucleus} \quad\is\quad \sigmaty{f}{\abs{F} \rightarrow \abs{F}}{\isnuclear{F}}.
  \end{equation*}
\end{defn}

\begin{prop}
  Every nucleus is monotonic.
\end{prop}
\begin{proof}
  Let $F$ be a frame and $j : \abs{F} \rightarrow \abs{F}$ a nucleus on it.
  Let $x~y : \abs{F}$ and suppose $x \sqsubseteq y$. We need to show that $j(x) \sqsubseteq j(y)$.
  First, notice that $x = \meet{x}{y}$ by antisymmetry since $x \sqsubseteq \meet{x}{y}$ by the fact
  that $\meet{\_}{\_}$ is a greatest lower and $\meet{x}{y} \sqsubseteq y$ by the fact that it is a
  lower bound. The result can now be derived as follows:
  \begin{align*}
    j(x) &\quad\sqsubseteq\quad j(\meet{x}{y})                 && [x = \meet{x}{y}]                       \\
         &\quad\sqsubseteq\quad \meet{j(x)}{j(y)}              && [\text{meets are preserved}]            \\
         &\quad\sqsubseteq\quad {j(y)}                         && [\text{\meet{\_}{\_} is a lower bound}]  .
  \end{align*}
\end{proof}

\begin{prop}
  The set of fixed points of a nucleus forms a poset.
\end{prop}
\begin{proof}
  \todo{
    Complete this rather trivial proof. The only interesting thing in it is antisymmetry.
  }
\end{proof}

Now, we are ready to prove the main theorem of this section.

\begin{thm}\label{thm:fixed-point-frame}
  The set of fixed points for a nucleus forms a frame.
\end{thm}
\begin{proof}
  \todo{
    Write and explain this proof completely and precisely.
  }
\end{proof}

%% As a brief digression, let us note that the reliance on the notion of ``black box''
%% perhaps gives the impression that this is a bizarre scenario that is not likely to occur
%% in real programming. We would like to stress that this is not the case. A program that the
%% programmer does not \emph{completely} grasp is essentially a black box; the distance
%% between the programmer's mental model of it and its actual behaviour has to be
%% investigated via experimentation. This is none other than the familiar process of
%% \emph{debugging a program} that is perhaps the only thing common to all kinds of
%% programming; programs do not come to exist in isolation, they come to exist through the
%% creative effort of a subject through the dialectical process of debugging~\cite{PDD}. This
%% discussion involving a black box can hence be taken to be about a program that behaves in
%% a way the programmer does not expect it to, meaning the programmer does not really know
%% ``what is inside''. Hence, the programmer must experiment with to understand the problem.
