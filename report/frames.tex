\chapter{Frames}\label{chap:frames}

We start by defining frames.

% TODO: recap the underlying idea of a frame.

\section{Partially ordered sets}

\begin{defn}[Poset]
  Let $Order(A) \is A \rightarrow A \rightarrow \Omega$. A poset is then defined as
  \begin{equation*}
    Poset \is \sigmaty{A}{\univ}{\posetstr{A}}
  \end{equation*}
  where
  \begin{align*}
    \posetstr{A} &\is \sigmaty{R}{Order(A)}{PosetAx(A, R)}\\
    PosetAx : &\pity{A}{\univ}{Order(A) \rightarrow \univ}\\
    PosetAx(A, R) \is &~\pity{x}{A}{R(x, x)}\\
                      \times&~\pity{x~y~z}{A}{R(x, y) \rightarrow R(y, z) \rightarrow R(x, z)}\\
                      \times&~\pity{x~y}{A}{R(x, y) \rightarrow R(y, x) \rightarrow x =_A y}\\
                      \times&~IsSet(A)
  \end{align*}
\end{defn}

Given a poset $P$, we will refer to its relation as $\_\sqsubseteq_P\_$ (in cases there might be
ambiguity) and the underlying set of $P$ as $\abs{P}$ Notice that the fourth component of
$PosetAx(A, R)$ requires the carrier set to be an h-set.

Given a poset $P$ we will talk about its \emph{downward-closed subsets}: sets
that include all elements below their elements. As mentioned, we view the
elements of our poset as \emph{finite} observations. By this perspective, we
will view a downward-closed subset as \emph{general observation}. TODO: explain
better and give examples

This notion of a downward-closed subset is expressed formally in the following
definition.

\begin{defn}[Downward-closed sets]
  We first define a predicate expression that a given subset of $P$ is downward closed:
  \begin{align*}
    DownwardClosed    &:  \pow{\abs{P}} \rightarrow \Omega\\
    DownwardClosed(U) &\is{} \pity{x~y}{\abs{P}}{x \in U \rightarrow y \sqsubseteq x \rightarrow y \in U}.
  \end{align*}
  Propositionality follows directly from the propositionality of $y \in U$.

  We then define the type of downward-closed subsets of a poset as:
  \begin{align*}
    \dcsubsetnm{} &: Poset \rightarrow  \univ\\
    \dcsubset{P}  &\is{} \sigmaty{U}{\pow{\abs{P}}}{IsDownwardClosed(P, U)}
  \end{align*}
\end{defn}

The codomain of $DownwardClosedSubset$ is actually $\Omega$. However, this merits an explicit
mention.

\begin{prop}\label{isSetDCSubset}
  $\dcsubset{P}$ is an h-set.
\end{prop}
\begin{proof}
  TODO.
  Follows from Proposition \ref{isOfHLevelSigma}.
\end{proof}

In fact, the type of downward-closed subsets of a given poset is itself a poset
when ordered under the set inclusion relation.

\begin{thm}(Poset of downward-closed subsets)
  Let $P$ be a poset. The type $\dcsubset{P}$ forms a poset under the
  inclusion relation.
\end{thm}
\begin{proof}
  The fact that $\dcsubset{P}$ is a set is given by
  Proposition \ref{isSetDCSubset} so it suffices to show that the poset axioms are
  satisfied. Reflexivity and transitivity are immediate. For antisymmetry, let $U,
  V \in \pow{\abs{P}}$ and assume $U \subseteq V$, $V \subseteq U$. By function extensionality, it suffices
  to show that for every $x : \abs{P}$, $U(x) = V(x)$. Since $U(x), V(x) : \Omega$, it is
  sufficient to show $U(x) \leftrightarrow V(x)$ which is immediate by assumptions.
\end{proof}

\subsection{Monotonic functions}

TODO.

\section{Definition of a frame}

We now proceed to define frames. As discussed previously, a frame is the algebraic
embodiment of the lattice of open sets. Accordingly, it has all finite meets and all
joins. Furthermore, there is one other axiom that we must require, which states the binary
meets distribute over arbitrary joins.

Notice that in a topological space, the following law holds:
\begin{equation*}
  U \cap \left( \bigcup V_0, V_1, \ldots \right) = \bigcup\left\{ (U \cap V_0), (U \cap V_1), (U \cap V_2) , \ldots \right\}.
\end{equation*}
due to set-theoretical properties. In the pointless situation, however, we must explicitly
require this.

\begin{defn}[Frame]
  A frame structure on some type $A$ consists of (1) a poset structure, (2) a top element
  (3) a binary meet operation, and (4) a join operation of arbitrary arity, which we
  define using families:
  \begin{equation*}
    \framestr{A} \is \posetstr{A} \times A \times (A \rightarrow A \rightarrow A) \times (\sub{A} \rightarrow A).
  \end{equation*}
  This structure must be subject to the following axioms
  \begin{equation*}
    \frameax{A} \is TODO
  \end{equation*}
\end{defn}

\section{Isomorphic frames are equal}
