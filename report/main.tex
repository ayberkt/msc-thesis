\input{template/template}

\title{Formal Topology in Univalent Foundations}
\subtitle{Subtitle}
\author{Ayberk Tosun}
\newcommand{\authorname}{Ayberk Tosun}

\supervisor{Thierry Coquand}
\departmentofsupervisor{Computer Science and Engineering}

\examiner{Nils Anders Danielsson}
\departmentofexaminer{Computer Science and Engineering}

\division{Logic and Types}

\keywords{topology, domain theory}

\newcommand{\reals}{\mathbb{R}}
\newcommand{\ball}[2]{\mathfrak{A}(#1, #2)}
\newcommand{\neighbourhood}[1]{\mathbf{N}(#1)}

\newcommand{\pity}[3]{\prod_{(#1~:~#2)} #3}
\newcommand{\sigmaty}[3]{\sum_{(#1~:~#2)} #3}
\newcommand{\univ}{\mathcal{U}}
\newcommand{\abs}[1]{\left| #1 \right|}
\newcommand{\pow}[1]{\mathcal{P}\left(#1\right)}

\newcommand{\dcsubsetnm}{\mathsf{DCSubset}}
\newcommand{\dcsubset}[1]{\dcsubsetnm{}\left(#1\right)}

\newcommand{\is}{:\equiv}

\newcommand{\paragraphsummary}[1]{\paragraph*{#1}}

\setmainfont{XITS}
\setmathfont{XITS Math}

\begin{document}

\maketitlepage{}

\begin{abstract}
  \lipsum[1-2]{}
\end{abstract}

\begin{acknowledgements}
  Foo bar.
\end{acknowledgements}

\makelists{}

\chapter{Introduction}

This thesis is about topology, the branch of mathematics that studies \emph{continuous}
functions. The notion of a continuous function pervades practically all of mathematics, as
pointed out by JJ Sylvester: ``if I were asked to name, in one word, the pole star round
which the mathematical firmament revolves, the central idea which pervades the whole
corpus of mathematical doctrine, I should point to Continuity as contained in our notions
of space, and say, it is this, it is this!''~\cite[pg. 27]{armstrong-topology}. Let us
then start by considering the question of what continuity is.

A continuous function is defined in elementary courses as a function for which ``small
changes to the input result in small changes to the output''. Usually right after this,
comes the $\epsilon$-$\delta$ definition of continuity. A function $f : \reals{} \rightarrow \reals{}$ is
\emph{continuous} if
\begin{equation*}\label{eq:cont-0}
  \forall x \in \reals{}.~ \forall \epsilon > 0.~ \exists \delta > 0.~ \forall y \in \reals{}.~
    0 < | x - y | < \delta \rightarrow | f(x) - f(y) | < \epsilon.
\end{equation*}

This definition embodies the idea that, to make $f(x)$ closer than $\epsilon$ to $f(y)$, it
suffices to make $x$ closer than $\delta$ to $y$, for some certain $\delta$. Intuitively, this
expresses the property that the function does not involve any abrupt changes.

\paragraphsummary{Generalise the notion of distance.}
The pinpoint the essence of continuity, we now start generalising this. First, notice that
the function
\begin{equation*}
  (\langle x, y \rangle \mapsto | x - y |) : \reals{} \times \reals{} \rightarrow \reals{}
\end{equation*}
is just a specific notion of distance between inhabitants of the set of interest, namely,
$\reals{}$. There are many other sets whose elements can be viewed as points in a space.
For instance $\reals{}^2$ with the usual notion of distance between two points in the
two-dimensional plane
\begin{equation*}
  d((x_0, y_0) , (x_1 , y_1)) = \sqrt{| x_0^2 - x_1^2 |} + \sqrt{| y_0^2 - y_1^2 |}.
\end{equation*}
We can therefore formulate continuity for any set $X$ for which we know how to measure the
distance between any two points meaning we require an appropriate function $d : X \times
X \rightarrow \reals{}$. Such a set endowed with a distance function is called a
\emph{metric space}. Let $X$ and $Y$ be two metric spaces. Now we define continuity for
any function $f : X \rightarrow Y$:
\begin{equation*}
  \forall x \in X.~ \forall \epsilon > 0.~ \exists \delta \in X.~ \forall y.~ d_X(x, y) < \epsilon \rightarrow d_Y(f(x), f(y)).
\end{equation*}

\paragraphsummary{Transition to balls.}
We could have written Definition~\ref{eq:cont-0} in an alternative way. Given some
$x, \epsilon \in \reals{}$ define
\begin{equation*}
  \ball{x}{\epsilon} := \{ y \in \reals{}~\|~| x - y | < \epsilon \}.
\end{equation*}
Now, the following is the same as Definition~\ref{eq:cont-0}.
\begin{equation*}
  \forall x \in \reals{}.~ \forall \epsilon > 0.~ \exists \delta > 0.~ \forall y \in \reals{}.~
    y \in \ball{x}{\epsilon} \rightarrow f(x) \in \ball{f(x)}{\delta},
\end{equation*}
which could even more compactly be expressed as
\begin{equation*}\label{eq:cont-5}
  \forall x \in \reals{}.~ \forall \epsilon > 0.~ \exists \delta > 0.~ f(\ball{x}{\epsilon}) \subseteq \ball{f(x)}{\delta}.
\end{equation*}

\paragraphsummary{Approximations.}
What is the meaning of $\ball{x}{\epsilon}$? The key idea is that, given some $x \in X$,
$\ball{x}{\epsilon}$ expresses the set of things that are closer than $\epsilon$ to $x$. One intuitive
reading of this is that $\ball{x}{\epsilon}$ denotes the \emph{set of approximations of $x$ with
a degree of accuracy of $\epsilon$}. By making $\epsilon$ increasingly smaller, we obtain an
increasingly more accurate approximation. Consider, for instance, $\pi$ whose digits we
cannot fully write down. We can, however, obtain arbitrarily precise expansions of it such
as $\ball{\pi}{10^{-10}}$.

\paragraphsummary{Distance is not needed.}
To see continuity in its full generality, we now notice that the notion of distance is not
crucial to what is expressed in Property \ref{eq:cont-5}. We have made use of the distance
function just to be able to express the approximation sets. Instead, we can simply work
with a specification of what the approximation sets are. The situation is as follows: we
are fishing in some lake and we would like to construct a theory of which fish are close
to which. One approach would be to find a sonar which would give us the precise distance
between every single fish in the lake. The other approach, that is the \emph{topological}
one, is to record whenever we catch a certain kind of fish after another. We do not know
the distances between them but this does not prevent us from studying the proximity of
different kinds of fish. Topology is to (analytic) geometry what the latter approach is to
the former.

We required certain properties a distance function must satisfy so that the set on which
operates can be viewed as a space. In that case, it is guaranteed for the approximation
sets to behave well as a result of the well-behavedness of the distance function. To be
able to directly start with the approximation sets, we must axiomatise the precise
behaviour of an ``approximation set''.

% TODO: motivate and define topological spaces.

% TODO: motivate and define pointless topology.

The goal of carrying out topology in type theory presents further challenges: we must do
it not only constructively but also \emph{predicatively}. To address this problem, we will
work with presentations of frames that we will call \emph{formal topologies}. The idea is
that a formal topology is like a formal proof system that allow us to deal with our
topology as though it were a proof system.

\chapter{Foundations}

\section{Homotopy levels}

\begin{prop}\label{isOfHLevelSigma}
  $\Sigma$ types preserve h-levels.
\end{prop}

\section{Subsets}

Given a type $A$, we would like to talk about all ``subsets'' of $A$. There are
two ways we can approach this: (1) via taking the characteristic function of the
subset as the subset, and (2) via $A$-codomained functions. In univalent
foundations, both of these approaches are equal.

In this development, there will be places where we will prefer one of these two
approaches over the other for convenience. Therefore we will define both of them
and show that they are equivalent.

\subsection{Subsets as power sets}

\begin{defn}(Power set)
  Let $A$ be a type. Its power set $\pow{A}$ is defined as:
  \begin{equation*}
    \pow{A} \is A \rightarrow \Omega
  \end{equation*}
\end{defn}

\begin{prop}\label{isSetPow}
  Given any type $A$, $\pow{A}$ is an h-set.
\end{prop}

\section{Structure identity principle}

\subsection{Subsets as families}

\chapter{Frames}

We start by defining frames.

% TODO: recap the underlying idea of a frame.

\section{Partially ordered sets}

\begin{defn}[Poset]
  Let $Order(A) := A \rightarrow A \rightarrow \Omega$. A poset is then defined as
  \begin{equation*}
    Poset := \sigmaty{A}{\univ}{\sigmaty{R}{Order(A)}{PosetAx(A, R)}}
  \end{equation*}
  where
  \begin{align*}
    PosetAx : &\pity{A}{\univ}{Order(A) \rightarrow \univ}\\
    PosetAx(A, R) :=&~\pity{x}{A}{R(x, x)}\\
                   \times&~\pity{x~y~z}{A}{R(x, y) \rightarrow R(y, z) \rightarrow R(x, z)}\\
                   \times&~\pity{x~y}{A}{R(x, y) \rightarrow R(y, x) \rightarrow x =_A y}\\
                   \times&~IsSet(A)
  \end{align*}
\end{defn}

Given a poset $P$, we will refer to its relation as $\_\sqsubseteq_P\_$ (in cases there might be
ambiguity) and the underlying set of $P$ as $\abs{P}$ Notice that the fourth component of
$PosetAx(A, R)$ requires the carrier set to be an h-set.

Given a poset $P$ we will talk about its \emph{downward-closed subsets}: sets
that include all elements below their elements. As mentioned, we view the
elements of our poset as \emph{finite} observations. By this perspective, we
will view a downward-closed subset as \emph{general observation}. TODO: explain
better and give examples

This notion of a downward-closed subset is expressed formally in the following
definition.

\begin{defn}[Downward-closed sets]
  We first define a predicate expression that a given subset of $P$ is downward closed:
  \begin{align*}
    DownwardClosed    &:  \pow{\abs{P}} \rightarrow \Omega\\
    DownwardClosed(U) &\is{} \pity{x~y}{\abs{P}}{x \in U \rightarrow y \sqsubseteq x \rightarrow y \in U}.
  \end{align*}
  Propositionality follows directly from the propositionality of $y \in U$.

  We then define the type of downward-closed subsets of a poset as:
  \begin{align*}
    \dcsubsetnm{} &: Poset \rightarrow  \univ\\
    \dcsubset{P}  &\is{} \sigmaty{U}{\pow{\abs{P}}}{IsDownwardClosed(P, U)}
  \end{align*}
\end{defn}

The codomain of $DownwardClosedSubset$ is actually $\Omega$. However, this merits an explicit
mention.

\begin{prop}\label{isSetDCSubset}
  $\dcsubset{P}$ is an h-set.
\end{prop}
\begin{proof}
  TODO.
  Follows from Proposition \ref{isOfHLevelSigma}.
\end{proof}

In fact, the type of downward-closed subsets of a given poset is itself a poset
when ordered under the set inclusion relation.

\begin{thm}(Poset of downward-closed subsets)
  Let $P$ be a poset. The type $\dcsubset{P}$ forms a poset under the
  inclusion relation.
\end{thm}
\begin{proof}
  The fact that $\dcsubset{P}$ is a set is given by
  Proposition \ref{isSetDCSubset} so it suffices to show that the poset axioms are
  satisfied. Reflexivity and transitivity are immediate. For antisymmetry, let $U,
  V \in \pow{\abs{P}}$ and assume $U \subseteq V$, $V \subseteq U$. By function extensionality, it suffices
  to show that for every $x : \abs{P}$, $U(x) = V(x)$. Since $U(x), V(x) : \Omega$, it is
  sufficient to show $U(x) \leftrightarrow V(x)$ which is immediate by assumptions.
\end{proof}

\subsection{Morphisms between posets}

TODO.

\section{Frames}

\begin{defn}[Frame]
\end{defn}

\chapter{Formal Topologies}

\makebackmatter{}

\end{document}
