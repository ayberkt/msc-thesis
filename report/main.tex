\input{/home/ayberkt/academic/thesis/report/template/template}
\usepackage{agda}

\usepackage{ebproof}
\usepackage{tikz-cd}

\tikzcdset{
  arrow style=tikz,
  diagrams={>={Straight Barb[scale=0.8]}}
}

\title{Formal Topology in Univalent Foundations}
\multilinetitle{Formal Topology\\ in Univalent Foundations}
\author{Ayberk Tosun}

\supervisor{Thierry Coquand}
\departmentofsupervisor{Computer Science and Engineering}

\examiner{Nils Anders Danielsson}
\departmentofexaminer{Computer Science and Engineering}

\division{Logic and Types}

\keywords{
  topology, formal topology, pointless topology, formal space, locale, locale theory,
  frame, homotopy type theory, univalent foundations
}

\definecolor{darkgreen}{rgb}{0,0.45,0}
\definecolor{darkred}{rgb}{0.45,0,0}
\definecolor{hottviolet}{rgb}{0.45,0,0.45}
\definecolor{hottblue}{rgb}{0,0.45,0.45}

\hypersetup{
  linktoc    = page,
  colorlinks = true,
  linkcolor  = darkgreen,
  citecolor  = hottblue,
  urlcolor   = hottviolet
}

\newcommand{\reals}{\mathbb{R}}
\newcommand{\nats}{\mathbb{N}}
\newcommand{\bool}{\mathbf{Bool}}
\newcommand{\ball}[2]{\mathfrak{A}(#1, #2)}
\newcommand{\neighbourhood}[1]{\mathbf{N}(#1)}

\newcommand{\oftyI}[2]{#1\hspace{0.1mm}:\hspace{0.1mm}#2}
\newcommand{\oftyII}[3]{#1~#2:#3}
\newcommand{\refl}{\mathsf{refl}}
\newcommand{\zero}{\mathsf{zero}}
\newcommand{\suc}[1]{\mathsf{suc}\left(#1\right)}

\newcommand{\fiber}[2]{\hyperref[defn:fiber]{ \mathsf{fiber} }\left(#1, #2\right)}
\newcommand{\isequiv}[1]{\hyperref[defn:equiv]{\mathsf{isEquiv}}\left(#1\right)}
\newcommand{\idtoeqvnm}{\mathsf{idToEquiv}}
\newcommand{\idtoeqv}[1]{\idtoeqvnm{}\left(#1\right)}
\newcommand{\idequiv}{\hyperref[defn:id-equiv]{\mathsf{idEqv}}}
\newcommand{\isdec}[1]{\mathsf{isDecidable}\left(#1\right)}
\newcommand{\isdisc}[1]{\mathsf{isDiscrete}\left(#1\right)}
\newcommand{\typequiv}[2]{#1 \hyperref[defn:equiv]{\simeq} #2}
\newcommand{\logequiv}[2]{#1 \hyperref[defn:iff]{\leftrightarrow} #2}
\newcommand{\exteq}[2]{#1 \hyperref[defn:exteq]{\sim} #2}

\newcommand{\unitty}{\mathsf{Unit}}

%% Homotopy levels.
\newcommand{\iscontr}[1]{\hyperref[defn:contr]{\mathsf{isContr}}\left(#1\right)}
\newcommand{\isprop}[1]{\hyperref[defn:hprop]{\mathsf{isProp}}\left(#1\right)}
\newcommand{\isset}[1]{\hyperref[defn:hset]{\mathsf{isSet}}\left(#1\right)}
\newcommand{\isofhlevel}[2]{%
  \hyperref[defn:hlevel]{\mathsf{isOfHLevel}}\left(#1, #2\right)
}

\newcommand{\intersectnm}{\hyperref[defn:intersection]{\cap}}
\newcommand{\intersect}[2]{#1 \intersectnm{} #2}

\newcommand{\pity}[3]{\prod_{(#1~:~#2)} #3}
\newcommand{\sigmaty}[3]{\sum_{(#1~:~#2)} #3}
\newcommand{\univ}{\mathcal{U}}
\newcommand{\isaprop}[1]{\hyperref[defn:hprop]{\mathsf{isProp}}\left(#1\right)}
\newcommand{\hprop}{\hyperref[defn:omega]{Ω}}
\newcommand{\isaset}[1]{\mathsf{IsSet}\left(#1\right)}
\newcommand{\abs}[1]{\left| #1 \right|}
\newcommand{\trunc}[1]{\left\| #1 \right\|}
\newcommand{\pownm}{\hyperref[defn:pow]{\mathcal{P}}}
\newcommand{\pow}[1]{\pownm{}\left(#1\right)}
\newcommand{\sub}[2]{\hyperref[defn:fam]{\mathsf{Fam}}_{#1}\left(#2\right)}
\newcommand{\indexnm}{\mathsf{index}}
\newcommand{\indexset}[1]{\indexnm{}\left(#1\right)}
\newcommand{\pair}[2]{\langle #1 , #2 \rangle}

\newcommand{\isdcnm}{\hyperref[defn:dc-subset]{\mathsf{isDownwardsClosed}}}
\newcommand{\isdc}[1]{\isdcnm{}\left(#1\right)}
\newcommand{\dcsubsetnm}{\hyperref[defn:dc-subset]{\mathsf{DCSubset}}}
\newcommand{\dcsubset}[1]{\dcsubsetnm{}\left(#1\right)}
\newcommand{\dcframe}[1]{#1\hyperref[thm:down-set-frame]{\downarrow}}

\newcommand{\ordernm}{\hyperref[defn:poset]{\mathsf{Order}}}
\newcommand{\order}[2]{\ordernm{}_{#1}\left(#2\right)}

\newcommand{\posetstrnm}{\hyperref[defn:poset]{\mathsf{PosetStr}}}
\newcommand{\posetstr}[2]{\posetstrnm{}_{#1}\left(#2\right)}

\DeclareMathOperator{\memfamnm}{\hyperref[defn:fam-mem]{\mathtt{\epsilon}}}
\newcommand{\memfam}[2]{#1 \memfamnm #2}

\newcommand{\posetaxnm}{\hyperref[defn:poset]{\mathsf{PosetAx}}}
\newcommand{\posetax}[1]{\posetaxnm{}\left(#1\right)}

\newcommand{\poset}{\hyperref[defn:poset]{\mathsf{Poset}}}

\newcommand{\posof}[1]{\mathsf{pos}\left(#1\right)}

\newcommand{\ismonotonicnm}{\hyperref[defn:mono]{\mathsf{isMonotonic}}}
\newcommand{\ismonotonic}[1]{\ismonotonicnm{}\left(#1\right)}
\DeclareMathOperator{\mononm}{\hyperref[defn:mono-map]{\rightarrow_m}}
\newcommand{\mono}[2]{#1 \mononm{} #2}

\newcommand{\isframehomonm}{\hyperref[defn:frame-homo]{\mathsf{isFrameHomo}}}
\newcommand{\isframehomo}[1]{\isframehomonm{}\left(#1\right)}

\DeclareMathOperator{\framehomonm}{\hyperref[defn:frame-homo]{\rightarrow_f}}
\newcommand{\framehomo}[2]{#1 \framehomonm{} #2}

\newcommand{\hasmono}[1]{\hyperref[defn:mono]{\mathsf{hasMono}}\left(#1\right)}
\newcommand{\hassim}[1]{\hyperref[defn:sim]{\mathsf{hasSim}}\left(#1\right)}

\newcommand{\vermono}{monotonicity}
\newcommand{\versim}{simulation}
\newcommand{\vernucleus}{\hyperref[defn:nucleus]{nucleus}}
\newcommand{\verframe}{\hyperref[defn:frame]{frame}}
\newcommand{\verposet}{\hyperref[defn:poset]{poset}}
\newcommand{\verintrsys}{\hyperref[defn:intr-sys]{interaction system}}
\newcommand{\veragda}{\textsc{Agda}}

\newcommand{\modname}[1]{{\color{AgdaModule} \texttt{#1}}}
\newcommand{\fnname}[1]{{\color{AgdaFunction} \texttt{#1}}}

\newcommand{\rulename}[1]{{\color{darkred} \mathsf{#1}}}
\newcommand{\ruledir}{{\color{darkred} \mathsf{dir}}}
\newcommand{\rulebranch}{{\color{darkred} \mathsf{branch}}}
\newcommand{\rulesquash}{{\color{darkred} \mathsf{squash}}}

\newcommand{\fix}[2]{\hyperref[thm:fixed-point-frame]{\mathsf{fix}}\left(#1, #2\right)}

\newcommand{\isflat}[1]{\hyperref[defn:flat]{\mathsf{isFlat}}\left(#1\right)}

\newcommand{\framestrnm}{\mathsf{FrameStr}}
\newcommand{\framestr}[1]{\framestrnm{}\left(#1\right)}
\newcommand{\frameax}[1]{\hyperref[defn:frame]{\mathsf{FrameAx}}\left(#1\right)}
\newcommand{\framenm}{\hyperref[defn:frame]{\mathsf{Frame}}}
\newcommand{\framety}[3]{\framenm{}_{#1, #2, #3}}

\newcommand{\treestrnm}{\hyperref[defn:intr-sys]{\mathsf{IntrStr}}}
\newcommand{\treestr}[1]{\treestrnm{}\left(#1\right)}
\newcommand{\intrsys}{\hyperref[defn:intr-sys]{\mathsf{IntrSys}}}

\newcommand{\stumpnm}{\mathsf{Stump}}
\newcommand{\stump}[1]{\stumpnm\left(#1\right)}

\newcommand{\refines}[2]{#1~\mathcal{R}~#2}

\newcommand{\disciplinestrnm}{\mathsf{DisciplineStr}}
\newcommand{\disciplinestr}[1]{\disciplinestrnm{}\left(#1\right)}

\newcommand{\rawframestrnm}{\hyperref[defn:frame]{\mathsf{RawFrameStr}}}
\newcommand{\rawframestr}[3]{\rawframestrnm{}_{#1, #2}\left(#3\right)}

\newcommand{\isnuclearnm}{\hyperref[defn:nucleus]{\mathsf{isNuclear}}}
\newcommand{\isnuclear}[1]{\isnuclearnm{}\left(#1\right)}
\newcommand{\nucleus}{\hyperref[defn:nucleus]{\mathsf{Nucleus}}}

\newcommand{\meet}[2]{#1 \wedge #2}
\newcommand{\joinnm}[3]{\bigvee}
\newcommand{\join}[3]{\joinnm{}_{#1~:~#2} #3}

\newcommand{\RHD}{\scalebox{1.2}{{\tt ▶}}}
\DeclareMathOperator{\covernm}{\scalebox{1.2}{{\tt ◀}}}
\newcommand{\covers}[2]{#1 \covernm{} #2}

\newcommand{\setof}[1]{\left\{ #1 \right\}}
\newcommand{\img}[2]{\setof{ #1\left( a \right) ~|~ a \in #2 }}

\newcommand{\representsnm}{\hyperref[defn:rep]{\mathsf{represents}}}
\newcommand{\represents}[3]{\representsnm{}\left(#1, #2, #3\right)}

\newcommand{\bF}{\mathbf{F}}
\newcommand{\bG}{\mathbf{G}}
\newcommand{\McF}{\mathcal{F}}
\newcommand{\MfU}{\mathfrak{U}}

\newcommand{\is}{:\equiv}
\newcommand{\idnm}{~=~}

%% Names.
\newcommand{\UF}{Univalent Foundations}

%% \newcommand{\paragraphsummary}[1]{{\color{orange} \textsc{#1}}}
\newcommand{\paragraphsummary}[1]{}
\newcommand{\todo}[1]{
  {\color{red} \textsf{TODO: #1}}
}

%% \setmainfont{XITS}
%% \setmathfont{XITS Math}
\setmonofont[Scale=0.85]{PragmataPro Mono Liga}

\begin{document}

\maketitlepage{}

\begin{abstract}
  Formal topology is the mathematical discipline that aims to reconstruct topology in
  type-theoretical terms, that is, constructively \emph{and} predicatively. Type theory
  has recently undergone a transformation through insights arising from its association
  with homotopy theory, resulting in a conceptual novelty for foundations of mathematics,
  namely, the formulation of the notion of a \emph{univalent} foundation.

  We investigate, in this thesis, the natural continuation of the line of work on formal
  topology into univalent type theory. We first recapitulate our finding that a naive
  approach to develop formal topology in \UF{} is problematic, and would require a form of
  the axiom of choice. We then present a solution to this problem that involves the use of
  higher inductive types. We hence sketch the beginnings of an approach towards developing
  formal topology in \UF{}. As a proof of concept, we develop the formal topology of the
  Cantor space and construct a proof that is compact.

  Our development has been fully assisted by the cubical extension of the \veragda{} proof
  assistant. The presentation in this thesis amounts to an informalisation of this formal
  development.
\end{abstract}

\begin{acknowledgements}
  \todo{Add acknowledgements here.}
\end{acknowledgements}

\makelists{}

\chapter{Introduction}\label{chap:intro}

\paragraphsummary{State what topology is about.}
This thesis is about topology, the branch of mathematics that studies \emph{continuous}
functions. The notion of a continuous function pervades practically all of mathematics, as
pointed out by JJ Sylvester: ``if I were asked to name, in one word, the pole star round
which the mathematical firmament revolves, the central idea which pervades the whole
corpus of mathematical doctrine, I should point to Continuity as contained in our notions
of space, and say, it is this, it is this!''~\cite[pg. 27]{armstrong-topology}. Let us
then start by considering the question of what continuity is.

\paragraphsummary{$\epsilon$-$\delta$ definition of continuity.}
A continuous function, in the context of real numbers, is a function for which
``small changes to the input result in small changes to the output''. This is
formally expressed in the following ``$\epsilon$-$\delta$ definition'' of continuity: a
function March$f : \reals{} \rightarrow \reals{}$ is
\emph{continuous} if
\begin{equation*}\label{eq:cont-0}
  \forall x \in \reals{}.~ \forall \epsilon > 0.~ \exists \delta > 0.~ \forall y \in \reals{}.~
    | x - y | < \delta \rightarrow | f(x) - f(y) | < \epsilon.
\end{equation*}
This definition embodies the idea that, to make $f(x)$ closer than $\epsilon$ to
$f(y)$, it suffices to make $x$ closer than $\delta$ to $y$, for some certain $\delta$.

\paragraphsummary{Generalise the notion of distance.}
To pinpoint the essence of continuity, we will generalise this. First, notice that the
function
\begin{equation*}
  d(\pair{x}{y}) = | x - y | : \reals{} \times \reals{} \rightarrow \reals{}
\end{equation*}
is just a \emph{special} notion of distance between inhabitants of the set of interest,
namely, $\reals{}$. There are many other sets whose elements can be viewed as points in a
space. $\reals{}^2$, for instance, with the usual notion of distance between two points in
the two-dimensional plane has the corresponding distance function:
\begin{equation*}
  d(\pair{x_0}{y_0} , \pair{x_1}{y_1}) = \sqrt{(x_0 - x_1)^2 + (y_0 - y_1)^2}.
\end{equation*}
We can therefore formulate continuity for any set $X$ which admits a notion of distance
between two points. Formally, we require an appropriate function $d : X \times X \rightarrow \reals{}$.
Such a set endowed with a distance function will be called \emph{metric space} if it
satisfies certain axioms that ensure that the distance function is well-behaved i.e.,
behaves properly like a notion of distance. For instance, we may expect that $d(x, y) =
d(y, x)$ for any two points $x, y \in X$. The specifics of these axioms are not central to
our focus so we will refrain from presenting those. Let $X$ and $Y$ be two metric spaces.
Now we define continuity for any function $f : X \rightarrow Y$:
\begin{equation}\label{cont-1}
  \forall x \in X.~ \forall \epsilon > 0.~ \exists \delta \in X.~ \forall y.~ d_X(x, y) < δ \rightarrow d_Y(f(x), f(y)) < ε.
\end{equation}
Now we can express continuity of a function on any set that behaves like a space.

\paragraphsummary{Transition to balls.}
Notice that we could have written (\ref{cont-1}) in an alternative way. Given
some $x \in X, \epsilon \in \reals{}$ define
\begin{equation*}
  \ball{x}{\epsilon} \quad\is\quad \{ y \in \reals{}~|~d(x, y) < \epsilon \}.
\end{equation*}
Now, the following is the same as (\ref{cont-1}).
\begin{equation*}
  \forall x \in X.~ \forall \epsilon > 0.~ \exists \delta > 0.~ \forall y \in X.~ y \in \ball{x}{\epsilon} \rightarrow f(y) \in \ball{f(x)}{\delta},
\end{equation*}
which could be expressed even more compactly as:
\begin{equation}\label{cont-2}
  \forall x \in \reals{}.~ \forall \epsilon > 0.~ \exists \delta > 0.~ f(\ball{x}{\epsilon}) \subseteq \ball{f(x)}{\delta}.
\end{equation}

\paragraphsummary{Open balls as approximation sets.}
After having written down continuity in this more compact way, we consider the question:
what is the meaning of $\ball{x}{\epsilon}$? Given some $x \in X$, $\ball{x}{\epsilon}$ expresses the set
of things that are closer than $\epsilon$ to $x$. One intuitive reading of this is that
$\ball{x}{\epsilon}$ denotes the \emph{set of approximations of $x$ with a degree of accuracy of
$\epsilon$}. As $\epsilon$ decreases, the accuracy with which the inhabitants of $\ball{x}{\epsilon}$
represent $x$ increases. Consider, for instance, $\pi$ whose digits we cannot fully write
down. We can, however, obtain arbitrarily precise expansions of it such as
$\ball{\pi}{10^{-10}}$, if we are willing to wait long enough.

\paragraphsummary{Distance is not needed.}
To see continuity in its full generality, we now notice that the notion of distance is not
crucial to what is expressed in (\ref{cont-2}). We have made use of the distance function
just to be able to specify the approximation sets. Instead, we can work directly with a
specification of what the approximation sets are. This is the main idea underlying
topology: instead of working with a notion of distance, we work directly with a
specification of the approximations.

\paragraphsummary{Axiomatisation of basis (0).}
We mentioned that certain metric space axioms ensure that the distance function is
well-behaved. When one defines approximation sets in terms of such a distance function, it
is guaranteed that they will be well-behaved as well. As we want to completely remove our
dependency on a notion of distance, we will have to axiomatise instead the behaviour of
approximation sets themselves.

\paragraphsummary{Axiomatisation of basis (1).}
Any approximation set for a real number must that real number itself. In other words, we
want approximations to give us ``error bars'' around some result and any error bar that
does not encircle the result is clearly ill-behaved.

\paragraphsummary{Axiomatisation of basis(2).}
Furthermore, if we have two approximation sets for a real number, say $A_0$ and $A_1$,
that real number must lie in $A_0 \cap A_1$. We can then pick out a \emph{finer}, more
accurate approximation $A_2$ for the number such that $A_2 \subseteq A_1 \cap A_2$. For instance,
$A_0$ might say that $\pi$ lies within $3.0$ and $3.2$ and $A_1$ that it lies within $3.1$
and $3.5$ from which we can tell that it lies within $3.1$ and $3.2$. This will be our
second requirement.

\paragraphsummary{Formal defn.~ of basis.}
The standard term in topology for such a class of sets is a \emph{basis for a topology}.
We now summarise it precisely in the following definition.
\begin{defn}[Basis for a topology]
  Given a set $X$, a basis for a topology on $X$ is a class $\mathcal{B}$ of subsets of
  $X$ such that:
  \begin{enumerate}
    \item $\forall x \in X.~ \exists B \in \mathcal{B}.~x \in B$, which intuitively says ``approximations
      approximate'', and
    \item $\forall x \in X.~ \exists B_0, B_1 \in \mathcal{B}.~ x \in B_0 \cap B_1 \rightarrow \exists B_2 \subseteq (B_0 \cap B_1).~
      x \in B_2$, which says ``approximations can be refined''.
  \end{enumerate}
\end{defn}

\paragraphsummary{Continuity with bases.}
Now, we can reformulate continuity in a more general way using basis elements. Given sets
$X, Y$ with bases $\mathcal{B}_X, \mathcal{B}_Y$, a function $f : X \rightarrow Y$ is continuous if
\begin{equation}\label{cont-basis}
  \forall x \in X.~ \forall V \in \mathcal{B}_Y.~ f(x) \in V \rightarrow \exists U \in \mathcal{B}_X.~
    f(U) \subseteq V.
\end{equation}
This says exactly the same thing as (\ref{cont-2}), but talking directly about some given
approximation sets rather than defining these through a notion of distance. In other
words, we previously said that a function is continuous if we can make $f(x)$ closer than
$\epsilon$ to $f(y)$ by making $x$ closer than $\delta$ to $y$. We are now saying that for every
set approximating some $f(x)$, there is a set approximating $x$ such that the image of
the latter is contained in the former, which corresponds directly to (\ref{cont-2}).

Given a basis $\mathcal{B}$ on set $X$, we say that the topology generated by
$\mathcal{B}$, that we call $\mathbf{T}_{\mathcal{B}}$, is the class of all possible
unions of $\mathcal{B}$. The idea here is that an element $U$ of
$\mathbf{T}_{\mathcal{B}}$ is like a property of the set that we are interested in
\emph{verifying}. To verify $U$, it suffices to show that it falls in one of the basis
elements that constitute $U$ and since all of these are approximation sets, $U$ can be
verified from a finite amount of data since approximations suffice to express it.

\paragraphsummary{Start motivating topologies.}
In reality, it is often the case that we deal directly with the bases that generates
topologies. However, we will not have reached full generality until we can talk about
\emph{topologies} i.e., properties of a set which can be verified via approximations.
We will call such properties \emph{finitely verifiable}. The topological term is ``open
set'' in the sense that the set does not contain its own boundary.

If we have two finitely verifiable properties, we should be able to verify them from an
approximations of degrees of accuracy of $ε₀$ and $ε₁$. Then, we can verify both by taking
the higher between them so we can verify their intersection. If we have an arbitrary
number of finitely verifiable properties, to verify the union it suffices to verify at
least one, so if an point is in this set it must be in one of the finitely verifiable
subsets, meaning the set itself is finitely verifiable. This brings us to the following
definition:
\begin{defn}[Topological space]\label{defn:topospace}
  A topology on a set $X$ is a class $\mathcal{T}$ of subsets of $X$ such that
  \begin{itemize}
    \item The trivial subsets $X, \emptyset \subseteq X$ are in $\mathcal{T}$,
    \item $\mathcal{T}$ is closed under finite intersections, and
    \item $\mathcal{T}$ is closed under arbitrary unions.
  \end{itemize}
\end{defn}

\paragraphsummary{Transition to pointless topology.}
From the perspective where we view topology as a mathematical theory of finitary
approximations, it makes further sense to take the finitely verifiable properties as
primitive and then investigate them directly instead of defining them as \emph{derived
notions} that are defined as certain kinds of ``sets of points''. This is analogous to
the distinction between analytic and synthetic geometry: in the former, one defines a
space as a set of points, whereas in the latter one works directly with constructs on
spaces such as angles, lines, and circles. This approach in which we start directly with
a given set of finitely verifiable properties is called \emph{pointless topology} in the
sense that we avoid mentioning the points until we really need them.

\paragraphsummary{Motivate frames.}
In pointless topology, we start with a set $\mathcal{O}$ of opens. We then axiomatise
directly how these opens shall behave.
\begin{enumerate}
  \item Corresponding to the set-inclusion partial order in the pointful case, we require that
    there be a partial order $\_\sqsubseteq\_ \subseteq \mathcal{O} \times \mathcal{O}$.
  \item Corresponding to the fact that open sets are closed under finite intersection, we
    require that there be a binary meet operation $\meet{\_}{\_} : \mathcal{O} \times
    \mathcal{O} \rightarrow \mathcal{O}$ and a nullary one $\top : \mathcal{O}$.
  \item Corresponding to the fact that open sets are closed under arbitary union, we
    require that there be a join operation of arbitrary arity: $\joinnm{}\_ :
    \pow{\mathcal{O}} \rightarrow \mathcal{O}$.
\end{enumerate}
In addition to this, require that these operations satisfy the distributivity law, which
we will elaborate on in Chapter~\ref{chap:frames}. Such a lattice that embodies a logic
of finitely verifiable properties is called a \emph{frame}.

\paragraphsummary{Explain the benefit of pointless topology.} A question that is natural
is: what does this give us that traditional topology lacks? In this thesis, we will look
at topology from the perspective of computer science therefore we would like to be able to
understand theorems of topology \emph{in computational terms}. Topology notoriously relies
on classical reasoning in many of its fundamental theorems such as the Tychonoff theorem.
By doing pointless topology, we will see that we can avoid classical reasoning and hence
gain a computational understanding of topology. This point was put eloquently by
Johnstone~\cite[pg.~46]{stone-spaces}:
\begin{quote}
  It is here that the real point of pointless topology begins to emerge; the difference
  between locales and spaces is one that we can (usually) afford to ignore if we are
  working in a ``classical'' universe with the axiom of choice available, but when (or if)
  we work in a context where choice principles are not allowed, then we have to take
  account of the difference—and usually it is locales, not spaces, which provide the right
  context in which to do topology. This is the point which, as I mentioned earlier,
  Andr\'{e} Joyal began to hammer home in the early 1970s; I can well remember how, at the
  time, his insistence that locales were the real stuff of topology, and spaces were
  merely figments of the classical mathematician's imagination, seemed (to me, and I
  suspect to others) like unmotivated fanaticism. I have learned better since then.
\end{quote}

\paragraphsummary{Motivate formal topology.}
The goal of carrying out topology in type theory presents further challenges: we must do
it not only constructively but also \emph{predicatively}. To address this problem, we will
work with presentations of frames that are called \emph{formal
topologies}~\cite{int-formal-spaces}. The idea is that a formal topology is like a
formal proof system that allows us to deal with our topology as though it were a proof
system. In addition to the benefit of constructively understanding the results of
topology, formal topology provides us the further advantage of being able to understand
them predicatively.

\paragraphsummary{Explain the goal of the thesis.}
In this thesis, we present a development of formal topology in \UF{}. By now, it has
become clear that univalence addresses many shortcomings of type theory therefore the
question of what novelties it presents for the question of topology in type theory is a
natural one. We focus on one particular approach to formal topology, implementing an idea
going back to Coquand~\cite{coq-posets} to endow posets with an ``interaction'' structure.

\paragraphsummary{Summarise the thesis structure.}
This thesis is structured as follows. In Chapter~\ref{chap:foundations}, we summarise the
fundamentals of \UF{}. In Chapter~\ref{chap:frames}, we present our development of frames
and constructs related to them in preparation for Chapter~\ref{chap:formal-topo} in which
we present our main development of formal topology in \UF{}.


\chapter{Foundations}\label{chap:foundations}

In this chapter, we provide the preliminary lemmas of \UF{} for the sake of
self-containment. It is intended to be a summary rather than an introduction.

\section{Homotopy levels}

\begin{defn}[Contractible]
  A type $A$ is called contractible if
  \begin{equation*}
    \iscontr{A} \quad\is\quad \sigmaty{x}{A}{\pity{y}{A}{x =_A y}}
  \end{equation*}
\end{defn}

This says precisely that type $A$ has \emph{exactly one} inhabitant.

\begin{defn}
  We will say that the homotopy level of a type $A$ is $\oftyI{n}{\mathbb{N}}$ if
  \begin{align*}
    \isofhlevel{A}{\zero{}} &\quad\is\quad \iscontr{A}\\
    \isofhlevel{A}{\suc{n}} &\quad\is\quad \pity{x~y}{A}{\isofhlevel{x =_A y}{n}}.
  \end{align*}
\end{defn}

Homotopy levels of one and two are of special interest. The former is the type of
\emph{propositions} i.e., types that are like propositions in the sense that they have
trivial proof structure: they are inhabited by at most one term.

\begin{defn}[Proposition]
  A type $A$ is a proposition (sometimes disambiguated as \emph{h-proposition}) if it has
  a homotopy level of one:
  \begin{equation*}
    \isprop{A} \quad\is\quad \isofhlevel{A}{1}.
  \end{equation*}
\end{defn}

\begin{defn}
  A type $A$ is a set (sometimes disambiguated as \emph{h-set}) if it a has a homotopy
  level of two:
  \begin{equation*}
    \isset{A} \quad\is\quad \isofhlevel{A}{2}.
  \end{equation*}
\end{defn}


\chapter{Frames}\label{chap:frames}

In this chapter, we develop the notion of a frame in \UF{}. In Chapter~\ref{chap:intro},
we explained that a frame is the algebra of a logic of finitely verifiable properties.
Recall that a frame consists of the following:
\begin{itemize}
  \item a set $O$ of \emph{opens},
  \item a partial order $\_\sqsubseteq\_ \subseteq O \times O$, corresponding to the set-inclusion order of the
    open subsets,
  \item finite meets, and
  \item arbitrary joins.
\end{itemize}

In addition to these, there is a law that is needed to ensure the correct interplay
between meets and joins. Suppose that we have the observable property $\phi$ and the family
of observable properties $\psi_0, \psi_1, \cdots$. Consider the expression:
\begin{equation*}
  A \cap (\bigcup_i B_i).
\end{equation*}
where $A$ is a set and $B$ is a family of sets. By set-theoretic reasoning, this is the
same thing as:
\begin{equation*}
  \bigvee_i (\phi \wedge \psi_i).
\end{equation*}
As we are trying to characterise the behaviour of open ``sets'' without defining them as
sets of points, we have to explicitly add this distributivity law into the definition of
frame:
\begin{center}
  \emph{binary meets must distribute over arbitrary joins.}
\end{center}

As a brief digression, let us note that it is natural to consider the question: what
happens if we leave out this requirement of distributivity? The resulting structure is
called a \emph{basic topology} and is studied in the work of Sambin

\paragraphsummary{Structure of chapter.}
We now start presenting our formal development of frames. We start with partially ordered
sets in Section~\ref{sec:poset}, which underlie frames. In Section~\ref{sec:frame}, we
present the definition of a frame. In Section~\ref{sec:frame-univ}, we present an
important theorem unique to \UF{}: isomorphic frames are equal. In Sections
\ref{sec:down-set-frame} and \ref{sec:nuclei}, we prove two important theorems in
preparation for the succeeding Chapter~\ref{chap:formal-topo} on formal topology: (1) the
set of downward-closed subsets of a poset forms a frame and (2) given a nucleus (a
technical notion to be introduced) on a frame, its set of fixed points is itself a frame.

\section{Partially ordered sets}\label{sec:poset}

\begin{defn}[Poset]\label{defn:poset}
  Given some $\oftyI{A}{\univ{}_m}$, let $\order{n}{A} \is A \rightarrow A \rightarrow \hprop{}_n$. Notice the
  generality of the universes: the codomain of the relation is permitted to be on a level
  different than that of the carrier set. A poset at carrier level $m$ and relation level
  $n$ is then defined as:
  \begin{equation*}
    \mathsf{Poset}_{m, n} \quad\is\quad \sigmaty{A}{\univ_m}{\posetstr{n}{A}},
  \end{equation*}
  \begin{center}
  where
  \end{center}
  \begin{align*}
    \posetstr{n}{A} \quad\is&\quad \sigmaty{R}{\order{n}{A}}{\posetax{A, R}}\\
    \posetaxnm \quad:&\quad \pity{A}{\univ{}_m}{\order{n}{A} \rightarrow \univ_{\max(m, n)}}\\
    \posetax{A, R} \quad\is&\quad ~\pity{x}{A}{R(x, x)}\\
                      \times&~\pity{x~y~z}{A}{R(x, y) \rightarrow R(y, z) \rightarrow R(x, z)}\\
                      \times&~\pity{x~y}{A}{R(x, y) \rightarrow R(y, x) \rightarrow x =_A y}\\
                      \times&~\isaset{A}
  \end{align*}
\end{defn}

\paragraphsummary{Clarify notation.}
Given a poset $P$, we will refer to its relation as $\_\sqsubseteq_P\_$ (in cases where there might
be ambiguity) and the underlying set of $P$ as $\abs{P}$. Notice that the fourth component
of $\posetax{A, R}$ requires the carrier set to be an \hyperref[defn:hset]{h-set}.

Given a poset $P$ we will talk about its \emph{downward-closed subsets}: sets that include
all elements below their elements. This notion embodies the idea of verification at a
certain stage of information. Take a certain element $x : \abs{P}$, that we view as a
stage of information. For some other $y~:~\abs{P}$, $y \sqsubseteq x$ expresses the idea that $y$ is
a \emph{more refined} stage of information i.e., it contains more information hence ruling
out more approximations meaning it admits \emph{less}. Let $U$ be a subset of $\abs{P}$.
The property that $U$ is downward-closed is then expressed as:
\begin{equation*}
  x \epsilon U \rightarrow y \sqsubseteq x \rightarrow y \epsilon U,
\end{equation*}
the intuitive reading of which is: $U$ contains all stages that are ramifications of the
stages it contains. This means that $U$ is an \emph{observable} property: it is secured at
a certain stage in the sense that the reception of more information does not disrupt it.
Let us write this down formally.
\begin{defn}[Downward-closed subset]\label{defn:dc-subset}
  We first define a predicate expressing that a given subset of $P$ is downwards-closed:
  \begin{align*}
    \isdcnm{}   &\quad:\quad  Poset_{m, n} \rightarrow \pow{\abs{P}} \rightarrow \Omega\\
    \isdc{P, U} &\quad\is{}\quad \pity{x~y}{\abs{P}}{x \in U \rightarrow y \sqsubseteq x \rightarrow y \in U}.
  \end{align*}
  By multiple appeals to Proposition~\ref{prop:pi-prop}, it suffices to show that the
  inner-most expression inside the nested $\prod$ type is propositional which is immediate
  since the codomain of $U$ is \hyperref[defn:hprop]{propositional}. We then define the
  type of downwards-closed subsets of a poset as:
  \begin{align*}
    \dcsubsetnm{} &\quad:\quad \mathsf{Poset}_{m, n} \rightarrow \univ_{\max(m+1, n)}\\
    \dcsubset{P}  &\quad\is{}\quad \sigmaty{U}{\pow{\abs{P}}}{\isdc{P, U}}.
  \end{align*}
\end{defn}

So far we have dealt with two notions of \emph{observable property} throughout the
development:
\begin{enumerate}
  \item element of a poset which we will eventually view like pointless versions of an
    open set with the order corresponding to the subset-inclusion order, and
  \item the notion of downwards-closed subset which expresses that a property of the poset
    of opens behaves like an observational property.
\end{enumerate}
We will now start relating these two by showing that the set of downwards-closed subsets
of a poset is itself a poset, and indeed, we will prove later (in
Sec.~\ref{sec:down-set-frame}) that it actually forms a frame meaning downwards-closed
subsets satisfy our expectations from properties we view as observable.

Let us start by showing that $\dcsubset{P}$ is a set.
\begin{prop}\label{isSetDCSubset}
  $\dcsubset{P}$ is a set for every poset $P$.
\end{prop}
\begin{proof}
  By Proposition~\ref{prop:sigma-set}, it suffices to show that $\pow{\abs{P}}$ is a set
  and $$\isdc{P, U}$$ is a set for every $\oftyI{U}{\pow{\abs{P}}}$. The former holds by
  Proposition~\ref{prop:pow-set}. For the latter, observe that every $\isdc{P, U}$ is a
  proposition by definition meaning it is also set by Proposition~\ref{prop:prop-is-set}.
\end{proof}

Now we can proceed to construct the poset of downwards-closed subsets.
\begin{thm}(Poset of downward-closed subsets)
  Let $P$ be a poset. The type $\dcsubset{P}$ forms a poset under the
  inclusion relation.
\end{thm}
\begin{proof}
  The fact that $\dcsubset{P}$ is a set is given by Proposition~\ref{isSetDCSubset} so it
  suffices to show that the poset axioms are satisfied. Reflexivity and transitivity are
  immediate. For antisymmetry, let $U, V \in \pow{\abs{P}}$ and assume $U \subseteq V$, $V \subseteq U$. By
  function extensionality, it suffices to show that for every $x : \abs{P}$, $U(x) =
  V(x)$. Since $\oftyII{U(x)}{V(x)}{\Omega}$, it is sufficient to show $U(x) \leftrightarrow V(x)$ which is
  immediate
  by assumptions.
\end{proof}

\subsection{Monotonic functions}

The morphisms between two partially ordered sets are monotonic functions.

\begin{defn}[Monotonic function]
  Let $P, Q$ be posets. A function $f : \abs{P} \rightarrow \abs{Q}$ is monotonic if the following
  type is inhabited:
  \begin{equation*}
    \ismonotonic{f} \quad\is\quad \pity{x~y}{\abs{P}}{x \sqsubseteq_P y \rightarrow f(x) \sqsubseteq_Q f(y)}.
  \end{equation*}
  We collect the type of monotonic functions between $P$ and $Q$ in the following type:
  \begin{equation*}
    \monotonicmap{P}{Q} \quad\is\quad \sigmaty{f}{\abs{P} \rightarrow \abs{Q}}{\ismonotonic{f}}
  \end{equation*}
\end{defn}

\begin{defn}[Poset isomorphism]
  An isomorphism between two posets is a monotonic function with a monotonic inverse.
\end{defn}

\section{Definition of a frame}\label{sec:frame}

We now proceed to define frames.
\begin{defn}[Frame]\label{defn:frame}
  A frame structure on some type $A$ consists of (1) a poset structure, (2) a top element
  (3) a binary meet operation, and (4) a join operation of arbitrary arity, which we
  define using families:
  \begin{equation*}
    \rawframestr{n}{o}{A} \quad\is\quad \posetstr{n}{A} \times A \times (A \rightarrow A \rightarrow A) \times (\sub{o}{A} \rightarrow A).
  \end{equation*}
  This raw structure must be subject to the following axioms
  \begin{align*}
    \frameax{\sqsubseteq, \top, \wedge, \bigvee} \quad&\is\quad
      \mathsf{IsTop}(\top) \times \mathsf{IsGLB}(\wedge) \times \mathsf{IsLUB}\left( \bigvee \right)
      \mathsf{IsDistr}(\wedge, \bigvee)\\
    \mathsf{isTop}(\top) \quad&\is\quad \pity{x}{A}{x \sqsubseteq \top}\\
    \mathsf{isGLB}(\wedge) \quad&\is\quad \pity{x~y}{A}{(x \wedge y \sqsubseteq x) \times (x \wedge y \sqsubseteq y)}\\
                       &\hspace{0.5em}\times\quad \pity{z~~}{A}{(z \sqsubseteq x) \times (z \sqsubseteq y) \rightarrow z \sqsubseteq x \wedge y}\\
    \mathsf{isLUB}\left(\bigvee\right) \quad&\is\quad
         \pity{F}{\sub{n}{A}}{\pity{x}{A}{x \epsilon F \rightarrow x \sqsubseteq \bigvee_i F_i}}\\
         &\hspace{0.5em}\times \pity{F}{\sub{n}{A}}{\pity{x}{A}{
               \left( \pity{y}{A}{y \epsilon F \rightarrow y \sqsubseteq x}\right) \rightarrow \bigvee_i F \sqsubseteq x
             }}\\
    \mathsf{isDistr}(\wedge, \bigvee) \quad&\is\quad
      \pity{x}{A}{\pity{F}{\sub{n}{A}}{
          x \wedge \bigvee_i F_i} =_A \bigvee_i \left( x \wedge F_i \right)
      }
  \end{align*}
\end{defn}

\begin{prop}
  For every raw frame structure $(\sqsubseteq, \top, \wedge, \bigvee)$, $\frameax{\sqsubseteq, \top, \wedge, \bigvee}$ is a proposition.
\end{prop}
\begin{proof}[Proof sketch]
  By Proposition~\ref{prop:sigma-prop}, it suffices to show that each component is an
  h-prop. For $\mathsf{isTop}$, $\mathsf{isGLB}$, and $\mathsf{isLUB}$ this can be
  concluded by using Proposition~\ref{prop:sigma-prop} and Proposition~\ref{prop:pi-prop}.
  For $\mathsf{isDistr}$, we use Proposition~\ref{prop:pi-prop} followed by the fact that
  the underlying set of a poset is an h-set (by the definition of $\posetaxnm{}$ from
  Definition~\ref{defn:poset}).
\end{proof}

\section{Isomorphic frames are equal}\label{sec:frame-univ}

\todo{
  Prove that isomorphic frames are equal using the structure identity principle developed
  in Section~\ref{sec:sip}. This will consist in showing that definition of a frame with
  frame isomorphism forms a standard notion of structure and that frame axioms are
  propositions.
}

\section{Frame of downward-closed subsets}\label{sec:down-set-frame}

We have shown how to construct the poset of downwards-closed subsets from a given poset.
We will now show that this in fact forms a frame. We will interpret this frame of
downward-closed subsets as taking a set of stages of information, ordered with respect to
how refined they are, and then focusing on the ones that are downward-closed i.e.,
containing their ramifications.

\begin{thm}
  Given a poset $P$, its poset of downward-closed subsets forms a frame.
\end{thm}
\begin{proof}
  We start by defining the following $\top, \wedge, and \bigvee$ operations:
  \begin{align*}
    \top       \quad&\is\quad \top_A   && \text{(as in Definition~\ref{defn:full-set})}               \\
    U \wedge V   \quad&\is\quad U \cap V && \text{(as in Definition~\ref{defn:set-intersection})}       \\
    \bigvee \bF{} \quad&\is\quad \lambda x.~ \trunc{\sigmaty{i}{\indexset{\bF{}}}{x \epsilon \bF{}_i}}
  \end{align*}
  Notice that $\top$ and $\cap$ can immediately be seen to be propositional whereas $\bigvee$ requires
  a truncation to be forced to be propositional.

  We now proceed to show that these subsets we have defined are indeed downwards-closed.
  $\top_A$ is trivially downward-closed. Downward-closure of $U \cap V$ follows immediately from
  the downward closure of $U$ and $V$. This is also immediate: if $x$ is in $\bigvee \bF{}$
  where $\bF{}$ is a family of downward-closed subsets, $x$ must belong to a particular
  $\bF{}_i$ which will contain any $y \sqsubseteq x$ meaning $\bigvee \bF{}$ itself will contain any $y \sqsubseteq
  x$. The LUB and GLB properties follow from the fact that subsets of a type form a
  lattice as shown in Chapter~\ref{chap:foundations}.

  It remains to be shown that the distributivity law is satisfied. Let $U$ be a
  downwards-closed subset and $\bF{}$, a family of downward-closed subsets. We must show
  \begin{align*}
    U \cap \left(\lambda x.~ \trunc{\sigmaty{i}{\indexset{\bF{}}}{x \epsilon \bF{}_i}}\right) &=
      \bigvee \left( \left( \lambda x.~ \trunc{\sigmaty{i}{\indexset{\bF{}}}{x \epsilon \bF{}_i}}\right) \cap U \right)\\
      &= \bigvee \left( \lambda x.~ \trunc{\sigmaty{i}{\indexset{\bF{}}}{x \epsilon \bF{}_i}} \cap x \epsilon U \right)
  \end{align*}
  which follows by antisymmetry.
\end{proof}

\section{Nuclei and their fixed-points}\label{sec:nuclei}

To prepare for formal topology, we will now define a technical notion called a
\emph{nucleus}.

\begin{defn}[Nucleus]\label{defn:nucleus}
  Let $F : \mathsf{Frame}_{m, n, o}$ and $j : \abs{F} \rightarrow \abs{F}$ and endofunction on it.
  We say that $F$ is nuclear if the following condition holds:
  \begin{align*}
    \isnuclear{j} \quad&\is\quad
       \pity{x~y}{\abs{F}}{j(\meet{x}{y}) = \meet{j(x)}{j(y)}}       \\
      &\hspace{0.6em}\times\quad \pity{x~~}{\abs{F}}{x \sqsubseteq j(x)}                     \\
      &\hspace{0.6em}\times\quad \pity{x~~}{\abs{F}}{j(j(x)) \sqsubseteq j(x)}.
  \end{align*}

  A nucleus is then just a $\sum$ type collecting nuclear endofunctions on a frame:
  \begin{equation*}
    \mathsf{Nucleus} \quad\is\quad \sigmaty{j}{\abs{F} \rightarrow \abs{F}}{\isnuclear{j}}.
  \end{equation*}
\end{defn}

\begin{prop}
  Every nucleus is monotonic.
\end{prop}
\begin{proof}
  Let $F$ be a frame and $j : \abs{F} \rightarrow \abs{F}$ a nucleus on it.
  Let $x~y : \abs{F}$ and suppose $x \sqsubseteq y$. We need to show that $j(x) \sqsubseteq j(y)$.
  First, notice that $x = \meet{x}{y}$ by antisymmetry since $x \sqsubseteq \meet{x}{y}$ by the fact
  that $\meet{\_}{\_}$ is a greatest lower and $\meet{x}{y} \sqsubseteq y$ by the fact that it is a
  lower bound. The result can now be derived as follows:
  \begin{align*}
    j(x) &\quad\sqsubseteq\quad j(\meet{x}{y})                 && [x = \meet{x}{y}]                       \\
         &\quad\sqsubseteq\quad \meet{j(x)}{j(y)}              && [\text{meets are preserved}]            \\
         &\quad\sqsubseteq\quad {j(y)}                         && [\text{$\meet{}{}$ is a lower bound}]  .
  \end{align*}
\end{proof}

\begin{prop}
  The set of fixed points of a nucleus forms a poset.
\end{prop}
\begin{proof}
  \todo{
    Complete this rather trivial proof. The only interesting thing in it is antisymmetry.
  }
\end{proof}

Now, we are ready to prove the main theorem of this section.

\begin{thm}\label{thm:fixed-point-frame}
  The set of fixed points for a nucleus forms a frame.
\end{thm}
\begin{proof}
  \todo{
    Write and explain this proof completely and precisely.
  }
\end{proof}

In the next chapter, we will see now nuclei are useful for generating a frame with a given
formal topology.

%% As a brief digression, let us note that the reliance on the notion of ``black box''
%% perhaps gives the impression that this is a bizarre scenario that is not likely to occur
%% in real programming. We would like to stress that this is not the case. A program that the
%% programmer does not \emph{completely} grasp is essentially a black box; the distance
%% between the programmer's mental model of it and its actual behaviour has to be
%% investigated via experimentation. This is none other than the familiar process of
%% \emph{debugging a program} that is perhaps the only thing common to all kinds of
%% programming; programs do not come to exist in isolation, they come to exist through the
%% creative effort of a subject through the dialectical process of debugging~\cite{PDD}. This
%% discussion involving a black box can hence be taken to be about a program that behaves in
%% a way the programmer does not expect it to, meaning the programmer does not really know
%% ``what is inside''. Hence, the programmer must experiment with to understand the problem.


\chapter{Formal Topology}\label{chap:formal-topo}

\paragraphsummary{Motivate formal topology.}
We remarked that the motivation for pointless topology is to attain a constructive
understanding of topology, and that this is a prerequisite for being able to express
topology in type theory in a natural way i.e., without postulating classical axioms. This
task of making type-theoretical sense of topology presents another challenge: we must be
able to develop our results in a completely predicative way as well. To address this, we
will use formal topologies which give us a way of \emph{presenting} frames. Instead of
working with frames directly, we will work with formal topologies from which the frames
are freely generated.

This impredicativity springs from the axioms for a \hyperref[defn:frame]{frame} that
quantify over subsets. The LUB axiom for instance requires us to say

\paragraphsummary{Introduce the tree type.}
Our definition of a formal topology will make use of the notion of an interaction system,
first formulated by Petersson and Synek~\cite{tree-sets}, the motivation of whom was to
generalise $\mathsf{W}$ types to be able to accommodate mutual recursion. Other names for
interaction systems include \emph{tree set constructors} (which is the name that Petersson
and Synek~\cite{tree-sets}), and \emph{indexed containers}~\cite{indexed-containers}. The
idea of doing formal topology with interaction systems is due to Coquand~\cite{coq-posets}
who was inspired by Dragalin~\cite{dragalin}.

This chapter corresponds to the Agda module {\color{hottviolet} \texttt{TreeType}} in the
formal development.

\section{Interaction systems}

\paragraphsummary{Explain the idea of the tree type.}
The fundamental idea of an interaction system is simple. Consider the progression of a
two-player game. First, there is a type of \emph{game states}; call it $A$:
\begin{equation*}
  A~:~\univ.
\end{equation*}
At each state of the game, there are certain moves the player can take. In other words,
for every game state $\oftyI{x}{A}$, there is a type of possible moves the player may take.
Formally, this is a function:
\begin{equation*}
  \oftyI{B}{A \rightarrow \univ}.
\end{equation*}
Furthermore, for every move the player may take, the opponent can take certain
counter-moves in response. Formally:
\begin{equation*}
  \oftyI{C}{\pity{x}{A}{B(x) \rightarrow \univ}}.
\end{equation*}
Finally, given the counter-move in response to a certain move at some state, we proceed to
a new game state. This is given by some function:
\begin{equation*}
  \oftyI{d}{\pity{x}{A}{\pity{y}{B(x)}{C(x, y) \rightarrow A}}}.
\end{equation*}

In four pieces, namely $(A, B, C, d)$, we characterise structures that involve some kind
of ``interaction''. Even though the game analogy is useful, interaction systems are more
general than that games: they express anything that is like a dialogue i.e., involving
two parties interacting with each other.

\paragraphsummary{Explain how we will use the tree type.}
To define a formal topology, we will require the type of states ($A$) to be not just a
set, but a poset. The idea of this is the same as in the case of frames, we would like to
view these states as stages of information ranked with respect to how refined they are.
In addition, we will expect such a poset equipped with an interaction system to satisfy
the following two properties:
\begin{enumerate}
\item the monotonicity property: for every state $\oftyI{x}{A}$, experiment
  $y$ on $x$, and outcome $z$ of $y$, $d(x, y, z) \sqsubseteq x$.
  \item the simulation property which states that at any state we simulate the previous
    states.
\end{enumerate}

Once we have imposed the requirement that the set of stages be a poset, it makes sense to
use some more suggestive terminology due to Martin-Löf\footnote{Attributed to
  Martin-Löf in the note~\cite{coq-posets}}.
\begin{itemize}
  \item $A$: a type of stages of \emph{knowledge}.
  \item $\oftyI{B}{A \rightarrow \univ{}}$: a type of \emph{experiments} $B(x)$ that one can perform at a
    certain stage $x$.
  \item $\oftyI{C}{\pity{x}{A}{B(x) \rightarrow \univ{}}}$: a type of \emph{outcomes} of an
    experiment $\oftyI{y}{B(x)}$ for some stage of knowledge $x$.
  \item $d$: a function that expresses the act of \emph{revising} one's belief state based
    on the outcome from an experiment performed at a stage of knowledge.
\end{itemize}
Once we adopt this view, what the monotonicity property says becomes much more clear: if
we perform an experiment $y$ while we have knowledge $x$ and revise our knowledge based on
some outcome $z$ of $y$, the new knowledge we arrive at must contain at least as much
information. In fact, if we would like to view things sensibly as experiments, it makes
sense to reserve this terminology to interaction structures that satisfy this monotonicity
property as experiments must always increase the knowledge state, if they have any effect
at all.

Now, to proceed towards the definition of formal topology, let us first formally define
interaction systems.
\paragraphsummary{Formally define the tree type.}
Let us now formally define the tree type.
\begin{defn}[Interaction system]\label{defn:intr-sys}
  For simplicity, we will require all types to be at the same level $m$ and omit this fact
  in the notation.
  \begin{align*}
    \treestr{A} &\quad\is\quad
      \sigmaty{B}{A \rightarrow \univ}{
        \sigmaty{C}{\pity{x}{A}{B(x) \rightarrow C}}{
          \pity{x}{A}{\pity{y}{B(x)}{C(x, y) \rightarrow A}}
        }
      }\\
    \mathsf{IntrSys} &\quad\is\quad \sigmaty{A}{~~~~\univ}{\treestr{A}}
  \end{align*}
  Given an interaction system $\mathcal{I}$, we will refer to its components as
  $B_{\mathcal{I}}, C_{\mathcal{I}}$, and $d_{\mathcal{I}}$ in context where the possibility of
  ambiguity is present.
\end{defn}

Given an interaction system $\mathcal{I}$, the monotonicity property is then formally
expressed as in the following definition.
\begin{defn}[Monotonicity property of an interaction system]\label{defn:mono}
  Given an interaction system $\mathcal{I}$, it is said to have the monotonicity property
  if the following type is inhabited
  \begin{equation*}
    \hasmono{\mathcal{I}} \quad\is\quad
      \pity{x}{A_{\mathcal{I}}}{\pity{y}{B_{\mathcal{I}}(x)}{\pity{z}{C_{\mathcal{I}}(x, y)}{
        d(x, y, z) \sqsubseteq x}}.
      }
  \end{equation*}
\end{defn}

Unlike the simpler monotonicity property, we have not yet fully explained the simulation
property. Let us first provide its formal definition.
\begin{defn}[Simulation property]\label{defn:sim}
  Given an interaction system $\mathcal{I}$, we will say that it satisfies the simulation
  property if the following type is inhabited:
  \begin{align*}
    &\mathsf{HasSim}\left( \mathcal{I} \right) \quad\is \\
    &\pity{x~x'}{\abs{D}}{
      x' \sqsubseteq x \rightarrow\\
      &\pity{y}{B(x)}{
        \sigmaty{y'}{B(x')}{
          \pity{z'}{C(x', y')}{
            \sigmaty{z}{C(x, y)}{
              d(x', y', z') \sqsubseteq d(x, y, z)
            }
          }
        }
      }
    }.
  \end{align*}
\end{defn}
What does this say intuitively? At more refined stages we can always find a counterpart to
any experiment from a less refined stage, in the sense that that experiment will lead to a
finer stage. In other words, as we perform certain experiments and proceed to more refined
stages, we do not lose the ability to perform the experiments we previously did not
perform.

\begin{defn}[Formal Topology]\label{defn:formal-topo}
  A \emph{formal topology} is simply an interaction system satisfying the \vermono{} and
  \versim{} properties.
  \begin{align*}
    \mathsf{FT} \quad\is\quad \sigmaty{\mathcal{I}}{\mathsf{IntrSys}}{
        \hasmono{\mathcal{I}} \times \hassim{\mathcal{I}}
    }
  \end{align*}
\end{defn}

\section{Cover relation}

The real reason we are interested in formal topologies is that the structure they contain,
along with the \vermono{} and \versim{} properties, induces a \emph{covering relation}.
This is a method going back Johnstone's~\cite{stone-spaces} adaptation of the notion of a
Grothendieck topology to the context of locale theory, that was subsequently developed by
Martin-L\"{o}f and Sambin~\cite{int-formal-spaces}.

The idea is as follows: given a set $A$ that we view like a set of \emph{basic opens}
(i.e., opens not made up using the join operator) we require a relation $\_\covernm{}\_ : A \rightarrow
\pow{A} \rightarrow \Omega$. This relation is expected to pointlessly express the relation of being an
\emph{open cover} i.e., $x \covernm{} V$ iff $V$ is an open cover of $x$: $x = \bigcup_i V_i$. In
other words, we are specifying which basic opens can be expressed as the union of which
others. The information contained by this relation is sufficient to generate the opens.
In our case, the covering relation will be determined by the interaction system.

The original formulation of formal topology by Sambin suffered from the problem that it
was not possible to define the coproduct of two frames using it. This problem was solved
by Coquand et al.~\cite{coq-sambin} by defining the covering relation inductively.

Now, we will define the cover relation on a given formal topology. However, our
presentation will recapitulate our development: we first explain our naive attempt, that
we found out to not work in \UF{}, and its remedied form that circumvents this problem by
using an HIT.
\begin{defn}[Foo]
  \todo{define.}
\end{defn}

One way of reading $\covers{a}{U}$ is as a relaxation of $a \in U$ to ``it is eventually the
case that $a \in U$''. In other words, it might not be that $a \in U$ but once the knowledge
at level $a$ has been attained, paths that do not lead to $U$ have been ruled out:
regardless of what experiments are run, they will eventually lead to a stage in $U$.

Let us consider the how we will go from a formal topology to a frame via its cover
relation. The cover relation has the following type:
\begin{equation*}
  \_\covernm{}\_ : A \rightarrow \pow{A} \rightarrow \univ{},
\end{equation*}
which we can flip to get
\begin{equation*}
  \_\RHD\_ : \pow{A} \rightarrow A \rightarrow \univ{}.
\end{equation*}
\emph{If only} this had codomain $\hprop{}$ rather than $\univ{}$, we would have been able
to write it like:
\begin{equation*}
  \_\RHD\_ : \pow{A} \rightarrow \pow{A},
\end{equation*}
but it does not as it has two constructors. We could then restrict this to get an
endofunction on the set of downwards-closed subsets:
\begin{equation*}
  \_\RHD\_ : \dcsubset{A} \rightarrow \dcsubset{A}.
\end{equation*}
which of course requires us to show that given a downwards-closed subset $U$,
$\covers{\_}{U}$ is a subset that is downwards-closed. But first, we have to deal with the
problem that the codomain of $\covers{}{}$ is not $\hprop{}$.

It is tempting to try to achieve this by truncating $\covers{}{}$. When we do this, it
becomes impossible to prove the idempotence law for the purported nucleus $\RHD :
\dcsubset{A} \rightarrow \dcsubset{A}$. Why exactly is that? To prove the idempotence law:
\begin{equation*}
  \trunc{\covers{a}{\trunc{\covers{-}{U}}}} \rightarrow \trunc{\covers{a}{U}},
\end{equation*}
we need a lemma that says, given any $a$, and subsets $U, V$,
\begin{center}
  if $\trunc{\covers{a}{U}}$ and $\trunc{\covers{a'}{V}}$, for every $a'$, then
  $\trunc{\covers{a}{V}}$.
\end{center}
The $\mathsf{dir}$ case is easily verified. The $\mathsf{branch}$ case, however, results
in a situation where we are trying to show
\begin{equation*}
  \trunc{\pity{c}{C(a, b)}{\covers{d(a, b, c)}{V}}}
\end{equation*}
whereas all we get from the inductive hypothesis is
\begin{equation*}
  \pity{c}{C(a, b)}{\trunc{\covers{d(a, b, c)}{V}}}.
\end{equation*}
An inference of the former from the latter would require the use of a well-studied
classical reasoning principle: the axiom of countable choice~\cite{axiom-of-choice}:
\begin{equation*}
  \pity{x}{A}{\trunc{B(x)}} \rightarrow \trunc{\pity{x}{A}{B(x)}}
\end{equation*}

A remedy for this situation was proposed by the supervisor. Instead of truncating the
naive form of the cover relation, we add a higher constructor that \emph{squashes} the
difference between the $\mathsf{dir}$ and $\mathsf{branch}$ constructors. We now provide
the revised form.
\begin{defn}[Cover relation]\label{defn:covering}
  Given a formal topology
  $\mathcal{F}$ on type $A$, and given $\oftyI{a}{A}$, $U : \pow{A}$, the type
  $\covers{a}{U}$ is inductively defined with the following two constructors:
  \[
  \begin{prooftree}
    \hypo{ a \in U }
    \infer1[$\ruledir{}$]{\covers{a}{U}}
  \end{prooftree}
  \qquad
  \begin{prooftree}
    \hypo{\oftyI{b}{B(a)}}
    \hypo{\pity{c}{C(a, b)}{\covers{d(a, b, c)}{U}}}
    \infer2[$\rulebranch{}$]{\covers{a}{U}}
  \end{prooftree}
  \]
  In addition to the constructors, the type $\covers{a}{U}$ contains the following path:
  \begin{equation*}
    \begin{prooftree}
      \hypo{\oftyI{p}{\covers{a}{U}}}
      \hypo{\oftyI{q}{\covers{a}{U}}}
      \infer2[$\rulesquash{}$]{p =_{\covers{a}{U}} q}
    \end{prooftree}
  \end{equation*}
\end{defn}

When trying to prove the idempotence law with this, we get from the inductive hypothesis a
family $\pity{c}{C(a, b)}{\covers{d(a, b, c)}{V}}$ where $\covers{d(a, b, c)}{V}$ is still
``squashed'', but this squashing is an integral part of $\covers{}{}$ rather than a
truncation that is imposed extrinsically upon it. This is sufficient for circumventing the
problem that would have required the axiom of countable choice, and allows us to
successfully complete the idempotence proof. The type of $\covers{\_}{\_}$ can be seen
directly to be $\pow{A} \rightarrow \pow{A} \rightarrow \Omega$ by the existence of the $\mathsf{squash}$
constructor.

Notice that, given a downwards-closed subset $U$, the subset of elements that are covered
by $U$ ($\covers{\_}{U}$) is itself downwards-closed.
\begin{prop}
  Let $\mathcal{F}$ be an \verintrsys{} and $U$ a downwards-closed subset of its
  underlying poset. $\covers{\_}{U}$ is a downwards-closed subset.
\end{prop}
\begin{proof}
  Let $\oftyII{x}{y}{A}$ such that $\covers{x}{U}$ and $y \sqsubseteq x$. We need to show
  $\covers{y}{U}$. Our proof proceeds by (higher) induction on the proof of
  $\covers{x}{U}$.
  \begin{itemize}
    \item Case $\mathsf{dir}$. It must be that $x \in U$ and hence by the downwards-closure
      of $U$, $y \in U$ meaning $\covers{y}{U}$ by $\mathsf{dir}$.
    \item Case $\mathsf{branch}$. We have some experiment $b$ on $x$ and a function
      $$\oftyI{f}{\pity{c}{C(a, b)}{\covers{d(a, b, c)}{U}}}.$$
      Using the $\mathsf{branch}$ rule, we are done if we can exhibit some experiment
      $\oftyI{b'}{B(y)}$ along with a function
      $$\oftyI{g}{\pity{c'}{C(a, b')}{\covers{d(y, b', c')}{U}}}.$$
      Pick $b'$ to be the experiment given by appealing to the \versim{} property with
      $b$. Let $\oftyI{c'}{C(a, b')}$. It remains to be shown that $\covers{d(y, b', c')}{U}$.
      By the inductive hypothesis, we are done if we can show that $z \sqsubseteq d(y, b', c')$
      for some $\covers{z}{U}$. Pick $z \is d(x, b, c)$ where $c$ is the outcome of $b$ given
      by the \versim{} property. We have $d(x, b, c) \sqsubseteq d(y, b', c')$, directly by the
      \versim{} property and that $\covers{d(x, b, c)}{U}$ by $f(c)$.
    \item Case $\mathsf{squash}$. We are done by appealing to the $\mathsf{squash}$ rule
      with both of the inductive hypothesis.
  \end{itemize}
\end{proof}

\begin{prop}\label{prop:lem1}
  Given a formal topology $\McF{}$, $\oftyII{a}{a'}{A_{\McF{}}}$ such that $a' \sqsubseteq a$, and a
  downwards-closed subset $U$ of $A_{\McF{}}$, if $\covers{a}{U}$ then $\covers{a'}{U}$.
\end{prop}
\begin{proof}[Proof sketch]
  Follows by straightforward induction where the downwards-closedness of $U$ is used
  in the base case and the simulation property is used in the inductive case.
\end{proof}

\begin{prop}\label{prop:lem3}
  Given a formal topology $\McF{}$, $\oftyII{a}{a'}{A_{\McF{}}}$ such that $a' \sqsubseteq a$, and
  downwards-closed subsets $U$ and $V$ of $A_{\McF{}}$, if $\covers{a'}{U}$ and
  $\covers{a}{V}$ then $\covers{a'}{U \cap V}$.
\end{prop}
\begin{proof}
  \todo{Complete}.
\end{proof}

\begin{prop}\label{prop:lem4}
  Let $\mathcal{F}$ be an \verintrsys{} and $U, V$ be subsets of its underlying poset. If
  $U \subseteq \covers{\_}{V}$ then $\covers{\_}{U} \subseteq \covers{\_}{V}$.
\end{prop}
\begin{proof}[Proof sketch]
  Follows by straightforward induction on the covering proof.
\end{proof}

Our method of obtaining a frame out of a formal topology comprises four steps.
\begin{enumerate}
  \item Start with a formal topology $\mathcal{T}$ on type $A$.
  \item $\mathcal{T}$ has an underlying poset $P$; construct its frame of downwards-closed
    subsets.
  \item Note: $\covers{\_}{\_}$ is a nucleus on the frame of downwards-closed subsets.
  \item We have shown (in Theorem~\ref{thm:fixed-point-frame}) that the set of
    fixed-points of every nucleus is a frame. The final frame is this fixed-point frame
    on the frame of downwards-closed subsets by the nucleus $\covers{\_}{\_}$.
\end{enumerate}
The only missing step is (3): we have not yet shown that the covering relation is a
nucleus. We will do exactly this in the following section.

\section{The covering relation is a nucleus}

$\_\RHD\_ : \pow{A} \rightarrow \pow{A}$ is a endofunction on the powerset of $A$. By restricting
our attention to subsets that are downwards-closed, we can view as an endofunction on a
frame
\begin{equation*}
  \oftyI{\_\RHD\_}{\dcsubset{A} \rightarrow \dcsubset{A}}.
\end{equation*}
The natural question to be asked is then: is this a \vernucleus{} on the frame of
downwards-closed subsets of $A$? Let us recapitulate what this means before we proceed
to the proof.
\begin{itemize}
  \item $N_0$: $\pity{a}{A}{\covers{a}{U \cap V} = (\covers{a}{U}) \cap (\covers{a}{V})}$,
  \item $N_1$: $U \subseteq \_ \covernm{} U$, and
  \item $N_2$: $\_ \covernm{} (\_ \covernm{} U) \subseteq \_ \covernm{} U$.
\end{itemize}

\begin{thm}\label{thm:covering-nucleus}
  The covering relation satisfies the nuclearity axioms.
\end{thm}
\begin{proof}
  $N_1$ is direct using the $\ruledir{}$ rule. $N_2$ is a direct application of
  Proposition~\ref{prop:lem4}.

  For $N_0$, we construct a proof by induction. Let $\oftyII{U}{V}{\pow{A}}$. We will show
  that $\_ \covernm{} (U \cap V) = (\_ \covernm{} U) \cap (\_ \covernm{} V)$ by antisymmetry. The $(\_ \covernm{} U) \cap
  (\_ \covernm{} V) \subseteq \_ \covernm{} (U \cap V)$ direction follows by Proposition~\ref{prop:lem3}. For the
  other direction, let $\oftyI{a}{A}$ such that $a \covernm{} U \cap V$. We proceed by induction on
  this proof.
  \begin{itemize}
    \item Case $\ruledir{}$. $a \in U \cap V$ meaning $a \in U$ and $a \in V$ hence we are done
      by an two applications of $\mathsf{dir}$.
    \item Case $\rulebranch{}$. Two appeals to the inductive hypothesis, followed by
      applications of the $\mathsf{branch}$ rule.
    \item Case $\rulesquash{}$. We combine the two inductive hypothesis using the
      $\mathsf{squash}$ rule.
  \end{itemize}
\end{proof}

\section{Generating a frame from a formal topology}

Now that we have shown the nuclearity of the covering relation $\covers{\_}{\_}$, we have
everything we need for the procedure of generating a \verframe{} from a formal topology.

Let $\McF{}$ be a formal topology. We know by Theorem~\ref{thm:down-set-frame} that the
set of downwards-closed subsets of the underlying poset of $\McF{}$ is a frame; denote
this by $\McF{}\downarrow$. As we know that $\covers{\_}{\_}$ is a nucleus on this frame (by
Theorem~\ref{thm:covering-nucleus}), we know that the set of fixed points for
$\covers{\_}{\_}$ is a frame as well; denote this by $L$. Now, notice that we can define
a map $\eta : A_{\McF{}} \rightarrow \abs{L}$ as follows:
\begin{align*}
  \eta    \quad&:\quad A_{\McF{}} \rightarrow \pow{A_{\McF{}}}\\
  \eta(x) \quad&\is\quad \covers{\_}{x\downarrow}
\end{align*}
where $x\downarrow$ denotes the \emph{downwards-closure} of $x$: $\{ y~|~y \sqsubseteq x \}$. So $y \in \eta(x) \equiv
\covers{y}{x\downarrow}$ is to say ``$y$ leads to a ramification of $x$''. In fact, one can see
that $\eta(x)$ is downwards-closed and a fixed point for $\covers{\_}{\_}$, meaning its type
can be refined to $\oftyI{\eta}{A_{\McF{}} \rightarrow \abs{L}}$.

\begin{defn}[$\eta$]
  Let $\McF{}$ be a formal topology and denote its cover relation by $\covers{\_}{\_}$.
  Let $L \is \fix{\dcsubset{A}}{\_\RHD\_}$. There exists a monotonic map from the
  underlying poset of $P$ of $\McF{}$ to the underlying poset of $L$:
  \begin{align*}
    \eta    \quad&:\quad P \rightarrow_m (\abs{L}, \_\sqsubseteq_L\_)\\
    \eta(a) \quad&\is\quad \covers{\_}{a\downarrow}.
  \end{align*}
  The fact that $\eta$ is monotonic follows from Proposition~\ref{prop:lem1}. It remains to
  be shown that it is a fixed point for $\_\RHD\_$. Let $\oftyI{a}{A_{\McF{}}}$. We need
  to show that $\covers{\_}{(\covers{\_}{a})} = \covers{\_}{a}$. We proceed by
  antisymmetry. $\covers{\_}{a\downarrow} \subseteq \covers{\_}{(\covers{\_}{a\downarrow})}$ follows by $N_1$. The
  other direction is a direct application of Proposition~\ref{prop:lem4}.
\end{defn}

\section{Formal topologies present}

\paragraphsummary{Explain the aim.} We are now ready to shift our focus on what can be
called main theorem of this thesis: our notion of a formal topology is capable of
presenting a frame. Let $\mathcal{F}$ be a formal topology and $F$ a frame. Consider a
monotonic function $f : A_{\mathcal{F}} \rightarrow \abs{F}$ on the underlying posets. We will
define a notion of $f$ representing $F$ in $L$.

\begin{defn}[Representation]\label{defn:rep}
  Given a formal topology $\mathcal{F} = (A, B, C, d)$, a frame $F$, and a function
  $\oftyI{f}{A \rightarrow \abs{F}}$ we say that $f$ represents $\mathcal{F}$ in $F$ if the
  following type is inhabited:
  \begin{equation*}
    \mathsf{represents}\left(\mathcal{F}, F, f\right) \quad\is\quad
      \pity{x}{A}{
        \pity{y}{B(x)}{
          f(x) \sqsubseteq \bigvee_{\oftyI{z}{C(x, y)}} f(d(x, y, z)).
        }
      }
  \end{equation*}
\end{defn}

To state the universal property, we will work with \emph{flat} monotonic maps. This is the
special case of the notion of a \emph{flat functor}~\cite{nlab-flat-functor} in the case
where the categories we are working with are posets. Consider a monotonic map $f : P \rightarrow Q$,
where $Q$ has (finite) products but $P$ does not. We would like to assert somehow that $f$
preserves finite meets but we cannot mention the meets of $P$ because they do not exist.
So what we want to do is to state that $f$ preserves products that \emph{do not exist yet}
which we can do by requiring the following conditions:
\begin{enumerate}
  \item $\top_Q = \bigvee f(P)$, and
  \item $\pity{a~b}{P}{f(a) \wedge f(b) =
           \bigvee f \left( \left\{ z~|~ z \sqsubseteq x\ \text{and}\ z \sqsubseteq y \right\} \right)}$
\end{enumerate}

In our case, we will of course be interested in monotonic maps whose codomains are frames.
\begin{defn}[Flat monotonic map]\label{defn:flat}
  Let $P$ be a \verposet{} and $F$ a \verframe{}. Denote by $I$ the type
  $\sigmaty{z}{\abs{F}}{z \sqsubseteq x \times z \sqsubseteq y}$.
  A monotonic map $f : P \rightarrow F$ from $P$
  to the underlying poset of the frame is called \emph{flat} if the following type is
  inhabited: 
  \begin{align*}
    \isflat{f} \quad&\is\quad \top_F = \bigvee f(P)\\
                &\hspace{0.6em}\times
                \pity{a~b}{P}{f(a) \wedge f(b) = \bigvee_{\oftyI{(i, p)}{I}} f(i)}.
  \end{align*}
\end{defn}

\begin{figure}
  \centering
  \caption{The universal property}
  \begin{tikzcd}[row sep=40pt, column sep=40pt]\label{fig:comm-diag}
    \McF{} \arrow[swap, rd, "f"] \arrow[r, "\eta"] & L \arrow[d, dashed, "g"] \\
                                                & R
  \end{tikzcd}
\end{figure}

Using flat monotonic maps, the universal property can now be stated.

\begin{thm}[Universal property for formal topologies]\label{thm:univ-prop}
  Given any formal topology $\McF{}$, frame $R$, flat monotonic map $f : A_{\McF{}} \rightarrow R$
  from the underlying \verposet{} of $\McF{}$ to the underlying \verposet{} of $R$, that
  \hyperref[defn:rep]{represents} $\McF{}$ in $R$, there exists a unique \textbf{frame
    homomorphism} $\oftyI{g}{L \rightarrow R}$ such that $f = g \circ \eta$, as summarised in
  Fig.~\ref{fig:comm-diag}

  We express this fully formally as follows:
  \begin{align*}
    \pity{\McF{}}{\mathsf{FT}_{n, n}}{
      &\pity{R}{\mathsf{Frame}_{n+1, n, n}}{
         \pity{f}{P \rightarrow_m R}{
           \isflat{f} \rightarrow \mathsf{represents}\left( \McF{} , F , f \right) \rightarrow\\
             &\iscontr{\sigmaty{g}{L \rightarrow_f R}{f = g \circ \eta}}
        }
      }
    }.
  \end{align*}
\end{thm}

Before proceeding to the proof, let us first prove a lemma of key importance.

\begin{lemma}\label{lem:main}
  Let $\McF{}$ be a formal topology. Denote by $L$ the set of downwards-closed subsets of
  $A_{\McF{}}$ that are fixed points for its covering nucleus. For any $\oftyI{U}{L}$,
  it is the case that:
  \begin{equation*}
    U = \bigvee^L \setof{ \eta(u) ~|~ u \in U }.
  \end{equation*}
\end{lemma}
\begin{proof}
  Let $\oftyI{U}{L}$. We proceed by antisymmetry. The $U \sqsubseteq \bigvee^L \setof{ \eta(u) ~|~ u \in U }$
  direction is immediate by an application for the $\ruledir{}$ rule. For the other
  direction, let $\oftyI{a}{A_{\McF{}}}$ and suppose that
  $a \in \bigvee^L \setof{ \eta(u) ~|~ u \in U }$. Recall that the join operation is defined by
  applying the nucleus in the fixed point frame (Theorem \ref{thm:fixed-point-frame}).
  This is to say that we have $\covers{a}{\bigcup \setof{ \eta(u) ~|~ u \in U }}$.
\end{proof}

\begin{proof}[Proof of Theorem~\ref{thm:univ-prop}]
  Let $\McF{}$ be a formal topology and $R$ a frame. Let $\oftyI{f}{P \rightarrow_m R}$ be a
  \emph{flat} monotonic map that represents $\McF{}$ in $R$
  (as defined in Defn.~\ref{defn:rep}). We will show the \emph{unique existence} of a
  \emph{frame homomorphism} $\oftyI{g}{L \rightarrow_f R}$ such that $f = g \circ \eta$.

  We choose
  \begin{align*}
    g    \quad&:\quad \abs{L} \rightarrow \abs{R} \\
    g(U) \quad&\is\quad \bigvee^R f(U)        .
  \end{align*}
  To show that the diagram commutes, it suffices by Proposition~\ref{prop:funext} to show
  $f(x) = g(\eta(x))$ for every $\oftyI{x}{A_{\McF{}}}$. Let $\oftyI{x}{A_{\McF{}}}$; we
  proceed by antisymmetry. The $f(x) \sqsubseteq g(\eta(x))$ direction is easy: $g(\eta(x)) = \bigvee^R f(\eta(x))$
  and $\bigvee^R$ is an upper bound of $f(\eta(x)$ so it suffices to show $f(x) \in f(\eta(x))$. This is
  immediate since $x \in \eta(x)$.

  The other direction is the interesting one. First, we prove a lemma in preparation:
  \emph{given any $\oftyII{a}{a'}{A_{\McF{}}}$ such that $\covers{a'}{a\downarrow}$, $f(a') \sqsubseteq f(a)$}.
  Let $\oftyII{a}{a'}{A_{\McF{}}}$ and assume $\covers{a'}{a}$. We proceed by induction on
  the proof of $\covers{a'}{a}$.
  \begin{itemize}
    \item Case: $\ruledir{}$. It must be that $a' \in a \downarrow$ i.e., $a' \sqsubseteq a$. We are done by
      the monotonicity of $f$.
    \item Case: $\rulebranch{}$. There must be some $\oftyI{b}{B_{\McF{}}(a')}$ and a
      function $h$ such that $\oftyI{h(c)}{\covers{d(a', b, c)}{a\downarrow}}$ for any
      $\oftyI{c}{C(a', b)}$. Notice that $f(a') \sqsubseteq \bigvee_{\oftyI{c}{C(a', b)}}f(d(a, b, c))$ by
      the assumption that $f$ \emph{represents} (Defn.~\ref{defn:rep}) so it remains to be
      shown $$\bigvee_{\oftyI{c}{C(a', b)}}d(a, b, c) \sqsubseteq f(a).$$ As
      $\bigvee_{\oftyI{c}{C(a', b)}}f(d(a, b, c))$ is a LUB, it suffices to show that $f(a)$ is an
      upper bound of the subset of $A_{\McF{}}$ delineated by $f(d(a', b, \_))$.
      Consider $f(d(a', b, c))$ for some $c$. We have that $\oftyI{h(c)}{\covers{d(a', b,
          c)}{a\downarrow}}$ meaning $f(d(a', b, c)) \sqsubseteq f(a)$ by the inductive hypothesis.
    \item Case $\rulesquash{}$. We combine the inductive hypotheses using the fact that
      the result type $f(a') \sqsubseteq f(a)$ is propositional.
  \end{itemize}

  Now, we want to show that $\bigvee^R f(\eta(x)) \sqsubseteq f(x)$. Since $\bigvee^R$ is a LUB it suffices to show
  that $f(x)$ is an upper bound of $\img{f}{\eta(x)}$. Let $$f(y) \epsilon \img{f}{\eta(x)}.$$ We need
  to show that $f(y) \sqsubseteq f(x)$. By the lemma we have just proven, it suffices to show that
  $\covers{y}{x\downarrow}$ and this holds directly because $f(y) \epsilon \img{f}{\eta(x)}$.

  There are two more things that have to be shown (1) $g$ is a frame homomorphism and (2)
  $g$ is unique.

  Let us first address (1). The fact that $g$ preserves the top element of $L$ is given
  by the flatness assumption: $g(\top) = \bigvee^R \img{f}{A_{\McF{}}} = \top_R$ by
  flatness (Defn.~\ref{defn:flat}). $g$ can also be seen to preserve binary meets by
  the following reasoning: let $\oftyII{U}{V}{L}$;
  \begin{align*}
    g(U \cap V) &\equiv \bigvee^R \{ f(a) ~|~ a \in U \cap V \} \\
             &= \bigvee^R \setof{ \bigvee \{ w ~|~ w \sqsubseteq u \times w \sqsubseteq v \} ~|~ (u, v) \in U \times V }
               && \text{[flatness]}                                                 \\
             &= \bigvee^R \setof{ f(u) \wedge f(v) ~|~ (u, v) \in U \times V }
               &&\text{[Prop.~\ref{prop:distr}]}                                    \\
             &= \left( \bigvee^R \img{f}{U} \right) \wedge \left( \bigvee^R \img{f}{V} \right)       \\
             &\equiv g(U) \wedge g(V)                                                         .
  \end{align*}
  Let us now show that $g$ preserves joins. Let $U$ be a family of inhabitants of
  $\abs{L}$.
  \begin{align*}
    g(\bigvee^L U) &\equiv \bigvee^R \setof{ f(a) ~|~ a \in \bigvee^L_i U_i } \\
             &= \bigvee^R \setof{ f(a) ~|~ \oftyI{(\_, (a, \_))}{\sigmaty{i}{I}{\sigmaty{x}{A_{\McF{}}}{x \in U_i}}} } &&\text{[Lemma~\ref{lem:flatten}]} \\
             &= \bigvee^R \setof{ \bigvee^L \img{f}{U_i} ~|~ U_i \in U } \\
             &\equiv \bigvee^R \setof{ g(U_i) ~|~ U_i \in U }.
  \end{align*}

  Finally, we conclude the proof by showing uniqueness of $g$: let $g'$ be a frame
  homomorphism from $L$ to $R$ that makes the diagram commute. We need to show that
  $g = g'$. Let $\oftyI{U}{\abs{L}}$. $g(U) = g'(U)$ by the following equational proof:
  \begin{align*}
    g(U) &\equiv \bigvee^R \setof{ f(u) ~|~ u \in U }
            &&\text{[Lemma~\ref{lem:main}]} \\
         &= g \left( \bigvee^L \setof{ \eta(u) ~|~ u \in U } \right) 
            &&\text{[$g$ is a frame homomorphism]} \\
         &= \bigvee^L \setof{ g(\eta(u)) ~|~ u \in U }
            &&\text{[$g \circ \eta = f$]} \\
         &= \bigvee^L \setof{ f(u) ~|~ u \in U }
            &&\text{[$g' \circ \eta = f$]} \\
         &= \bigvee^L \setof{ g'(\eta(u)) ~|~ u \in U }
            &&\text{[$g'$ is a frame homomorphism]} \\
         &= g'\left( \bigvee^L \setof{ \eta(u) ~|~ u \in U } \right)
            &&\text{[Lemma~\ref{lem:main}]} \\
         &= g'(U).
  \end{align*}
\end{proof}


%% \input{examples}

\chapter{Conclusion and Further Work}\label{chap:conc}


\makebackmatter{}

\end{document}
