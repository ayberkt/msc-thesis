\input{template/template}

\title{Formal Topology in Univalent Foundations}
\author{Ayberk Tosun}
\newcommand{\authorname}{Ayberk Tosun}

\supervisor{Thierry Coquand}
\departmentofsupervisor{Computer Science and Engineering}

\examiner{Nils Anders Danielsson}
\departmentofexaminer{Computer Science and Engineering}

\division{Logic and Types}

\keywords{topology, domain theory}

\newcommand{\reals}{\mathbb{R}}

\begin{document}

\maketitlepage{}

\begin{abstract}
  \lipsum[1-2]{}
\end{abstract}

\begin{acknowledgements}
  Foo bar.
\end{acknowledgements}

\makelists{}

\chapter{Introduction}

This thesis is about topology, the branch of mathematics that studies \emph{continuous}
functions. The notion of a continuous function pervades practically all of mathematics, as
pointed out by JJ Sylvester: ``if I were asked to name, in one word, the pole star round
which the mathematical firmament revolves, the central idea which pervades the whole
corpus of mathematical doctrine, I should point to Continuity as contained in our notions
of space, and say, it is this, it is this!''~\cite[pg. 27]{armstrong_basic_2011}. Let us
then start by considering the question of what continuity is.

A continuous function is defined in elementary courses as a function for which ``small
changes to the input result in small changes to the output''. Usually right after this,
comes the $\epsilon$-$\delta$ definition of continuity. A function $f : \reals{} \rightarrow \reals{}$ is
\emph{continuous} if
\begin{equation*}
  \forall x, y \in \reals{}.\ \forall \epsilon > 0.\ \exists \delta > 0.\ 0 < | x - y | < \delta \rightarrow | f(x) - f(y) | < \epsilon.
\end{equation*}

This definition embodies the idea that, to make $f(x)$ closer than $\epsilon$ to $f(y)$, it
suffices to make $x$ closer than $\delta$ to $y$, for some certain $\delta$. Consider for instance,
a function whose graph has an jump, or an \emph{abrupt change} at some point:
\begin{align*}
  g    &: \reals{} \rightarrow \reals{}\\
  g(x) &= \mathsf{if}\ x = 3\ \mathsf{then}\ 42\ \mathsf{else}\ x.
\end{align*}
Consider inputs $x = 2$ and $y = 3$ and $\epsilon = 1$. $| f(2) - f(3) | = | 2 - 42 | = 40$. It
is clear that the consequent $40 < 1$

\paragraph{Generalise the notion of distance.}

\paragraph{Distance is not needed.}

\lipsum[1-3]

\end{document}
