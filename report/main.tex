\input{/home/ayberkt/academic/thesis/report/template/template}
\usepackage{agda}

\usepackage{ebproof}
\usepackage{tikz-cd}

\tikzcdset{
  arrow style=tikz,
  diagrams={>={Straight Barb[scale=0.8]}}
}

\title{Formal Topology in Univalent Foundations}
\multilinetitle{Formal Topology\\ in Univalent Foundations}
\author{Ayberk Tosun}

\supervisor{Thierry Coquand}
\departmentofsupervisor{Computer Science and Engineering}

\examiner{Nils Anders Danielsson}
\departmentofexaminer{Computer Science and Engineering}

\division{Logic and Types}

\keywords{
  topology, formal topology, pointless topology, formal space, locale, locale theory,
  frame, homotopy type theory, univalent foundations
}

\definecolor{darkgreen}{rgb}{0,0.45,0}
\definecolor{darkred}{rgb}{0.45,0,0}
\definecolor{hottviolet}{rgb}{0.45,0,0.45}
\definecolor{hottblue}{rgb}{0,0.45,0.45}

\hypersetup{
  linktoc    = page,
  colorlinks = true,
  linkcolor  = darkgreen,
  citecolor  = hottblue,
  urlcolor   = hottviolet
}

\newtheorem{ax}{Axiom}

\newcommand{\reals}{\mathbb{R}}
\newcommand{\nats}{\mathbb{N}}
\newcommand{\bool}{\mathbf{Bool}}
\newcommand{\ball}[2]{\mathfrak{A}(#1, #2)}
\newcommand{\neighbourhood}[1]{\mathbf{N}(#1)}

\newcommand{\oftyI}[2]{#1\hspace{0.1mm}:\hspace{0.1mm}#2}
\newcommand{\oftyII}[3]{#1, #2:#3}
\newcommand{\refl}{\mathsf{refl}}
\newcommand{\zero}{\mathsf{zero}}
\newcommand{\suc}[1]{\mathsf{suc}\left(#1\right)}

\newcommand{\fiber}[2]{\hyperref[defn:fiber]{ \mathsf{fiber} }\left(#1, #2\right)}
\newcommand{\isequiv}[1]{\hyperref[defn:equiv]{\mathsf{isEquiv}}\left(#1\right)}
\newcommand{\idtoeqvnm}{\hyperref[defn:id-to-equiv]{\mathsf{idToEquiv}}}
\newcommand{\idtoeqv}[2]{\idtoeqvnm{}\left(#1, #2\right)}
\newcommand{\idequivnm}{\hyperref[defn:id-equiv]{\mathsf{idEqv}}}
\newcommand{\idequiv}[1]{\idequivnm{}\left(#1\right)}
\newcommand{\isdec}[1]{\mathsf{isDecidable}\left(#1\right)}
\newcommand{\isdisc}[1]{\mathsf{isDiscrete}\left(#1\right)}
\newcommand{\typequiv}[2]{#1 \hyperref[defn:equiv]{\simeq} #2}
\newcommand{\logequiv}[2]{#1 \hyperref[defn:iff]{\leftrightarrow} #2}
\newcommand{\exteq}[2]{#1 \hyperref[defn:exteq]{\sim} #2}

\newcommand{\unitty}{\mathsf{Unit}}

%% Homotopy levels.
\newcommand{\iscontr}[1]{\hyperref[defn:contr]{\mathsf{isContr}}\left(#1\right)}
\newcommand{\isprop}[1]{\hyperref[defn:hprop]{\mathsf{isProp}}\left(#1\right)}
\newcommand{\isset}[1]{\hyperref[defn:hset]{\mathsf{isSet}}\left(#1\right)}
\newcommand{\isofhlevel}[2]{%
  \hyperref[defn:hlevel]{\mathsf{isOfHLevel}}\left(#1, #2\right)
}

\newcommand{\intersectnm}{\hyperref[defn:intersection]{\cap}}
\newcommand{\intersect}[2]{#1 \intersectnm{} #2}

\newcommand{\pity}[3]{\prod_{(#1~:~#2)} #3}
\newcommand{\sigmaty}[3]{\sum_{(#1~:~#2)} #3}
\newcommand{\univ}{\mathcal{U}}
\newcommand{\isaprop}[1]{\hyperref[defn:hprop]{\mathsf{isProp}}\left(#1\right)}
\newcommand{\hprop}{\hyperref[defn:omega]{Ω}}
\newcommand{\isaset}[1]{\mathsf{IsSet}\left(#1\right)}
\newcommand{\abs}[1]{\left| #1 \right|}
\newcommand{\trunc}[1]{\left\| #1 \right\|}
\newcommand{\pownm}{\hyperref[defn:pow]{\mathcal{P}}}
\newcommand{\pow}[1]{\pownm{}\left(#1\right)}
\newcommand{\sub}[2]{\hyperref[defn:fam]{\mathsf{Fam}}_{#1}\left(#2\right)}
\newcommand{\indexnm}{\mathsf{index}}
\newcommand{\indexset}[1]{\indexnm{}\left(#1\right)}
\newcommand{\pair}[2]{\langle #1 , #2 \rangle}

\newcommand{\subsetof}[2]{#1 \hyperref[defn:inclusion]{\subseteq} #2}

\newcommand{\isdcnm}{\hyperref[defn:dc-subset]{\mathsf{isDownwardsClosed}}}
\newcommand{\isdc}[1]{\isdcnm{}\left(#1\right)}
\newcommand{\dcsubsetnm}{\hyperref[defn:dc-subset]{\mathsf{DCSubset}}}
\newcommand{\dcsubset}[1]{\dcsubsetnm{}\left(#1\right)}
\newcommand{\dcframe}[1]{#1\hyperref[thm:down-set-frame]{\downarrow}}

\newcommand{\ordernm}{\hyperref[defn:poset]{\mathsf{Order}}}
\newcommand{\order}[2]{\ordernm{}_{#1}\left(#2\right)}

\newcommand{\posetstrnm}{\hyperref[defn:poset]{\mathsf{PosetStr}}}
\newcommand{\posetstr}[2]{\posetstrnm{}_{#1}\left(#2\right)}

\DeclareMathOperator{\memfamnm}{\hyperref[defn:fam-mem]{\mathtt{\epsilon}}}
\newcommand{\memfam}[2]{#1 \memfamnm #2}
\newcommand{\mempow}[2]{#1 \hyperref[defn:pow]{\in} #2}

\newcommand{\posetaxnm}{\hyperref[defn:poset]{\mathsf{PosetAx}}}
\newcommand{\posetax}[1]{\posetaxnm{}\left(#1\right)}

\newcommand{\poset}{\hyperref[defn:poset]{\mathsf{Poset}}}

\newcommand{\posof}[1]{\mathsf{pos}\left(#1\right)}

\newcommand{\ismonotonicnm}{\hyperref[defn:mono]{\mathsf{isMonotonic}}}
\newcommand{\ismonotonic}[1]{\ismonotonicnm{}\left(#1\right)}
\DeclareMathOperator{\mononm}{\hyperref[defn:mono-map]{\rightarrow_m}}
\newcommand{\mono}[2]{#1 \mononm{} #2}

\newcommand{\isframehomonm}{\hyperref[defn:frame-homo]{\mathsf{isFrameHomo}}}
\newcommand{\isframehomo}[1]{\isframehomonm{}\left(#1\right)}

\DeclareMathOperator{\framehomonm}{\hyperref[defn:frame-homo]{\rightarrow_f}}
\newcommand{\framehomo}[2]{#1 \framehomonm{} #2}

\newcommand{\hasmono}[1]{\hyperref[defn:mono]{\mathsf{hasMono}}\left(#1\right)}
\newcommand{\hassim}[1]{\hyperref[defn:sim]{\mathsf{hasSim}}\left(#1\right)}

\newcommand{\vermono}{monotonicity}
\newcommand{\versim}{simulation}
\newcommand{\vernucleus}{\hyperref[defn:nucleus]{nucleus}}
\newcommand{\verframe}{\hyperref[defn:frame]{frame}}
\newcommand{\verposet}{\hyperref[defn:poset]{poset}}
\newcommand{\verintrsys}{\hyperref[defn:intr-sys]{interaction system}}
\newcommand{\veragda}{\textsc{Agda}}

\newcommand{\modname}[1]{{\color{AgdaModule} \texttt{#1}}}
\newcommand{\fnname}[1]{{\color{AgdaFunction} \texttt{#1}}}

\newcommand{\rulename}[1]{{\color{darkred} \mathsf{#1}}}
\newcommand{\ruledir}{{\color{darkred} \mathsf{dir}}}
\newcommand{\rulebranch}{{\color{darkred} \mathsf{branch}}}
\newcommand{\rulesquash}{{\color{darkred} \mathsf{squash}}}

\newcommand{\fix}[2]{\hyperref[thm:fixed-point-frame]{\mathsf{fix}}\left(#1, #2\right)}

\newcommand{\isflat}[1]{\hyperref[defn:flat]{\mathsf{isFlat}}\left(#1\right)}

\newcommand{\framestrnm}{\mathsf{FrameStr}}
\newcommand{\framestr}[1]{\framestrnm{}\left(#1\right)}
\newcommand{\frameax}[1]{\hyperref[defn:frame]{\mathsf{FrameAx}}\left(#1\right)}
\newcommand{\framenm}{\hyperref[defn:frame]{\mathsf{Frame}}}
\newcommand{\framety}[3]{\framenm{}_{#1, #2, #3}}

\newcommand{\treestrnm}{\hyperref[defn:intr-sys]{\mathsf{IntrStr}}}
\newcommand{\treestr}[1]{\treestrnm{}\left(#1\right)}
\newcommand{\intrsys}{\hyperref[defn:intr-sys]{\mathsf{IntrSys}}}

\newcommand{\stumpnm}{\mathsf{Stump}}
\newcommand{\stump}[1]{\stumpnm\left(#1\right)}

\newcommand{\refines}[2]{#1~\mathcal{R}~#2}

\newcommand{\disciplinestrnm}{\mathsf{DisciplineStr}}
\newcommand{\disciplinestr}[1]{\disciplinestrnm{}\left(#1\right)}

\newcommand{\rawframestrnm}{\hyperref[defn:frame]{\mathsf{RawFrameStr}}}
\newcommand{\rawframestr}[3]{\rawframestrnm{}_{#1, #2}\left(#3\right)}

\newcommand{\isnuclearnm}{\hyperref[defn:nucleus]{\mathsf{isNuclear}}}
\newcommand{\isnuclear}[1]{\isnuclearnm{}\left(#1\right)}
\newcommand{\nucleus}{\hyperref[defn:nucleus]{\mathsf{Nucleus}}}

\newcommand{\meet}[2]{#1 \wedge #2}
\newcommand{\joinnm}[3]{\bigvee}
\newcommand{\join}[3]{\joinnm{}_{#1~:~#2} #3}

\newcommand{\RHD}{\scalebox{1.2}{{\tt ▶}}}
\DeclareMathOperator{\covernm}{\scalebox{1.2}{{\tt ◀}}}
\newcommand{\covers}[2]{#1 \covernm{} #2}

\newcommand{\setof}[1]{\left\{ #1 \right\}}
\newcommand{\img}[2]{\setof{ #1\left( a \right) ~|~ a \in #2 }}

\newcommand{\representsnm}{\hyperref[defn:rep]{\mathsf{represents}}}
\newcommand{\represents}[3]{\representsnm{}\left(#1, #2, #3\right)}

\newcommand{\bF}{\mathbf{F}}
\newcommand{\bG}{\mathbf{G}}
\newcommand{\McF}{\mathcal{F}}
\newcommand{\MfU}{\mathfrak{U}}

\newcommand{\is}{:\equiv}
\newcommand{\idnm}{~=~}

%% Names.
\newcommand{\UF}{Univalent Foundations}

%% \newcommand{\paragraphsummary}[1]{{\color{orange} \textsc{#1}}}
\newcommand{\paragraphsummary}[1]{}
\newcommand{\todo}[1]{
  {\color{red} \textsf{TODO: #1}}
}

%% \setmainfont{XITS}
%% \setmathfont{XITS Math}
\setmonofont[Scale=0.85]{PragmataPro Mono Liga}

\begin{document}

\maketitlepage{}

\begin{abstract}
  Formal topology is the mathematical discipline that aims to reconstruct topology in
  type-theoretical terms, that is, constructively \emph{and} predicatively. Type theory
  has recently undergone a transformation through insights arising from its association
  with homotopy theory, resulting in a conceptual novelty for foundations of mathematics,
  namely, the formulation of the notion of a \emph{univalent} foundation.

  We investigate, in this thesis, the natural continuation of the line of work on formal
  topology into univalent type theory. We first recapitulate our finding that a naive
  approach to develop formal topology in \UF{} is problematic, and would require a form of
  the axiom of choice. We then present a solution to this problem that involves the use of
  higher inductive types. We hence sketch the beginnings of an approach towards developing
  formal topology in \UF{}. As a proof of concept, we develop the formal topology of the
  Cantor space and construct a proof that is compact.

  Our development has been carried out using the cubical extension of the \veragda{} proof
  assistant~\cite{cubical-agda}. The presentation in this thesis amounts to an
  informalisation of this formal development.
\end{abstract}

\begin{acknowledgements}
  \todo{Add acknowledgements here.}
\end{acknowledgements}

\makelists{}

\chapter{Introduction}\label{chap:intro}

This thesis is about topology, the branch of mathematics that studies \emph{continuous}
functions. The notion of a continuous function pervades practically all of mathematics as
pointed out by Sylvester~\cite[pg.~27]{armstrong-topology}: ``if I were asked to name, in
one word, the pole star round which the mathematical firmament revolves, the central idea
which pervades the whole corpus of mathematical doctrine, I should point to Continuity as
contained in our notions of space, and say, it is this, it is this!''. Let us then start
by considering the question of what continuity is.

A continuous function, in the context of real numbers, is a function for which ``small
changes to the input result in small changes to the output''. This is formally expressed
in the following ``$\epsilon$-$\delta$'' definition of continuity: a function $f : \reals{} \rightarrow
\reals{}$ is
\emph{continuous} if
\begin{equation*}\label{eq:cont-0}
  \forall x \in \reals{}.~ \forall \epsilon > 0.~ \exists \delta > 0.~ \forall y \in \reals{}.~
    | x - y | < \delta \rightarrow | f(x) - f(y) | < \epsilon.
\end{equation*}
This definition embodies the idea that, to make $f(x)$ closer than $\epsilon$ to
$f(y)$, it suffices to make $x$ closer than $\delta$ to $y$, for some certain $\delta$.

To pinpoint the essence of continuity, we will consider it in general form. $\reals{}$ is
just one instance of a \emph{metric space}, a set that admits a well-behaved notion of
distance, its distance function being:
\begin{alignat*}{2}
  d       \quad&:\quad  && \reals{} \times \reals{} \rightarrow \reals{} \\
  d(x, y) \quad&:=\quad && | x - y |                      .
\end{alignat*}
This is just \emph{one} instance of a notion of distance on a set that is viewed as a set
of points. We can speak of the continuity of a function between any two metric spaces. Let
$X$ and $Y$ be two metric spaces. We now define continuity for any function $f : X \rightarrow Y$:
\begin{equation}\label{cont-1}
  \forall x \in X.~ \forall \epsilon > 0.~ \exists \delta > 0.~ \forall y \in X.~ d_X(x, y) < \delta \rightarrow d_Y(f(x), f(y)) < \epsilon.
\end{equation}

Notice that we could have written (\ref{cont-1}) in an alternative way. Given some $x \in X,
\epsilon \in \reals{}$ define
\begin{equation*}
  \ball{x}{\epsilon} \quad\is\quad \setof{ y \in X ~|~ d_X(x, y) < \epsilon }.
\end{equation*}
Now, the following is the same as (\ref{cont-1}):
\begin{equation*}
  \forall x \in X.~ \forall \epsilon > 0.~ \exists \delta > 0.~ \forall y \in X.~ y \in \ball{x}{\delta} \rightarrow f(y) \in \ball{f(x)}{\epsilon},
\end{equation*}
which could be expressed even more compactly as:
\begin{equation}\label{cont-2}
  \forall x \in X.~ \forall \epsilon > 0.~ \exists \delta > 0.~ f(\ball{x}{\delta}) \subseteq \ball{f(x)}{\epsilon}.
\end{equation}

After having written down continuity in this more compact way, we consider the question:
what is the meaning of $\ball{x}{\epsilon}$? Given some $x \in X$, $\ball{x}{\epsilon}$ expresses the set
of points that are \emph{closer than} $\epsilon$ to $x$. One intuitive reading of this is:
$\ball{x}{\epsilon}$ denotes the set of approximations of $x$ with a degree of accuracy of $\epsilon$.
As $\epsilon$ decreases, the accuracy with which $\ball{x}{\epsilon}$ represents $x$ increases. What is
crucial here is that, as $x$ might be infinite, we cannot (decidably) judge the membership
of an element in set $\setof{ x }$, but we can judge its membership in set $\ball{x}{\epsilon}$
so this is an approximation of ``infinite precision'' properties by finitely observable
ones. This situation is familiar from \emph{experimental} sciences: even though we might
not be able to pin down an exact (i.e., with $\epsilon \equiv 0$) value $v$, we can approximate $v$ by
conducting more accurate experiments, hence narrowing the value down to a tight
$\ball{v}{\epsilon}$.

To see continuity in its full generality, we now notice that the notion of distance is not
crucial to what is expressed in (\ref{cont-2}). We have made use of the distance function
just to be able to specify the approximation sets. Instead, we can work directly with a
specification of what the approximation sets are. This is the main idea underlying
topology: instead of working with a notion of distance, we work directly with a
specification of the approximations.

The definition of a metric space ensures that the distance function is well-behaved. When
one defines approximation sets in terms of such a distance function, they are necessarily
well-behaved: they satisfy one's expectations from a reasonable approximation structure.
As we want to remove our dependency on a notion of distance, we will have to axiomatise
the behaviour of approximation sets themselves instead.

The standard term for such a class of sets embodying an approximation structure is a
\emph{basis for a topology}~\cite{munkres}. We now summarise it precisely in the following
definition.
\begin{defn}[Basis for a topology]
  Given a set $X$, a basis for a topology on $X$ is a class $\mathcal{B}$ of subsets of
  $X$ such that:
  \begin{enumerate}
  \item $\forall x \in X.~\exists B \in \mathcal{B}.~x \in B$, which intuitively says ``there is an
    approximation for all points'', and
  \item $\forall x \in X.~\forall B_0, B_1 \in \mathcal{B}.~x \in B_0 \cap B_1 \rightarrow
    \exists B_2 \in \mathcal{B}.~ B_2 \subseteq (B_0 \cap B_1) \wedge x \in B_2$, which says
    ``approximations can be combined to yield finer approximations''.
  \end{enumerate}
\end{defn}

\paragraphsummary{Continuity with bases.}
Now, we can reformulate continuity in a more general way, without relying on a notion of
distance. Given sets $X, Y$ with bases $\mathcal{B}_X, \mathcal{B}_Y$, a function $f : X \rightarrow
Y$ is continuous if
\begin{equation}\label{cont-basis}
  \forall x \in X.~ \forall V \in \mathcal{B}_Y.~ f(x) \in V \rightarrow \exists U \in \mathcal{B}_X.~
    f(U) \subseteq V.
\end{equation}
This expresses the idea in (\ref{cont-2}), but talking directly about some given
approximation structure rather than defining this structure through a notion of distance.
In other words, we previously said that a function is continuous if we can make $f(x)$
closer than $\epsilon$ to $f(y)$ by making $x$ closer than $\delta$ to $y$. We are now saying that for
every set approximating some $f(x)$, there is a set approximating $x$ such that the image
of the latter is contained in the former, which corresponds to the idea in (\ref{cont-2}).

Given a basis $\mathcal{B}$ on set $X$, we say that the topology generated by
$\mathcal{B}$ is the class of all possible unions of $\mathcal{B}$. The idea here is that
an element $U$ of the topology is like a property of the points that we are interested in
\emph{verifying}. To verify $x \in U$, it suffices to show that $x$ lies in one of the basis
elements that constitute $U$ and since the basis elements are approximation sets, the fact
that $x$ has property $U$ can be verified from a finite approximation of $x$, which
requires only a finite amount of data (``finite precision'').

In reality, it is often the case that we deal directly with bases that generate
topologies. However, we will not have reached full generality until we can talk about
\emph{topologies} directly i.e., properties of a set which can be verified via
approximations. We will call such properties finitely verifiable or
\emph{observable}~\cite{abramsky-thesis}. The topological term is ``open set'' in the
sense that the set does not contain its own boundary.

If we have two finitely verifiable properties $U$ and $V$, and we know that a point
satisfies both of them, there should be some degrees of accuracy, $ε₀$ and $ε₁$, at which
$U$ and $V$ are verified. Then, we can verify both by taking the lower between them so we
can verify their intersection. On the other hand, if we have an arbitrary number of
finitely verifiable properties, to verify membership in the union, it suffices to verify
membership in just one of the constituents, so if a point is in this set it must be in one
of the finitely verifiable subsets, meaning the set itself is finitely verifiable as the
degree of accuracy required to check membership in the constituent set is sufficient to
check membership in the union. This brings us to the following definition of a topological
space~\cite{munkres}.
\begin{defn}[Topological space]\label{defn:topospace}
  A topology on a set $X$ is a class $\mathcal{T}$ of subsets of $X$ such that
  \begin{itemize}
    \item The trivial subsets $X, \emptyset \subseteq X$ are in $\mathcal{T}$,
    \item $\mathcal{T}$ is closed under finite intersections, and
    \item $\mathcal{T}$ is closed under arbitrary unions.
  \end{itemize}
\end{defn}

By this definition, we start with a set $X$ of \emph{points} on which we are interested in
making observations. The topology we attach to $X$ specifies these observable sets. The
view of topology we will be concerned with in this thesis is called \emph{pointless}
topology~\cite{johnstone-the-point}. In pointless topology, we start with a primitive set
of opens instead, and we study these directly. In other words, we reverse the conceptual
order of open sets being notions derived from points: we view points as being derived from
the opens instead. What is remarkable is that we can go a long way without mentioning the
points~\cite{johnstone-the-point}.

So how do we express the idea of open sets \emph{not} as sets of points? We simply
axiomatise the behaviour of the lattice of open sets. Let $\mathcal{O}$ be a set of
opens, that are some unspecified primitive entities.
\begin{enumerate}
  \item Corresponding to the set-inclusion partial order in the pointful case, we require
    that there be a partial order $\_\sqsubseteq\_ \subseteq \mathcal{O} \times \mathcal{O}$.
  \item Corresponding to the fact that open sets are closed under finite intersection, we
    require that there be a binary meet operation $\meet{\_}{\_} : \mathcal{O} \times
    \mathcal{O} \rightarrow \mathcal{O}$ and a nullary one $\top : \mathcal{O}$.
  \item Corresponding to the fact that open sets are closed under arbitary union, we
    require that there be a join operation of arbitrary arity: $\joinnm{}\_ :
    \mathcal{P}(\mathcal{O}) \rightarrow \mathcal{O}$.
\end{enumerate}
In addition to these, we have to require that these operations satisfy an infinite
distributivity law, on which we will elaborate in Chapter~\ref{chap:frames}. The official
name for such a lattice that embodies a logic of observable or finitely verifiable
properties is \emph{frame}~\cite{vickers}. Indeed, frames form a category whose morphisms
are frame homomorphisms; objects of the opposite category are called \emph{locales}.
Therefore, frames and locales are synonymous
\emph{as long as no mentions to morphisms are made}~\cite{vickers, stone-spaces}

A natural question is then: what do we gain by studying topology in a pointless way? In
this thesis, we will look at topology from the perspective of computer science therefore
we would like to be able to understand theorems of topology in computational terms.
Topology notoriously relies on classical reasoning in many of its fundamental theorems
such as the Tychonoff theorem. By doing topology pointlessly, we can avoid classical
reasoning and hence gain a computational understanding of it---most saliently of such
important theorems such as Tychonoff~\cite{coq-tychonoff}. This point was put eloquently
by Johnstone~\cite[pg.~46]{stone-spaces}:
\begin{quote}
  It is here that the real point of pointless topology begins to emerge; the difference
  between locales and spaces is one that we can (usually) afford to ignore if we are
  working in a ``classical'' universe with the axiom of choice available, but when (or if)
  we work in a context where choice principles are not allowed, then we have to take
  account of the difference—and usually it is locales, not spaces, which provide the right
  context in which to do topology. This is the point which, as I mentioned earlier,
  Andr\'{e} Joyal began to hammer home in the early 1970s; I can well remember how, at the
  time, his insistence that locales were the real stuff of topology, and spaces were
  merely figments of the classical mathematician's imagination, seemed (to me, and I
  suspect to others) like unmotivated fanaticism. I have learned better since then.
\end{quote}

In this thesis, we are interested in investigating topology in the context of type theory.
The first prerequisite for this is that we be able to develop topology constructively,
that is, without relying on classical principles. As pointed out by Johnstone, pointless
topology can help us here as it allows us to understand topology in constructive terms.
The goal of carrying out topology in type theory, however, presents further challenges: we
have to understand topology not only constructively but also \emph{predicatively}.

This is the subject matter of the branch of mathematics known as formal topology, first
instigated by Martin-Löf and Sambin~\cite{int-formal-spaces} in the early days of type
theory. The idea is that a formal topology recasts the notion of a frame into a form that
resembles a formal proof system. In addition to enabling the importation of
proof-theoretical ideas, such a formal presentation has the virtue of being
\emph{predicative} hence enabling the development of frames in type theory.

In this thesis, we present a development of formal topology in univalent type
theory~\cite{hottbook}. By now, it has become clear that univalence addresses certain
severe shortcomings of type theory. The question of what novelties it presents for formal
topology is therefore a natural one. In attempting to answer this question, we follow a
particular approach to formal topology, implementing an idea of Coquand~\cite{coq-posets}
to define formal topologies as posets endowed with ``interaction''
structures~\cite{tree-sets, hancock-interaction-systems}.

This thesis is structured as follows. In Chapter~\ref{chap:foundations}, we summarise the
fundamentals of univalent type theory. In Chapter~\ref{chap:frames}, we present our
development of frames and constructs related to them. In Chapter~\ref{chap:formal-topo},
we present our main development of formal topology in univalent type theory. In
Chapter~\ref{chap:cantor}, we present a prime example of a formal topology: the Cantor
space. As a proof of concept for the formal study of topological properties in univalent
type theory, we prove an important property of the Cantor space, namely, that it is
compact.


\chapter{Foundations}\label{chap:foundations}

In this chapter, we provide the preliminary constructions and theorems of \UF{} for the
sake of self-containment. It is intended to be a summary rather than an introduction to
HoTT/UF.

\section{The identity type}

\todo{
  Do a couple of things here.
  \begin{enumerate}
    \item Define the identity type and present its introduction and elimination rules.
    \item Give a brief summary of Cubical Type Theory and the fact that that's what really
      we are making use of.
    \item Remark that the refl is a built-in in CTT.
    \item Explain that the details of CTT need not concern us as we will be viewing it as
      just one particular implementation of HoTT.
  \end{enumerate}
}

\section{Pi types}

\todo{Complete.}

\section{Sigma types}

\todo{Complete.}

\section{Homotopy levels}

\begin{prop}\label{isOfHLevelSigma}
  $\Sigma$ types preserve h-levels.
\end{prop}
\begin{proof}
  \todo{Sketch the proof}.
\end{proof}

\section{Propositions}

We collect propositional types in $\hprop{}$.

\begin{defn}\label{omega}
  $\hprop{} \is \sigmaty{A}{\univ}{\isaprop{A}}$
\end{defn}

Technically, to assert that some $A~:~\hprop{}$ is inhabited, we must project out the
first component. However, we will engage in the understandable notational abuse of taking
this to be implicit.

\section{Propositional truncation}

\section{Subsets}

Given a type $A$, we would like to talk about all ``subsets'' of $A$. There are
two ways we can approach this: (1) via taking the characteristic function of the
subset as the subset, and (2) via $A$-codomained functions. In univalent
foundations, both of these approaches are equal.

In this development, there will be places where we will prefer one of these two
approaches over the other for convenience. Therefore we will define both of them
and show that they are equivalent.

\subsection{Subsets as power sets}

\begin{defn}(Power set)\label{defn:pow}
  Let $A$ be a type. Its power set $\pow{A}$ is defined as:
  \begin{equation*}
    \pow{A} \quad\is\quad A \rightarrow \Omega
  \end{equation*}
\end{defn}

\begin{prop}\label{isSetPow}
  Given any type $A$, $\pow{A}$ is an h-set.
\end{prop}

\begin{defn}[Full subset]\label{defn:full-set}
  The empty subset on any type $A$ is defined as $\lambda \_ \rightarrow \top$. We will refer to this as
  $\top_A$.
\end{defn}

\begin{defn}[Set intersection]\label{defn:set-intersection}
  Given subsets $U, V : \pow{A}$, their intersection is defined as:
  \begin{equation*}
    U \cap V \quad\is\quad x \mapsto x \epsilon U \times x \epsilon V
  \end{equation*}
\end{defn}

\subsection{Subsets as families}

\begin{defn}(Subset as a family of elements)\label{defn:fam}
  Let $A$ be a type. We define a subset of $A$ as a family of its elements as:
  \begin{equation*}
    \sub{A} \quad\is\quad \sigmaty{I}{\univ}{I \rightarrow A},
  \end{equation*}
  where $I$ is viewed as a type of index elements.

  Given some $F : \sub{A}$ we will refer the index type of $\mathcal{F}$ as
  $\indexset{F}$.
\end{defn}

\subsection{Equivalence of power sets and subsets as families}

\todo{
  Prove that Definition~\ref{defn:pow} and Definition~\ref{defn:fam} are actually equal to
  each other
}

\section{Structure identity principle}\label{sec:sip}


\chapter{Frames}\label{chap:frames}

In this chapter, we develop the notion of a frame in \UF{}. In Chapter~\ref{chap:intro},
we explained that a frame is the algebra of a logic of finitely verifiable properties.
Recall that a frame consists of the following:
\begin{itemize}
  \item a set $O$ of \emph{opens},
  \item a partial order $\_\sqsubseteq\_ \subseteq O \times O$, corresponding to the set-inclusion order of the
    open subsets,
  \item finite meets, and
  \item arbitrary joins.
\end{itemize}

In addition to these, there is a law that is needed to ensure the correct interplay
between meets and joins. Suppose that we have the observable property $\phi$ and the family
of observable properties $\psi_0, \psi_1, \cdots$. Consider the expression:
\begin{equation*}
  A \cap (\bigcup_i B_i).
\end{equation*}
where $A$ is a set and $B$ is a family of sets. By set-theoretic reasoning, this is the
same thing as:
\begin{equation*}
  \bigvee_i (\phi \wedge \psi_i).
\end{equation*}
As we are trying to characterise the behaviour of open ``sets'' without defining them as
sets of points, we have to explicitly add this distributivity law into the definition of
frame:
\begin{center}
  \emph{binary meets must distribute over arbitrary joins.}
\end{center}

As a brief digression, let us note that it is natural to consider the question: what
happens if we leave out this requirement of distributivity? The resulting structure is
called a \emph{basic topology} and is studied in the work of Sambin

\paragraphsummary{Structure of chapter.}
We now start presenting our formal development of frames. We start with partially ordered
sets in Section~\ref{sec:poset}, which underlie frames. In Section~\ref{sec:frame}, we
present the definition of a frame. In Section~\ref{sec:frame-univ}, we present an
important theorem unique to \UF{}: isomorphic frames are equal. In Sections
\ref{sec:down-set-frame} and \ref{sec:nuclei}, we prove two important theorems in
preparation for the succeeding Chapter~\ref{chap:formal-topo} on formal topology: (1) the
set of downward-closed subsets of a poset forms a frame and (2) given a nucleus (a
technical notion to be introduced) on a frame, its set of fixed points is itself a frame.

\section{Partially ordered sets}\label{sec:poset}

\begin{defn}[Poset]\label{defn:poset}
  Given some $\oftyI{A}{\univ{}_m}$, let $\order{n}{A} \is A \rightarrow A \rightarrow \hprop{}_n$. Notice the
  generality of the universes: the codomain of the relation is permitted to be on a level
  different than that of the carrier set. A poset at carrier level $m$ and relation level
  $n$ is then defined as:
  \begin{equation*}
    \mathsf{Poset}_{m, n} \quad\is\quad \sigmaty{A}{\univ_m}{\posetstr{n}{A}},
  \end{equation*}
  \begin{center}
  where
  \end{center}
  \begin{align*}
    \posetstr{n}{A} \quad\is&\quad \sigmaty{R}{\order{n}{A}}{\posetax{A, R}}\\
    \posetaxnm \quad:&\quad \pity{A}{\univ{}_m}{\order{n}{A} \rightarrow \univ_{\max(m, n)}}\\
    \posetax{A, R} \quad\is&\quad ~\pity{x}{A}{R(x, x)}\\
                      \times&~\pity{x~y~z}{A}{R(x, y) \rightarrow R(y, z) \rightarrow R(x, z)}\\
                      \times&~\pity{x~y}{A}{R(x, y) \rightarrow R(y, x) \rightarrow x =_A y}\\
                      \times&~\isaset{A}
  \end{align*}
\end{defn}

\paragraphsummary{Clarify notation.}
Given a poset $P$, we will refer to its relation as $\_\sqsubseteq_P\_$ (in cases where there might
be ambiguity) and the underlying set of $P$ as $\abs{P}$. Notice that the fourth component
of $\posetax{A, R}$ requires the carrier set to be an \hyperref[defn:hset]{h-set}.

Given a poset $P$ we will talk about its \emph{downward-closed subsets}: sets that include
all elements below their elements. This notion embodies the idea of verification at a
certain stage of information. Take a certain element $x : \abs{P}$, that we view as a
stage of information. For some other $y~:~\abs{P}$, $y \sqsubseteq x$ expresses the idea that $y$ is
a \emph{more refined} stage of information i.e., it contains more information hence ruling
out more approximations meaning it admits \emph{less}. Let $U$ be a subset of $\abs{P}$.
The property that $U$ is downward-closed is then expressed as:
\begin{equation*}
  x \epsilon U \rightarrow y \sqsubseteq x \rightarrow y \epsilon U,
\end{equation*}
the intuitive reading of which is: $U$ contains all stages that are ramifications of the
stages it contains. This means that $U$ is an \emph{observable} property: it is secured at
a certain stage in the sense that the reception of more information does not disrupt it.
Let us write this down formally.
\begin{defn}[Downward-closed subset]\label{defn:dc-subset}
  We first define a predicate expressing that a given subset of $P$ is downwards-closed:
  \begin{align*}
    \isdcnm{}   &\quad:\quad  Poset_{m, n} \rightarrow \pow{\abs{P}} \rightarrow \Omega\\
    \isdc{P, U} &\quad\is{}\quad \pity{x~y}{\abs{P}}{x \in U \rightarrow y \sqsubseteq x \rightarrow y \in U}.
  \end{align*}
  By multiple appeals to Proposition~\ref{prop:pi-prop}, it suffices to show that the
  inner-most expression inside the nested $\prod$ type is propositional which is immediate
  since the codomain of $U$ is \hyperref[defn:hprop]{propositional}. We then define the
  type of downwards-closed subsets of a poset as:
  \begin{align*}
    \dcsubsetnm{} &\quad:\quad \mathsf{Poset}_{m, n} \rightarrow \univ_{\max(m+1, n)}\\
    \dcsubset{P}  &\quad\is{}\quad \sigmaty{U}{\pow{\abs{P}}}{\isdc{P, U}}.
  \end{align*}
\end{defn}

So far we have dealt with two notions of \emph{observable property} throughout the
development:
\begin{enumerate}
  \item element of a poset which we will eventually view like pointless versions of an
    open set with the order corresponding to the subset-inclusion order, and
  \item the notion of downwards-closed subset which expresses that a property of the poset
    of opens behaves like an observational property.
\end{enumerate}
We will now start relating these two by showing that the set of downwards-closed subsets
of a poset is itself a poset, and indeed, we will prove later (in
Sec.~\ref{sec:down-set-frame}) that it actually forms a frame meaning downwards-closed
subsets satisfy our expectations from properties we view as observable.

Let us start by showing that $\dcsubset{P}$ is a set.
\begin{prop}\label{isSetDCSubset}
  $\dcsubset{P}$ is a set for every poset $P$.
\end{prop}
\begin{proof}
  By Proposition~\ref{prop:sigma-set}, it suffices to show that $\pow{\abs{P}}$ is a set
  and $$\isdc{P, U}$$ is a set for every $\oftyI{U}{\pow{\abs{P}}}$. The former holds by
  Proposition~\ref{prop:pow-set}. For the latter, observe that every $\isdc{P, U}$ is a
  proposition by definition meaning it is also set by Proposition~\ref{prop:prop-is-set}.
\end{proof}

Now we can proceed to construct the poset of downwards-closed subsets.
\begin{thm}(Poset of downward-closed subsets)\label{thm:dc-poset}
  Let $P$ be a poset. The type $\dcsubset{P}$ forms a poset under the
  inclusion relation.
\end{thm}
\begin{proof}
  The fact that $\dcsubset{P}$ is a set is given by Proposition~\ref{isSetDCSubset} so it
  suffices to show that the poset axioms are satisfied. Reflexivity and transitivity are
  immediate. For antisymmetry, let $U, V \in \pow{\abs{P}}$ and assume $U \subseteq V$, $V \subseteq U$. By
  function extensionality, it suffices to show that for every $x : \abs{P}$, $U(x) =
  V(x)$. Since $\oftyII{U(x)}{V(x)}{\Omega}$, it is sufficient to show $U(x) \leftrightarrow V(x)$ which is
  immediate
  by assumptions.
\end{proof}

\subsection{Monotonic functions}

The morphisms between two partially ordered sets are monotonic functions.

\begin{defn}[Monotonic function]
  Let $P, Q$ be posets. A function $f : \abs{P} \rightarrow \abs{Q}$ is monotonic if the following
  type is inhabited:
  \begin{equation*}
    \ismonotonic{f} \quad\is\quad \pity{x~y}{\abs{P}}{x \sqsubseteq_P y \rightarrow f(x) \sqsubseteq_Q f(y)}.
  \end{equation*}
  We collect the type of monotonic functions between $P$ and $Q$ in the following type:
  \begin{equation*}
    \monotonicmap{P}{Q} \quad\is\quad \sigmaty{f}{\abs{P} \rightarrow \abs{Q}}{\ismonotonic{f}}
  \end{equation*}
\end{defn}

\begin{defn}[Poset isomorphism]
  An isomorphism between two posets is a monotonic function with a monotonic inverse.
\end{defn}

\section{Definition of a frame}\label{sec:frame}

We now proceed to define frames.
\begin{defn}[Frame]\label{defn:frame}
  A frame structure on some type $A$ consists of (1) a poset structure, (2) a top element
  (3) a binary meet operation, and (4) a join operation of arbitrary arity, which we
  define using families:
  \begin{equation*}
    \rawframestr{n}{o}{A} \quad\is\quad \posetstr{n}{A} \times A \times (A \rightarrow A \rightarrow A) \times (\sub{o}{A} \rightarrow A).
  \end{equation*}
  This raw structure must be subject to the following axioms
  \begin{align*}
    \frameax{\sqsubseteq, \top, \wedge, \bigvee} \quad&\is\quad
      \mathsf{IsTop}(\top) \times \mathsf{IsGLB}(\wedge) \times \mathsf{IsLUB}\left( \bigvee \right)
      \mathsf{IsDistr}(\wedge, \bigvee)\\
    \mathsf{isTop}(\top) \quad&\is\quad \pity{x}{A}{x \sqsubseteq \top}\\
    \mathsf{isGLB}(\wedge) \quad&\is\quad \pity{x~y}{A}{(x \wedge y \sqsubseteq x) \times (x \wedge y \sqsubseteq y)}\\
                       &\hspace{0.5em}\times\quad \pity{z~~}{A}{(z \sqsubseteq x) \times (z \sqsubseteq y) \rightarrow z \sqsubseteq x \wedge y}\\
    \mathsf{isLUB}\left(\bigvee\right) \quad&\is\quad
         \pity{F}{\sub{n}{A}}{\pity{x}{A}{x \epsilon F \rightarrow x \sqsubseteq \bigvee_i F_i}}\\
         &\hspace{0.5em}\times \pity{F}{\sub{n}{A}}{\pity{x}{A}{
               \left( \pity{y}{A}{y \epsilon F \rightarrow y \sqsubseteq x}\right) \rightarrow \bigvee_i F \sqsubseteq x
             }}\\
    \mathsf{isDistr}(\wedge, \bigvee) \quad&\is\quad
      \pity{x}{A}{\pity{F}{\sub{n}{A}}{
          x \wedge \bigvee_i F_i} =_A \bigvee_i \left( x \wedge F_i \right)
      }
  \end{align*}
\end{defn}

We will use the notation $\abs{F}$ for referring to the underlying set of a frame, as we
do for posets. Similarly, we will refer to the underlying partial order as $\_\sqsubseteq_F\_$, in
possibly ambiguous contexts.

\begin{prop}
  For every raw frame structure $(\sqsubseteq, \top, \wedge, \bigvee)$, $\frameax{\sqsubseteq, \top, \wedge, \bigvee}$ is a proposition.
\end{prop}
\begin{proof}[Proof sketch]
  By Proposition~\ref{prop:sigma-prop}, it suffices to show that each component is an
  h-prop. For $\mathsf{isTop}$, $\mathsf{isGLB}$, and $\mathsf{isLUB}$ this can be
  concluded by using Proposition~\ref{prop:sigma-prop} and Proposition~\ref{prop:pi-prop}.
  For $\mathsf{isDistr}$, we use Proposition~\ref{prop:pi-prop} followed by the fact that
  the underlying set of a poset is an h-set (by the definition of $\posetaxnm{}$ from
  Definition~\ref{defn:poset}).
\end{proof}

\section{Isomorphic frames are equal}\label{sec:frame-univ}

\todo{
  Prove that isomorphic frames are equal using the structure identity principle developed
  in Section~\ref{sec:sip}. This will consist in showing that definition of a frame with
  frame isomorphism forms a standard notion of structure and that frame axioms are
  propositions.
}

\section{Frame of downward-closed subsets}\label{sec:down-set-frame}

We have constructed, in Theorem~\ref{thm:dc-poset}, the poset of downwards-closed subsets,
where the relation is the subset inclusion relation. We will now construct the
\emph{frame} of downwards-closed subsets, in which the meet is subset intersection and the
join is subset union.

\begin{thm}
  Given a poset $P$, the poset of downwards-closed subsets of $P$ (as constructed in
  Theorem~\ref{thm:dc-poset}), is a frame.
\end{thm}
\begin{proof}
  We start by defining the following $\top, \wedge, and \bigvee$ operations:
  \begin{align*}
    \top       \quad&\is\quad \top_A   && \text{(as constructed in Defn.~\ref{defn:full-set})} \\
    U \wedge V   \quad&\is\quad U \cap V && \text{(as constructed in Defn.~\ref{defn:intersection})}\\
    \bigvee \bF{} \quad&\is\quad \lambda x.~ \trunc{\sigmaty{i}{\indexset{\bF{}}}{x \epsilon \bF{}_i}}
                         && \text{(using truncation as defined in Defn.~\ref{defn:truncation})}
  \end{align*}
  $\top$ and $\cap$ are propositional by construction whereas $\bigvee$ requires a truncation to be
  forced to be propositional. Downwards-closure and the LUB/GLB properties are easy to
  verify. We focus on showing that the distributivity law is satisfied. Let $U$ be a
  downwards-closed subset and $\bF{}$, a family of downward-closed subsets. We must show
  \begin{align*}
    U \cap \left(\lambda x.~ \trunc{\sigmaty{i}{\indexset{\bF{}}}{x \epsilon \bF{}_i}}\right)
      &= \bigvee \left( U \cap \left( \lambda x.~ \trunc{\sigmaty{i}{\indexset{\bF{}}}{x \epsilon \bF{}_i}}\right) \right)\\
      &\equiv \bigvee \left( \lambda x.~ \trunc{\sigmaty{i}{\indexset{\bF{}}}{x \epsilon \bF{}_i}} \times x \epsilon U \right)
  \end{align*}
  which follows by antisymmetry.

  \todo{revise and expand}.
\end{proof}

\section{Nuclei and their fixed points}\label{sec:nuclei}

To prepare for formal topology, we will now define a technical notion called a
\emph{nucleus}. Nuclei are used to describe quotient frames of a frame, which one views as
subspaces of the space corresponding to that frame. They are presented by Johnstone
in~\cite[Sec.~II.2]{stone-spaces}.

The reason we are interested in nuclei is that in Chapter~\ref{chap:formal-topo} we will
be focusing on a particular nucleus on the frame of downward-closed subsets. It is this
nucleus that will allow us to describe the topological structure of our frame by letting
us specify laws that are expected to hold in the resulting frame.
\begin{defn}[Nucleus]\label{defn:nucleus}
  Let $F : \mathsf{Frame}_{m, n, o}$ and $j : \abs{F} \rightarrow \abs{F}$ and endofunction on it.
  We say that $F$ is \emph{nuclear} if the following condition holds:
  \begin{align*}
    \isnuclearnm{}\quad&:\quad (\abs{F} \rightarrow \abs{F}) \rightarrow \Omega                   \\
    \isnuclear{j} \quad&\is\quad
       \pity{x~y}{\abs{F}}{j(\meet{x}{y}) = \meet{j(x)}{j(y)}}  \\
      &\hspace{0.55em}\times\quad \pity{x~~}{\abs{F}}{x \sqsubseteq j(x)}          \\
      &\hspace{0.55em}\times\quad \pity{x~~}{\abs{F}}{j(j(x)) \sqsubseteq j(x)}.
  \end{align*}
  The propositionality is, as usual, a consequence of Proposition~\ref{prop:sigma-prop},
  Proposition~\ref{prop:pi-prop}, and the fact that the carrier set is a set (by the
  definition of $\posetaxnm{}$ from Defn.~\ref{defn:poset}).

  The type of nuclei is then just the $\sum$ type collecting all nuclear endofunctions on a
  frame:
  \begin{equation*}
    \mathsf{Nucleus} \quad\is\quad \sigmaty{j}{\abs{F} \rightarrow \abs{F}}{\isnuclear{j}}.
  \end{equation*}
\end{defn}

\begin{prop}
  Every nucleus is monotonic.
\end{prop}
\begin{proof}
  Let $F$ be a frame and $j : \abs{F} \rightarrow \abs{F}$ a nucleus on it. Let
  $\oftyII{x}{y}{\abs{F}}$ and suppose $x \sqsubseteq y$. We need to show that $j(x) \sqsubseteq j(y)$. First,
  notice that $x = \meet{x}{y}$ by antisymmetry since $\meet{x}{y} \sqsubseteq x$ and $x \sqsubseteq
  \meet{x}{y}$ as $\meet{x}{y}$ is greater than any other lower bound and $x$ is a lower
  bound as it is less than both itself and $y$.
  \begin{align*}
    j(x) &\quad\sqsubseteq\quad j(\meet{x}{y})                 && [x = \meet{x}{y}]                      \\
         &\quad\sqsubseteq\quad \meet{j(x)}{j(y)}              && [\text{nuclei preserve meets}]         \\
         &\quad\sqsubseteq\quad {j(y)}                         && [\text{$\meet{}{}$ is a lower bound}]  .
  \end{align*}
\end{proof}

We will be interested in the type of inhabitants of a frame that are \emph{fixed} points
for a given nucleus on the frame, i.e., given a frame $F$ and a nucleus $j$ on it,
the type $$\sigmaty{x}{F}{j(x) = x}.$$

\begin{prop}
  The set of fixed points of a nucleus forms a poset.
\end{prop}
\begin{proof}[Proof sketch]
  The proof amounts to forgetting the information of being a fixed point. For
  antisymmetry, we use Proposition~\ref{prop:sigma-prop} along with the fact that the
  carrier set is an h-set (by the definition of $\posetaxnm{}$ from
  Defn.~\ref{defn:poset}).
\end{proof}

Now, we are ready to prove the main theorem of this section: this poset of fixed points
for a nucleus on a frame is itself a frame. The proof we will present has been adapted to
the type-theoretic setting from Johnstone's proof in \cite[II.2.2, pg.~49]{stone-spaces}.

\begin{thm}\label{thm:fixed-point-frame}
  The set of fixed points for a nucleus $j$ on some frame $F$ forms a frame.
\end{thm}
\begin{proof}
  The binary meets and the top elements are taken directly from the frame $F$. We need
  to show that they are fixed points for $j$. Let $\oftyII{x}{y}{\abs{F}}$.
\end{proof}

In the next chapter, we will make use of nuclei to a generate frame from a formal
topology.

\section{Comparison to the Agda formalisation}


\chapter{Formal Topology}\label{chap:formal-topo}

\paragraphsummary{Motivate formal topology.}
Our motivation for pointless topology was that it allows us to interpret topology in
constructive terms. The goal of developing topology in type theory presents further
challenges: we must be able to develop our results in a completely predicative way as
well. The definition of a frame we have seen suffers from impredicativity. We will not be
able to instantiate it to topologies we are interested in. The culprit is the join
operator.

\paragraphsummary{Explain the impredicativity of the join operator.}
TODO: explain and give examples.

\paragraphsummary{Introduce the tree type.}
To present frames, we will make use of the idea of tree set constructors, originally due
to Petersson and Synek~\cite{tree-sets}, who were trying to generalise Martin-Löf's
$\mathsf{W}$ type so that it can accommodate mutually recursive types. It is remarkable
that these structures embody the essence of a Post system in type theory. Tree set
constructors are also called interaction systems~\cite{hancock-interaction-systems} and
indexed containers~\cite{indexed-containers}; we will call it the tree type. This chapter
corresponds to the Agda module \texttt{TreeType} in the formal development.

\section{Petersson and Synek's tree type}

\paragraphsummary{Explain the idea of the tree type.}
The fundamental idea of an interaction system is simple. Consider the progression of a
two-player game. First, there is a type of \emph{game states}; call it $A$:
\begin{equation*}
  A~:~\univ.
\end{equation*}
At each state of the game, there are certain moves the player can take. In other words,
for every game state $x~:~A$, there is a type of possible moves the player may take.
Formally, this is a function:
\begin{equation*}
  B~:~A \rightarrow \univ.
\end{equation*}
Furthermore, for every move the player may take, the opponent can take certain
counter-moves in response. Formally:
\begin{equation*}
  C~:~\pity{x}{A}{B(x) \rightarrow \univ}.
\end{equation*}
Finally, given the counter-move in response to a certain move at some state, we proceed to
a new game state. This is given by some function:
\begin{equation*}
  d~:~\pity{x}{A}{\pity{y}{B(x)}{C(x, y) \rightarrow A}}
\end{equation*}
In four pieces, namely $(A, B, C, d)$, we express a ``game-like system'' in a very general
way. Even though the game analogy is very useful, the tree type is more general: it
expresses \emph{anything} that is like a dialogue i.e., two subjects interacting with each
other.

\paragraphsummary{Explain how we will use the tree type.}
Our presentation of frames will be based on this notion of interaction system, based upon
an idea of Coquand~\cite{coq-posets}. We will have to impose three additional requirements
on an interaction system.
\begin{enumerate}
  \item The type $A$ is equipped with a partial order. We will view this order as ranking
    states with respect to how \emph{refined} they are. This might be counter-intuitive at
    first: the more informative the states are, the smaller they will be. The sense of
    this ordering is: the more informative a state is, there less sequences of
    interactions there will be that pass through that state. It is like an open ball that
    encircles its center more tightly; hence it is smaller.
  \item The ordering on $A$ satisfies the \emph{monotonicity} requirement: for every state
    $x~:~A$, $d(x) \sqsubseteq x$. In other words, the states that we proceed to via interaction are
    always at least as informative as the previous ones.
\end{enumerate}
We will call an interaction system that satisfies (1) and (2) an \emph{discipline}, (in the
sense that the states resemble a discipline of knowledge, which we will explain later).
The final requirement is (3):
\begin{enumerate}
  \item The poset satisfies the \emph{simulation property} which states that at any state
    we simulate the previous states.
\end{enumerate}

\paragraphsummary{Formally define the tree type.}
Let us now formally define the tree type.
\begin{defn}[Tree type]
  \begin{align*}
    \treestr{A} &\quad\is\quad
      \sigmaty{B}{A \rightarrow \univ}{
        \sigmaty{C}{\pity{x}{A}{B(x) \rightarrow C}}{
          \pity{x}{A}{\pity{y}{B(x)}{C(x, y) \rightarrow A}}
        }
      }\\
    \mathsf{Tree} &\quad\is\quad \sigmaty{A}{~~~~\univ}{\treestr{A}}
  \end{align*}
  Given a tree type structure $\mathcal{T}$, we will refer to its components as
  $A_{\mathcal{T}}, B_{\mathcal{T}}, C_{\mathcal{T}}$, and $d_{\mathcal{T}}$.
\end{defn}

\paragraphsummary{Formally define disciplines.}
\begin{defn}[Discipline]
  \begin{align*}
    \mathsf{Discipline} \quad&\is\quad \sigmaty{P}{\poset{}}{\disciplinestr{P}}                 \\
    \disciplinestr{P}   \quad&\is\quad \hspace{-0.9em}\sigmaty{\mathcal{T}}{\treestr{A}}{
      \pity{x}{A_{\mathcal{T}}}{\pity{y}{B_{\mathcal{T}}(x)}{
          \pity{z}{C_{\mathcal{T}}(x, y)}{d_{\mathcal{T}}(x, y, z) \sqsubseteq x}}
      }
    }
  \end{align*}
  We will refer to this as the monotonicity property \emph{of a discipline}. This is not
  to be confused with the monotonicity of a monotonic map.
\end{defn}

\paragraphsummary{Transition to talking about the simulation property.}
The only remaining thing towards our definition of formal topology is the aforementioned
simulation property. It requires a couple of auxiliary concepts which we will now define.

\paragraphsummary{Formally define the simulation property.}
We will first define this formally and justify it conceptually afterwards.
\begin{defn}[Simulation property]
  Given a discipline $D$, we will say that it satisfies the simulation property if the
  following type is inhabited:
  \begin{equation*}
    \pity{x~x'}{\abs{D}}{
      x' \sqsubseteq x \rightarrow \pity{y}{B(x)}{
        \sigmaty{y'}{B(x')}{
          \pity{z'}{C(x', y')}{
            \sigmaty{z}{C(x, y)}{
              d(x', y', z') \sqsubseteq d(x, y, z)
            }
          }
        }
      }
    }.
  \end{equation*}
\end{defn}

\paragraphsummary{Intuitive justification.}
What does this say intuitively? At more refined stages we can always find a counterpart to
any experiment from a less refined stage, in the sense that that experiment will lead to a
more refined stage. To put it more succinctly:
\begin{quote}
  \emph{lower stages can always simulate upper stages}.
\end{quote}

\paragraphsummary{Define formal topology.}
Once the property of simulation has been defined it is easy to state our
definition of a formal topology.
\begin{defn}[Formal topology]
  A formal topology is a discipline satisfying the simulation property.
\end{defn}

\section{Cover relation}

\paragraphsummary{Motivate and provide historical summary.} The reason we defined the
notion of a formal topology is that it admits a cover relation. This method goes back to
Johnstone's~\cite{stone-spaces} adaptation of the notion of a Grothendieck topology, that
was subsequently developed by Martin-L\"{o}f and Sambin~\cite{int-formal-spaces}. The
original formulation of Sambin suffered from the problem that it was not possible to
define the coproduct of two frames using it. This problem was solved by Coquand, Sambin,
and others~\cite{coq-sambin} by defining the cover relation inductively. It is this method
that we will follow in this development.

\paragraphsummary{Defn.~of coverage.}
First, we define the coverage relation on a given formal topology.
\begin{defn}[Coverage relation]
  \[
  \begin{prooftree}
    \hypo{ a \epsilon U }
    \infer1[\textsf{dir}]{\covers{a}{U}}
  \end{prooftree}
  \qquad
  \begin{prooftree}
    \hypo{
      \pity{b}{B(a)}{\left( \pity{c}{B(a, b)}{d(a, b, c) \triangleleft U} \right) \rightarrow a \triangleleft U}
    }
    \infer1[\textsf{branch}]{\covers{a}{U}}
  \end{prooftree}
  \]
\end{defn}

\paragraphsummary{Cover is a nucleus.}
This coverage relation gives us a way of obtaining a frame from a formal topology. Let us
look at the type of the coverage relation:
\begin{equation*}
  \_\covernm{}\_ : A \rightarrow \pow{A} \rightarrow \univ{}
\end{equation*}
which we can flip to get
\begin{equation*}
  \_\triangleright\_ : \pow{A} \rightarrow A \rightarrow \univ{}
\end{equation*}
which can be written as
\begin{equation*}
  \_\triangleright\_ : \pow{A} \rightarrow \pow{A}
\end{equation*}
so we have an endofunction. We can restrict this to the subset of $\pow{A}$ that is
downwards-closed
\begin{equation*}
  \_\triangleright\_ : \dcsubset{A} \rightarrow \dcsubset{A},
\end{equation*}
which of course requires us to show that given a downwards-closed subset $U$,
$\covers{\_}{U}$ is a subset that is downwards-closed.

\begin{thm}
  Given any subset $U$, $\covers{\_}{U}$ is a downwards-closed subset.
\end{thm}
\begin{proof}
  \todo{Complete the proof.}
\end{proof}

Now, our method of obtaining a frame out of a formal topology is the following.
\begin{enumerate}
  \item Start with a formal topology $\mathcal{T}$ on type $A$.
  \item $\mathcal{T}$ has an underlying poset $P$; construct its frame of downwards-closed
    subsets.
  \item Note: $\covers{\_}{\_}$ is a nucleus on the frame of downwards-closed subsets.
  \item We have shown (in Theorem~\ref{thm:fixed-point-frame}) that the set of
    fixed-points of every nucleus is a frame. The final frame is this fixed-point frame
    on the frame of downwards-closed subsets by the nucleus $\covers{\_}{\_}$.
\end{enumerate}

\section{Discipline presentations present}

\paragraphsummary{Explain the aim.}
We are now ready to shift our focus on what can be called main theorem of this
thesis: our notion of a formal topology is capable of presenting a frame. We
will see that this presents an interesting challenge, arising from the
peculiarity of \UF{}.

\paragraphsummary{Explain the intuition of representation.}
Let $A$ be a formal topology and $L$ a frame. Consider a monotonic function $f : \abs{A} \rightarrow
\abs{L}$ on the underlying posets. We will define a notion of $f$ representing $F$ in $L$
which is to say $f$ encodes crucial information of $F$.

\begin{defn}[Representation]
  \begin{equation*}
  \pity{x}{A}{\pity{y}{B(x)}{f(x) \sqsubseteq \left( \bigvee_{z~:~C(x, y)} f(d(x, y, z))}} \right)
  \end{equation*}
\end{defn}

\todo{State and prove the main theorem.}


%% \chapter{The Cantor formal topology}\label{chap:examples}

Foo bar...


\chapter{Conclusion}\label{chap:conc}

The aim of formal topology is to make type-theoretical sense out of topology. The recent
formulation of \emph{univalent} type theories present interesting novelties for the future
of type-theoretical mathematics. The meaning of these novelties for existing bodies of
mathematical knowledge is an active area of research and it is fair to say that they have
not yet been fully explored. Considering this, we have set out to investigate a
natural question: what will formal topology look like in the context of univalent type
theory?

The first outcome of this investigation was a negative result: we have come to the
conclusion that an \emph{as is} development of formal topology in univalent type theory is
problematic as a consequence of the distinction between \emph{property} and
\emph{structure}, rendered visible through the homotopical view of types. In particular,
our development led to a situation in which the proof structure of the covering relation
had to be collapsed for it to behave as a property, and at the same time, kept intact for
it to be used in succeeding proofs.

We have then found, somewhat unexpectedly, that this problem can be circumvented by using
HITs. Using HITs, we were able to reconcile, in our specific case, the imbalance between
property and structure. This allowed us to successfully carry out a rudimentary
development of formal topology in a univalent setting: we have been able to formulate a
notion of formal topology and then prove its universal property. We hence conclude that
formal topology is not only possible in univalent type theory but is possible in a way
that makes non-trivial use of its conceptual novelties.

HITs have been previously used for similar purposes, and have allowed the circumvention of
choice principles. The relevant work known to the author includes the higher
inductive-inductive definition of the Cauchy reals in the HoTT
book~\cite[Defn.~11.3.2]{hottbook} and the reconstruction of the partiality monad by
Altenkirch, Danielsson, and Kraus~\cite{partiality-revisited}. The latter seems to be
especially relevant to our use of HITs, as the view of topology as a theory of observable
properties is closely related to partial computations (see, for instance,
\cite{synthetic-topology, shulman-logic-of-space}).

In addition to providing a preliminary answer to the question of doing formal topology in
univalent type theory, we have carried out Coquand's idea of doing topology with
interaction systems on posets~\cite{coq-posets} to a further extent. We have found this to
work well in the context of univalent type theory. Apart from addressing issues of
univalence, this has been the secondary contribution of our development. To the author's
knowledge, this thesis presents the first formalised development of formal topologies as
interaction systems.

As a proof of concept of our approach to formal topology, we have implemented a
fundamental example: the Cantor space. Furthermore, we have proven that the formal Cantor
topology is compact. This exemplifies what we mean by ``making type-theoretical sense of
topology'': we have stated and then proven a non-trivial theorem of topology in a way that
is completely constructive and predicative.

After all, our development remains but an \emph{esquisse}. There remains much more work to
be done and it will be the undertaking of this that will put this approach to the test. We
now discuss some of these.

First and foremost, there are many more topological and locale-theoretic results to be
reconstructed in the setting of our development. For instance, one of the fundamental
theorems of topology is the Tychonoff theorem, formal versions of which have been
constructed~\cite{coq-tychonoff, vickers-tychonoff}. It is an interesting question if
these will fit well within our approach. Furthermore, a crucial class of spaces in
pointless topology is the \emph{Stone spaces}~\cite{stone-spaces}; indeed, the notion of a
Stone space is what brought the field into existence in the first place. Developing the
theory of Stone spaces in the context our development is an important direction for
further work.

Another limitation of our development is that we have not been able to explore the notion
of a \emph{formal point}, which can be introduced as a notion derived from the opens in
formal topology~\cite[pg.~94]{coq-sambin}. A prime example of this is the space of real
numbers, defined as the formal points of the formal topology of real numbers. It would be
interesting to explore such a definition of real numbers in univalent type theory.

Furthermore, formal topology has important connections to the theory of domains. As
pointed out by Sambin, domain theory can be viewed as a ``branch of formal
topology''~\cite{sambin-domains}. It was within our plans to provide a formal-topological
reconstruction of domain theory in univalent type theory, as an application of this
development. We have not been able to accomplish this due to time constraints. This would
be a very interesting example as it would likely make non-trivial use of the constructive
properties of our development. Domain theory in a univalent setting has recently been
investigated by de Jong~\cite{de-jong-domains}. A comparison of the two approaches might
prove beneficial.

Let us also discuss some of the lower-level shortcomings of our development that could be
more readily addressed.

As remarked in Chapter~\ref{chap:formal-topo}, our method for generating a frame from a
formal topology works only on formal topologies whose carrier set level and order level
are equal---this is a restriction we have imposed for the sake of presentational simplicity.
To reach full generality, these levels have to be generalised.

Furthermore, our statement of the universal property has not been modular. We have proven
that a frame generated from a formal topology using our method presents the frame. The
statement of this theorem, however, does not express what it means for an arbitrary formal
topology to present an arbitrary frame. Instead, we have stated this in an ad hoc way, for
the specific case of our generated frame. Ideally, this should be stated for \emph{any}
frame with a lifting map, our current statement then being one instance of it. Most
importantly, this would allow us to attempt to prove (which we think would be a successful
attempt) that two frames satisfying the universal property are isomorphic. As we proved
that isomorphic frames are equal, this would mean two such frames are \emph{equal}.

A further direction to take for the modularisation of the universal property would be to
state it using categorical terms. Conceptually, the universal property ought to amount to
the existence of an adjoint pair of a free and a forgetful functor. However, it is not
clear what the category of formal topologies is, as the notion of a morphism between two
interaction systems is not clear. Therefore, the resolution of this issue is likely not
straightforward.

We believe that the type-theoretical distillation of topology enables a conceptual
clarification if it. Thanks to the invention of univalent type theory, type-theoretical
mathematics has been invigorated, and is now rapidly progressing. We hope the discipline
of formal topology can benefit from this progress, ideally through a renewed interest by
the type theory community.


\makebackmatter{}

\chapter{Agda Formalisation}\label{app:agda-form}

\section{The \modname{Basis} module}

{
  \footnotesize
  \setmathfont{PragmataPro Mono Liga}
  \input{literate-latex/Basis}
}

\section{The \modname{Poset} module}

{
  \footnotesize
  \setmathfont{PragmataPro Mono Liga}
  \input{literate-latex/Poset}
}

\section{The \modname{Frame} module}

\footnotesize
\input{literate-latex/Frame}

\section{The \modname{Nucleus} module}

{
  \footnotesize
  \input{literate-latex/Nucleus}
}

\section{The \modname{Cover} module}

{
  \footnotesize
  \input{literate-latex/Cover}
}

\section{The \modname{CoverFormsNucleus} module}

{
  \footnotesize
  \input{literate-latex/CoverFormsNucleus}
}

\section{The \modname{FormalTopology} module}

{
  \footnotesize
  \input{literate-latex/FormalTopology}
}

\section{The \modname{UniversalProperty} module}

{
  \footnotesize
  \input{literate-latex/UniversalProperty}
}


\end{document}
