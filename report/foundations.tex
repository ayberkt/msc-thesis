\chapter{Foundations}\label{chap:foundations}

We present in this chapter the basics of univalent type theory. We assume that the reader
is familiar with standard (intensional) type theory and refrain from presenting the formal
calculus from scratch; for this, the reader is referred to Appendix A of the HoTT
Book~\cite{hottbook}. Instead, we focus on the fundamental notions and theorems of
univalent type theory.

Our presentation in this chapter has been heavily influenced by three resources: (1) the
HoTT Book~\cite{hottbook}, (2) Martín Escardó's introductory exposition of HoTT/UF in
\veragda{}~\cite{escardo-uf-intro}, and (3) the \texttt{cubical} library of
\veragda{}~\cite{agda-cubical}.

\section{Formal vs.~informal mathematics}

Our goal in this thesis is to use type theory in an informal way, much like how set theory
is frequently used without committing to a specific one of its implementations. The work
to be presented is in fact the \emph{informalisation} of a completely formalised
development, for which the formal system of choice is Cubical Type Theory\footnote{%
  The implementation of Cubical Type Theory in the cubical extension of
  \veragda{}~\cite{cubical-agda}, to be specific.
}
(CTT)~\cite{CCHM}. CTT is just one instance of a univalent foundation, and its details
will therefore not concern us in this thesis. We will instead abstract over the
implementation, and use univalent type theory as a \emph{practical}
foundation~\cite{PFOM, PFPL} whose formal details are left unspecified. The reader
interested in the details of the formal development can find the (cubical) \veragda{}
formalisation in Appendix~\ref{app:agda-form}.

To keep our informal presentation precise, we will closely follow the notational
conventions of the HoTT Book~\cite{hottbook}, as explained in
\cite[Sec.~I.1]{hottbook}. For instance, we will often make use of pattern matching: we
will write $\oftyI{(x, \_)}{\sigmaty{a}{A}{B(a)}}$ rather than using the projection
notation for projecting out the first component. There will be some minor notational
deviations and abuses of language, which will be addressed locally, in the relevant
sections.

Finally, we note that we will be introducing new definitional equalities in various
definitions. For the convenience of the reader who is reading this thesis electronically,
the definienda have been made clickable, which is marked by the
{\color{darkgreen} dark green} colour.

\section{Univalent foundations: the idea}

Starting with the work of Hofmann and Streicher~\cite{hofmann-streicher}, the idea that
types (as conceived in intensional type theory) have connections to homotopy theory had
been brewing in the type theory community, the key idea being that types and their types
of equality proofs can be viewed as spaces (identified up to homotopy) and
their spaces of \emph{paths}. The solidification of this idea started around 2006, through
the efforts of various researchers~\cite{warren-fmcs, uppsala-awodey, uppsala-garner, uppsala-van-den-berg, voevodsky-very-short}
exploring this connection from different perspectives. It was around this time that the
concept of univalence emerged. We will refrain from addressing the history of the subject
in detail.

Among these researchers, Vladimir Voevodsky~\cite{voevodsky-bernays} emphasised the
possibility of transforming type theory through insights arising from this line of work.
He coined the term ``univalent foundations'' to refer to his vision of mathematical
foundations that support univalent mathematics.

So what is univalence and what are the novelties it presents for mathematical foundations?
Foundations have historically suffered problems with the notion of equality. In
mathematics, superficial differences between structures are immaterial. For instance, we
never want to talk about the difference between isomorphic graphs. It has therefore been
common practice in mathematics to work with the ``right'' notion of equality between
structures: homeomorphism for topological spaces, isometry for metric spaces, equivalence
for categories, group isomorphism for groups, and so on.

A reasonable expectation from a foundational system for mathematics is that it be able to
accommodate the common practices occurring in its high-level use in a natural way. Both
set theory and type theory have suffered the problem that they could not accommodate this
practice well. In the context of type theory, however, this problem can be seen more
clearly: the implementation this informal practice in type theory gave rise to what is
known as ``setoid hell''~\cite{homotopy-heaven}.

Homotopy type theory, being the first of univalent type theories, is arguably the first
foundational system that validates this practice of working with the ``right'' equality.
As remarked by Escardó in his introductory exposition of HoTT/UF~\cite{escardo-uf-intro},
once the type corresponding to a certain algebraic structure has been written down in
univalent type theory, the identity type corresponds \emph{automatically} to the type of
isomorphisms for that structure. It is quite remarkable that this can be achieved by a
minor modification to type theory: first, make sure that the rich structures of types are
not trivialised by the likes of Axiom K, and then add a simple axiom---the univalence
axiom---allowing one to prove that \emph{equivalent} types are equal.

One immediate application of this axiom is that it allows one to prove function
extensionality, that is commonly added as a postulate to type theory. The fact that one
has to extend type theory with a postulate for something as natural as function
extensionality can be considered a deficiency of type theory, and that univalence renders
it into a theorem can therefore be considered an noteworthy improvement.

\section{Equivalence and univalence}

At the heart of univalent type theory lies the notion of \emph{type equivalence}. We build
up to its definition (Defn.~\ref{defn:equiv}) in this section.

Following the HoTT book~\cite[Sec.~I.1.12]{hottbook}, we denote by $x =_A y$ the type of
identity proofs (or ``identifications'') between terms $x$ and $y$. This is given by the
constructor $$\oftyI{\mathsf{refl}}{\pity{x}{A}{x =_A x}},$$ whose elimination counterpart
is the usual path induction rule. As mentioned before, we will abstract over such
technicalities and engage in a high-level use of univalent type theory.

We start by delineating the class of types that have exactly one inhabitant.
\begin{defn}[Contractible]\label{defn:contr}
  A type $A$ is called contractible if it has exactly one inhabitant:
  \begin{equation*}
    \iscontr{A} \quad\is\quad \sigmaty{c}{A}{\pity{x}{A}{c =_A x}}.
  \end{equation*}
\end{defn}

\begin{defn}[Fiber]\label{defn:fiber}
  Given types $\oftyI{A}{\univ{}_m}$, $\oftyI{B}{\univ{}_n}$, a function $f : A
  \rightarrow B$, and some $\oftyI{y}{B}$, the fiber of $f$ over $y$ is the type of all
  inhabitants of $A$ that are mapped by $f$ to $y$:
  \begin{equation*}
    \fiber{f}{y} \quad\is\quad \sigmaty{x}{A}{f(x) =_B y}.
  \end{equation*}
\end{defn}

The following definition of type equivalence was first formulated by Voevodsky.
\begin{defn}[Equivalence]\label{defn:equiv}
  Given types $\oftyI{A}{\univ{}_m}$, $\oftyI{B}{\univ{}_n}$, a function $f : A \rightarrow B$ is an
  \emph{equivalence} if $\fiber{f}{y}$ is a contractible type for every $\oftyI{y}{B}$:
  \begin{equation*}
    \isequiv{f} \quad\is\quad \pity{y}{B}{\iscontr{\fiber{f}{y}}}.
  \end{equation*}
\end{defn}

We will denote by $\typequiv{A}{B}$ the type of equivalences between types $A$ and $B$.
\begin{defn}[Equivalence of types]
  Given $\oftyI{A}{\univ{}_m}$, $\oftyI{B}{\univ{}_n}$,
  \begin{equation*}
    \typequiv{A}{B} \quad\is\quad \sigmaty{f}{A \rightarrow B}{\isequiv{f}}.
  \end{equation*}
\end{defn}

One may wonder why the notion of type equivalence is not the familiar categorical notion
of isomorphism, that is, a function between types that is invertible. We will address this
point in Section~\ref{subsec:prop}.

\begin{defn}[Identity equivalence]\label{defn:id-equiv}
  The identity function on every type $A$ can easily be seen to be an equivalence.
  Therefore there exists a function:
  \begin{equation*}
    \oftyI{\idequivnm}{\pity{A}{\univ{}_n}{\typequiv{A}{A}}}.
  \end{equation*}
\end{defn}

Given two types $\oftyII{A}{B}{\univ{}_n}$, and a proof $\oftyI{p}{A = B}$ that they are
equal, we can clearly prove that they are equivalent.
\begin{defn}\label{defn:id-to-equiv}
  Given any $\oftyII{A}{B}{\univ{}_n}$ with a proof $\oftyI{p}{A = B}$, there exists a
  function:
  \begin{equation*}
    \oftyI{\idtoeqvnm{}}{\pity{A, B}{\univ{}_n}{A = B \rightarrow \typequiv{A}{B}}}.
  \end{equation*}
\end{defn}

In other words, it is justified intuitively that equality is stronger than equivalence.
The famous univalence axiom states, essentially, that equivalence is as strong as
equality. Formally, this amounts to saying that $\idtoeqvnm{}$ is an equivalence.

\begin{ax}[Univalence]\label{ax:ua}
  The following type has an inhabitant:
  \begin{equation*}
    \pity{A, B}{\univ{}_n}{\isequiv{\idtoeqv{A}{B}}}.
  \end{equation*}
\end{ax}

The addition of this simple axiom into type theory has some surprising consequences.
Perhaps most importantly, it allows us to \emph{prove} function extensionality as
mentioned.
\begin{defn}[Extensional equality]\label{defn:exteq}
  Given types $\oftyI{A}{\univ{}_m}$, $\oftyI{B}{\univ{}_n}$, and functions
  $\oftyII{f}{g}{A \rightarrow B}$, $f$ and $g$ are said to be extensionally equal if the following
  type is inhabited:
  \begin{equation*}
    \exteq{f}{g} \quad\is\quad \pity{x}{A}{f(x) = g(x)}.
  \end{equation*}
\end{defn}

\begin{prop}[Function extensionality]\label{prop:funext}
  Two functions are equal whenever they are extensionally equal.
\end{prop}

This result was first proven by Voevodsky\footnote{%
  Attributed to Voevodsky in \cite{escardo-uf-intro}.
}.

\section{Homotopy levels}

One of the fundamental observations arising from the association of type theory with
homotopy theory is that we can classify types with respect to the nontrivial homotopy
structure they bear. Let $A$ be a type. Given any $\oftyII{x}{y}{A}$, the identity type $x
= y$ is the type of equality proofs between $x$ and $y$. Given, then, some
$\oftyII{p}{q}{x = y}$, we can talk about the type $p = q$ of equality proofs between the
equality proofs. We can repeat this process \emph{ad infinitum} meaning any type $A$
induces an infinite tower of types of equality proofs. When we use the term
\emph{dimension}, we are referring to levels of this tower. A space containing no
nontrivial homotopy above dimension $n$ is called a homotopy $n$-type (or said to be of
homotopy level $n$).

We then recursively define a predicate expressing that a given type has homotopy level
$n$. The idea is that we increase the dimension by one (i.e., go from talking about a type
to talking about its type of equality proofs) at each step.
\begin{defn}[Homotopy level]\label{defn:hlevel}
  We will say that the homotopy level of a type $A$ is $\oftyI{n}{\mathbb{N}}$ if
  $\isofhlevel{A}{n}$ as inhabited:
  \begin{align*}
    \isofhlevel{A}{\zero{}} &\quad\is\quad \iscontr{A}\\
    \isofhlevel{A}{\suc{n}} &\quad\is\quad \pity{x~y}{A}{\isofhlevel{x =_A y}{n}}.
  \end{align*}
\end{defn}

The hierarchy of $n$-types is upwards-closed as expressed in the following proposition.
\begin{prop}\label{prop:level-up}
  Given any type $A$, the following type has an inhabitant:
  \begin{equation*}
    \pity{n}{\mathbb{N}}{\isofhlevel{A}{n} \rightarrow \isofhlevel{A}{\suc{n}}}.
  \end{equation*}
\end{prop}

Furthermore, $\sum$ and $\prod$ types respect h-levels.
\begin{prop}\label{prop:is-of-level-pi}
  Given an arbitrary type $\oftyI{A}{\univ{}_m}$ and an $A$-indexed family of $n$-types
  $\oftyI{B}{A \rightarrow \univ{}_o}$, $\pity{x}{A}{B(x)}$ is an $n$-type. In formal
  terms:
  \begin{equation*}
    \left( \pity{x}{A}{\isofhlevel{B(x)}{n}} \right)
      \rightarrow \isofhlevel{\pity{x}{A}{B(x)}}{n}.
  \end{equation*}
\end{prop}

\begin{prop}\label{prop:is-of-level-sigma}
  Given an $n$-type $\oftyI{A}{\univ{}_m}$ and an $A$-indexed family
  $\oftyI{B}{A \rightarrow \univ{}_o}$ of $n$-types, $\sigmaty{x}{A}{B(x)}$ is an $n$-type.
  This is formally expressed by the type below.
  \begin{equation*}
      \isofhlevel{A}{n}
    \rightarrow \left( \pity{x}{A}{\isofhlevel{B(x)}{n}} \right)
    \rightarrow \isofhlevel{\sigmaty{x}{A}{B(x)}}{n}
  \end{equation*}
\end{prop}

\subsection{Propositions}\label{subsec:prop}

The homotopy level of one is of special interest: it is the class of types that behave
like logical propositions, in the sense that their proof structures are trivial; they can
inhabited by at most one term. This is precisely the property we desire in those types
that we view as expressing \emph{properties}: we are interested, not in what they are
inhabited by, but in whether they are inhabited or not.

\begin{defn}[Proposition]\label{defn:prop'}
  A type $A$ is a proposition (sometimes disambiguated as an \emph{h-proposition} if it
  has a homotopy level of one:
  \begin{equation*}
    \isprop{A} \quad\is\quad \isofhlevel{A}{1}.
  \end{equation*}
\end{defn}

An equivalent and much more intuitive way of expressing propositionality is the following.
\begin{defn}[Proposition (official)]\label{defn:hprop}
  \begin{equation*}
    \isprop{A} \quad\is\quad \pity{x~y}{A}{x =_A y}.
  \end{equation*}
\end{defn}
\begin{prop}
  Definitions \ref{defn:prop'} and \ref{defn:hprop} are equivalent.
\end{prop}
Due to this equivalence, we will use definitions \ref{defn:prop'} and \ref{defn:hprop}
interchangeably. Which definition we are referring to should be clear in context.

We collect propositional types at level $n$ in the type $\hprop{}_n$.
\begin{defn}[$\hprop{}$]\label{defn:omega}
  $\hprop{}_n$ is the type of all propositional types at universe $n$:
  \begin{equation*}
    \hprop{}_n \quad\is\quad \sigmaty{A}{\univ{}_n}{\isprop{A}}.
  \end{equation*}
\end{defn}

When asserting that a proposition $\oftyI{A}{\hprop{}_n}$ is inhabited, we have to project
out the first component to be completely precise. We will engage in the excusable
notational abuse of denoting this projection by $A$ itself. In the \veragda{}
formalisation, we use the notation of the \libcub{}~\cite{agda-cubical} library in which
this is denoted \fnname{[} $A$ \fnname{]}.

Once we introduce this delineation of the class of propositional types, we have to be
careful about the distinction between \emph{property} and \emph{structure}. Ideally, types
that are thought of as expressing propositions should always be propositional. For
instance, the definition of equivalence we gave (in Defn.~\ref{defn:equiv}) has the
crucial property that it is propositional. This would not always be the
case~\cite{hottbook} if we were to use the usual notion of equivalence as an invertible
function.

Many types that are expected to behave like propositions behave naturally like
propositions so it suffices to prove their propositionality; for those that do not,
univalent type theory provides a mechanism, called \emph{propositional truncation}, for
forcing them to be propositional. A substantial component of the work presented in this
thesis pertains to problems related to the use of this mechanism.

Now, let us state some simple facts about the class of propositional types. Observe the
two corollaries, of propositions \ref{prop:is-of-level-pi} and
\ref{prop:is-of-level-sigma} respectively.

\begin{prop}\label{prop:pi-prop}
  Given $\oftyI{A}{\univ{}_m}$, $\oftyI{B}{A \rightarrow \univ{}_n}$, the type $\pity{x}{A}{B(x)}$
  is a proposition whenever $B(x)$ is proposition every $\oftyI{x}{A}$.
\end{prop}

\begin{prop}\label{prop:sigma-prop}
  Given types $\oftyI{A}{\univ{}_m}$ and $\oftyI{B}{A \rightarrow \univ{}_n}$, if $A$ is a
  proposition and $B$ is a family of propositions, then $\sigmaty{x}{A}{B(x)}$ is a
  proposition.
\end{prop}

The type expressing that a given type $A$ is a proposition, is itself a proposition.
\begin{prop}
  Given $\oftyI{A}{\univ{}_n}$, the type $\isprop{A}$ is propositional.
\end{prop}

Note also the following fact.
\begin{prop}\label{prop:to-subtype}
  Given some type $A$, a family of propositions $\oftyI{B}{A \rightarrow \hprop{}}$, and two
  inhabitants $\oftyII{(x, p)}{(y, q)}{\sigmaty{a}{A}{B(a)}}$, if $x = y$ then
  $(x, p) = (y, q)$.
\end{prop}
This can in fact be strengthened to an equivalence but this form will be sufficient for
our purposes. Such a type $\sigmaty{a}{A}{B(a)}$, where $B$ is a family of propositions,
can be thought of as a ``subtype'' of $A$: the restriction of $A$ to its inhabitants that
satisfy the \emph{property} $B$.

As we view propositional types as embodying properties, it is natural to expect the right
notion of equivalence between them to be logical equivalence.
\begin{defn}\label{defn:iff}
  Types $\oftyI{A}{\univ{}_m}$ and $\oftyI{B}{\univ{}_n}$ are logically equivalent
  (denoted $\logequiv{A}{B}$) if the following type is inhabited:
  \begin{equation*}
    \logequiv{A}{B} \quad\is\quad (A \rightarrow B) \times (B \rightarrow A).
  \end{equation*}
\end{defn}

We can indeed show that logically equivalent h-propositions are equivalent.
\begin{prop}\label{prop:iff-equiv}
  Given propositions $\oftyI{P}{\univ{}_m}$, $\oftyI{Q}{\univ{}_n}$, if $\logequiv{P}{Q}$
  then $\typequiv{A}{B}$.
\end{prop}

As we mentioned before, Voevodsky's notion of type equivalence enjoys the crucial property
that it is propositional. Let us consider the familiar notion of equivalence between
types.
\begin{defn}[Type isomorphism]\label{defn:type-iso}
  Given types $A$ and $B$, the type of type isomorphisms between them is denoted by
  $\typiso{A}{B}$:
  \begin{alignat*}{2}
    \istypiso{f}   \quad&\is\quad && \sigmaty{g}{B \rightarrow A}{\exteq{g \circ f}{\idfn{A}} \times \exteq{f \circ g}{\idfn{B}}} \\
    \typiso{A}{B} \quad&\is\quad &&\sigmaty{f}{A \rightarrow B}{\istypiso{f}}                          .
  \end{alignat*}
\end{defn}

This would not be a viable definition of type equivalence in univalent type theory as it
suffers the problem that it is not propositional: although the inverse a function is
necessarily unique, the invertibility proofs of are not unique in the general case when we
are dealing with $n$-types. Therefore, Definition~\ref{defn:type-iso} is logically
equivalent but not equivalent to Definition~\ref{defn:equiv}.

\begin{prop}\label{prop:iso-equiv-equiv}
  Definitions \ref{defn:type-iso} and \ref{defn:equiv} are logically equivalent.
\end{prop}


Notice, however, that this logical equivalence is an equivalence in the special where the
domain and the codomain are h-sets.

When we have some $\oftyI{f}{A \rightarrow B}$ such that $\isequiv{f}$, we will speak of the inverse
of $f$ and denote this $\invequiv{f}$. This expression refers implicitly to this logical
equivalence. The distinction between $\typequiv{A}{B}$ and $\typiso{A}{B}$ will be
especially important in Section~\ref{sec:frame-univ}.

\subsection{Sets}

The other homotopy class of interest is the class of types whose proof structures are not
necessarily trivial but the proof structures of their types of equality proofs are
trivial. Put more simply, inhabitants of such types are equal to each other in
\emph{at most one way}. This is the class of types that are called sets.
\begin{defn}[Set]\label{defn:hset}
  A type $A$ is a set (disambiguated as h-set) if its homotopy level is two:
  \begin{equation*}
    \isset{A} \quad\is\quad \isofhlevel{A}{2}.
  \end{equation*}
\end{defn}

The structures we present in chapters \ref{chap:frames} and \ref{chap:formal-topo} have
underlying partially ordered \emph{sets}. When implementing these in univalent type
theory, they must be required to be h-sets. A more accurate name for a ``poset'' with a
carrier set bearing higher dimensional homotopy structure would be \emph{$\infty$-poset}.

Observe the following corollary of Proposition~\ref{prop:level-up}.
\begin{prop}\label{prop:prop-is-set}
  Every proposition is a set.
\end{prop}

We also note that the type $\hprop{}$ of h-propositions is an h-set.
\begin{prop}\label{prop:hprop-set}
  $\hprop{}$ is an h-set.
\end{prop}

As we have done for the case of h-propositions, we consider the special cases
of propositions \ref{prop:is-of-level-pi} and \ref{prop:is-of-level-sigma} in the
h-set case.

\begin{prop}\label{prop:pi-set}
  Given $\oftyI{A}{\univ{}_m}$, $\oftyI{B}{A \rightarrow \univ{}_n}$, the type $\pity{x}{A}{B(x)}$
  is a set whenever $B(x)$ is a set for every $\oftyI{x}{A}$.
\end{prop}

\begin{prop}\label{prop:sigma-set}
  Given types $\oftyI{A}{\univ{}_m}$ and $\oftyI{B}{A \rightarrow \univ{}_n}$, if $A$ is a set and
  $B(a)$ is a set for every $\oftyI{a}{A}$, then $\sigmaty{x}{A}{B(x)}$ is a set.
\end{prop}

Notice also that the type expressing that a given a type is a set is a proposition.
\begin{prop}\label{prop:set-prop}
  Given any type $A$, the type $\isset{A}$ is a proposition.
\end{prop}

We conclude this section by presenting a highly useful sufficient condition for setness.
Let us write down the usual definition of a decidable type.
\begin{defn}\label{defn:decidable}
  A type $A$ is called is decidable if it satisfies the law of excluded middle:
  \begin{equation*}
    \isdec{A} \quad\is\quad \pity{x~y}{A}{A + \neg A}.
  \end{equation*}
\end{defn}

A type is called discrete if its type of equality proofs is decidable.
\begin{defn}\label{defn:discrete}
  A type $A$ is called discrete if the following type is inhabited:
  \begin{equation*}
    \isdisc{A} \quad\is\quad \pity{x~y}{A}{\isdec{x =_A y}}.
  \end{equation*}
\end{defn}

The following sufficient condition for setness is due to Michael Hedberg~\cite{hedberg}
and is known as Hedberg's theorem.
\begin{thm}[Hedberg]\label{thm:hedberg}
  Every discrete type is a set.
\end{thm}

\section{Powersets}\label{sec:pow}

Given a type $A$, we would like to talk about its powerset: the type of \emph{properties}
inhabitants of $A$ might have. This is nothing but the type $A \rightarrow \hprop{}$. Note that the
codomain is $\hprop{}$ rather than $\univ{}$; it is here that the property-structure
distinction starts to become visible: one would have to use $A \rightarrow \univ{}$ for this purpose
in non-univalent type theory. Of course, $\hprop{}$ could be defined in non-univalent type
theory as well, but non-univalent type theory is too restrictive for one to engage in
sophisticated developments that maintain this distinction between structure and property.
Most importantly, there are no methods for turning structure into property in
non-univalent type theory meaning one would not be able to express many crucial
inhabitants of $A \rightarrow \hprop{}$ as they would involve non-trivial proof structure.

We represent the subsets of inhabitants of a type using the powerset. This is just one
such representation; we will present another common approach in Section~\ref{sec:fam}.

\begin{defn}\label{defn:pow}
  Given a type $\oftyI{A}{\univ{}_n}$, the \emph{powerset} of $A$ is the type of all
  predicates on $A$: $\pow{A} \is A \rightarrow \hprop{}_n$. We introduce a bit of syntactic sugar
  and write membership in a subset as: $\mempow{x}{U} \is U(x)$.
\end{defn}

Our use of the word ``set'' in ``powerset'' is justified by the following proposition.
\begin{prop}\label{prop:pow-set}
  Given \emph{any} type $A$, $\pow{A}$ is an h-set.
\end{prop}
\begin{proof}
  By Proposition~\ref{prop:pi-set}, suffices to show that the codomain, $\hprop{}$, is
  a set. This is given by Proposition~\ref{prop:hprop-set}.
\end{proof}

We can define the usual inclusion order on the powerset of a given a type.
\begin{defn}\label{defn:inclusion}
  Given a type $\oftyI{A}{\univ{}_n}$ and subsets $\oftyII{U}{V}{\pow{A}}$,
  $U$ is a subset of $B$ if the following type is inhabited:
  \begin{equation*}
    \subsetof{U}{V} \quad\is\quad \pity{x}{A}{x \in U \rightarrow x \in V}.
  \end{equation*}
  The propositionality follows from Prop.~\ref{prop:pi-prop}.
\end{defn}

\begin{defn}[Full subset]\label{defn:entire-subset}
  Given a type $\oftyI{A}{\univ{}_n}$, the subset containing all inhabitants of $A$ is
  defined as the constant function:
  \begin{equation*}
    \entire{A} \quad\is\quad \lambda\_.~\unitty{}_n.
  \end{equation*}
\end{defn}

Notice that the definition of $\pownm{}$ requires the involved h-proposition to live in
the same universe as $A$. It is for this reason that we generalise the $\unitty{}$ type to
live in the universe level we provide it as an argument; we denote this using a subscript.

\begin{defn}\label{defn:intersection}
  Given a type $\oftyI{A}{\univ{}_n}$ and subsets $\oftyII{U}{V}{\pow{A}}$, the subset
  delineating those elements that are in \emph{both} of $U$ and $V$ is defined as:
  \begin{alignat*}{2}
    \intersect{\_}{\_}  \quad&:\quad   && \pow{A} \rightarrow \pow{A} \rightarrow \pow{A}         \\
    \intersect{U}{V}    \quad&\is\quad && \lambda x.~ \mempow{x}{U} \times \mempow{x}{V} .
  \end{alignat*}
  We note that $x \in \intersect{U}{V}$ is propositional by the propositionality of
  $\mempow{x}{U}$ and $\mempow{x}{V}$ combined via Proposition~\ref{prop:sigma-prop}.
\end{defn}

\section{Families}\label{sec:fam}

In type theory, we have two common notions of ``subset of inhabitants of a type''. One of
these is the powerset as we presented in Sec.~\ref{sec:pow}. We now briefly summarise the
other approach in which we view a subset of a type $A$ as an $A$-valued function (on some
domain).

\begin{defn}[Family]\label{defn:fam}
  Given a type $A$, an $I$-indexed family of inhabitants of $A$ is simply a function:
  $I \rightarrow A$. We collect the type of all families on $A$ in the following type:
  \begin{equation*}
    \sub{o}{A} \quad\is\quad \sigmaty{I}{\univ{}_o}{I \rightarrow A}.
  \end{equation*}
  Notice that this is parameterised by a level $o$, being the level of the index type.
  Given a family $\oftyI{U \equiv (I, f)}{\sub{o}{A}}$, the evaluation of $U$ at $\oftyI{i}{I}$
  will be denoted $U_i$.
\end{defn}

\paragraph*{Notational clarification.} An inhabitant of $\sub{o}{A}$ has the form
$(I, f)$ as defined in Defn.~\ref{defn:fam}. However, we will use some more suggestive
notation for pairs of this form. It is quite convenient to be able to talk about a family
as though it were a concrete set, so we will use the notation $\{ x_i ~|~ i \in I \}$ as a
substitute for ``$I$-indexed family whose $i$th projection is denoted $x_i$''. Furthermore,
we will talk quite often about join operators $\oftyI{\bigvee\_}{\sub{o}{A} \rightarrow A}$. Instead of
writing $\bigvee (I, \lambda i.~e(i))$ for the application of a join operator to a family, we will
use the more familiar notation:
\begin{equation}\label{eqn:join-syntax}
  \bigvee_i e.
\end{equation}
What is meant by such notational sugar is expected to be obvious to the reader.

\begin{defn}[Family membership]\label{defn:fam-mem}
  Given a type $A$, some $\oftyI{x}{A}$, and a family $U \equiv (I, f)$, we say
  that $x$ is a member of $U$ if the fiber of $f$ over $x$ is inhabited:
  \begin{equation*}
    \memfam{x}{U} \quad\is\quad \fiber{f}{x}.
  \end{equation*}
\end{defn}

Let us note at this point that, technically, we would like to work with families whose
functions are injective since repeated elements in the image are not desirable. If we were
to add this requirement, we could prove $\memfam{x}{U}$ to be propositional. To keep the
presentation simple, however, we will not do this; this is not a fact we need for the work
presented in this thesis.

\begin{defn}[Image over a family]\label{defn:fam-img}
  Given a type $A$, some family $U \equiv (I, f)$ on $A$, and a function $\oftyI{g}{A \rightarrow B}$,
  the image of $g$ over $U$ is nothing but the family $(I, g \circ f)$. We will denote this
  $\img{g}{U}$ or $\setof{ g(f(i)) ~|~ \oftyI{i}{I} }$.
\end{defn}

Given an inhabitant $U$ of the powerset of some type $A$, we can represent $U$ as a family
as defined below.
\begin{defn}[Familification of a powerset]
  Let $A$ be a type and $\oftyI{U}{\pow{A}}$. The family associated with $U$ is
  the pair $(I, f)$ where
  \begin{align*}
    I \quad&\is\quad \sigmaty{x}{A}{\mempow{x}{U}} \\
    f \quad&\is\quad \mathsf{pr}_1        .
  \end{align*}
\end{defn}

As the representation we are working with will be clear to the reader from the context, we
will engage in a form of notational abuse and will not explicitly denote that we are
working with the familification of a powerset. In the \veragda{} formalisation, this is
denoted $\AgdaFunction{⟪} U \AgdaFunction{⟫}$.

\section{Higher Inductive Types}

In ordinary type theory, the inductive definition of a type amounts to a specification of
its points. In univalent type theory, we are working not just with $2$-types but with
arbitrary $n$-types. Therefore, univalent type theory generalises the mechanism for the
inductive generation of types by generalising points ($0$-paths) to $n$-paths. In other
words, it provides the means to define a type not just by specifying its points, but also
the paths between the points, paths between those paths, and then paths between those and
so on. Such an inductive type is called a \emph{higher inductive type} (HIT for short).

Let us give an example of what an HIT definition might look like. The usual $\unitty{}$
type would be inductively defined by the specification of a single constructor:
\begin{equation*}
  \begin{prooftree}
    \infer0{\oftyI{\rulename{\star}}{\unitty{}}}
  \end{prooftree}
\end{equation*}
whose only equality is the trivial one:
$\oftyI{\mathsf{refl}_{\rulename{\star}}}{\rulename{\star} =_{\unitty{}} \rulename{\star}}$. This means
that $\unitty{}$ contains no nontrivial paths.

HITs allow us to insert nontrivial paths into a type. For instance, we can add a new
equality of $\rulename{\star}$ to itself, that we will call $\rulename{loop}$; the resulting
type is called the $\mathsf{Circle}$.
\begin{equation*}
  \begin{prooftree}
    \infer0[]{\oftyI{\rulename{base}}{\mathsf{Circle}}}
  \end{prooftree}
  \hspace{2em}
  \begin{prooftree}
    \infer0[]{\oftyI{\rulename{loop}}{\mathsf{base} = \mathsf{base}}}
  \end{prooftree}
\end{equation*}

Of course, when generalising the machinery for defining inductive types in this way, we
have to be careful about the corresponding \emph{introduction} and \emph{elimination}
rules. The introduction rules are easy to deal with as they are given by the constructors.
The elimination rules, however, are quite subtle and it is technically intricate to
completely address the problem of extracting (dependent) eliminators from higher inductive
definitions. Even in the HoTT Book~\cite[Sec.~6.2]{hottbook}, this issue is not addressed
in a completely technical way. Similarly, we will consider in the following section, a
useful instance of an HIT and present its recursion principle. Hopefully, this should
provide sufficient intuition for the reader; it certainly suffices for the purposes of
our development.

HITs turn out to be immensely useful and in fact we will explain later that they are used
in a crucial way in this thesis: the main construction of Chapter~\ref{chap:formal-topo}
would not have been possible without them (i.e., it is not known to us how it would have
been possible without them).

\section{Truncation}

A particular HIT of interest will be the \emph{propositional truncation} type. This will
allow us to \emph{forget} the proof structure of a type, or in other words, force a
non-propositional type to be propositional by collapsing its proof structure.

This turns out to be quite convenient as it gives us a way of turning structure into
property. Consider for instance two propositions $\oftyII{P}{Q}{\hprop{}}$. We would like
to be able to express their logical disjunction and we would like this to be a proposition
itself. It would clearly not be propositional if we were to define their disjunction as
the sum type $P + Q$ since inhabitants of sum types contain the information of which one
of the summands they come from. For this to become propositional, we have to somehow
\emph{forget} this information; propositional truncation allows us to do precisely this.

We now give the definition of propositional truncation as an HIT.
\begin{defn}[Propositional truncation]\label{defn:truncation}
  Given a type $A$, its propositional truncation $\trunc{A}$ is an HIT given by one point
  constructor, $\rulename{|}\_\rulename{|}$, and one path constructor,
  $\rulename{trunc}$.
  \begin{equation*}
    \begin{prooftree}
      \hypo{\oftyI{x}{A}}
      \infer1{\oftyI{\rulename{|}x\rulename{|}}{\trunc{A}}}
    \end{prooftree}
    \qquad
    \begin{prooftree}
      \hypo{\oftyI{x}{\trunc{A}}}
      \hypo{\oftyI{y}{\trunc{A}}}
      \infer2{\oftyI{\rulename{trunc}}{x = y}}
    \end{prooftree}
  \end{equation*}
  The recursion principle for propositional truncation is the following: given any two
  types $A$ and $B$, to show $\trunc{A} \rightarrow B$, it suffices to show $\isprop{B}$ and $A \rightarrow
  B$. Formally,
  \begin{equation*}
    \isprop{B} \rightarrow (A \rightarrow B) \rightarrow \trunc{A} \rightarrow B.
  \end{equation*}
\end{defn}
The $\rulename{trunc}$ rule is often called $\rulename{squash}$ but we choose this name to
prevent ambiguity as we will later introduce another HIT featuring a rule called
$\rulename{squash}$.

Intuitively, why does the recursion principle have this form? To define a function
$\nats{} \rightarrow B$ on the natural numbers using their elimination principle, we need to
construct two terms: (1) an inhabitant of $B$ for the base case, and (2) a
$\nats{}$-indexed family of inhabitants of $B \rightarrow B$ for the inductive case; these
correspond to the types of the (non-dependent) constructors $\oftyI{\zero{}}{\nats{}}$ and
$\oftyI{\sucnm{}}{\nats{} \rightarrow \nats{}}$. In the case of truncation, we have a (dependent)
constructor of the form $\oftyI{\rulename{trunc}}{\pity{x~y}{\trunc{A}}{x = y}}$.
Accordingly, in addition to some term of type $A \rightarrow B$ for the $\rulename{|}\_\rulename{|}$
constructor, we now have to require a $(B \times B)$-indexed family of equality proofs: for
each $\oftyII{x}{y}{B}$, a proof of $x = y$, which is none other than the requirement of
$B$ to be propositional.

Observe that this corresponds to the intuitive distinction between \emph{property} and
\emph{structure}: we are allowed to use the inhabitant of a propositionally truncated type
only if it will be used in the proof of another property i.e., only in a context where the
structure will not become exposed.

When we use propositional truncation, we will not explicitly refer to these two
constructors. Instead, we will simply mention that some $\trunc{A} \rightarrow B$ is enabled by the
propositionality of $B$.

This definition of propositional truncation as an HIT brings us to the end of this
chapter. The foundational notions we have collected so far suffice to carry out an
investigation of formal topology in univalent foundations.
