\chapter{Foundations}\label{chap:foundations}

In this chapter, we provide the preliminary constructions and theorems of \UF{} for the
sake of self-containment. It is intended to be a summary rather than an introduction to
HoTT/UF.

\section{The identity type}

\todo{
  Do a couple of things here.
  \begin{enumerate}
    \item Define the identity type and present its introduction and elimination rules.
    \item Give a brief summary of Cubical Type Theory and the fact that that's what really
      we are making use of.
    \item Remark that the refl is a built-in in CTT.
    \item Explain that the details of CTT need not concern us as we will be viewing it as
      just one particular implementation of HoTT.
  \end{enumerate}
}

\section{Pi types}

\todo{Complete.}

\section{Sigma types}

\todo{Complete.}

\section{Homotopy levels}

\begin{prop}\label{isOfHLevelSigma}
  $\Sigma$ types preserve h-levels.
\end{prop}
\begin{proof}
  \todo{Sketch the proof}.
\end{proof}

\section{Propositions}

We collect propositional types in $\Omega$.

\begin{defn}\label{omega}
  $\Omega \is \sigmaty{A}{\univ}{\isaprop{A}}$
\end{defn}

Technically, to assert that some $A~:~\Omega$ holds, we must project out the first component.
However, in informal writing we will engage in the notational abuse of taking this to be
implicit.

\section{Propositional truncation}

\section{Subsets}

Given a type $A$, we would like to talk about all ``subsets'' of $A$. There are
two ways we can approach this: (1) via taking the characteristic function of the
subset as the subset, and (2) via $A$-codomained functions. In univalent
foundations, both of these approaches are equal.

In this development, there will be places where we will prefer one of these two
approaches over the other for convenience. Therefore we will define both of them
and show that they are equivalent.

\subsection{Subsets as power sets}

\begin{defn}(Power set)\label{defn:pow}
  Let $A$ be a type. Its power set $\pow{A}$ is defined as:
  \begin{equation*}
    \pow{A} \quad\is\quad A \rightarrow \Omega
  \end{equation*}
\end{defn}

\begin{prop}\label{isSetPow}
  Given any type $A$, $\pow{A}$ is an h-set.
\end{prop}

\begin{defn}[Full subset]\label{defn:full-set}
  The empty subset on any type $A$ is defined as $\lambda \_ \rightarrow \top$. We will refer to this as
  $\top_A$.
\end{defn}

\begin{defn}[Set intersection]\label{defn:set-intersection}
  Given subsets $U, V : \pow{A}$, their intersection is defined as:
  \begin{equation*}
    U \cap V \quad\is\quad x \mapsto x \epsilon U \times x \epsilon V
  \end{equation*}
\end{defn}

\subsection{Subsets as families}

\begin{defn}(Subset as a family of elements)\label{defn:fam}
  Let $A$ be a type. We define a subset of $A$ as a family of its elements as:
  \begin{equation*}
    \sub{A} \quad\is\quad \sigmaty{I}{\univ}{I \rightarrow A},
  \end{equation*}
  where $I$ is viewed as a type of index elements.

  Given some $F : \sub{A}$ we will refer the index type of $\mathcal{F}$ as
  $\indexset{F}$.
\end{defn}

\subsection{Equivalence of power sets and subsets as families}

\todo{
  Prove that Definition~\ref{defn:pow} and Definition~\ref{defn:fam} are actually equal to
  each other
}

\section{Structure identity principle}\label{sec:sip}
