\chapter{Foundations}\label{chap:foundations}

We present in this chapter basics of \UF{}. We assume that the author is familiar with
standard (intensional) type theory and refrain from building the formal calculus from
scratch. Instead, we summarise the additions that must be made to type theory to make it a
suitable context for the pursuit of \emph{univalent} mathematics.

Our presentation in this chapter has been heavily influenced by both the HoTT
book~\cite{hottbook} and Martín Escardó's introductory exposition of HoTT/UF in
\veragda{}~\cite{escardo-uf-intro}.

\section{Notation}

We refrain from formally presenting the calculus we will be working from scratch. This is
both for the sake of simplicity as well as for philosophical reasons: as we will be using
\UF{} in a practical way, we would like to keep the foundational system \emph{conceptual}
and use it as a practical foundation as opposed to a formal calculus. Indeed, the work to
be presented is the \emph{informalisation} of a completely formalised development and the
formal system used for that is Cubical Type Theory (CTT)~\cite{CCHM}. However, CTT is just
one specific instance of a univalent foundation. We would like to abstract over the
implementation, and work with a univalen foundations whose formal details are
unspecified---much like how set theory is used in ``everyday'' mathematics.

In terms of notation, we will mostly use the conventions from the HoTT
book~\cite{hottbook}, as explained in Section~I.1. Like the HoTT book, we will often make
use of pattern matching; for instance, we will write $\oftyI{(x,
  \_)}{\sigmaty{a}{A}{B(a)}}$ rather than using the projection notation for projecting out
the first component. We will have a minor deviation when it comes to the unit type, as we
will explain later. To make this more visible, we will denote the unit type as
$\unitty{}$.

\section{Equivalence and univalence of types}

\begin{defn}[Contractible]\label{defn:contr}
  A type $A$ is called contractible if it has exactly one inhabitant:
  \begin{equation*}
    \iscontr{A} \quad\is\quad \sigmaty{c}{A}{\pity{y}{A}{c =_A y}}.
  \end{equation*}
\end{defn}

\begin{defn}[Fiber]\label{defn:fiber}
  Given types $\oftyI{A}{\univ{}_m}$, $\oftyI{B}{\univ{}_n}$, a function $f : A
  \rightarrow B$, and some $\oftyI{y}{B}$, the fiber of $f$ over $y$ is the type of all
  inhabitants of $A$ that are mapped by $f$ to $y$:
  \begin{equation*}
    \fiber{f}{y} \quad\is\quad \sigmaty{x}{A}{f(x) =_B y}.
  \end{equation*}
\end{defn}

The following definition of type equivalence was first formulated by Voevodsky.
\begin{defn}[Equivalence]\label{defn:equiv}
  Given types $\oftyI{A}{\univ{}_m}$, $\oftyI{B}{\univ{}_n}$, a function $f : A \rightarrow B$ is an
  \emph{equivalence} if $\fiber{f}{y}$ is a contractible type for every $\oftyI{y}{B}$:
  \begin{equation*}
    \isequiv{f} \quad\is\quad \pity{y}{B}{\iscontr{\fiber{f}{y}}}.
  \end{equation*}
\end{defn}

We will denote by $A \simeq B$ the type of equivalences between types $A$ and $B$.
\begin{defn}[Equivalence of types]
  Given $\oftyI{A}{\univ{}_m}$, $\oftyI{B}{\univ{}_n}$,
  \begin{equation*}
    \typequiv{A}{B} \quad\is\quad \sigmaty{f}{A \rightarrow B}{\isequiv{f}}
  \end{equation*}
\end{defn}

\begin{defn}[Identity equivalence]\label{defn:id-equiv}
  The identity function on every type $A$ is an equivalence:
  \begin{equation*}
    \oftyI{\idequiv{}}{\typequiv{A}{B}}
  \end{equation*}
\end{defn}

Given two types $\oftyI{A~B}{\univ{}_n}$ and a proof $\oftyI{p}{A = B}$ that
they are equal, we can clearly prove that they are equivalent.
\begin{defn}
  \begin{equation*}
    \oftyI{\mathsf{idToEquiv}}{A = B \rightarrow A \simeq B}
  \end{equation*}
\end{defn}
In other words, it is justified intuitively that equality is stronger than equivalence.
Voevodsky's famous Univalence Axiom states, essentially, that equivalence is as strong as
equality. Formally, this amounts to saying that $\mathsf{idToEquiv}$ is an equivalence:
\begin{equation*}
  \isequiv{\idtoeqvnm{}},
\end{equation*}
an immediate consequence of which is that the type of identity proofs is equivalent to the
type of equivalences:
\begin{equation*}
  \pity{A~B}{\univ{}_n}{(A = B) \simeq (A \simeq B)}.
\end{equation*}

The consequences of adding this simple axiom into type theory are quite remarkable.
Perhaps most importantly, it allows us to \emph{prove} function extensionality.
\begin{defn}[Extensional equality]\label{defn:exteq}
  Given types $\oftyI{A}{\univ{}_m}$ and $\oftyI{B}{\univ{}_n}$, and functions
  $\oftyII{f}{g}{A \rightarrow B}$, $f$ and $g$ are said to be extensionally equal if
  the following type is inhabited:
  \begin{equation*}
    f \sim g \quad\is\quad \pity{x}{A}{f(x) = g(x)}.
  \end{equation*}
\end{defn}

\begin{prop}[Function extensionality]\label{prop:funext}
  Two functions are equal whenever they are extensionally equal.
\end{prop}

We omit the proof as it is rather involved and the content is not interesting for our
purposes. The reader is referred to Sec.~4.9 of the HoTT book~\cite{hottbook}.

\section{Homotopy levels}

One of the most interesting observations that arise from associating type theory with
homotopy theory is that we can classify types with respect to the non-trivial homotopy
structure they bear. A space containing no nontrivial homotopy above dimension $n$ is
called a homotopy $n$-type (or said to be of homotopy level $n$).

We then recursively define a predicate expressing that a given type has homotopy level
$n$. The idea is that we increase the dimension by one at each step.
\begin{defn}[Homotopy level]\label{defn:hlevel}
  We will say that the homotopy level of a type $A$ is $\oftyI{n}{\mathbb{N}}$ if
  $\isofhlevel{A}{n}$ as inhabited:
  \begin{align*}
    \isofhlevel{A}{\zero{}} &\quad\is\quad \iscontr{A}\\
    \isofhlevel{A}{\suc{n}} &\quad\is\quad \pity{x~y}{A}{\isofhlevel{x =_A y}{n}}.
  \end{align*}
\end{defn}

The hierarchy of $n$-types is upwards-closed as expressed in the following proposition.
\begin{prop}\label{prop:level-up}
  Given any type $A$, the following type has an inhabitant.
  \begin{equation*}
    \pity{n}{\mathbb{N}}{\isofhlevel{A}{n} \rightarrow \isofhlevel{A}{n+1}}
  \end{equation*}
\end{prop}

Furthermore, $\sum$ and $\prod$ types respect h-levels.

\begin{prop}\label{prop:is-of-level-pi}
  Given an \emph{arbitrary} type $\oftyI{A}{\univ{}_m}$ and an $A$-indexed family of
  $n$-types $\oftyI{B}{A \rightarrow \univ{}_n}$, $\pity{x}{A}{B(x)}$ is an $n$-type. In formal
  terms:
  \begin{equation*}
    \left( \pity{x}{A}{\isofhlevel{B(x)}{n}} \right)
      \rightarrow \isofhlevel{\pity{x}{A}{B(x)}}{n}
  \end{equation*}
\end{prop}

\begin{prop}\label{prop:is-of-level-sigma}
  Given an $n$-type $\oftyI{A}{\univ{}}$ and a $A$-indexed family $\oftyI{B}{A \rightarrow \univ{}}$
  of $n$-types, $\sigmaty{x}{A}{B(x)}$ is an $n$-type. Formally,
  \begin{equation*}
    \isofhlevel{A}{n} \rightarrow
      \pity{x}{A}{\isofhlevel{B(x)}{n}} \rightarrow \isofhlevel{\sigmaty{x}{A}{B(x)}}{n}
  \end{equation*}
\end{prop}

\subsection{Propositions}

The homotopy level of one is of special interest: it is the class of types that are like
propositions in the sense that their own proof structure is trivial: they are inhabited by
at most one term. This is precisely the property we desire in those types that we view
like expressing \emph{properties}: we are interested in whether they are inhabited or not,
rather than what they are inhabited by.

\begin{defn}[Proposition]\label{defn:prop'}
  A type $A$ is a proposition (sometimes disambiguated as an \emph{h-proposition}) if it
  has a homotopy level of one:
  \begin{equation*}
    \isprop{A} \quad\is\quad \isofhlevel{A}{1}.
  \end{equation*}
\end{defn}

An equivalent way of expressing propositionality would be the following.
\begin{defn}[Proposition (official)]\label{defn:hprop}
  \begin{equation*}
    \isprop{A} \quad\is\quad \pity{x~y}{A}{x =_A y}.
  \end{equation*}
\end{defn}

\begin{prop}
  Definitions \ref{defn:prop'} and \ref{defn:hprop} are equivalent.
\end{prop}

Due to this equivalence, we will sometimes use definitions \ref{defn:prop'} and
\ref{defn:prop} interchangeably.

We collect propositional types at level $n$ in the type $\hprop{}_n$.
\begin{defn}[$\hprop{}$]\label{defn:omega}
  $\hprop{}_n$ is the type of all types at universe $n$ that are propositional:
  \begin{equation*}
    \hprop{}_n \quad\is\quad \sigmaty{A}{\univ{}_n}{\isprop{A}}.
  \end{equation*}
\end{defn}

When asserting that a proposition $\oftyI{A}{\hprop{}_n}$ is inhabited, we have to project
out the first component to be completely precise. We will engage in the excusable
notational abuse of denoting this projection by $A$ itself.

Notice that $\hprop{}$ itself is always an h-set.
\begin{prop}\label{prop:hprop-set}
  $\hprop{}$ is an h-set.
\end{prop}

Once we introduce this delineation of the class of propositional types, we have to make
sure that types that are thought of as properties are actually propositional types. This
is usually easy to do, that is, most types that are expected to behave like propositions
behave naturally like propositions, so it suffices to prove this fact. However, there are
cases when they actually \emph{are not} propositional. A substantial component of the work
presented in this thesis is devoted to the problem of forcing certain types to behave
propositionally.

Now, let us state some simple facts about the class of propositional types. Observe the
two corollaries, of propositions \ref{prop:is-of-level-pi} and
\ref{prop:is-of-level-sigma} respectively.

\begin{prop}\label{prop:pi-prop}
  Given $\oftyI{A}{\univ{}_m}$, $\oftyI{B}{A \rightarrow \univ{}_n}$, the type $\pity{x}{A}{B(x)}$
  is a proposition whenever $B(x)$ is a proposition for every $\oftyI{x}{A}$.
\end{prop}
\begin{proof}
  Direct application of Proposition~\ref{prop:is-of-level-pi} with $n \is 1$.
\end{proof}

\begin{prop}\label{prop:sigma-prop}
  Given types $\oftyI{A}{\univ{}_m}$ and $\oftyI{B}{\univ{}_n}$, if $A$ is an
  h-proposition and $B(a)$ is an h-proposition for every $\oftyI{a}{A}$, then
  $\sigmaty{x}{A}{B}$ is an h-proposition.
\end{prop}
\begin{proof}
  Direct application of Proposition~\ref{prop:is-of-level-sigma}.
\end{proof}

The type expressing that a given a type $A$ is a proposition, is itself a proposition.
\begin{prop}
  Given $\oftyI{A}{\univ{}_n}$, the type $\isprop{A}$ is propositional.
\end{prop}
\begin{proof}
  By Proposition~\ref{prop:pi-prop}, it suffices to show $x = y$ is a proposition, for any
  $\oftyII{x}{y}{A}$, which follows by the upwards-closure of h-levels
  (Proposition~\ref{prop:level-up}).
\end{proof}

As we view propositional types as embodying logical propositions, it is natural to
expect that the right notion of equivalence between them is logical equivalence.
\begin{defn}\label{defn:iff}
  Types $\oftyI{A}{\univ{}_m}$ and $\oftyI{B}{\univ{}_m}$ are logically equivalent
  (denoted $\logequiv{A}{B}$) iff there exists functions both ways:
  \begin{equation*}
    \logequiv{A}{B} \quad\is\quad (A \rightarrow B) \times (B \rightarrow A).
  \end{equation*}
\end{defn}
In \UF{}, it is sufficient to show logical equivalence to establish equivalence between
propositional types.
\begin{prop}\label{prop:iff-equiv}
  Given propositions $\oftyI{P}{\univ{}_n}$, $\oftyI{Q}{\univ{}_n}$, if $\logequiv{P}{Q}$
  then $\typequiv{A}{B}$.
\end{prop}
\begin{proof}[Proof sketch]
  The two directions of $\logequiv{P}{Q}$ must constitute an isomorphism as the section
  and the retraction are trivial by the propositionality.
\end{proof}

\subsection{Sets}

The other homotopy class of interest is the class of types whose proof structures are not
trivial, in that there may be distinct terms that inhabit them, but the proof structure of
the type of equality proofs between their terms is trivial. Put more simply, inhabitants
of a set are equal to each other in \emph{at most one way}.
\begin{defn}[Set]\label{defn:hset}
  A type $A$ is a set if its homotopy level is two:
  \begin{equation*}
    \isset{A} \quad\is\quad \isofhlevel{A}{2}.
  \end{equation*}
\end{defn}

The structures we present in chapters \ref{chap:frames} and \ref{chap:formal-topo} have
underlying partially ordered \emph{sets}. When implementing these in \UF{}, these must be
required to be h-sets. A more accurate name for a ``poset'' whose carrier set bears
higher dimensional homotopy structure would be \emph{$\infty$-poset}.

Observe the following corollary of Proposition~\ref{prop:level-up}.
\begin{prop}\label{prop:prop-is-set}
  Every proposition is a set.
\end{prop}

As we have done for the case of h-propositions, we consider the special cases
of propositions \ref{prop:is-of-level-pi} and \ref{prop:is-of-level-sigma} in the
h-set case.

\begin{prop}\label{prop:pi-set}
  Given $\oftyI{A}{\univ{}_m}$, $\oftyI{B}{A \rightarrow \univ{}_n}$, the type $\pity{x}{A}{B(x)}$
  is a set whenever $B(x)$ is a set for every $\oftyI{x}{A}$.
\end{prop}

\begin{prop}\label{prop:sigma-set}
  Given types $\oftyI{A}{\univ{}_m}$ and $\oftyI{B}{\univ{}_n}$, if $A$ is an h-set and
  $B(a)$ is an h-set for every $\oftyI{a}{A}$, then $\sigmaty{x}{A}{B}$ is an h-set.
\end{prop}

\section{Powersets}\label{sec:pow}

Given a type $A$, we would like to talk about its powerset: the type of \emph{properties}
the inhabitants of $A$ might have. This is nothing but the type $A \rightarrow \hprop{}$.
Inhabitants of the powerset of a type are one way of representing subsets of inhabitants
of a type in type theory. We will present the other approach in Section~\ref{sec:fam}.

\begin{defn}\label{defn:pow}
  Given a type $\oftyI{A}{\univ{}_n}$, the \emph{powerset} of $A$ is the type of all
  predicates on $A$:
  \begin{equation*}
    \pow{A} \quad\is\quad A \rightarrow \hprop{}_n.
  \end{equation*}
\end{defn}

The proposition expressing the membership of an element $x$ in a subset is then just the
application $U(x)$. For the sake of clarity, we will use a bit of syntactic sugar and
write this as $x \in U \is U(x)$ for subsets.

Our use of the word ``set'' in ``powerset'' is justified by the following proposition.
\begin{prop}\label{prop:pow-set}
  Given \emph{any} type $A$, $\pow{A}$ is an h-set.
\end{prop}
\begin{proof}
  By Proposition~\ref{prop:pi-set}, suffices to show that the codomain, $\hprop{}$, is
  a set. This is given by Proposition~\ref{prop:hprop-set}.
\end{proof}

Powersets are ordered under the usual inclusion order.
\begin{defn}\label{defn:inclusion}
  Given a type $\oftyI{A}{\univ{}_n}$ and subsets $\oftyII{U}{V}{\pow{A}}$, the inclusion
  relation is defined as follows:
  \begin{equation*}
    U \subseteq V \quad\is\quad \pity{x}{A}{x \in U \rightarrow x \in V}.
  \end{equation*}
\end{defn}

\begin{defn}[Full subset]\label{defn:entire-subset}
  Given a type $\oftyI{A}{\univ{}_n}$, the subset containing the entirety of inhabitants
  of $A$ is defined as the constant function $\lambda\_.~\unitty{}_n$.
\end{defn}

Notice that, by the definition of $\pownm{}$, the type has to live in the same universe as
$A$. It is for this reason that $\unitty{}$, as we have defined it, takes as argument the
universe level we would like it to inhabit and inhabits that universe.

\begin{defn}\label{defn:intersection}
  Given a type $\oftyI{A}{\univ{}_n}$ and subsets $\oftyII{U}{V}{\pow{A}}$, the subset
  delineating those elements that are in \emph{both} of $U$ and $V$ is defined as:
  \begin{align*}
    \_\cap\_  \quad&:\quad   \pow{A} \rightarrow \pow{A} \rightarrow \pow{A} \\
    U \cap V  \quad&\is\quad \lambda x.~ x \in U \times x \in V  .
  \end{align*}
  We note that $x \in U \cap V$ is propositional by the propositionality of $x \in U$ and $x \in V$
  combined via Proposition~\ref{prop:sigma-prop}.
\end{defn}

\section{Families}\label{sec:fam}

In type theory, we have two notions of ``subset of a type''. One of these is powersets as
we have presented in Sec.~\ref{sec:pow}. We now briefly summarise the other approach in
which we view a subset of a type $A$ as an $A$-valued function (on some domain).

\begin{defn}[Family]\label{defn:fam}
  Given a type $A$, a family of inhabitants of $A$ with index set $I$ is simply a
  function: $I \rightarrow A$. We collect the type of all families on $A$ in the following type:
  \begin{equation*}
    \sub{o}{A} \quad\is\quad \sigmaty{I}{\univ{}_o}{I \rightarrow A}.
  \end{equation*}
  Notice that this is parameterised by a level $o$, being the level of the index type.
  Given a family $\oftyI{U \equiv (I, f)}{\sub{o}{A}}$, we will denote the evaluation of it at
  $\oftyI{i}{I}$ as $U_i$.
\end{defn}

\paragraph*{Notational clarification.} An inhabitant of $\sub{o}{A}$ has the form
$(I, f)$ as defined in Defn.~\ref{defn:fam}. However, we will use some more suggestive
notation for pairs of this form. It is quite convenient to be able to talk about a family
as though it were a concrete set, so we will use the notation $\{ x_i ~|~ i \in I \}$ as a
substitute for ``$I$-indexed family whose $i$th projection is denoted $x_i$''. Furthermore,
we will talk quite often about join operators $\oftyI{\bigvee\_}{\sub{o}{A} \rightarrow A}$. Instead of
writing, $\bigvee (I, \lambda i.~e(i))$ for the application of a join operator to a family, we will
use the more familiar notation of
\begin{equation}\label{eqn:join-syntax}
  \bigvee_i e
\end{equation}
What is meant by such notational sugar is expected to be obvious to the reader.

In the \veragda{} formalisation, we use similar notational conventions. For instance, the
syntax we have exemplified in (\ref{eqn:join-syntax}) is implemented in the
\modname{JoinSyntax} submodule of the \modname{Frame} module.
\begin{code}%
\>[0]\AgdaKeyword{module}\AgdaSpace{}%
\AgdaModule{JoinSyntax}\AgdaSpace{}%
\AgdaSymbol{(}\AgdaBound{A}\AgdaSpace{}%
\AgdaSymbol{:}\AgdaSpace{}%
\AgdaFunction{Type}\AgdaSpace{}%
\AgdaGeneralizable{ℓ₀}\AgdaSymbol{)}\AgdaSpace{}%
\AgdaSymbol{\{}\AgdaBound{ℓ₂}\AgdaSpace{}%
\AgdaSymbol{:}\AgdaSpace{}%
\AgdaPostulate{Level}\AgdaSymbol{\}}\AgdaSpace{}%
\AgdaSymbol{(}\AgdaBound{join}\AgdaSpace{}%
\AgdaSymbol{:}\AgdaSpace{}%
\AgdaFunction{Fam}\AgdaSpace{}%
\AgdaBound{ℓ₂}\AgdaSpace{}%
\AgdaBound{A}\AgdaSpace{}%
\AgdaSymbol{→}\AgdaSpace{}%
\AgdaBound{A}\AgdaSymbol{)}\AgdaSpace{}%
\AgdaKeyword{where}\<%
\\
%
\\[\AgdaEmptyExtraSkip]%
\>[0][@{}l@{\AgdaIndent{0}}]%
\>[2]\AgdaFunction{join-of}\AgdaSpace{}%
\AgdaSymbol{:}\AgdaSpace{}%
\AgdaSymbol{\{}\AgdaBound{I}\AgdaSpace{}%
\AgdaSymbol{:}\AgdaSpace{}%
\AgdaFunction{Type}\AgdaSpace{}%
\AgdaBound{ℓ₂}\AgdaSymbol{\}}\AgdaSpace{}%
\AgdaSymbol{→}\AgdaSpace{}%
\AgdaSymbol{(}\AgdaBound{I}\AgdaSpace{}%
\AgdaSymbol{→}\AgdaSpace{}%
\AgdaBound{A}\AgdaSymbol{)}\AgdaSpace{}%
\AgdaSymbol{→}\AgdaSpace{}%
\AgdaBound{A}\<%
\\
%
\>[2]\AgdaFunction{join-of}\AgdaSpace{}%
\AgdaSymbol{\{}\AgdaArgument{I}\AgdaSpace{}%
\AgdaSymbol{=}\AgdaSpace{}%
\AgdaBound{I}\AgdaSymbol{\}}\AgdaSpace{}%
\AgdaBound{f}\AgdaSpace{}%
\AgdaSymbol{=}\AgdaSpace{}%
\AgdaBound{join}\AgdaSpace{}%
\AgdaSymbol{(}\AgdaBound{I}\AgdaSpace{}%
\AgdaOperator{\AgdaInductiveConstructor{,}}\AgdaSpace{}%
\AgdaBound{f}\AgdaSymbol{)}\<%
\\
%
\\[\AgdaEmptyExtraSkip]%
%
\>[2]\AgdaKeyword{syntax}\AgdaSpace{}%
\AgdaFunction{join-of}\AgdaSpace{}%
\AgdaSymbol{(λ}\AgdaSpace{}%
\AgdaBound{i}\AgdaSpace{}%
\AgdaSymbol{→}\AgdaSpace{}%
\AgdaBound{e}\AgdaSymbol{)}\AgdaSpace{}%
\AgdaSymbol{=}\AgdaSpace{}%
\AgdaFunction{⋁⟨}\AgdaSpace{}%
\AgdaBound{i}\AgdaSpace{}%
\AgdaFunction{⟩}\AgdaSpace{}%
\AgdaBound{e}\<%
\end{code}

The bracketed index $\mathtt{⟨~i~⟩}$ is intended to be a plain text approximation of
subscripting.

\begin{defn}[Family membership]\label{defn:fam-mem}
  Given a type $A$, some $\oftyI{x}{A}$, and a family $U \equiv (I, f)$, we say
  that $x$ is a member of $U$ if the fiber of $f$ over $x$ is inhabited:
  \begin{equation*}
    \memfam{x}{U} \quad\is\quad \fiber{f}{x}.
  \end{equation*}
\end{defn}

\begin{defn}[Image over a family]\label{defn:fam-img}
  Given a type $A$, some family $U \equiv (I, f)$, and a function $\oftyI{g}{A \rightarrow B}$, the
  image of $g$ over $U$ is nothing but the family $(I, g \circ f)$.
  We will denote this $\img{g}{U}$ or $\setof{ g(f(i)) ~|~ \oftyI{i}{I} }$.
\end{defn}

\paragraph*{Remark on Agda notation.}
Again, we follow a similar notational sugaring in the \veragda{} formalisation. As the
curly bracket symbols, \texttt{\{} and \texttt{\}}, are not available in \veragda{}, we
approximate them as $\texttt{\textlquill}$ and $\texttt{\textrquill}$ for the comprehension notation. One
instance of such a syntax declaration can be found in the \modname{Family} module.

\begin{code}%
\>[0]\AgdaFunction{img}\AgdaSpace{}%
\AgdaSymbol{:}\AgdaSpace{}%
\AgdaSymbol{\{}\AgdaBound{X}\AgdaSpace{}%
\AgdaSymbol{:}\AgdaSpace{}%
\AgdaFunction{Type}\AgdaSpace{}%
\AgdaGeneralizable{ℓ₀}\AgdaSymbol{\}}\AgdaSpace{}%
\AgdaSymbol{\{}\AgdaBound{Y}\AgdaSpace{}%
\AgdaSymbol{:}\AgdaSpace{}%
\AgdaFunction{Type}\AgdaSpace{}%
\AgdaGeneralizable{ℓ₁}\AgdaSymbol{\}}\AgdaSpace{}%
\AgdaSymbol{→}\AgdaSpace{}%
\AgdaSymbol{(}\AgdaBound{g}\AgdaSpace{}%
\AgdaSymbol{:}\AgdaSpace{}%
\AgdaBound{X}\AgdaSpace{}%
\AgdaSymbol{→}\AgdaSpace{}%
\AgdaBound{Y}\AgdaSymbol{)}\AgdaSpace{}%
\AgdaSymbol{→}\AgdaSpace{}%
\AgdaSymbol{(}\AgdaBound{ℱ}\AgdaSpace{}%
\AgdaSymbol{:}\AgdaSpace{}%
\AgdaFunction{Sub}\AgdaSpace{}%
\AgdaGeneralizable{ℓ₂}\AgdaSpace{}%
\AgdaBound{X}\AgdaSymbol{)}\AgdaSpace{}%
\AgdaSymbol{→}\AgdaSpace{}%
\AgdaFunction{Sub}\AgdaSpace{}%
\AgdaGeneralizable{ℓ₂}\AgdaSpace{}%
\AgdaBound{Y}\<%
\\
\>[0]\AgdaFunction{img}\AgdaSpace{}%
\AgdaBound{g}\AgdaSpace{}%
\AgdaBound{ℱ}\AgdaSpace{}%
\AgdaSymbol{=}\AgdaSpace{}%
\AgdaSymbol{(}\AgdaFunction{index}\AgdaSpace{}%
\AgdaBound{ℱ}\AgdaSymbol{)}\AgdaSpace{}%
\AgdaOperator{\AgdaInductiveConstructor{,}}\AgdaSpace{}%
\AgdaBound{g}\AgdaSpace{}%
\AgdaOperator{\AgdaFunction{∘}}\AgdaSpace{}%
\AgdaSymbol{(}\AgdaOperator{\AgdaFunction{\AgdaUnderscore{}\$\AgdaUnderscore{}}}\AgdaSpace{}%
\AgdaBound{ℱ}\AgdaSymbol{)}\<%
\\
%
\\[\AgdaEmptyExtraSkip]%
\>[0]\AgdaKeyword{syntax}\AgdaSpace{}%
\AgdaFunction{img}\AgdaSpace{}%
\AgdaSymbol{(λ}\AgdaSpace{}%
\AgdaBound{x}\AgdaSpace{}%
\AgdaSymbol{→}\AgdaSpace{}%
\AgdaBound{e}\AgdaSymbol{)}\AgdaSpace{}%
\AgdaBound{ℱ}\AgdaSpace{}%
\AgdaSymbol{=}\AgdaSpace{}%
\AgdaFunction{⁅}\AgdaSpace{}%
\AgdaBound{e}\AgdaSpace{}%
\AgdaFunction{∣}\AgdaSpace{}%
\AgdaBound{x}\AgdaSpace{}%
\AgdaFunction{ε}\AgdaSpace{}%
\AgdaBound{ℱ}\AgdaSpace{}%
\AgdaFunction{⁆}\<%
\end{code}

Given a powerset $U$ on some type $A$, we can represent subset $U$ represents as a family
in a very simple way as given below.
\begin{defn}[Familification of a powerset]
  Let $A$ be a type and $\oftyI{U}{\pow{A}}$. The family associated with $U$ is
  the pair $(I, f)$ where
  \begin{align*}
    I \quad&\is\quad \sigmaty{x}{A}{U(x)} \\
    f \quad&\is\quad \mathsf{pr}_1        .
  \end{align*}
\end{defn}

As what representation we are working with will be clear to the reader from the context,
we will engage in a form notational abuse and will not explicitly denote that we are
working with the familification of a powerset. In the \veragda{} formalisation, this is
denoted $\mathtt{⟪} U \mathtt{⟫}$.

\section{Higher Inductive Types (HITs)}

In ordinary type theory, the inductive definition of a type amounts to a specification of
its points. In \UF{} we are working with not just with $0$-types but with arbitrary
$n$-types. Therefore, we accordingly generalise our mechanism for the inductive
specification of types by generalising $0$-constructors to $n$-constructors. In other
words, we add the means to define a type not just with its points but also with the paths
between the points, paths between those paths, and then paths between those and so on.
Such an inductive type is called a \emph{higher inductive type} (HIT for short).

Let us give an example of what an HIT definition might look like. The usual $\unitty{}$
type would be inductively defined by the specification of a single constructor:
\begin{equation*}
  \begin{prooftree}
    \infer0{\oftyI{\star}{\unitty{}}}
  \end{prooftree}
\end{equation*}
whose only equality is the trivial one: $\oftyI{\mathsf{refl}_\star}{\star =_{\unitty{}} \star}$.

Using HITs, we can add, for instance, a non-trivial equality of $\star$ to itself. The
resulting type is called the $\mathsf{Circle}$

\begin{equation*}
  \begin{prooftree}
    \infer0{\oftyI{\mathsf{base}}{\mathsf{Circle}}}
  \end{prooftree}
  \hspace{2em}
  \begin{prooftree}
    \infer0{\oftyI{\mathsf{loop}}{\mathsf{base} =_{\mathsf{Circle}} \mathsf{base}}}
  \end{prooftree}
\end{equation*}

HITs turn out to be immensely useful and in fact we will explain later that they are used
in a crucial way in this thesis: the main construction of Chapter~\ref{chap:formal-topo}
would not have possible without them.

\section{Truncation}

A particular HIT of interest will be the \emph{propositional truncation} type. This will
allow us to \emph{forget} the proof structure of a type, or in other words, force a
non-propositional type to be propositional by collapsing all of its different proofs into
a single one.

This turns out to be extremely useful as it gives us a way making proof-relevant things
proof-irrelevant. Consider for instance two propositions $\oftyII{P}{Q}{\Omega_n}$. We would
like to be able to express their logical disjunction, $P + Q$, and we would also like it
to be a proposition itself. It will clearly not be propositional due to the fundamental,
proof-relevant nature of type theory: the inhabitant will contain the information of which
one of $P$ or $Q$ it comes from. For this to become propositional, we have to
\emph{forget} this information; propositional truncation allows us to do precisely this.

We now give the definition of propositional truncation as an HIT.

\begin{defn}[Propositional truncation]\label{defn:truncation}
  Given a type $A$, its propositional truncation $\trunc{A}$ is a HIT given by
  one constructor, $\rulename{inj}$, and one path, $\rulename{trsquash}$.
  \begin{equation*}
    \begin{prooftree}
      \hypo{\oftyI{x}{A}}
      \infer1[$\rulename{inj}$]{\oftyI{x}{\trunc{A}}}
    \end{prooftree}
    \qquad
    \begin{prooftree}
      \hypo{\oftyI{x}{\trunc{A}}}
      \hypo{\oftyI{y}{\trunc{A}}}
      \infer2[$\rulename{trsquash}$]{x = y}
    \end{prooftree}
  \end{equation*}
  The recursion principle for propositional truncation is the following: given any two
  types $A$ and $B$, to show $\trunc{A} \rightarrow B$ it suffices to show $\isprop{B}$ and $A \rightarrow B$.
  Formally,
  \begin{equation*}
    \isprop{B} \rightarrow (A \rightarrow B) \rightarrow \trunc{A} \rightarrow B.
  \end{equation*}
\end{defn}

Observe that this corresponds to the intuitive distinction between \emph{property} and
\emph{structure}: we are allowed to use the inhabitant of a propositionally truncated type
only if it will be used in the proof of another property.

This definition of propositional truncation as an HIT brings us to the end of this
chapter. With all the foundational notions we have defined, we are now ready to carry out
an investigation of formal topology in \UF{}.
