\chapter{Foundations}\label{chap:foundations}

In this chapter, we provide the preliminary lemmas of \UF{} for the sake of
self-containment. It is intended to be a summary rather than an introduction.

\section{Homotopy levels}

\begin{defn}[Contractible]
  A type $A$ is called contractible if
  \begin{equation*}
    \iscontr{A} \quad\is\quad \sigmaty{x}{A}{\pity{y}{A}{x =_A y}}
  \end{equation*}
\end{defn}

This says precisely that type $A$ has \emph{exactly one} inhabitant.

\begin{defn}
  We will say that the homotopy level of a type $A$ is $\oftyI{n}{\mathbb{N}}$ if
  \begin{align*}
    \isofhlevel{A}{\zero{}} &\quad\is\quad \iscontr{A}\\
    \isofhlevel{A}{\suc{n}} &\quad\is\quad \pity{x~y}{A}{\isofhlevel{x =_A y}{n}}.
  \end{align*}
\end{defn}

Homotopy levels of one and two are of special interest. The former is the type of
\emph{propositions} i.e., types that are like propositions in the sense that they have
trivial proof structure: they are inhabited by at most one term.

\begin{defn}[Proposition]
  A type $A$ is a proposition (sometimes disambiguated as \emph{h-proposition}) if it has
  a homotopy level of one:
  \begin{equation*}
    \isprop{A} \quad\is\quad \isofhlevel{A}{1}.
  \end{equation*}
\end{defn}

\begin{defn}
  A type $A$ is a set (sometimes disambiguated as \emph{h-set}) if it a has a homotopy
  level of two:
  \begin{equation*}
    \isset{A} \quad\is\quad \isofhlevel{A}{2}.
  \end{equation*}
\end{defn}
