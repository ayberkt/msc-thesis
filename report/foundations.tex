\chapter{Foundations}\label{chap:foundations}

In this chapter, we provide the preliminary lemmas of \UF{} for the sake of
self-containment. It is intended to be a summary rather than an introduction.

\section{Homotopy levels}

\begin{defn}[Contractible]
  A type $A$ is called contractible if
  \begin{equation*}
    \iscontr{A} \quad\is\quad \sigmaty{x}{A}{\pity{y}{A}{x =_A y}}
  \end{equation*}
\end{defn}

This says precisely that type $A$ has \emph{exactly one} inhabitant.

\begin{defn}
  We will say that the homotopy level of a type $A$ is $\oftyI{n}{\mathbb{N}}$ if
  \begin{align*}
    \isofhlevel{A}{\zero{}} &\quad\is\quad \iscontr{A}\\
    \isofhlevel{A}{\suc{n}} &\quad\is\quad \pity{x~y}{A}{\isofhlevel{x =_A y}{n}}.
  \end{align*}
\end{defn}

Homotopy levels of one and two are of special interest. The former is the type of
\emph{propositions} i.e., types that are like propositions in the sense that they have
trivial proof structure: they are inhabited by at most one term.

\begin{defn}
  A type $A$ is a set (sometimes disambiguated as \emph{h-set}) if it a has a homotopy
  level of two:
  \begin{equation*}
    \isset{A} \quad\is\quad \isofhlevel{A}{2}.
  \end{equation*}
\end{defn}

\section{Propositions}

The homotopy level of one is of special interest: it is the class of types that are like
propositions in the sense that they have trivial proof structure: they are inhabited by at
most one term.

\begin{defn}[Proposition]
  A type $A$ is a proposition (sometimes disambiguated as \emph{h-proposition}) if it has
  a homotopy level of one:
  \begin{equation*}
    \isprop{A} \quad\is\quad \isofhlevel{A}{1}.
  \end{equation*}
\end{defn}

We will be collecting the class of propositional types in the following type of $\Omega$.

\begin{defn}[$\Omega$]
  $\Omega_n$ is the type of all types at universe $n$ that are propositional.
  \begin{equation*}
    \sigmaty{A}{\univ{}_n}{\isprop{A}}.
  \end{equation*}
\end{defn}

When asserting that a proposition $\oftyI{A}{\Omega_n}$ is inhabited, we have to technically
project out the first component. We will engage in the excusable notational abuse of
denoting this projection by $A$ itself.

Once we introduce this delineation of the class of propositional types, we have to make
sure that types that we view as \emph{properties} of structures are actually propositional
types. This is usually easy to do, that is, most types that are expected to behave like
propositions behave naturally like propositions, so it suffices to prove this fact.
However, there are cases when they actually \emph{are not} propositional. A substantial
component of the work presented in this thesis is devoted to the problem of ensuring
certain types behave propositionally as we will explain later.

Now, let us prove some simple facts about the class of propositional types.

\begin{prop}\label{thm:pi-prop}
  Given $\oftyI{A}{\univ{}_m}$, $\oftyI{B}{A \rightarrow \univ{}_n}$, the type $\pity{x}{A}{B(x)}$
  is a proposition when every $B(x)$ is a proposition for every $\oftyI{x}{A}$.
\end{prop}
\begin{proof}
  Let $\oftyI{A}{\univ{}_m}$, $\oftyI{B}{A \rightarrow \univ{}_n}$ and assume that $B(x)$ is a
  proposition for every $\oftyI{x}{A}$. Let $\oftyII{f}{g}{\pity{x}{A}{B(x)}}$. By
  function extensionality, it suffices to show that $f(x) = g(x)$ for every
  $\oftyI{x}{A}$. Let $\oftyI{x}{A}$. $f(x) = g(x)$ is given directly by the fact that
  $B(x)$ is propositional.
\end{proof}

\begin{prop}[Propositionality is propositional]
  Given $\oftyI{A}{\univ{}_n}$, the type $\isprop{A}$ is propositional.
\end{prop}

\section{Sets}
