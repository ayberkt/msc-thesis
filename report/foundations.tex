\chapter{Foundations}\label{chap:foundations}

In this chapter, we provide the preliminary constructions and theorems of \UF{} for the
sake of self-containment. It is intended to be a summary rather than an introduction to
HoTT/UF.

\section{Homotopy levels}

\begin{prop}\label{isOfHLevelSigma}
  $\Sigma$ types preserve h-levels.
\end{prop}

\section{Propositions}

We collect propositional types in $\Omega$.

\begin{defn}\label{omega}
  $\Omega \is \sigmaty{A}{\univ}{\isaprop{A}}$
\end{defn}

Technically, to assert that some $A~:~\Omega$ holds, we must project out the first component.
However, in informal writing we will engage in the notational abuse of taking this to be
implicit.

\section{Subsets}

Given a type $A$, we would like to talk about all ``subsets'' of $A$. There are
two ways we can approach this: (1) via taking the characteristic function of the
subset as the subset, and (2) via $A$-codomained functions. In univalent
foundations, both of these approaches are equal.

In this development, there will be places where we will prefer one of these two
approaches over the other for convenience. Therefore we will define both of them
and show that they are equivalent.

\subsection{Subsets as power sets}

\subsection{Subsets as families}

\begin{defn}(Power set)
  Let $A$ be a type. Its power set $\pow{A}$ is defined as:
  \begin{equation*}
    \pow{A} \is A \rightarrow \Omega
  \end{equation*}
\end{defn}

\begin{prop}\label{isSetPow}
  Given any type $A$, $\pow{A}$ is an h-set.
\end{prop}

\section{Structure identity principle}
