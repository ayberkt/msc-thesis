\chapter{Formal Topology}\label{chap:formal-topo}

We remarked that the motivation for pointless topology is to attain a constructive
understanding of topology, and that this is a prerequisite for being able to express
topology in type theory in a natural way i.e., without postulating classical axioms. This
task of making type-theoretical sense of topology presents another challenge: we must be
able to develop our results in a completely predicative way as well. To address this, we
will use formal topologies which give us a way of \emph{presenting} frames. Instead of
working with frames directly, we will work with formal topologies from which the frames
are freely generated.

Our definition of a formal topology will make use of the notion of an interaction system,
first formulated by Petersson and Synek~\cite{tree-sets}, the motivation of whom was to
generalise $\mathsf{W}$ types to be able to accommodate mutual induction. Other names for
interaction systems are \emph{tree set constructors} (which is the name that Petersson and
Synek~\cite{tree-sets} use), and \emph{indexed containers}~\cite{indexed-containers}. The
idea of doing formal topology with interaction systems is due to Coquand~\cite{coq-posets}
who was inspired by Dragalin~\cite{dragalin}.

This chapter corresponds to several modules in the \veragda{} formalisation. Sections
\ref{sec:intr-sys}, \ref{sec:cover}, \ref{sec:cover-nucleus}, and \ref{sec:universal-prop}
correspond, respectively, to modules \modname{FormalTopology}, \modname{Cover},
\modname{CoverFormsNucleus}, and \modname{UniversalProperty}.

\section{Interaction systems}\label{sec:intr-sys}

\paragraphsummary{Explain the idea of the tree type.}
The fundamental idea of an interaction system is simple. Consider the progression of a
two-player game. First, there is a type of \emph{game states}; call it $A$:
\begin{equation*}
  A~:~\univ.
\end{equation*}
At each state of the game, there are certain moves the player can take. In other words,
for every game state $\oftyI{a}{A}$, there is a type of possible moves the player may take.
Formally, this is a function:
\begin{equation*}
  \oftyI{B}{A \rightarrow \univ}.
\end{equation*}
Furthermore, for every move the player may take, the opponent can take certain
counter-moves in response. Formally:
\begin{equation*}
  \oftyI{C}{\pity{a}{A}{B(a) \rightarrow \univ}}.
\end{equation*}
Finally, given the counter-move in response to a certain move at some state, we proceed to
a new game state. This is given by some function:
\begin{equation*}
  \oftyI{d}{\pity{a}{A}{\pity{b}{B(a)}{C(a, b) \rightarrow A}}}.
\end{equation*}

In four pieces, namely $(A, B, C, d)$, we characterise structures that involve some kind
of ``interaction''. Even though the game analogy is useful, interaction systems are more
general than games: they express anything that can be viewed as a \emph{dialogue} i.e.,
involving two parties interacting with each other.
Hancock~\cite{hancock-interaction-systems} provides a table providing various structures
that can be viewed as interaction systems. We reproduce this (with minor notational
modifications) in Table~\ref{tab:hancock}.

\begin{table}[tbp]
  \centering
  \caption{%
    Peter Hancock's~\cite{hancock-interaction-systems} table (with notational
    modifications) providing a list of examples of interaction systems.
  }
  \label{tab:hancock}
  \begin{tabular}{l | l | l | l}
    $\oftyI{a}{A}$ & $\oftyI{b}{B(a)}$ & $\oftyI{c}{C(a, b)}$ & $\oftyI{d(a, b, c)}{A}$ \\\hline
    sort           & constructor       & selector             & component sort          \\
    statement      & inference         & premise              & premise statement       \\
    neighbourhood  & partition         & part                 & new neighbourhood       \\
    game           & attack            & defence              & new state               \\
    interrogation  & question          & answer               & new state               \\
    interface      & call              & return               & new state               \\
    universe       & observation       & reading              & new state               \\
    knowledge      & experience        & result               & new state               \\
    dialogue       & thesis            & antithesis           & synthesis               \\
  \end{tabular}
\end{table}

To define the notion of a formal topology, we will require the type $A$ of states to be
not just a set, but a poset. The idea is the same as in the case of frames: we would like
to view these states as stages of information ranked with respect to how refined they are.
In addition, we will expect such a poset equipped with an interaction system to satisfy
the following two properties:
\begin{enumerate}
\item the \textbf{monotonicity} property: for every state $\oftyI{a}{A}$, move
  $b$ at state $a$, and counter-move $c$ in response to $b$, $d(a, b, c) \sqsubseteq a$.
  \item the \textbf{simulation} property which states that at any state, we can simulate
    the previous states (in a sense that we will explain in detail later).
\end{enumerate}

Once we have imposed the requirement that the set of stages be a poset, it makes sense to
use some more suggestive terminology due to Per Martin-Löf
\footnote{%
  Attributed to Martin-Löf in the note~\cite{coq-posets}.
}%
$^{,}$%
\footnote{%
  Although this terminology is due to Martin-Löf, it is the present author's opinion that
  it is particularly sensible to use this terminology in the context where we are imposing
  an ordering on states that satisfies the monotonicity property. This terminology is
  freely used for \emph{all} interaction systems in the literature (as exemplified by
  Table~\ref{tab:hancock} as well). Our justification is that it makes no sense to call
  $B$ a type of experiments if the experiments might be \emph{taking away} existing
  knowledge i.e., not satisfying the monotonicity property.
}.
\begin{itemize}
  \item $A$: a type of stages of \emph{knowledge}.
  \item $\oftyI{B}{A \rightarrow \univ{}}$: a type of \emph{experiments} $B(a)$ that one can perform
    at a certain stage of knowledge $a$.
  \item $\oftyI{C}{\pity{a}{A}{B(a) \rightarrow \univ{}}}$: a type of \emph{outcomes} of an
    experiment $\oftyI{b}{B(a)}$ at some stage of knowledge $a$.
  \item $d$: a function that expresses the act of \emph{revising} one's knowledge state
    based on the outcome of an experiment performed at a stage of knowledge.
\end{itemize}

Once we adopt this view, what the monotonicity property says becomes much more clear: if
we perform an experiment $b$ while we have knowledge $a$ and revise our knowledge based on
some outcome $c$ of $b$, the new state of knowledge we arrive at must contain at least as
much information. In fact, if we would like to view things sensibly as experiments, it
makes sense to reserve this terminology to interaction structures that have an ordering
that satisfies the monotonicity property

Now, to proceed towards the definition of a formal topology, let us first formally define
interaction systems.
\begin{defn}[Interaction system]\label{defn:intr-sys}
  For simplicity, we will require all types to be at the same level and omit the level
  in the notation.
  \begin{align*}
    \treestr{A} &\quad\is\quad
      \sigmaty{B}{A \rightarrow \univ}{
        \sigmaty{C}{\pity{a}{A}{B(a) \rightarrow C}}{
          \pity{a}{A}{\pity{b}{B(a)}{C(a, b) \rightarrow A}}
        }
      }\\
    \intrsys{} &\quad\is\quad \sigmaty{A}{~~~~\univ}{\treestr{A}}
  \end{align*}
  Given an interaction system $\mathcal{I}$, we will refer to its components as
  $A_{\mathcal{I}}$, $B_{\mathcal{I}}$, $C_{\mathcal{I}}$, and $d_{\mathcal{I}}$ in
  contexts where the possibility of ambiguity is present.
\end{defn}

Given an interaction system $\mathcal{I}$, the monotonicity property is then formally
expressed as in Definition~\ref{defn:mono}.
\begin{defn}[Monotonicity property of an interaction system]\label{defn:mono}
  An interaction system $\mathcal{I}$, whose type $A_{\mathcal{I}}$ of stages is equipped
  with a poset structure, is said to have the monotonicity property if the following type
  is inhabited:
  \begin{equation*}
    \hasmono{\mathcal{I}, \sqsubseteq} \quad\is\quad
      \pity{a}{A_{\mathcal{I}}}{%
        \pity{b}{B_{\mathcal{I}}(a)}{%
          \pity{c}{C_{\mathcal{I}}(a, b)}{%
              d(a, b, c) \sqsubseteq a
          }
        }
      }.
  \end{equation*}
\end{defn}

We have not yet fully explained the simulation property. Let us first provide its formal
definition.
\begin{defn}[Simulation property]\label{defn:sim}
  Given an interaction system $\mathcal{I}$, whose type $A_{\mathcal{I}}$ of stages is a
  poset, we will say that it satisfies the simulation property if the following type is
  inhabited:
  \begin{align*}
    &\hassim{\mathcal{I}, \sqsubseteq} \quad\is \\
    &\hspace{2em}\pity{a~a'}{A}{%
      a' \sqsubseteq a \rightarrow\\
      &\hspace{4em}\pity{b}{B(a)}{
        \sigmaty{b'}{B(a')}{
          \pity{z'}{C(a', b')}{
            \sigmaty{z}{C(a, b)}{
              d(a', b', c') \sqsubseteq d(a, b, c)
            }
          }
        }
      }
    }.
  \end{align*}
\end{defn}
What does this say intuitively? It says that at more refined stages we can always find a
counterpart to any experiment from a less refined stage, in the sense that that experiment
will lead to a finer stage. In other words, as we choose to perform certain experiments
and proceed to more refined stages, we do not lose the ability to perform the experiments
we previously chose not to perform.

\begin{defn}[Formal Topology]\label{defn:formal-topo}
  A \emph{formal topology} is simply an interaction system, whose type $A$ of stages is a
  poset, that satisfies the monotonicity and simulation properties.
  \begin{align*}
    \mathsf{FT} \quad\is\quad \sigmaty{P}{\poset{}_{n, n}}{%
      \sigmaty{\mathcal{I}}{\treestr{\abs{P}}}
        \hasmono{\mathcal{I}, \sqsubseteq_P} \times \hassim{\mathcal{I}, \sqsubseteq_P}
    }
  \end{align*}
\end{defn}

\section{Cover relation}\label{sec:cover}

The real reason we are interested in formal topologies is that the structure they contain,
along with the \vermono{} and \versim{} properties, induces a well-behaved \emph{covering
  relation}. This is a method going back Johnstone's~\cite{stone-spaces} adaptation of the
notion of a Grothendieck topology~\cite{artin} to the context of locale theory, that was
subsequently developed by Martin-L\"{o}f and Sambin~\cite{int-formal-spaces}.

The idea is as follows: given a set $A$ that we view like a set of \emph{basic opens}
(i.e., opens not made up using the join operator), we require a relation:
\begin{equation*}
  \_\covernm{}\_ : A \rightarrow \pow{A} \rightarrow \hprop{}
\end{equation*}
that is expected to pointlessly express the relation of being an \emph{open cover} i.e.,
$x \covernm{} U$ iff $U$ is an open cover of $x$: $x = \bigvee_i U_i$. In other words, we are
specifying which basic opens can be expressed as the join of which others. The information
contained by this relation is sufficient to generate the topology (i.e., the frame).

The original formulation of formal topology by Sambin suffers a particular problem: it is
not known how to define, predicatively, the coproduct of two frames using it. This problem
was solved by Coquand, Sambin et al.~\cite{coq-sambin} by defining the covering relation
inductively. We follow this approach in our development. In fact, as we will explain
later, we will define this relation not just as an inductive type but as a \emph{higher}
inductive one.

We will now proceed to define the cover relation on a given formal topology. However, our
presentation will recapitulate the progression our development: we first explain our naive
attempt, that we found out not to work in \UF{}, and after that its remedied form (thanks
to a suggestion by Thierry Coquand~\cite{another-way-out}) that circumvents this problem
through the use of HITs.
\begin{defn}[Naive cover relation]\label{defn:naive-cover}
  Given a formal topology
  $\mathcal{F}$ on type $A$, and given $\oftyI{a}{A}$, $U : \pow{A}$, the type
  $\covers{a}{U}$ is inductively defined with the following two constructors.
  \[
  \begin{prooftree}
    \hypo{ a \in U }
    \infer1[$\ruledir{}$]{\covers{a}{U}}
  \end{prooftree}
  \qquad
  \begin{prooftree}
    \hypo{\oftyI{b}{B(a)}}
    \hypo{\pity{c}{C(a, b)}{\covers{d(a, b, c)}{U}}}
    \infer2[$\rulebranch{}$]{\covers{a}{U}}
  \end{prooftree}
  \]
\end{defn}

One way of reading $\covers{a}{U}$ is as a relaxation of $a \in U$ to ``it is eventually the
case that $a \in U$''. In other words, it might not be that $a \in U$ but once stage $a$ has
been reached, all paths that do not lead to $U$ have been ruled out: regardless of what
experiments are run, they will eventually lead to a stage in $U$.

Let us now turn our attention to the question of how we will go from a formal topology to
a frame via its cover relation. The cover relation has the following type:
\begin{equation*}
  \_\covernm{}\_ : A \rightarrow \pow{A} \rightarrow \univ{},
\end{equation*}
which we can flip to get
\begin{equation*}
  \_\RHD\_ : \pow{A} \rightarrow A \rightarrow \univ{}.
\end{equation*}
\emph{If only} this had codomain $\hprop{}$ rather than $\univ{}$, we would have been able
to write it like:
\begin{equation*}
  \_\RHD\_ : \pow{A} \rightarrow \pow{A},
\end{equation*}
but this is not the case as the relation has been defined using more than one
constructors. We could then restrict this to get an endofunction on the set of
downwards-closed subsets:
\begin{equation*}
  \_\RHD\_ : \dcsubset{A} \rightarrow \dcsubset{A}
\end{equation*}
which of course requires us to show that given a downwards-closed subset $U$,
$\covers{\_}{U}$ is a subset that is downwards-closed. But first, we have to deal with the
problem that the result type of $\covers{}{}$ is not $\hprop{}$.

It is tempting to try to achieve this by truncating $\covers{\_}{\_}$. When we do this,
however, it becomes impossible to prove the idempotence law for the purported nucleus
$\RHD : \dcsubset{A} \rightarrow \dcsubset{A}$. Why exactly is that? To prove the idempotence law
\begin{equation*}
  \trunc{\covers{a}{\trunc{\covers{-}{U}}}} \rightarrow \trunc{\covers{a}{U}},
\end{equation*}
we need a lemma that says, given any $a$, and subsets $U, V$,
\begin{center}
  if $\trunc{\covers{a}{U}}$ and $\trunc{\covers{a'}{V}}$, for every $a'$, then
  $\trunc{\covers{a}{V}}$.
\end{center}
The $\ruledir{}$ case is easily verified. The $\rulebranch{}$ case, however, results in a
situation where we are trying to show
\begin{equation*}
  \trunc{\pity{c}{C(a, b)}{\covers{d(a, b, c)}{V}}}
\end{equation*}
whereas all we get from the inductive hypothesis is
\begin{equation*}
  \pity{c}{C(a, b)}{\trunc{\covers{d(a, b, c)}{V}}}.
\end{equation*}
An inference of the former from the latter would require the use of a classical reasoning
principle looking a lot like the axiom of choice~\cite[pg.~119, 3.8.3]{hottbook}:
\begin{equation*}
  \pity{x}{A}{\trunc{B(x)}} \rightarrow \trunc{\pity{x}{A}{B(x)}},
\end{equation*}
with the difference that neither $A$ nor $B$ are required to be h-sets.

To remedy this, we will refrain from truncating the naive form of the cover relation, and
we will instead add a higher constructor that \emph{squashes} the difference between the
$\ruledir{}$ and $\rulebranch{}$ constructors. We now provide the revised form (we repeat
the rules from Defn.~\ref{defn:naive-cover} for the sake of self-containment).
\begin{defn}[Cover relation]\label{defn:covering}
  Given a formal topology
  $\mathcal{F}$ on type $A$, and given $\oftyI{a}{A}$, $U : \pow{A}$, the type
  $\covers{a}{U}$ is inductively defined with the following two constructors:
  \[
  \begin{prooftree}
    \hypo{ a \in U }
    \infer1[$\ruledir{}$]{\covers{a}{U}}
  \end{prooftree}
  \qquad
  \begin{prooftree}
    \hypo{\oftyI{b}{B(a)}}
    \hypo{\pity{c}{C(a, b)}{\covers{d(a, b, c)}{U}}}
    \infer2[$\rulebranch{}$.]{\covers{a}{U}}
  \end{prooftree}
  \]
  In addition to the constructors, the type $\covers{a}{U}$ contains the following path:
  \begin{equation*}
    \begin{prooftree}
      \hypo{\oftyI{p}{\covers{a}{U}}}
      \hypo{\oftyI{q}{\covers{a}{U}}}
      \infer2[$\rulesquash{}$.]{p = q}
    \end{prooftree}
  \end{equation*}
\end{defn}

The definition of this cover relation can be found in the module \modname{Cover} in the
\veragda{} formalisation.

When trying to prove the idempotence law with this, we get from the inductive hypothesis a
family $\pity{c}{C(a, b)}{\covers{d(a, b, c)}{V}}$ where $\covers{d(a, b, c)}{V}$ is still
``squashed'', but this squashing is an integral part of $\covers{\_}{\_}$ rather than a
truncation that is imposed extrinsically upon it. This is sufficient for circumventing the
problem that would have required a form of choice, and allows us to successfully complete
the idempotence proof. The type of $\covers{\_}{\_}$ can be seen \emph{directly} to be
$\pow{A} \rightarrow \pow{A} \rightarrow \hprop{}$ by the existence of the $\rulesquash{}$ constructor.

Notice that, given a downwards-closed subset $U$, the subset of elements that are covered
by $U$ ($\covers{\_}{U}$) is itself downwards-closed.
\begin{prop}
  Let $\mathcal{F}$ be a formal topology and $U$ a downwards-closed subset of it.
  $\covers{\_}{U}$ is a downwards-closed subset.
\end{prop}
\begin{proof}
  Let $\oftyII{a}{a'}{A}$ such that $\covers{a}{U}$ and $a' \sqsubseteq a$. We need to show
  $\covers{a'}{U}$. Our proof proceeds by (higher) induction on the proof of
  $\covers{a}{U}$.
  \begin{itemize}
    \item Case $\ruledir{}$. It must be that $a \in U$ and hence, by the downwards-closure of
      $U$, $a' \in U$ meaning $\covers{a'}{U}$ by $\ruledir{}$.
    \item Case $\rulebranch{}$. We have some experiment $b$ on $a$ and a function
      $$\oftyI{f}{\pity{c}{C(a, b)}{\covers{d(a, b, c)}{U}}}.$$ Using the $\rulebranch{}$
      rule, we are done if we can exhibit some experiment $\oftyI{b'}{B(a')}$ along with a
      function
      \begin{equation*}
        \oftyI{g}{\pity{c'}{C(a, b')}{\covers{d(a', b', c')}{U}}}.
      \end{equation*}
      Pick $b'$ to be the experiment obtained by appealing to the \versim{} property with
      $b$. Let $\oftyI{c'}{C(a, b')}$. It remains to be shown that
      $\covers{d(a', b', c')}{U}$. By the inductive hypothesis, we are done if we can show that
      $a_0 \sqsubseteq d(a', b', c')$ for some $\covers{a_0}{U}$. Pick $a_0 \is d(a, b, c)$ where $c$ is the
      outcome of $b$ given by the \versim{} property. We have $d(a, b, c) \sqsubseteq d(a', b', c')$,
      directly by the \versim{} property and that $\covers{d(a, b, c)}{U}$ by $f(c)$.
    \item Case $\rulesquash{}$. We are done by appealing to the $\rulesquash{}$ rule with
      both of the inductive hypothesis.
  \end{itemize}
\end{proof}

\begin{prop}\label{prop:lem1}
  Given a formal topology $\McF{}$, $\oftyII{a}{a'}{A_{\McF{}}}$ such that $a' \sqsubseteq a$, and a
  downwards-closed subset $U$ of $A_{\McF{}}$, if $\covers{a}{U}$ then $\covers{a'}{U}$.
\end{prop}
\begin{proof}[Proof sketch]
  Follows by straightforward induction where the downwards-closedness of $U$ is used
  in the base case and the simulation property is used in the inductive case.
\end{proof}

\begin{prop}\label{prop:lem3}
  Given a formal topology $\McF{}$, $\oftyII{a}{a'}{A_{\McF{}}}$ such that $a' \sqsubseteq a$, and
  downwards-closed subsets $U$ and $V$ of $A_{\McF{}}$, if $\covers{a'}{U}$ and
  $\covers{a}{V}$ then $\covers{a'}{U \cap V}$.
\end{prop}
\begin{proof}
  \todo{complete}.
\end{proof}

\begin{prop}\label{prop:lem4}
  Let $\mathcal{F}$ be a formal topology and $U, V$ be subsets of its underlying poset. If
  $U \subseteq \covers{\_}{V}$ then $\covers{\_}{U} \subseteq \covers{\_}{V}$.
\end{prop}
\begin{proof}[Proof sketch]
  Follows by straightforward induction on the covering proof.
\end{proof}

Let us now summarise our method of obtaining a frame out of a formal topology which
comprises four steps.
\begin{enumerate}
  \item Start with a formal topology $\mathcal{T}$ on type $A$.
  \item $\mathcal{T}$ has an underlying poset $P$; construct its frame of downwards-closed
    subsets.
  \item Note: $\covers{\_}{\_}$ is a nucleus on the frame of downwards-closed subsets.
  \item We have shown (in Theorem~\ref{thm:fixed-point-frame}) that the set of
    fixed-points of every nucleus is a frame. The final frame is this fixed-point frame
    on the frame of downwards-closed subsets by the nucleus $\covers{\_}{\_}$.
\end{enumerate}
The only missing step is (3): we have not yet shown that the covering relation is a
nucleus. We will do exactly this in the following section.

\section{The covering relation is a nucleus}\label{sec:cover-nucleus}

$\_\RHD\_ : \pow{A} \rightarrow \pow{A}$ is an endofunction on the powerset of $A$. By restricting
our attention to subsets that are downwards-closed, we can view this as an endofunction on
the frame of downwards-closed subsets:
\begin{equation*}
  \oftyI{\_\RHD\_}{\dcsubset{A} \rightarrow \dcsubset{A}}.
\end{equation*}
The natural question to be asked is then: is this a nucleus (Defn.~\ref{defn:nucleus}) on
the frame of downwards-closed subsets of $A$? Let us recapitulate what this means before
we answer the question positively.
\begin{itemize}
  \item $N_0$: $\pity{a}{A}{\covers{a}{U \cap V} = (\covers{a}{U}) \cap (\covers{a}{V})}$,
  \item $N_1$: $U \subseteq \_ \covernm{} U$, and
  \item $N_2$: $\_ \covernm{} (\_ \covernm{} U) \subseteq \_ \covernm{} U$.
\end{itemize}

\begin{thm}\label{thm:covering-nucleus}
  The covering relation satisfies the nuclearity axioms.
\end{thm}
\begin{proof}
  $N_1$ is direct using the $\ruledir{}$ rule. $N_2$ is a direct application of
  Proposition~\ref{prop:lem4}.

  For $N_0$, we construct a proof by induction. Let $\oftyII{U}{V}{\pow{A}}$. We will show
  that $\_ \covernm{} (U \cap V) = (\_ \covernm{} U) \cap (\_ \covernm{} V)$ by antisymmetry. The $(\_ \covernm{} U) \cap
  (\_ \covernm{} V) \subseteq \_ \covernm{} (U \cap V)$ direction follows by Proposition~\ref{prop:lem3}. For the
  other direction, let $\oftyI{a}{A}$ such that $a \covernm{} U \cap V$. We proceed by induction on
  this proof.
  \begin{itemize}
    \item Case $\ruledir{}$. $a \in \intersect{U}{V}$ meaning $a \in U$ and $a \in V$ hence we
      are done by two applications of $\ruledir{}$.
    \item Case $\rulebranch{}$. Two appeals to the inductive hypothesis, followed by
      applications of the $\rulebranch{}$ rule.
    \item Case $\rulesquash{}$. We combine the two inductive hypothesis using the
      $\rulesquash{}$ rule.
  \end{itemize}
\end{proof}

\section{Generating a frame from a formal topology}

Now that we have shown the nuclearity of the covering relation $\covers{\_}{\_}$, we have
everything we need for the procedure of generating a frame from a formal topology.

Let $\McF{}$ be a formal topology. We know by Theorem~\ref{thm:down-set-frame} that the
set of downwards-closed subsets of the underlying poset of $\McF{}$ is a frame; denote
this by $\McF{}\downarrow$. As we know that $\covers{\_}{\_}$ is a nucleus on this frame (by
Theorem~\ref{thm:covering-nucleus}), we know that the set of fixed points for
$\covers{\_}{\_}$ is a frame as well; denote this by $L$. Now, notice that we can define
a map $\eta : A_{\McF{}} \rightarrow \abs{L}$ as follows:
\begin{align*}
  \eta    \quad&:\quad A_{\McF{}} \rightarrow \pow{A_{\McF{}}}\\
  \eta(x) \quad&\is\quad \covers{\_}{x\downarrow}
\end{align*}
where $x\downarrow$ denotes the \emph{downwards-closure} of $x$: $\{ y~|~y \sqsubseteq x \}$. So $y \in \eta(x) \equiv
\covers{y}{x\downarrow}$ is to say ``$y$ leads to a ramification of $x$''. In fact, one can see
that $\eta(x)$ is downwards-closed and a fixed point for $\covers{\_}{\_}$, meaning its type
can be refined to $\oftyI{\eta}{A_{\McF{}} \rightarrow \abs{L}}$.

\begin{defn}[$\eta$]
  Let $\McF{}$ be a formal topology and denote its cover relation by $\covers{\_}{\_}$.
  Let $L \is \fix{\dcsubset{A}}{\_\RHD\_}$. There exists a monotonic map from the
  underlying poset of $P$ of $\McF{}$ to the underlying poset of $L$:
  \begin{align*}
    \eta    \quad&:\quad P \rightarrow_m (\abs{L}, \_\sqsubseteq_L\_)\\
    \eta(a) \quad&\is\quad \covers{\_}{a\downarrow}.
  \end{align*}
  The fact that $\eta$ is monotonic follows from Proposition~\ref{prop:lem1}. It remains to
  be shown that it is a fixed point for $\_\RHD\_$. Let $\oftyI{a}{A_{\McF{}}}$. We need
  to show that $\covers{\_}{(\covers{\_}{a})} = \covers{\_}{a}$. We proceed by
  antisymmetry. $\covers{\_}{a\downarrow} \subseteq \covers{\_}{(\covers{\_}{a\downarrow})}$ follows by $N_1$. The
  other direction is a direct application of Proposition~\ref{prop:lem4}.
\end{defn}

\section{Formal topologies present}\label{sec:universal-prop}

\paragraphsummary{Explain the aim.} We are now ready to shift our focus on what can be
called main theorem of this thesis: our notion of a formal topology is capable of
presenting a frame. Let $\mathcal{F}$ be a formal topology and $F$ a frame. Consider a
monotonic function $f : A_{\mathcal{F}} \rightarrow \abs{F}$ on the underlying posets. We will
define a notion of $f$ representing $\McF{}$ in $F$.

\begin{defn}[Representation]\label{defn:rep}
  Given a formal topology $\mathcal{F} = (A, B, C, d)$, a frame $F$, and a function
  $\oftyI{f}{A \rightarrow \abs{F}}$, we say that $f$ represents $\mathcal{F}$ in $F$ if the
  following type is inhabited:
  \begin{equation*}
    \represents{\mathcal{F}}{F}{f} \quad\is\quad
      \pity{a}{A}{
        \pity{b}{B(x)}{
          f(a) \sqsubseteq \bigvee_{\oftyI{z}{C(a, b)}} f(d(a, b, c)).
        }
      }
  \end{equation*}
\end{defn}

To state the universal property, we will work with \emph{flat} monotonic maps. This is the
special case of the notion of a \emph{flat functor}~\cite{nlab-flat-functor} in the case
where the categories we are working with are posets. Consider a monotonic map $f : P \rightarrow Q$,
where $Q$ has finite meets but $P$ does not. We would like to assert somehow that $f$
preserves these finite meets but we cannot mention the meets of $P$ as they do not
exist. So what we want to do is to state that $f$ preserves those meets that \emph{do not
exist yet} which we can do by requiring the following conditions:
\begin{enumerate}
  \item $\top_Q = \bigvee \img{f}{P}$, and
  \item $\pity{x~y}{P}{f(x) \wedge f(y) = \bigvee \setof{ f(z) ~|~ z \sqsubseteq x\ \text{and}\ z \sqsubseteq y }}$.
\end{enumerate}
In our case, we will of course be interested in monotonic maps whose codomains are frames.

\begin{defn}[Flat monotonic map]\label{defn:flat}
  Let $P$ be a poset and $F$ a frame. Denote by $I$ the type $\sigmaty{z}{\abs{F}}{z \sqsubseteq x \times
    z \sqsubseteq y}$. A monotonic map $f : P \rightarrow F$ from $P$ to the underlying poset of the frame is
  called \emph{flat} if the following type is inhabited:
  \begin{alignat*}{5}
    \isflat{f} \quad&\is\quad && &&\top_F &&= &&\bigvee f(P)\\
      &\hspace{0.4em}\times &&\pity{a~b}{P}{&&f(a) \wedge f(b) &&= &&\bigvee_{\oftyI{(i, p)}{I}} f(i)}.
  \end{alignat*}
\end{defn}

\begin{figure}
  \centering
  \caption{The universal property}
  \begin{tikzcd}[row sep=40pt, column sep=40pt]\label{fig:comm-diag}
    \McF{} \arrow[swap, rd, "f"] \arrow[r, "\eta"] & L \arrow[d, dashed, "g"] \\
                                                & R
  \end{tikzcd}
\end{figure}

Using flat monotonic maps, the universal property can now be stated.

\begin{thm}[Universal property for formal topologies]\label{thm:univ-prop}
  Given any formal topology $\McF{}$, frame $R$, flat monotonic map $f : A_{\McF{}} \rightarrow R$
  from the underlying poset of $\McF{}$ to the underlying poset of $R$, that represents
  $\McF{}$ in $R$, there exists a unique \textbf{frame homomorphism} $\oftyI{g}{L \rightarrow R}$
  such that $f = g \circ \eta$, as summarised in Fig.~\ref{fig:comm-diag}

  We express this fully formally as follows:
  \begin{align*}
    \pity{\McF{}}{\mathsf{FT}_{n, n}}{
      &\pity{R}{\framety{n+1}{n}{n}}{
         \pity{f}{\mono{P}{R}}{
           \isflat{f} \rightarrow \represents{\McF{}}{F}{f} \rightarrow\\
             &\iscontr{\sigmaty{g}{L \rightarrow_f R}{f = g \circ \eta}}
        }
      }
    }.
  \end{align*}
\end{thm}

Before proceeding to the proof, let us first prove a lemma of key importance.

\begin{lemma}\label{lem:main}
  Let $\McF{}$ be a formal topology. Denote by $L$ the set of downwards-closed subsets of
  $A_{\McF{}}$ that are fixed points for its covering nucleus. For any $\oftyI{U}{L}$,
  it is the case that:
  \begin{equation*}
    U = \bigvee^L \setof{ \eta(u) ~|~ u \in U }.
  \end{equation*}
\end{lemma}
\begin{proof}
  Let $\oftyI{U}{L}$. We proceed by antisymmetry. The $U \sqsubseteq \bigvee^L \setof{ \eta(u) ~|~ u \in U }$
  direction is immediate by an application for the $\ruledir{}$ rule. For the other
  direction, let $\oftyI{a}{A_{\McF{}}}$ and suppose that
  $a \in \bigvee^L \setof{ \eta(u) ~|~ u \in U }$. Recall that the join operation is defined by
  applying the nucleus in the fixed point frame (Theorem \ref{thm:fixed-point-frame}).
  This is to say that we have $\covers{a}{\bigcup \setof{ \eta(u) ~|~ u \in U }}$.
\end{proof}

\begin{proof}[Proof of Theorem~\ref{thm:univ-prop}]
  Let $\McF{}$ be a formal topology and $R$ a frame. Let $\oftyI{f}{P \rightarrow_m R}$ be a
  \emph{flat} monotonic map that represents $\McF{}$ in $R$
  (as defined in Defn.~\ref{defn:rep}). We will show the \emph{unique existence} of a
  \emph{frame homomorphism} $\oftyI{g}{L \rightarrow_f R}$ such that $f = g \circ \eta$.

  We choose
  \begin{align*}
    g    \quad&:\quad \abs{L} \rightarrow \abs{R} \\
    g(U) \quad&\is\quad \bigvee^R f(U)        .
  \end{align*}
  To show that the diagram commutes, it suffices by Proposition~\ref{prop:funext} to show
  $f(x) = g(\eta(x))$ for every $\oftyI{x}{A_{\McF{}}}$. Let $\oftyI{x}{A_{\McF{}}}$; we
  proceed by antisymmetry. The $f(x) \sqsubseteq g(\eta(x))$ direction is easy: $g(\eta(x)) = \bigvee^R f(\eta(x))$
  and $\bigvee^R$ is an upper bound of $f(\eta(x)$ so it suffices to show $f(x) \in f(\eta(x))$. This is
  immediate since $x \in \eta(x)$.

  The other direction is the interesting one. First, we prove a lemma in preparation:
  \emph{given any $\oftyII{a}{a'}{A_{\McF{}}}$ such that $\covers{a'}{a\downarrow}$, $f(a') \sqsubseteq f(a)$}.
  Let $\oftyII{a}{a'}{A_{\McF{}}}$ and assume $\covers{a'}{a}$. We proceed by induction on
  the proof of $\covers{a'}{a}$.
  \begin{itemize}
    \item Case: $\ruledir{}$. It must be that $a' \in a \downarrow$ i.e., $a' \sqsubseteq a$. We are done by
      the monotonicity of $f$.
    \item Case: $\rulebranch{}$. There must be some $\oftyI{b}{B_{\McF{}}(a')}$ and a
      function $h$ such that $\oftyI{h(c)}{\covers{d(a', b, c)}{a\downarrow}}$ for any
      $\oftyI{c}{C(a', b)}$. Notice that $f(a') \sqsubseteq \bigvee_{\oftyI{c}{C(a', b)}}f(d(a, b, c))$ by
      the assumption that $f$ \emph{represents} (Defn.~\ref{defn:rep}) so it remains to be
      shown $$\bigvee_{\oftyI{c}{C(a', b)}}d(a, b, c) \sqsubseteq f(a).$$ As
      $\bigvee_{\oftyI{c}{C(a', b)}}f(d(a, b, c))$ is a LUB, it suffices to show that $f(a)$ is an
      upper bound of the subset of $A_{\McF{}}$ delineated by $f(d(a', b, \_))$.
      Consider $f(d(a', b, c))$ for some $c$. We have that $\oftyI{h(c)}{\covers{d(a', b,
          c)}{a\downarrow}}$ meaning $f(d(a', b, c)) \sqsubseteq f(a)$ by the inductive hypothesis.
    \item Case $\rulesquash{}$. We combine the inductive hypotheses using the fact that
      the result type $f(a') \sqsubseteq f(a)$ is propositional.
  \end{itemize}

  Now, we want to show that $\bigvee^R f(\eta(x)) \sqsubseteq f(x)$. Since $\bigvee^R$ is a LUB it suffices to show
  that $f(x)$ is an upper bound of $\img{f}{\eta(x)}$. Let $$f(y) \epsilon \img{f}{\eta(x)}.$$ We need
  to show that $f(y) \sqsubseteq f(x)$. By the lemma we have just proven, it suffices to show that
  $\covers{y}{x\downarrow}$ and this holds directly because $f(y) \epsilon \img{f}{\eta(x)}$.

  There are two more things that have to be shown (1) $g$ is a frame homomorphism and (2)
  $g$ is unique.

  Let us first address (1). The fact that $g$ preserves the top element of $L$ is given
  by the flatness assumption: $g(\top) = \bigvee^R \img{f}{A_{\McF{}}} = \top_R$ by
  flatness (Defn.~\ref{defn:flat}). $g$ can also be seen to preserve binary meets by
  the following reasoning: let $\oftyII{U}{V}{L}$;
  \begin{align*}
    g(U \cap V) &\equiv \bigvee^R \{ f(a) ~|~ a \in U \cap V \} \\
             &= \bigvee^R \setof{ \bigvee \{ w ~|~ w \sqsubseteq u \times w \sqsubseteq v \} ~|~ (u, v) \in U \times V }
               && \text{[flatness]}                                                 \\
             &= \bigvee^R \setof{ f(u) \wedge f(v) ~|~ (u, v) \in U \times V }
               &&\text{[Prop.~\ref{prop:distr}]}                                    \\
             &= \left( \bigvee^R \img{f}{U} \right) \wedge \left( \bigvee^R \img{f}{V} \right)       \\
             &\equiv g(U) \wedge g(V)                                                         .
  \end{align*}
  Let us now show that $g$ preserves joins. Let $U$ be a family of inhabitants of
  $\abs{L}$.
  \begin{align*}
    g(\bigvee^L U) &\equiv \bigvee^R \setof{ f(a) ~|~ a \in \bigvee^L_i U_i } \\
             &= \bigvee^R \setof{ f(a) ~|~ \oftyI{(\_, (a, \_))}{\sigmaty{i}{I}{\sigmaty{x}{A_{\McF{}}}{x \in U_i}}} } &&\text{[Lemma~\ref{lem:flatten}]} \\
             &= \bigvee^R \setof{ \bigvee^L \img{f}{U_i} ~|~ U_i \in U } \\
             &\equiv \bigvee^R \setof{ g(U_i) ~|~ U_i \in U }.
  \end{align*}

  Finally, we conclude the proof by showing uniqueness of $g$: let $g'$ be a frame
  homomorphism from $L$ to $R$ that makes the diagram commute. We need to show that
  $g = g'$. Let $\oftyI{U}{\abs{L}}$. $g(U) = g'(U)$ by the following equational proof:
  \begin{align*}
    g(U) &\equiv \bigvee^R \setof{ f(u) ~|~ u \in U }
            &&\text{[Lemma~\ref{lem:main}]} \\
         &= g \left( \bigvee^L \setof{ \eta(u) ~|~ u \in U } \right) 
            &&\text{[$g$ is a frame homomorphism]} \\
         &= \bigvee^L \setof{ g(\eta(u)) ~|~ u \in U }
            &&\text{[$g \circ \eta = f$]} \\
         &= \bigvee^L \setof{ f(u) ~|~ u \in U }
            &&\text{[$g' \circ \eta = f$]} \\
         &= \bigvee^L \setof{ g'(\eta(u)) ~|~ u \in U }
            &&\text{[$g'$ is a frame homomorphism]} \\
         &= g'\left( \bigvee^L \setof{ \eta(u) ~|~ u \in U } \right)
            &&\text{[Lemma~\ref{lem:main}]} \\
         &= g'(U).
  \end{align*}
\end{proof}
