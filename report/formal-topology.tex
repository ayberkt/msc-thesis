\chapter{Formal Topology}\label{chap:formal-topo}

\paragraphsummary{Motivate formal topology.}
Our motivation for pointless topology was that it allows us to interpret topology in
constructive terms. The goal of developing topology in type theory presents further
challenges: we must be able to develop our results in a completely predicative way as
well. The definition of a frame we have seen suffers from impredicativity. We will not be
able to instantiate it to topologies we are interested in. The culprit is the join
operator.

\paragraphsummary{Explain the impredicativity of the join operator.}
TODO: explain and give examples.

\paragraphsummary{Introduce the tree type.}
To present frames, we will make use of the idea of tree set constructors, originally due
to Petersson and Synek~\cite{tree-sets}, who were trying to generalise Martin-Löf's
$\mathsf{W}$ type so that it can accommodate mutually recursive types. It is remarkable
that these structures embody the essence of a Post system in type theory. Tree set
constructors are also called interaction systems~\cite{hancock-interaction-systems} and
indexed containers~\cite{indexed-containers}; we will call it the tree type. This chapter
corresponds to the Agda module \texttt{TreeType} in the formal development.

\section{Petersson and Synek's tree type}

\paragraphsummary{Explain the idea of the tree type.}
The fundamental idea of an interaction system is simple. Consider the progression of a
two-player game. First, there is a type of \emph{game states}; call it $A$:
\begin{equation*}
  A~:~\univ.
\end{equation*}
At each state of the game, there are certain moves the player can take. In other words,
for every game state $x~:~A$, there is a type of possible moves the player may take.
Formally, this is a function:
\begin{equation*}
  B~:~A \rightarrow \univ.
\end{equation*}
Furthermore, for every move the player may take, the opponent can take certain
counter-moves in response. Formally:
\begin{equation*}
  C~:~\pity{x}{A}{B(x) \rightarrow \univ}.
\end{equation*}
Finally, given the counter-move in response to a certain move at some state, we proceed to
a new game state. This is given by some function:
\begin{equation*}
  d~:~\pity{x}{A}{\pity{y}{B(x)}{C(x, y) \rightarrow A}}
\end{equation*}
In four pieces, namely $(A, B, C, d)$, we express a ``game-like system'' in a very general
way. Even though the game analogy is very useful, the tree type is more general: it
expresses \emph{anything} that is like a dialogue i.e., two subjects interacting with each
other.

\paragraphsummary{Explain how we will use the tree type.}
Our presentation of frames will be based on this notion of interaction system, based upon
an idea of Coquand~\cite{coq-posets}. We will have to impose three additional requirements
on an interaction system.
\begin{enumerate}
  \item The type $A$ is equipped with a partial order. We will view this order as ranking
    states with respect to how \emph{refined} they are. This might be counter-intuitive at
    first: the more informative the states are, the smaller they will be. The sense of
    this ordering is: the more informative a state is, there less sequences of
    interactions there will be that pass through that state. It is like an open ball that
    encircles its center more tightly; hence it is smaller.
  \item The ordering on $A$ satisfies the \emph{monotonicity} requirement: for every state
    $x~:~A$, $d(x) \sqsubseteq x$. In other words, the states that we proceed to via interaction are
    always at least as informative as the previous ones.
\end{enumerate}
We will call an interaction system that satisfies (1) and (2) an \emph{discipline}, (in the
sense that the states resemble a discipline of knowledge, which we will explain later).
The final requirement is (3):
\begin{enumerate}
  \item The poset satisfies the \emph{simulation property} which states that at any state
    we simulate the previous states.
\end{enumerate}

\paragraphsummary{Formally define the tree type.}
Let us now formally define the tree type.
\begin{defn}[Tree type]
  For simplicity, we will require all types to be at the same level $m$ and omit this fact
  in the notation.
  \begin{align*}
    \treestr{A} &\quad\is\quad
      \sigmaty{B}{A \rightarrow \univ}{
        \sigmaty{C}{\pity{x}{A}{B(x) \rightarrow C}}{
          \pity{x}{A}{\pity{y}{B(x)}{C(x, y) \rightarrow A}}
        }
      }\\
    \mathsf{Tree} &\quad\is\quad \sigmaty{A}{~~~~\univ}{\treestr{A}}
  \end{align*}
  Given a tree type structure $\mathcal{T}$, we will refer to its components as
  $B_{\mathcal{T}}, C_{\mathcal{T}}$, and $d_{\mathcal{T}}$.
\end{defn}

\paragraphsummary{Formally define disciplines.}
\begin{defn}[Discipline]
  \begin{align*}
    \mathsf{Discipline} \quad&\is\quad \sigmaty{P}{\poset{}}{\disciplinestr{P}}                 \\
    \disciplinestr{P}   \quad&\is\quad \hspace{-0.9em}\sigmaty{\mathcal{T}}{\treestr{\abs{P}}}{
      \pity{x}{\abs{P}}{\pity{y}{B_{\mathcal{T}}(x)}{
          \pity{z}{C_{\mathcal{T}}(x, y)}{d_{\mathcal{T}}(x, y, z) \sqsubseteq x}}
      }
    }
  \end{align*}
  We will refer to this as the monotonicity property \emph{of a discipline}. This is not
  to be confused with the monotonicity of a monotonic map.
\end{defn}

\paragraphsummary{Transition to talking about the simulation property.}
The only remaining thing towards our definition of formal topology is the aforementioned
simulation property. It requires a couple of auxiliary concepts which we will now define.

\paragraphsummary{Formally define the simulation property.}
We will first define this formally and justify it conceptually afterwards.
\begin{defn}[Simulation property]
  Given a discipline $D$, we will say that it satisfies the simulation property if the
  following type is inhabited:
  \begin{equation*}
    \pity{x~x'}{\abs{D}}{
      x' \sqsubseteq x \rightarrow \pity{y}{B(x)}{
        \sigmaty{y'}{B(x')}{
          \pity{z'}{C(x', y')}{
            \sigmaty{z}{C(x, y)}{
              d(x', y', z') \sqsubseteq d(x, y, z)
            }
          }
        }
      }
    }.
  \end{equation*}
\end{defn}

\paragraphsummary{Intuitive justification.}
What does this say intuitively? At more refined stages we can always find a counterpart to
any experiment from a less refined stage, in the sense that that experiment will lead to a
more refined stage. To put it more succinctly:
\begin{quote}
  \emph{lower stages can always simulate upper stages}.
\end{quote}

\paragraphsummary{Define formal topology.}
Once the property of simulation has been defined it is easy to state our
definition of a formal topology.
\begin{defn}[Formal topology]
  A formal topology is a discipline satisfying the simulation property.
\end{defn}

\section{Cover relation}

\paragraphsummary{Motivate and provide historical summary.} The reason we defined the
notion of a formal topology is that it admits a cover relation. This method goes back to
Johnstone's~\cite{stone-spaces} adaptation of the notion of a Grothendieck topology, that
was subsequently developed by Martin-L\"{o}f and Sambin~\cite{int-formal-spaces}. The
original formulation of Sambin suffered from the problem that it was not possible to
define the coproduct of two frames using it. This problem was solved by Coquand, Sambin,
and others~\cite{coq-sambin} by defining the cover relation inductively. It is this method
that we will follow in this development.

\paragraphsummary{Defn.~of coverage.}
First, we define the coverage relation on a given formal topology.
\begin{defn}[Coverage relation]
  \[
  \begin{prooftree}
    \hypo{ a \epsilon U }
    \infer1[\textsf{dir}]{\covers{a}{U}}
  \end{prooftree}
  \qquad
  \begin{prooftree}
    \hypo{
      \pity{b}{B(a)}{\left( \pity{c}{B(a, b)}{d(a, b, c) \triangleleft U} \right) \rightarrow a \triangleleft U}
    }
    \infer1[\textsf{branch}]{\covers{a}{U}}
  \end{prooftree}
  \]
\end{defn}

\paragraphsummary{Cover is a nucleus.}
This coverage relation gives us a way of obtaining a frame from a formal topology. Let us
look at the type of the coverage relation:
\begin{equation*}
  \_\covernm{}\_ : A \rightarrow \pow{A} \rightarrow \univ{}
\end{equation*}
which we can flip to get
\begin{equation*}
  \_\triangleright\_ : \pow{A} \rightarrow A \rightarrow \univ{}
\end{equation*}
which can be written as
\begin{equation*}
  \_\triangleright\_ : \pow{A} \rightarrow \pow{A}
\end{equation*}
so we have an endofunction. We can restrict this to the subset of $\pow{A}$ that is
downwards-closed
\begin{equation*}
  \_\triangleright\_ : \dcsubset{A} \rightarrow \dcsubset{A},
\end{equation*}
which of course requires us to show that given a downwards-closed subset $U$,
$\covers{\_}{U}$ is a subset that is downwards-closed.

\begin{thm}
  Given any subset $U$, $\covers{\_}{U}$ is a downwards-closed subset.
\end{thm}
\begin{proof}
  \todo{Complete the proof.}
\end{proof}

Now, our method of obtaining a frame out of a formal topology is the following.
\begin{enumerate}
  \item Start with a formal topology $\mathcal{T}$ on type $A$.
  \item $\mathcal{T}$ has an underlying poset $P$; construct its frame of downwards-closed
    subsets.
  \item Note: $\covers{\_}{\_}$ is a nucleus on the frame of downwards-closed subsets.
  \item We have shown (in Theorem~\ref{thm:fixed-point-frame}) that the set of
    fixed-points of every nucleus is a frame. The final frame is this fixed-point frame
    on the frame of downwards-closed subsets by the nucleus $\covers{\_}{\_}$.
\end{enumerate}

\section{Discipline presentations present}

\paragraphsummary{Explain the aim.}
We are now ready to shift our focus on what can be called main theorem of this
thesis: our notion of a formal topology is capable of presenting a frame. We
will see that this presents an interesting challenge, arising from the
peculiarity of \UF{}.

\paragraphsummary{Explain the intuition of representation.}
Let $A$ be a formal topology and $L$ a frame. Consider a monotonic function $f : \abs{A} \rightarrow
\abs{L}$ on the underlying posets. We will define a notion of $f$ representing $F$ in $L$
which is to say $f$ encodes crucial information of $F$.

\begin{defn}[Representation]
  \begin{equation*}
  \pity{x}{A}{\pity{y}{B(x)}{f(x) \sqsubseteq \left( \bigvee_{z~:~C(x, y)} f(d(x, y, z))}} \right)
  \end{equation*}
\end{defn}

\todo{State and prove the main theorem.}
