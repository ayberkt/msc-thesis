\chapter{Introduction}\label{chap:intro}

This thesis is about topology, the branch of mathematics that studies \emph{continuous}
functions. The notion of a continuous function pervades practically all of mathematics as
pointed out by Sylvester~\cite[pg.~27]{armstrong-topology}: ``if I were asked to name, in
one word, the pole star round which the mathematical firmament revolves, the central idea
which pervades the whole corpus of mathematical doctrine, I should point to Continuity as
contained in our notions of space, and say, it is this, it is this!''. Let us then start
by considering the question of what continuity is.

A continuous function, in the context of real numbers, is a function for which ``small
changes to the input result in small changes to the output''. This is formally expressed
in the following ``$\epsilon$-$\delta$'' definition of continuity: a function $f : \reals{} \rightarrow
\reals{}$ is
\emph{continuous} if
\begin{equation*}\label{eq:cont-0}
  \forall x \in \reals{}.~ \forall \epsilon > 0.~ \exists \delta > 0.~ \forall y \in \reals{}.~
    | x - y | < \delta \rightarrow | f(x) - f(y) | < \epsilon.
\end{equation*}
This definition embodies the idea that, to make $f(x)$ closer than $\epsilon$ to
$f(y)$, it suffices to make $x$ closer than $\delta$ to $y$, for some certain $\delta$.

To pinpoint the essence of continuity, we will consider it in general form. $\reals{}$ is
just one instance of a \emph{metric space}, a set that admits a well-behaved notion of
distance, its distance function being:
\begin{alignat*}{2}
  d       \quad&:\quad  && \reals{} \times \reals{} \rightarrow \reals{} \\
  d(x, y) \quad&:=\quad && | x - y |                      .
\end{alignat*}
This is just \emph{one} instance of a notion of distance on a set that is viewed as a set
of points. We can speak of the continuity of a function between any two metric spaces. Let
$X$ and $Y$ be two metric spaces. We now define continuity for any function $f : X \rightarrow Y$:
\begin{equation}\label{cont-1}
  \forall x \in X.~ \forall \epsilon > 0.~ \exists \delta > 0.~ \forall y \in X.~ d_X(x, y) < \delta \rightarrow d_Y(f(x), f(y)) < \epsilon.
\end{equation}

Notice that we could have written (\ref{cont-1}) in an alternative way. Given some $x \in X,
\epsilon \in \reals{}$ define
\begin{equation*}
  \ball{x}{\epsilon} \quad\is\quad \setof{ y \in X ~|~ d_X(x, y) < \epsilon }.
\end{equation*}
Now, the following is the same as (\ref{cont-1}):
\begin{equation*}
  \forall x \in X.~ \forall \epsilon > 0.~ \exists \delta > 0.~ \forall y \in X.~ y \in \ball{x}{\delta} \rightarrow f(y) \in \ball{f(x)}{\epsilon},
\end{equation*}
which could be expressed even more compactly as:
\begin{equation}\label{cont-2}
  \forall x \in X.~ \forall \epsilon > 0.~ \exists \delta > 0.~ f(\ball{x}{\delta}) \subseteq \ball{f(x)}{\epsilon}.
\end{equation}

After having written down continuity in this more compact way, we consider the question:
what is the meaning of $\ball{x}{\epsilon}$? Given some $x \in X$, $\ball{x}{\epsilon}$ expresses the set
of points that are \emph{closer than} $\epsilon$ to $x$. One intuitive reading of this is:
$\ball{x}{\epsilon}$ denotes the set of approximations of $x$ with a degree of accuracy of $\epsilon$.
As $\epsilon$ decreases, the accuracy with which $\ball{x}{\epsilon}$ represents $x$ increases. What is
crucial here is that, as $x$ might be infinite, we cannot (decidably) judge the membership
of an element in set $\setof{ x }$, but we can judge its membership in set $\ball{x}{\epsilon}$
so this is an approximation of ``infinite precision'' properties by finitely observable
ones. This situation is familiar from \emph{experimental} sciences: even though we might
not be able to pin down an exact (i.e., with $\epsilon \equiv 0$) value $v$, we can approximate $v$ by
conducting more accurate experiments, hence narrowing the value down to a tight
$\ball{v}{\epsilon}$.

To see continuity in its full generality, we now notice that the notion of distance is not
crucial to what is expressed in (\ref{cont-2}). We have made use of the distance function
just to be able to specify the approximation sets. Instead, we can work directly with a
specification of what the approximation sets are. This is the main idea underlying
topology: instead of working with a notion of distance, we work directly with a
specification of the approximations.

The definition of a metric space ensures that the distance function is well-behaved. When
one defines approximation sets in terms of such a distance function, they are necessarily
well-behaved: they satisfy one's expectations from a reasonable approximation structure.
As we want to remove our dependency on a notion of distance, we will have to axiomatise
the behaviour of approximation sets themselves instead.

The standard term for such a class of sets embodying an approximation structure is a
\emph{basis for a topology}~\cite{munkres}. We now summarise it precisely in the following
definition.
\begin{defn}[Basis for a topology]
  Given a set $X$, a basis for a topology on $X$ is a class $\mathcal{B}$ of subsets of
  $X$ such that:
  \begin{enumerate}
  \item $\forall x \in X.~\exists B \in \mathcal{B}.~x \in B$, which intuitively says ``there is an
    approximation for all points'', and
  \item $\forall x \in X.~\forall B_0, B_1 \in \mathcal{B}.~x \in B_0 \cap B_1 \rightarrow
    \exists B_2 \in \mathcal{B}.~ B_2 \subseteq (B_0 \cap B_1) \wedge x \in B_2$, which says
    ``approximations can be combined to yield finer approximations''.
  \end{enumerate}
\end{defn}

\paragraphsummary{Continuity with bases.}
Now, we can reformulate continuity in a more general way, without relying on a notion of
distance. Given sets $X, Y$ with bases $\mathcal{B}_X, \mathcal{B}_Y$, a function $f : X \rightarrow
Y$ is continuous if
\begin{equation}\label{cont-basis}
  \forall x \in X.~ \forall V \in \mathcal{B}_Y.~ f(x) \in V \rightarrow \exists U \in \mathcal{B}_X.~
    f(U) \subseteq V.
\end{equation}
This expresses the idea in (\ref{cont-2}), but talking directly about some given
approximation structure rather than defining this structure through a notion of distance.
In other words, we previously said that a function is continuous if we can make $f(x)$
closer than $\epsilon$ to $f(y)$ by making $x$ closer than $\delta$ to $y$. We are now saying that for
every set approximating some $f(x)$, there is a set approximating $x$ such that the image
of the latter is contained in the former, which corresponds to the idea in (\ref{cont-2}).

Given a basis $\mathcal{B}$ on set $X$, we say that the topology generated by
$\mathcal{B}$ is the class of all possible unions of $\mathcal{B}$. The idea here is that
an element $U$ of the topology is like a property of the points that we are interested in
\emph{verifying}. To verify $x \in U$, it suffices to show that $x$ lies in one of the basis
elements that constitute $U$ and since the basis elements are approximation sets, the fact
that $x$ has property $U$ can be verified from a finite approximation of $x$, which
requires only a finite amount of data (``finite precision'').

In reality, it is often the case that we deal directly with bases that generate
topologies. However, we will not have reached full generality until we can talk about
\emph{topologies} directly i.e., properties of a set which can be verified via
approximations. We will call such properties finitely verifiable or
\emph{observable}~\cite{abramsky-thesis}. The topological term is ``open set'' in the
sense that the set does not contain its own boundary.

If we have two finitely verifiable properties $U$ and $V$, and we know that a point
satisfies both of them, there should be some degrees of accuracy, $ε₀$ and $ε₁$, at which
$U$ and $V$ are verified. Then, we can verify both by taking the lower between them so we
can verify their intersection. On the other hand, if we have an arbitrary number of
finitely verifiable properties, to verify membership in the union, it suffices to verify
membership in just one of the constituents, so if a point is in this set it must be in one
of the finitely verifiable subsets, meaning the set itself is finitely verifiable as the
degree of accuracy required to check membership in the constituent set is sufficient to
check membership in the union. This brings us to the following definition of a topological
space~\cite{munkres}.
\begin{defn}[Topological space]\label{defn:topospace}
  A topology on a set $X$ is a class $\mathcal{T}$ of subsets of $X$ such that
  \begin{itemize}
    \item The trivial subsets $X, \emptyset \subseteq X$ are in $\mathcal{T}$,
    \item $\mathcal{T}$ is closed under finite intersections, and
    \item $\mathcal{T}$ is closed under arbitrary unions.
  \end{itemize}
\end{defn}

By this definition, we start with a set $X$ of \emph{points} on which we are interested in
making observations. The topology we attach to $X$ specifies these observable sets. The
view of topology we will be concerned with in this thesis is called \emph{pointless}
topology~\cite{johnstone-the-point}. In pointless topology, we start with a primitive set
of opens instead, and we study these directly. In other words, we reverse the conceptual
order of open sets being notions derived from points: we view points as being derived from
the opens instead. What is remarkable is that we can go a long way without mentioning the
points~\cite{johnstone-the-point}.

So how do we express the idea of open sets \emph{not} as sets of points? We simply
axiomatise the behaviour of the lattice of open sets. Let $\mathcal{O}$ be a set of
opens, that are some unspecified primitive entities.
\begin{enumerate}
  \item Corresponding to the set-inclusion partial order in the pointful case, we require
    that there be a partial order $\_\sqsubseteq\_ \subseteq \mathcal{O} \times \mathcal{O}$.
  \item Corresponding to the fact that open sets are closed under finite intersection, we
    require that there be a binary meet operation $\meet{\_}{\_} : \mathcal{O} \times
    \mathcal{O} \rightarrow \mathcal{O}$ and a nullary one $\top : \mathcal{O}$.
  \item Corresponding to the fact that open sets are closed under arbitary union, we
    require that there be a join operation of arbitrary arity: $\joinnm{}\_ :
    \mathcal{P}(\mathcal{O}) \rightarrow \mathcal{O}$.
\end{enumerate}
In addition to these, we have to require that these operations satisfy an infinite
distributivity law, on which we will elaborate in Chapter~\ref{chap:frames}. The official
name for such a lattice that embodies a logic of observable or finitely verifiable
properties is \emph{frame}~\cite{vickers}. Indeed, frames form a category whose morphisms
are frame homomorphisms; objects of the opposite category are called \emph{locales}.
Therefore, frames and locales are synonymous
\emph{as long as no mentions to morphisms are made}~\cite{vickers, stone-spaces}

A natural question is then: what do we gain by studying topology in a pointless way? In
this thesis, we will look at topology from the perspective of computer science therefore
we would like to be able to understand theorems of topology in computational terms.
Topology notoriously relies on classical reasoning in many of its fundamental theorems
such as the Tychonoff theorem. By doing topology pointlessly, we can avoid classical
reasoning and hence gain a computational understanding of it---most saliently of such
important theorems such as Tychonoff~\cite{coq-tychonoff}. This point was put eloquently
by Johnstone~\cite[pg.~46]{stone-spaces}:
\begin{quote}
  It is here that the real point of pointless topology begins to emerge; the difference
  between locales and spaces is one that we can (usually) afford to ignore if we are
  working in a ``classical'' universe with the axiom of choice available, but when (or if)
  we work in a context where choice principles are not allowed, then we have to take
  account of the difference—and usually it is locales, not spaces, which provide the right
  context in which to do topology. This is the point which, as I mentioned earlier,
  Andr\'{e} Joyal began to hammer home in the early 1970s; I can well remember how, at the
  time, his insistence that locales were the real stuff of topology, and spaces were
  merely figments of the classical mathematician's imagination, seemed (to me, and I
  suspect to others) like unmotivated fanaticism. I have learned better since then.
\end{quote}

In this thesis, we are interested in investigating topology in the context of type theory.
The first prerequisite for this is that we be able to develop topology constructively,
that is, without relying on classical principles. As pointed out by Johnstone, pointless
topology can help us here as it allows us to understand topology in constructive terms.
The goal of carrying out topology in type theory, however, presents further challenges: we
have to understand topology not only constructively but also \emph{predicatively}.

This is the subject matter of the branch of mathematics known as formal topology, first
instigated by Martin-Löf and Sambin~\cite{int-formal-spaces} in the early days of type
theory. The idea is that a formal topology recasts the notion of a frame into a form that
resembles a formal proof system. In addition to enabling the importation of
proof-theoretical ideas, such a formal presentation has the virtue of being
\emph{predicative} hence enabling the development of frames in type theory.

In this thesis, we present a development of formal topology in univalent type
theory~\cite{hottbook}. By now, it has become clear that univalence addresses certain
severe shortcomings of type theory. The question of what novelties it presents for formal
topology is therefore a natural one. In attempting to answer this question, we follow a
particular approach to formal topology, implementing an idea of Coquand~\cite{coq-posets}
to define formal topologies as posets endowed with ``interaction''
structures~\cite{tree-sets, hancock-interaction-systems}.

This thesis is structured as follows. In Chapter~\ref{chap:foundations}, we summarise the
fundamentals of univalent type theory. In Chapter~\ref{chap:frames}, we present our
development of frames and constructs related to them. In Chapter~\ref{chap:formal-topo},
we present our main development of formal topology in univalent type theory. In
Chapter~\ref{chap:cantor}, we present a prime example of a formal topology: the Cantor
space. As a proof of concept for the formal study of topological properties in univalent
type theory, we prove an important property of the Cantor space, namely, that it is
compact.
