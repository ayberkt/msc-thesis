\chapter{Introduction}\label{chap:intro}

\paragraphsummary{State what topology is about.}
This thesis is about topology, the branch of mathematics that studies \emph{continuous}
functions. The notion of a continuous function pervades practically all of mathematics, as
pointed out by JJ Sylvester: ``if I were asked to name, in one word, the pole star round
which the mathematical firmament revolves, the central idea which pervades the whole
corpus of mathematical doctrine, I should point to Continuity as contained in our notions
of space, and say, it is this, it is this!''~\cite[pg. 27]{armstrong-topology}. Let us
then start by considering the question of what continuity is.

\paragraphsummary{$\epsilon$-$\delta$ definition of continuity.}
A continuous function, in the context of real numbers, is a function for which
``small changes to the input result in small changes to the output''. This is
formally expressed in the following ``$\epsilon$-$\delta$ definition'' of continuity: a
function March$f : \reals{} \rightarrow \reals{}$ is
\emph{continuous} if
\begin{equation*}\label{eq:cont-0}
  \forall x \in \reals{}.~ \forall \epsilon > 0.~ \exists \delta > 0.~ \forall y \in \reals{}.~
    | x - y | < \delta \rightarrow | f(x) - f(y) | < \epsilon.
\end{equation*}
This definition embodies the idea that, to make $f(x)$ closer than $\epsilon$ to
$f(y)$, it suffices to make $x$ closer than $\delta$ to $y$, for some certain $\delta$.

\paragraphsummary{Generalise the notion of distance.}
To pinpoint the essence of continuity, we will generalise this. First, notice that the
function
\begin{equation*}
  d(\pair{x}{y}) = | x - y | : \reals{} \times \reals{} \rightarrow \reals{}
\end{equation*}
is just a \emph{special} notion of distance between inhabitants of the set of interest,
namely, $\reals{}$. There are many other sets whose elements can be viewed as points in a
space. $\reals{}^2$, for instance, with the usual notion of distance between two points in
the two-dimensional plane has the corresponding distance function:
\begin{equation*}
  d(\pair{x_0}{y_0} , \pair{x_1}{y_1}) = \sqrt{(x_0 - x_1)^2 + (y_0 - y_1)^2}.
\end{equation*}
We can therefore formulate continuity for any set $X$ which admits a notion of distance
between two points. Formally, we require an appropriate function $d : X \times X \rightarrow \reals{}$.
Such a set endowed with a distance function will be called \emph{metric space} if it
satisfies certain axioms that ensure that the distance function is well-behaved i.e.,
behaves properly like a notion of distance. For instance, we may expect that $d(x, y) =
d(y, x)$ for any two points $x, y \in X$. The specifics of these axioms are not central to
our focus so we will refrain from presenting those. Let $X$ and $Y$ be two metric spaces.
Now we define continuity for any function $f : X \rightarrow Y$:
\begin{equation}\label{cont-1}
  \forall x \in X.~ \forall \epsilon > 0.~ \exists \delta \in X.~ \forall y.~ d_X(x, y) < δ \rightarrow d_Y(f(x), f(y)) < ε.
\end{equation}
Now we can express continuity of a function on any set that behaves like a space.

\paragraphsummary{Transition to balls.}
Notice that we could have written (\ref{cont-1}) in an alternative way. Given
some $x \in X, \epsilon \in \reals{}$ define
\begin{equation*}
  \ball{x}{\epsilon} \quad\is\quad \{ y \in \reals{}~|~d(x, y) < \epsilon \}.
\end{equation*}
Now, the following is the same as (\ref{cont-1}).
\begin{equation*}
  \forall x \in X.~ \forall \epsilon > 0.~ \exists \delta > 0.~ \forall y \in X.~ y \in \ball{x}{\epsilon} \rightarrow f(y) \in \ball{f(x)}{\delta},
\end{equation*}
which could be expressed even more compactly as:
\begin{equation}\label{cont-2}
  \forall x \in \reals{}.~ \forall \epsilon > 0.~ \exists \delta > 0.~ f(\ball{x}{\epsilon}) \subseteq \ball{f(x)}{\delta}.
\end{equation}

\paragraphsummary{Open balls as approximation sets.}
After having written down continuity in this more compact way, we consider the question:
what is the meaning of $\ball{x}{\epsilon}$? Given some $x \in X$, $\ball{x}{\epsilon}$ expresses the set
of things that are closer than $\epsilon$ to $x$. One intuitive reading of this is that
$\ball{x}{\epsilon}$ denotes the \emph{set of approximations of $x$ with a degree of accuracy of
$\epsilon$}. As $\epsilon$ decreases, the accuracy with which the inhabitants of $\ball{x}{\epsilon}$
represent $x$ increases. Consider, for instance, $\pi$ whose digits we cannot fully write
down. We can, however, obtain arbitrarily precise expansions of it such as
$\ball{\pi}{10^{-10}}$, if we are willing to wait long enough.

\paragraphsummary{Distance is not needed.}
To see continuity in its full generality, we now notice that the notion of distance is not
crucial to what is expressed in (\ref{cont-2}). We have made use of the distance function
just to be able to specify the approximation sets. Instead, we can work directly with a
specification of what the approximation sets are. This is the main idea underlying
topology: instead of working with a notion of distance, we work directly with a
specification of the approximations.

\paragraphsummary{Axiomatisation of basis (0).}
We mentioned that certain metric space axioms ensure that the distance function is
well-behaved. When one defines approximation sets in terms of such a distance function, it
is guaranteed that they will be well-behaved as well. As we want to completely remove our
dependency on a notion of distance, we will have to axiomatise instead the behaviour of
approximation sets themselves.

\paragraphsummary{Axiomatisation of basis (1).}
Any approximation set for a real number must that real number itself. In other words, we
want approximations to give us ``error bars'' around some result and any error bar that
does not encircle the result is clearly ill-behaved.

\paragraphsummary{Axiomatisation of basis(2).}
Furthermore, if we have two approximation sets for a real number, say $A_0$ and $A_1$,
that real number must lie in $A_0 \cap A_1$. We can then pick out a \emph{finer}, more
accurate approximation $A_2$ for the number such that $A_2 \subseteq A_1 \cap A_2$. For instance,
$A_0$ might say that $\pi$ lies within $3.0$ and $3.2$ and $A_1$ that it lies within $3.1$
and $3.5$ from which we can tell that it lies within $3.1$ and $3.2$. This will be our
second requirement.

\paragraphsummary{Formal defn.~ of basis.}
The standard term in topology for such a class of sets is a \emph{basis for a topology}.
We now summarise it precisely in the following definition.
\begin{defn}[Basis for a topology]
  Given a set $X$, a basis for a topology on $X$ is a class $\mathcal{B}$ of subsets of
  $X$ such that:
  \begin{enumerate}
    \item $\forall x \in X.~ \exists B \in \mathcal{B}.~x \in B$, which intuitively says ``approximations
      approximate'', and
    \item $\forall x \in X.~ \exists B_0, B_1 \in \mathcal{B}.~ x \in B_0 \cap B_1 \rightarrow \exists B_2 \subseteq (B_0 \cap B_1).~
      x \in B_2$, which says ``approximations can be refined''.
  \end{enumerate}
\end{defn}

\paragraphsummary{Continuity with bases.}
Now, we can reformulate continuity in a more general way using basis elements. Given sets
$X, Y$ with bases $\mathcal{B}_X, \mathcal{B}_Y$, a function $f : X \rightarrow Y$ is continuous if
\begin{equation}\label{cont-basis}
  \forall x \in X.~ \forall V \in \mathcal{B}_Y.~ f(x) \in V \rightarrow \exists U \in \mathcal{B}_X.~
    f(U) \subseteq V.
\end{equation}
This says exactly the same thing as (\ref{cont-2}), but talking directly about some given
approximation sets rather than defining these through a notion of distance. In other
words, we previously said that a function is continuous if we can make $f(x)$ closer than
$\epsilon$ to $f(y)$ by making $x$ closer than $\delta$ to $y$. We are now saying that for every
set approximating some $f(x)$, there is a set approximating $x$ such that the image of
the latter is contained in the former, which corresponds directly to (\ref{cont-2}).

Given a basis $\mathcal{B}$ on set $X$, we say that the topology generated by
$\mathcal{B}$, that we call $\mathbf{T}_{\mathcal{B}}$, is the class of all possible
unions of $\mathcal{B}$. The idea here is that an element $U$ of
$\mathbf{T}_{\mathcal{B}}$ is like a property of the set that we are interested in
\emph{verifying}. To verify $U$, it suffices to show that it falls in one of the basis
elements that constitute $U$ and since all of these are approximation sets, $U$ can be
verified from a finite amount of data since approximations suffice to express it.

\paragraphsummary{Start motivating topologies.}
In reality, it is often the case that we deal directly with the bases that generates
topologies. However, we will not have reached full generality until we can talk about
\emph{topologies} i.e., properties of a set which can be verified via approximations.
We will call such properties \emph{finitely verifiable}. The topological term is ``open
set'' in the sense that the set does not contain its own boundary.

If we have two finitely verifiable properties, we should be able to verify them from an
approximations of degrees of accuracy of $ε₀$ and $ε₁$. Then, we can verify both by taking
the higher between them so we can verify their intersection. If we have an arbitrary
number of finitely verifiable properties, to verify the union it suffices to verify at
least one, so if an point is in this set it must be in one of the finitely verifiable
subsets, meaning the set itself is finitely verifiable. This brings us to the following
definition:
\begin{defn}[Topological space]\label{defn:topospace}
  A topology on a set $X$ is a class $\mathcal{T}$ of subsets of $X$ such that
  \begin{itemize}
    \item The trivial subsets $X, \emptyset \subseteq X$ are in $\mathcal{T}$,
    \item $\mathcal{T}$ is closed under finite intersections, and
    \item $\mathcal{T}$ is closed under arbitrary unions.
  \end{itemize}
\end{defn}

\paragraphsummary{Transition to pointless topology.}
From the perspective where we view topology as a mathematical theory of finitary
approximations, it makes further sense to take the finitely verifiable properties as
primitive and then investigate them directly instead of defining them as \emph{derived
notions} that are defined as certain kinds of ``sets of points''. This is analogous to
the distinction between analytic and synthetic geometry: in the former, one defines a
space as a set of points, whereas in the latter one works directly with constructs on
spaces such as angles, lines, and circles. This approach in which we start directly with
a given set of finitely verifiable properties is called \emph{pointless topology} in the
sense that we avoid mentioning the points until we really need them.

\paragraphsummary{Motivate frames.}
In pointless topology, we start with a set $\mathcal{O}$ of opens. We then axiomatise
directly how these opens shall behave.
\begin{enumerate}
  \item Corresponding to the set-inclusion partial order in the pointful case, we require that
    there be a partial order $\_\sqsubseteq\_ \subseteq \mathcal{O} \times \mathcal{O}$.
  \item Corresponding to the fact that open sets are closed under finite intersection, we
    require that there be a binary meet operation $\meet{\_}{\_} : \mathcal{O} \times
    \mathcal{O} \rightarrow \mathcal{O}$ and a nullary one $\top : \mathcal{O}$.
  \item Corresponding to the fact that open sets are closed under arbitary union, we
    require that there be a join operation of arbitrary arity: $\joinnm{}\_ :
    \pow{\mathcal{O}} \rightarrow \mathcal{O}$.
\end{enumerate}
In addition to this, require that these operations satisfy the distributivity law, which
we will elaborate on in Chapter~\ref{chap:frames}. Such a lattice that embodies a logic
of finitely verifiable properties is called a \emph{frame}.

\paragraphsummary{Explain the benefit of pointless topology.} A question that is natural
is: what does this give us that traditional topology lacks? In this thesis, we will look
at topology from the perspective of computer science therefore we would like to be able to
understand theorems of topology \emph{in computational terms}. Topology notoriously relies
on classical reasoning in many of its fundamental theorems such as the Tychonoff theorem.
By doing pointless topology, we will see that we can avoid classical reasoning and hence
gain a computational understanding of topology. This point was put eloquently by
Johnstone~\cite[pg.~46]{stone-spaces}:
\begin{quote}
  It is here that the real point of pointless topology begins to emerge; the difference
  between locales and spaces is one that we can (usually) afford to ignore if we are
  working in a ``classical'' universe with the axiom of choice available, but when (or if)
  we work in a context where choice principles are not allowed, then we have to take
  account of the difference—and usually it is locales, not spaces, which provide the right
  context in which to do topology. This is the point which, as I mentioned earlier,
  Andr\'{e} Joyal began to hammer home in the early 1970s; I can well remember how, at the
  time, his insistence that locales were the real stuff of topology, and spaces were
  merely figments of the classical mathematician's imagination, seemed (to me, and I
  suspect to others) like unmotivated fanaticism. I have learned better since then.
\end{quote}

\paragraphsummary{Motivate formal topology.}
The goal of carrying out topology in type theory presents further challenges: we must do
it not only constructively but also \emph{predicatively}. To address this problem, we will
work with presentations of frames that are called \emph{formal
topologies}~\cite{int-formal-spaces}. The idea is that a formal topology is like a
formal proof system that allows us to deal with our topology as though it were a proof
system. In addition to the benefit of constructively understanding the results of
topology, formal topology provides us the further advantage of being able to understand
them predicatively.

\paragraphsummary{Explain the goal of the thesis.}
In this thesis, we present a development of formal topology in \UF{}. By now, it has
become clear that univalence addresses many shortcomings of type theory therefore the
question of what novelties it presents for the question of topology in type theory is a
natural one. We focus on one particular approach to formal topology, implementing an idea
going back to Coquand~\cite{coq-posets} to endow posets with an ``interaction'' structure.

\paragraphsummary{Summarise the thesis structure.}
This thesis is structured as follows. In Chapter~\ref{chap:foundations}, we summarise the
fundamentals of \UF{}. In Chapter~\ref{chap:frames}, we present our development of frames
and constructs related to them in preparation for Chapter~\ref{chap:formal-topo} in which
we present our main development of formal topology in \UF{}.
