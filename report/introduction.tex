\chapter{Introduction}\label{chap:intro}

This thesis is about topology, the branch of mathematics that studies \emph{continuous}
transformations between topological spaces. The raison d'être of topological spaces is to
allow the formulation of the notion of continuity that pervades practically all of
mathematics, as pointed out by Sylvester~\cite[pg.~27]{armstrong-topology}: ``if I were
asked to name, in one word, the pole star round which the mathematical firmament revolves,
the central idea which pervades the whole corpus of mathematical doctrine, I should point
to Continuity as contained in our notions of space, and say, it is this, it is this!''.
Let us then start by considering the question of what topology is.

The usual definition of a topological space is the following~\cite{munkres}.
\begin{defn}[Topological space]\label{defn:topospace}
  A topology on a set $X$ is a class $\mathcal{T}$ of subsets of $X$ such that
  \begin{itemize}
    \item The trivial subsets $X, \emptyset \subseteq X$ are in $\mathcal{T}$,
    \item $\mathcal{T}$ is closed under \emph{finite} intersections, and
    \item $\mathcal{T}$ is closed under \emph{arbitrary} unions.
  \end{itemize}
  A topological space is a set $X$ equipped with a topology $\mathcal{T}$.
\end{defn}

When a set $X$ forms a topological space i.e.~is equipped with some topology
$\mathcal{T}$, its elements are called \emph{points} and the members of $\mathcal{T}$ are
called \emph{open sets} of $X$. A noteworthy thing about this definition is the asymmetry
between intersection and union: a topology is required to be closed only under finite
intersection whereas it \emph{must} be closed under arbitrary union.

This asymmetry is familiar to computer scientists from computability theory. Let $P$ be a
program whose internal structure we do not have access to. One could say it is a ``black
box'' although this is not to say it is a scenario occurring only in theory; black boxes
\emph{do} occur in practice. A compiled program whose source code one does not have access
to, or even, a program whose source code is not familiar to one, can be seen as examples
of black boxes from one's perspective.

One way or another, suppose that we are in a situation in which we have to understand the
behaviour of some program $P$ by running it and examining its output. For the sake of
example, suppose that $P$ does something simple such as printing a sequence of integers
(which we do not know is finite or infinite). We run $P$ and observe that it starts
printing a sequence of integers:
\begin{equation}\label{eqn:prg-output-1}
  \mathtt{7 \quad 8 \quad 11 \quad 2 \quad 2 \quad 42 \quad \cdots}
\end{equation}

We then observe the output and judge whether or not $P$ satisfies certain properties.
Among the properties of $P$, some are special in that we can observe the fact that $P$
satisfies them. Consider, for instance, the property:
\begin{equation}\label{eqn:obs-prop}
  \text{``$P$ eventually prints $\mathtt{42}$''}.
\end{equation}
The knowledge about the behaviour of $P$ we can obtain from (\ref{eqn:prg-output-1}) is
sufficient to judge that $P$ satisfies this property. In other words, at the point of time
at which we make the observation (\ref{eqn:prg-output-1}), we have gathered complete
evidence for the fact that $P$ satisfies this property: there is no need to make any
further examination. Whenever a program satisfies Property~\ref{eqn:obs-prop}, there will
be such a \emph{finite} prefix of the output of the program by observing which we will
able judge for sure that $P$ satisfies this property. We refer to such properties as
\emph{observable properties}.

Some properties of $P$, on the other hand, are not observable. Consider, for instance,
the property:
\begin{equation}\label{eqn:non-obs-prop}
  \text{``$P$ prints at most two $\mathtt{2}$s''}.
\end{equation}
Suppose that $P$ is a program such that, after printing the sequence of integers in
(\ref{eqn:prg-output-1}), starts to continually print $\mathtt{0}$. Although $P$ then
certainly satisfies this property, we will not be able to judge on the basis of a finite
observation that it does so. We will gain more and more information about the behaviour of
$P$ but no amount of finite information will allow us to \emph{rule out} the possibility
that $P$ does not print a third $\mathtt{2}$.

Let us then make precise what exactly we mean by an ``observable property''.
\begin{defn}[Observable property (informal)]
  Let $P$ be a program that prints a possibly infinite sequence $\sigma$ of integers. We say
  that an extensional property $\phi$ of $P$ is observable if and only if:
  \begin{center}
    if $P$ satisfies the property $\phi$, there exists a finite prefix $\sigma|_i$ of the output
    $\sigma$ of $P$ such that $P$ is \emph{verified} to satisfy $\phi$ on the basis of $\sigma|_i$:
    no extension of $\sigma|_i$ can disrupt the fact that $P$ satisfies $\phi$.
  \end{center}
\end{defn}

Another name for the class of observable properties in this context is ``semidecidable''
although we will use the term ``observable'' as we will soon be generalising it to things
that are not programs. Let us now consider some properties of the class of observable
properties themselves.

Let $\phi_1, \cdots ,\phi_n$ be a \emph{finite} number of observable properties and assume that
\begin{equation*}
  \phi_1 \wedge \cdots \wedge \phi_n
\end{equation*}
holds. There must be \emph{stages}
\begin{equation*}
  m_1, \cdots , m_n
\end{equation*}
such that $\phi_k$ is verified at stage $m_k$. $\phi_1 \wedge \cdots \wedge \phi_n$ must then be verified at stage
\begin{equation*}
  \max(m_1, \cdots, m_n).
\end{equation*}
This is to say: if $\phi_1, \cdots, \phi_n$ are \emph{observable} then so is
$\phi_1 \wedge \cdots \wedge \phi_n$. In other words,
\begin{center}
  \emph{the class of observable properties is closed under finite intersection}.
\end{center}

Similarly, let $\{~\psi_i ~|~ i \in I~\}$ be an \emph{arbitrary} number of observable
properties such that $\bigvee_i \psi_i$ holds. As the disjunction holds, it must be the case that
some $\psi_i$ holds, meaning it must be verified at some stage $m$ as it is observable. $\bigvee_i
\psi_i$ is hence verified at stage $m$. This is to say: if $\{~\psi_i ~|~ i \in I~\}$ are
\emph{observable} then so is $\bigvee_i \psi_i$. In other words,
\begin{center}
  \emph{the class of observable properties is closed under arbitrary union}.
\end{center}

We have given an informal argument that the class of observable properties of programs is
a topology. This characterises be the perspective towards topology we will adopt in this
thesis: we will view it as a mathematical theory of observable properties. This
perspective towards topology has been developed by many researchers. Although it
originates in the work of Dana Scott~\cite{scott-original}, it was first made explicit by
Michael Smyth~\cite{smyth-handbook}. It was then subsequently developed by,
most saliently,
  Abramsky~\cite{abramsky-thesis},
  Vickers~\cite{vickers},
  Escardó~\cite{synthetic-topology}, and
  Taylor~\cite{taylor-asd}.

When one approaches topology from such a computational perspective, it is natural to
consider the question: if a topology is like a system of observable properties, what are
then the points? In other words, what are the observations about? The idea is that the
points of the topological space are programs whereas the open sets are observable
properties of these programs---but there is a subtlety: this \emph{pointful} view is not
entirely compatible with our computational view of topology.

We said that we are viewing programs as black boxes meaning we do not have full
understanding of their behaviour; we only have partial understanding that we form on the
bases of finite examinations. To work with the points directly then is to trivialise the
fact that we cannot, in general, fully understand the behaviour of a program by partial
examinations.

A view of topology more compatible with this perspective is what is called
\emph{pointless} topology~\cite{johnstone-the-point}, originating from the seminal work of
Marshall Stone~\cite{stone--original}. In pointless topology, points take a back seat to
open sets: instead of starting with a set of points, from which the notion of an open set
is derived, we start with a primitive set of observable properties (called the ``opens''),
and we then axiomatise the behaviour of these to reflect that they behave as a system of
observable properties. In other words, we axiomatise the behaviour of the lattice of open
sets of a topological space directly, that is, as an arbitrary lattice rather than a
lattice of subsets. The points can then be defined in terms of the opens but what is
interesting is that one can entertain important questions of topology without mentioning
the points at all~\cite{johnstone-the-point}.

Why is this view of topology more compatible with our perspective? We said previously that
if we are to understand the behaviour of a black-box program, we have to do this by
examining its output up to some finite stage simply because these are the manifestations
of the program's behaviour we can computationally access. The pointless vantage point can
then be motivated by the fact that we only want to write down things that we can
computationally access (i.e.~construct).

We have not yet precisely defined what a pointless topology is so let us do that now. As
mentioned, the idea is to axiomatise the behaviour of the lattice of open sets. Let
$\mathcal{O}$ be a set of opens, that are some unspecified primitive entities.
\begin{enumerate}
  \item Corresponding to the set-inclusion partial order in the pointful case, we require
    that there be a partial order $\_\sqsubseteq\_ \subseteq \mathcal{O} \times \mathcal{O}$.
  \item Corresponding to the fact that open sets are closed under finite intersection, we
    require that there be a binary meet operation $\meet{\_}{\_} : \mathcal{O} \times
    \mathcal{O} \rightarrow \mathcal{O}$ and a nullary one $\top : \mathcal{O}$.
  \item Corresponding to the fact that open sets are closed under arbitary union, we
    require that there be a join operation of arbitrary arity: $\joinnm{}\_ :
    \mathcal{P}(\mathcal{O}) \rightarrow \mathcal{O}$.
\end{enumerate}
In addition to these, we have to require that these operations satisfy an infinite
distributivity law, on which we will elaborate in Chapter~\ref{chap:frames}. The official
name for such a lattice that embodies a logic of observable or finitely verifiable
properties is \emph{frame}~\cite{vickers}. Indeed, frames form a category whose morphisms
are frame homomorphisms; objects of the opposite category are called \emph{locales}.
Therefore, frames and locales are synonymous
\emph{as long as no mentions to morphisms are made}~\cite{vickers, stone-spaces}

The precise relationship between the pointful and the pointless approaches is an
\emph{adjunction} called \emph{Stone duality}, that is not an equivalence in general. More
precisely, this adjunction is between the category of topological spaces and the category
of frames, and it comprises the following functor pair: the functor taking a topological
space to its frame of open subsets, and the functor taking a frame to the topological
spaces formed by its set of points (which we will not define here). Spaces are equivalent
to their pointless representation exactly when they satisfy the property of being sober.
We will not go into this topic in detail; The interested reader is referred to the work of
Johstone~\cite{johnstone-the-point, stone-spaces}.


We have provided some preliminary motivation for pointless topology that applies in the
context where we are viewing it as a theory of observable properties. Does this motivation
apply also from the perspective of general topology? Topology notoriously relies on
classical reasoning in many of its fundamental theorems such as the Tychonoff theorem. A
more concrete advantage of pointless topology is that by doing topology pointlessly, we
can avoid classical reasoning and hence gain a computational understanding of it---most
saliently of its crucial theorems such as the
Tychonoff theorem~\cite{coq-tychonoff, vickers-tychonoff}. This point was put eloquently
by Johnstone~\cite[pg.~46]{stone-spaces}:
\begin{quote}
  It is here that the real point of pointless topology begins to emerge; the difference
  between locales and spaces is one that we can (usually) afford to ignore if we are
  working in a ``classical'' universe with the axiom of choice available, but when (or if)
  we work in a context where choice principles are not allowed, then we have to take
  account of the difference—and usually it is locales, not spaces, which provide the right
  context in which to do topology. This is the point which, as I mentioned earlier,
  Andr\'{e} Joyal began to hammer home in the early 1970s; I can well remember how, at the
  time, his insistence that locales were the real stuff of topology, and spaces were
  merely figments of the classical mathematician's imagination, seemed (to me, and I
  suspect to others) like unmotivated fanaticism. I have learned better since then.
\end{quote}

This thesis is concerned with investigating topology in the context of type theory,
\emph{without} modifying its logic by the addition of postulates. The first prerequisite
for this is that we be able to develop topology constructively, that is, without relying
on classical principles. As pointed out by Johnstone, pointless topology can help us here
as it allows us to understand topology in constructive terms. The goal of carrying out
topology in type theory, however, presents further challenges: we have to understand
topology not only constructively but also \emph{predicatively}.

This is the subject matter of the branch of mathematics known as formal topology, first
instigated by Martin-Löf and Sambin~\cite{int-formal-spaces} in the early days of type
theory. The idea is that a formal topology recasts the notion of a frame into a form that
resembles a formal proof system. In addition to enabling the importation of
proof-theoretical ideas, such a formal presentation has the virtue of being
\emph{predicative} hence enabling the development of pointless topology in type theory.

Our goal in this thesis is to carry out formal topology in the context of univalent type
theory~\cite{hottbook, escardo-uf-intro}. By now, it has become clear that univalence
addresses certain severe shortcomings of type theory, and presents a new way of engaging
in mathematical developments in it. The question of what novelties it presents for formal
topology is therefore a natural one. In attempting to answer this question, we follow a
particular approach to formal topology, implementing an idea of Coquand~\cite{coq-posets}
to define formal topologies as posets endowed with ``interaction''
structures~\cite{tree-sets, hancock-interaction-systems}.

In summary, this thesis presents two contributions. The primary contribution is an answer
to the question how formal topology can be done in univalent type theory. In particular,
we find that an as is development of formal topology in univalent type theory is
problematic (at least ostensibly), although this issue can be remedied by using higher
inductive types (similar to how the Cauchy reals are constructed in the HoTT
book~\cite{hottbook}). The secondary contribution is the development of Coquand's idea of
doing formal topology with interaction systems to a further extent. To the author's
knowledge, this thesis presents the first formalised development of this approach to
formal topology.

This thesis is structured as follows. In Chapter~\ref{chap:foundations}, we summarise the
fundamentals of univalent type theory. In Chapter~\ref{chap:frames}, we present our
development of frames and constructs related to them. In Chapter~\ref{chap:formal-topo},
we present our main development of formal topology in univalent type theory. In
Chapter~\ref{chap:cantor}, we present a prime example of a formal topology: the Cantor
space. As a proof of concept for the formal study of topological properties in univalent
type theory, we prove an important property of the Cantor space, namely, that it is
compact.
