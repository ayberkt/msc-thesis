\chapter{The Cantor Space}\label{chap:cantor}

Now that we have pinned down a notion of formal topology, let us make things a bit more
concrete by looking at a prime example of a formal topology: the Cantor space. This
chapter corresponds to the \modname{CantorSpace} module in the \veragda{} formalisation.

For our purposes, the Cantor space is the type of all boolean sequences: $\nats{} \rightarrow
\bool{}$. If we were following a pointful road to the topology of the Cantor space, we
would start by saying the points of this space are sequences
$\oftyI{f}{\nats{} \rightarrow \bool{}}$. As we taking a \emph{pointless} road, we will start by
stating what the opens are: (finite) lists of booleans. Recalling the discussion about
approximations from Chapter~\ref{chap:intro}, it is clear why this is the case: a
function $\oftyI{f}{\nats{} \rightarrow \bool{}}$ is an infinite object, meaning we will never be
able to fully write it down. If we are to somehow understand its behaviour, this has to
by \emph{experimenting with it}, namely, examining its output up to some
$\oftyI{n}{\nats{}}$. In classical mathematics, we can indeed examine such a function
up to infinity as the infinity of time is trivialised by the law of excluded middle.
It is clear that this does not fit well with computational intuitions. If we write a
program to analyse a bitstream, and expect to get any sort of useful output from this
program, it has to analyse the given bitstream up to some finite $n$. This is precisely
the point of pointless topology: infinite things make computational sense only by finitary
examinations, which is what the opens are. Accordingly, we now start writing down the
topology of the Cantor space as an interaction system.

\begin{figure}
  \caption[Visualisation of the Cantor space]{%
    A visualisation of the Cantor space. Nodes correspond to stages of information (i.e.,
    observed prefixes) about a sequence as there is a unique path from each node to the
    root. Dashed lines denote possible outcomes whereas plain lines denote actual ones.
    The encircled node is the stage of information at which the prefix $\emptylist{} \frown 0 \frown
    0 \frown 1$ has been observed.
  }
  \label{fig:cantor}
  \centering
  \begin{forest}
    for tree={l sep=20pt}
    [\texttt{[]}
      [1, edge=dashed
        [0, edge=dashed
          [0, edge=dashed
            [ $\vdots$, edge=dashed ] [ $\vdots$, edge=dashed ] ]
          [1, edge=dashed
            [ $\vdots$, edge=dashed ] [ $\vdots$, edge=dashed ] ] ]
        [1, edge=dashed
          [0, edge=dashed [ $\vdots$~, edge=dashed ] [ $\vdots$, edge=dashed ] ]
          [1, edge=dashed
            [ $\vdots$, edge=dashed ] [ $\vdots$~, edge=dashed ] ]
        ]
      ]
      [\textbf{0}, edge=thick,
        [\textbf{0}, edge=thick
          [0, edge=dashed
            [ $\vdots$, edge=dashed ] [ $\vdots$, edge=dashed ] ]
          [\textbf{1}, edge=thick, circle, draw={red, radius=0.3pt, thick}
            [ $\vdots$, edge=dashed ]
            [ $\vdots$, edge=dashed ] ] ]
        [1, edge=dashed
          [0, edge=dashed
            [ $\vdots$, edge=dashed ]
            [ $\vdots$, edge=dashed ] ]
          [ 1, edge=dashed
            [ $\vdots$, edge=dashed ]
            [ $\vdots$, edge=dashed ] ] ]
      ] 
    ]
  \end{forest}
\end{figure}

\section{The interaction structure}

As we have explained in Section~\ref{sec:intr-sys}, we need to start defining the Cantor
space by giving its interaction structure.
\begin{description}
  \item[Stages:] finite lists of booleans. A list of length $n$ is the stage of
    information at which first $n$ bits of the sequence have been learned.
  \item[Experiments:] at any stage, there is only one experiment that can be performed:
    asking for the next bit. The sequences can be thought of as a black-box computing
    device with a button on it; whenever the button is pressed, it emits a new bit.
  \item[Outcomes:] the outcome of the experiment of asking for a new bit is its emission.
  \item[Revision:] when we receive bit $\oftyI{b}{\bool{}}$ at stage $bs$, we proceed to
    the next stage by appending $b$ to $bs$. We will denote this $bs \frown b$.
\end{description}

To write this down formally, we first need to define the inductive type of (snoc) lists.

\begin{defn}[Lists]\label{defn:list}
  The type $\listty{\bool{}}$ of lists of booleans is inductively defined by two constructors:
  \[
    \begin{prooftree}
      \infer0{\oftyI{\emptylist{}}{\listty{\bool{}}}}
    \end{prooftree}
    \qquad
    \begin{prooftree}
      \hypo{\oftyI{b}{\bool{}}}
      \hypo{\oftyI{bs}{\listty{\bool{}}}}
      \infer2{\oftyI{bs \frown b}{\listty{\bool{}}}}
    \end{prooftree}
  \]
\end{defn}

\begin{prop}\label{prop:list-set}
  $\listty{A}$ is an h-set for every discrete (\todo{reference}) type $A$.
\end{prop}
\begin{proof}[Proof sketch.]
  It is easy to show that equality on $\listty{\bool{}}$ is decidable. It is therefore a set
  by Hedberg's theorem.
\end{proof}

\todo{Add Hedberg's theorem to the foundations chapter.}

\begin{defn}[The Cantor poset]\label{defn:cantor-poset}
  The set $\listty{\bool{}}$ forms a poset under the following order. Let
  $\oftyII{bs_0}{bs_1}{\listty{\bool{}}}$;
  \begin{equation*}
    bs_1 \le bs_0 \quad\is\quad \sigmaty{bs_2}{\listty{\bool{}}}{bs_1 = bs_0 \concat bs_2}
  \end{equation*}
  We will refer to such a $bs_2$, that is the part of $bs_1$ after $bs_0$, as the
  \emph{difference} of $bs_1$ from $bs_0$. The fact that $\listty{\bool{}}$ is a set is
  given in Proposition~\ref{prop:list-set}. Reflexivity is immediate by picking $bs_2 \is
  \emptylist{}$. Transitivity boils down to to the associativity of $\concat$. For
  antisymmetry, let $\oftyII{bs}{bs'}{\listty{\bool{}}}$ such that $bs \le bs'$ and $bs' \le
  bs$. The result is immediate in the cases where the difference of either proof
  $\emptylist$, and the case where it is not is impossible. This latter fact involves a
  non-trivial amount of bureaucracy to prove in a completely precise way but it is
  intuitively apparent.

  Finally, note that the relation $\_\le\_$ is h-propositional (without the need for
  truncation!). This follows from the injectivity of $bs_0 \concat \_$.
\end{defn}

Now let us write down the interaction system that will underlies the Cantor space.
\begin{defn}[The Cantor interaction system]\label{defn:cantor-is}
  We will call the Cantor interaction system $\mathcal{C}$. It is defined as
  $\mathcal{C} \is (A, B, C, d)$, where
  \begin{alignat*}{2}
    A           \quad&\is\quad && \listty{\bool{}} \\
    B(bs)       \quad&\is\quad && \unitty{} \\
    C(bs, \star)    \quad&\is\quad && \bool{}   \\
    d(bs, \star, b) \quad&\is\quad && bs \frown b    .
  \end{alignat*}
\end{defn}

Definition~\ref{defn:cantor-is} provides the \emph{data} of the topology. Let us now prove
that this interaction system is indeed a formal topology.
\begin{thm}[The Cantor formal topology]\label{thm:cantor-topo}
  The Cantor interaction system on the Cantor poset forms a formal topology, that is, it
  satisfies the monotonicity (Defn.~\ref{defn:mono}) and the simulation
  (Defn.~\ref{defn:sim}) properties.
\end{thm}
\begin{proof}
  Monotonicity is immediate since, given a boolean $b$, the operation $\_ \frown b$ of snocing
  it onto a list of booleans clearly gets us in a more refined stage, witnessed by the
  difference of $\emptylist{} \frown b$.

  For the simulation property (Defn.~\ref{defn:sim}), let $\oftyII{bs_0}{bs_1}{\listty{\bool{}}}$
  and assume that $bs_1 \le bs_0$. Note that the experiment type is trivial so it suffices
  to show that for any outcome of $\oftyI{b_1}{C(bs_1, \star)}$, there exists some
  $\oftyI{b_0}{C(bs_0, \star)}$ such that
  \begin{equation*}
    d(bs_1, \star, b_1) \sqsubseteq d(bs_0, \star, b_0).
  \end{equation*}
  Unfolding the judgemental equalities in this statement, we can restate our goal as
  follows: for any boolean $b_1$ there exists some boolean $b_0$ such that
  \begin{equation*}
    bs_1 \frown b_1 \le bs_0 \frown b_0.
  \end{equation*}
  Let $b_1$ be an arbitrary boolean. We shall now construct such a $b_0$. We proceed by
  case analysis on the difference of $bs_1$ from $bs_2$.
  \begin{itemize}
    \item Case: $\emptylist{}$. This means that $bs_0 = bs_1$. We choose
      $b_0 \is b_1$. It is clear that $bs_1 \frown b_1 \le bs_0 \frown b_1$ by reflexivity.
    \item Case: $bs \frown b$. In this case, $bs_1 = bs_0 \concat{} (bs \frown b)$ and we know that
      $bs \frown b$ is nonempty. Call the first element (counting from left to right) of $bs \frown
      b$, $b_h$, and the rest of it, $bs_t$ ($h$ for ``head'' and $t$ for ``tail''). We
      choose $b_0 \is b_h$. It remains to be shown that $bs_1 \frown b_1 \le bs_0 \frown b_h$. Observe
      the following equality:
      \begin{align*}
        bs_1 \frown b_1 \quad&=\quad (bs_0 \concat{} (bs \frown b)) \frown b_1             \\
                   \quad&=\quad (bs_0 \concat{} (b_h \concat{} bs_t)) \frown b_1 \\
                   \quad&=\quad ((bs_0 \concat{} b_h) \concat{} bs_t) \frown b_1 \\
                   \quad&=\quad (bs_0 \concat{} b_h) \concat{} (bs_t \frown b_1) .
      \end{align*}
      This means that $bs_1 \frown b_1 \le bs_0 \frown b_h$ is witnessed by the difference
      $bs_t \frown b_1$.
  \end{itemize}
\end{proof}

\section{The Cantor space is compact}

An extremely important property of topological spaces in general topology is that of
\emph{compactness}. A space $X$ is called compact if
\begin{center}
  each open cover of $X$ has a finite subcover.
\end{center}
We remarked that an open cover of space $X$ is a collection $\{ V_i ~|~ i \in I \}$ of open
sets of $X$ such that $X = \bigcup_i V_i$. As we are viewing open sets as observable properties,
this means that an open cover $\setof{ V_i ~|~ i \in I }$ is a way of decomposing $X$ into
$\abs{I}$-many verifiable properties. If $X$ is compact, every time we can decompose $X$
into such a collection of open subsets, we can find a \textbf{finite subset} of this
collection that remains a decomposition of $X$ into observable properties. In other words,
even though $X$ might not be finite, its behaviour can somehow be reduced to finitely many
observable properties.

As the representation of topologies we have been working with \emph{is} directly the open
covers of open sets, it will be not too hard for us to state this property for a formal
topology. Let us first define the following auxiliary function.
\begin{defn}\label{defn:fin-cover}
  Given a poset $\oftyI{P}{\poset{}_{n, n}}$ and $\oftyI{xs}{\listty{\abs{P}}}$, we define
  the following function expressing the subset of inhabitants of $\abs{P}$ that are below
  at least one element of $xs$.
  \begin{align*}
    \fincovernm{}           \quad&:\quad   \listty{A} \rightarrow \pow{A}                            \\
    \fincover{\emptylist{}} \quad&\is\quad \lambda \_.~\bot_n                                       \\
    \fincover{xs' \frown x}      \quad&\is\quad \lambda ys.~\trunc{ys \sqsubseteq_P xs + ys \in \fincover{xs'}}
  \end{align*}
\end{defn}

\begin{defn}[Compactness]\label{defn:compact}
  A given formal topology $\McF{}$ is compact if the following type is inhabited:
  \begin{align*}
    &\iscompact{\McF{}} \quad\is\quad \\
    &\hspace{2em}
    \pity{a}{A}{\pity{(U, \_)}{\dcsubset{A}}{\\
        &\hspace{4em}
        \covers{a}{U} \rightarrow
          \trunc{%
            \sigmaty{as}{\listty{A}}{\covers{a}{\fincover{as}} \times \fincover{as} \subseteq U}
          }
      }
    }
  \end{align*}
\end{defn}

Notice that the result $\sum$ type must be truncated due to the existence of the
$\rulesquash{}$ constructor. One could argue that it would have to be truncated anyway
since compactness is a property and should therefore be propositional. However, one could
consider refining this property so that it uniquely characterises the finite subcover.
This would render the property naturally propositional, hence freeing us from the need to
truncate it.

We now proceed to prove that the Cantor space is compact. We will first prove three
small lemmas for the sake of clarity.

\begin{lemma}\label{lem:comp1}
  Given any formal topology $\McF{}$ and downward-closed subsets $U$ and $V$ of its
  underlying poset, if $\subsetof{U}{V}$ then $\subsetof{\covers{\_}{U}}{\covers{\_}{V}}$.
\end{lemma}
\begin{proof}
  Corollary of Proposition~\ref{prop:lem4}.
\end{proof}

\begin{lemma}\label{lem:comp2}
  Given any formal topology $\McF{}$, subsets $U$ and $V$ of $A_{\McF{}}$, and lists
  $\oftyII{as_0}{as_1}{\listty{A_{\McF{}}}}$,
  $\subsetof{\fincover{xs}}{\fincover{xs \concat ys}}$ and
  $\subsetof{\fincover{ys}}{\fincover{xs \concat ys}}$.
\end{lemma}
\begin{proof}
  Straightforward induction. Notice, however that the recursion principle of propositional
  truncation has to be invoked.
\end{proof}

\begin{lemma}
  Given any $\oftyI{bs}{\bitlist{}}$ and $\oftyII{bss_0}{bss_1}{\listty{\bitlist{}}}$, if
  $bs \in \fincover{bss_0 \concat bss_1}$ then it is either $bs \in \fincover{bss_0}$ or
  $bs \in \fincover{bss_1}$.
\end{lemma}
\begin{proof}
  Result follows by induction on $bss_1$.
\end{proof}

We are now ready to prove the compactness of the Cantor space.

\begin{thm}[Compactness of the Cantor space]
  $\iscompact{\mathcal{C}}$ is inhabited.
\end{thm}
\begin{proof}
  Let $\oftyI{bs}{\bitlist{}}$, and $U$, a downwards-closed subset of the Cantor poset
  such that $\covers{bs}{U}$. We need to show that $U$ has an finite subcover i.e., there
  exists some $\oftyI{bss}{\listty{\bitlist{}}}$ such that $\covers{bss}{\fincover{bs}}$
  and $\fincover{bs} \subseteq U$. We proceed by induction on the proof of $\covers{bs}{U}$.

  Case: $\ruledir{}$. Choose $bss \is [\ bs\ ]$, namely, the singleton list consisting of
  $bs$. The first property follows by the $\ruledir{}$ rule and reflexivity whilst the
  second one holds by the downwards-closure of $U$.

  Case: $\rulebranch{}$. We know that $\covers{bs \frown b}{U}$ for \emph{any} $b$, by the
  right premise of the $\rulebranch{}$ rule. Now, notice that we can appeal to the
  inductive hypothesis with both of $bs \frown \bitI{}$ and $bs \frown \bitO{}$, as we know
  $\covers{bs \frown \bitI{}}{U}$ and $\covers{bs \frown \bitO{}}{U}$. This gives us some $bss_0$
  and $bss_1$ such that not only
    \begin{equation*}
      \covers{bs \frown \bitI}{\fincover{bss_0}}
      \quad \text{and} \quad
      \covers{bs \frown \bitO{}}{\fincover{bss_1}},
    \end{equation*}
    but also
    \begin{equation*}
      \subsetof{\fincover{bs \frown \bitI{}}}{U}
      \quad \text{and} \quad
      \subsetof{\fincover{bs \frown \bitO{}}}{U}.
    \end{equation*}
    We now pick $bss \is bss_0 \concat bss_1$. It remains to be shown that
    \begin{equation*}
      \covers{bs}{\fincover{bss_0 \concat bss_1}}
      \quad\text{and}\quad
      \subsetof{\fincover{bss_0 \concat bss_1}}{U}.
    \end{equation*}

    Let us first show the left conjunct. By the $\rulebranch{}$ rule, it suffices to
    show $$\covers{bs \frown b}{\fincover{bss_0 \concat bss_1}}$$ for any $\oftyI{b}{\bool{}}$.
    Let $b$ be an arbitrary Boolean. By Lemma~\ref{lem:comp1}, and the fact that
    $\fincover{bss_0} \subseteq \fincover{bss_0 \concat bss_1}$ (by Lemma~\ref{lem:comp2}),
    it suffices to show $\covers{bs \frown b}{\fincover{bss_0}}$. Both $b \equiv \bitI{}$ and
    $b \equiv \bitO{}$ cases are immediate by assumption.

    Now, let us address the second conjunct. \todo{finish.}
\end{proof}
