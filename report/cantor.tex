\chapter{The Cantor Space}\label{chap:cantor}

Now that we have pinned down a notion of formal topology, let us make things a bit more
concrete by looking at a prime example of a formal topology: the Cantor space. This
chapter corresponds to the \modname{CantorSpace} module in the \veragda{} formalisation.

For our purposes, the Cantor space is the type of all boolean sequences: $\nats{} \rightarrow
\bool{}$. If we were following a pointful road to the topology of the Cantor space, we
would start by saying the points of this space are sequences
$\oftyI{f}{\nats{} \rightarrow \bool{}}$. As we taking a \emph{pointless} road, we will start by
stating what the opens are: (finite) lists of booleans. Recalling the discussion about
approximations from Chapter~\ref{chap:intro}, it is clear why this is the case: a
function $\oftyI{f}{\nats{} \rightarrow \bool{}}$ is an infinite object, meaning we will never be
able to fully write it down. If we are to somehow understand its behaviour, this has to
by \emph{experimenting with it}, namely, examining its output up to some
$\oftyI{n}{\nats{}}$. In classical mathematics, we can indeed examine such a function
up to infinity as the infinity of time is trivialised by the law of excluded middle.
It is clear that this does not fit well with computational intuitions. If we write a
program to analyse a bitstream, and expect to get any sort of useful output from this
program, it has to analyse the given bitstream up to some finite $n$. This is precisely
the point of pointless topology: infinite things make computational sense only by finitary
examinations, which is what the opens are. Accordingly, we now start writing down the
topology of the Cantor space as an interaction system.

\begin{figure}
  \caption[Visualisation of the Cantor space]{%
    A visualisation of the Cantor space. Nodes correspond to stages of information (i.e.,
    observed prefixes) about a sequence as there is a unique path from each node to the
    root. Dashed lines denote possible outcomes whereas plain lines denote actual ones.
    The encircled node is the stage of information at which the prefix $\emptylist{} \frown 0 \frown
    0 \frown 1$ has been observed.
  }
  \label{fig:cantor}
  \centering
  \begin{forest}
    for tree={l sep=20pt}
    [\texttt{[]}
      [1, edge=dashed
        [0, edge=dashed
          [0, edge=dashed
            [ $\vdots$, edge=dashed ] [ $\vdots$, edge=dashed ] ]
          [1, edge=dashed
            [ $\vdots$, edge=dashed ] [ $\vdots$, edge=dashed ] ] ]
        [1, edge=dashed
          [0, edge=dashed [ $\vdots$~, edge=dashed ] [ $\vdots$, edge=dashed ] ]
          [1, edge=dashed
            [ $\vdots$, edge=dashed ] [ $\vdots$~, edge=dashed ] ]
        ]
      ]
      [\textbf{0}, edge=thick,
        [\textbf{0}, edge=thick
          [0, edge=dashed
            [ $\vdots$, edge=dashed ] [ $\vdots$, edge=dashed ] ]
          [\textbf{1}, edge=thick, circle, draw={red, radius=0.3pt, thick}
            [ $\vdots$, edge=dashed ]
            [ $\vdots$, edge=dashed ] ] ]
        [1, edge=dashed
          [0, edge=dashed
            [ $\vdots$, edge=dashed ]
            [ $\vdots$, edge=dashed ] ]
          [ 1, edge=dashed
            [ $\vdots$, edge=dashed ]
            [ $\vdots$, edge=dashed ] ] ]
      ] 
    ]
  \end{forest}
\end{figure}

\section{The interaction structure}

As we have explained in Section~\ref{sec:intr-sys}, we need to start defining the Cantor
space by giving its interaction structure.
\begin{description}
  \item[Stages:] finite lists of booleans. A list of length $n$ is the stage of
    information at which first $n$ bits of the sequence have been learned.
  \item[Experiments:] at any stage, there is only one experiment that can be performed:
    asking for the next bit. The sequences can be thought of as a black-box computing
    device with a button on it; whenever the button is pressed, it emits a new bit.
  \item[Outcomes:] the outcome of the experiment of asking for a new bit is its emission.
  \item[Revision:] when we receive bit $\oftyI{b}{\bool{}}$ at stage $bs$, we proceed to
    the next stage by appending $b$ to $bs$. We will denote this $bs \frown b$.
\end{description}

To write this down formally, we first need to define the inductive type of (snoc) lists.

\begin{defn}[Lists]\label{defn:list}
  The type $\listty{}$ of lists of booleans is inductively defined by two constructors:
  \[
    \begin{prooftree}
      \infer0{\oftyI{\emptylist{}}{\listty{}}}
    \end{prooftree}
    \qquad
    \begin{prooftree}
      \hypo{\oftyI{b}{\bool{}}}
      \hypo{\oftyI{bs}{\listty{}}}
      \infer2{\oftyI{bs \frown b}{\listty{}}}
    \end{prooftree}
  \]
\end{defn}

\begin{prop}\label{prop:list-set}
  $\listty$ is an h-set.
\end{prop}
\begin{proof}[Proof sketch.]
  It is easy to show that equality on $\listty{}$ is decidable. It is therefore a set
  by Hedberg's theorem.
\end{proof}

\todo{Add Hedberg's theorem to the foundations chapter.}

\begin{defn}[The Cantor poset]\label{defn:cantor-poset}
  The set $\listty{}$ forms a poset under the following order. Let
  $\oftyII{bs_0}{bs_1}{\listty{}}$;
  \begin{equation*}
    bs_1 \le bs_0 \quad\is\quad \sigmaty{bs_2}{\listty{}}{bs_1 = bs_0 \concat bs_2}
  \end{equation*}
  We will refer to such a $bs_2$, that is the part of $bs_1$ after $bs_0$, as the
  \emph{difference} of $bs_1$ from $bs_0$. The fact that $\listty$ is a set is given in
  Proposition~\ref{prop:list-set}. Reflexivity is immediate by picking $bs_2 \is
  \emptylist{}$. Transitivity boils down to to the associativity of $\concat$. For
  antisymmetry, let $\oftyII{bs}{bs'}{\listty{}}$ such that $bs \le bs'$ and $bs' \le bs$. The
  result is immediate in the cases where the difference of either proof $\emptylist$, and
  the case where it is not is impossible. This latter fact involves a non-trivial amount
  of bureaucracy to prove in a completely precise way but it is intuitively apparent.

  Finally, note that the relation $\_\le\_$ is h-propositional (without the need for
  truncation!). This follows from the injectivity of $bs_0 \concat \_$.
\end{defn}

Now let us write down the interaction system that will underlies the Cantor space.
\begin{defn}[The Cantor interaction system]\label{defn:cantor-is}
  We will call the Cantor interaction system $\mathcal{C}$. It is defined as
  $\mathcal{C} \is (A, B, C, d)$, where
  \begin{alignat*}{2}
    A           \quad&\is\quad && \listty{} \\
    B(bs)       \quad&\is\quad && \unitty{} \\
    C(bs, \star)    \quad&\is\quad && \bool{}   \\
    d(bs, \star, b) \quad&\is\quad && bs \frown b    .
  \end{alignat*}
\end{defn}

Definition~\ref{defn:cantor-is} provides the \emph{data} of the topology. Let us now prove
that this interaction system is indeed a formal topology.
\begin{thm}[The Cantor formal topology]\label{thm:cantor-topo}
  The Cantor interaction system on the Cantor poset forms a formal topology, that is, it
  satisfies the monotonicity (Defn.~\ref{defn:mono}) and the simulation
  (Defn.~\ref{defn:sim}) properties.
\end{thm}
\begin{proof}
  Monotonicity is immediate since, given a boolean $b$, the operation $\_ \frown b$ of snocing
  it onto a list of booleans clearly gets us in a more refined stage, witnessed by the
  difference of $\emptylist{} \frown b$.

  For the simulation property (Defn.~\ref{defn:sim}), let $\oftyII{bs_0}{bs_1}{\listty{}}$
  and assume that $bs_1 \le bs_0$. Note that the experiment type is trivial so it suffices
  to show that for any outcome of $\oftyI{b_1}{C(bs_1, \star)}$, there exists some
  $\oftyI{b_0}{C(bs_0, \star)}$ such that
  \begin{equation*}
    d(bs_1, \star, b_1) \sqsubseteq d(bs_0, \star, b_0).
  \end{equation*}
  Unfolding the judgemental equalities in this statement, we can restate our goal as
  follows: for any boolean $b_1$ there exists some boolean $b_0$ such that
  \begin{equation*}
    bs_1 \frown b_1 \le bs_0 \frown b_0.
  \end{equation*}
  Let $b_1$ be an arbitrary boolean. We shall now construct such a $b_0$. We proceed by
  case analysis on the difference of $bs_1$ from $bs_2$.
  \begin{itemize}
    \item Case: $\emptylist{}$. This means that $bs_0 = bs_1$. We choose
      $b_0 \is b_1$. It is clear that $bs_1 \frown b_1 \le bs_0 \frown b_1$ by reflexivity.
    \item Case: $bs \frown b$. In this case, $bs_1 = bs_0 \concat{} (bs \frown b)$ and we know that
      $bs \frown b$ is nonempty. Call the first element (counting from left to right) of $bs \frown
      b$, $b_h$, and the rest of it, $bs_t$ ($h$ for ``head'' and $t$ for ``tail''). We
      choose $b_0 \is b_h$. It remains to be shown that $bs_1 \frown b_1 \le bs_0 \frown b_h$. Observe
      the following equality:
      \begin{align*}
        bs_1 \frown b_1 \quad&=\quad (bs_0 \concat{} (bs \frown b)) \frown b_1             \\
                   \quad&=\quad (bs_0 \concat{} (b_h \concat{} bs_t)) \frown b_1 \\
                   \quad&=\quad ((bs_0 \concat{} b_h) \concat{} bs_t) \frown b_1 \\
                   \quad&=\quad (bs_0 \concat{} b_h) \concat{} (bs_t \frown b_1) .
      \end{align*}
      This means that $bs_1 \frown b_1 \le bs_0 \frown b_h$ is witnessed by the difference
      $bs_t \frown b_1$.
  \end{itemize}
\end{proof}

\section{The Cantor space is compact}

\begin{defn}[Compactness]
\end{defn}

\begin{thm}[Compactness of the Cantor space]
\end{thm}
\begin{proof}
\end{proof}
