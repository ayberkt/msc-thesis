\chapter{The Cantor Space}\label{chap:cantor}

Now that we have pinned down a notion of formal topology, let us make things a bit more
concrete by looking at a prime example of a formal topology: the Cantor space. This
chapter corresponds to the \modname{CantorSpace} module in the \veragda{} formalisation.

For our purposes, the Cantor space is the type of boolean sequences: $\nats{} \rightarrow \bool{}$.
If we were following a pointful road to the topology of the Cantor space, we would start
by saying the points of this space are sequences $\oftyI{f}{\nats{} \rightarrow \bool{}}$. As we are
taking a \emph{pointless} road instead, we will start by stating what the opens are:
(finite) lists of booleans. Recalling the discussion about approximations from
Chapter~\ref{chap:intro}, it is clear why this is the case: a function $\oftyI{f}{\nats{}
  \rightarrow \bool{}}$ is an infinite object, meaning we will never be able to fully write it down
(i.e., its extension). If we are to somehow understand its behaviour, this has to by
\emph{experimenting with it}, namely, examining its output up to some
$\oftyI{n}{\nats{}}$. In classical mathematics, the ability to examine such a function up
to infinity is provided by the law of excluded middle. Although this may be useful for
some purposes, it is clear that it does not fit well with the computational view of
topology we would like to adopt: if we write a program to analyse a bitstream, and expect
to get any sort of useful output from this program, it has to analyse the given bitstream
up to some finite $n$. This is precisely the point of pointless topology: infinite things
make computational sense only by finitary examinations, which is what the opens are.
Accordingly, we now start writing down the underlying interaction system of the Cantor
topology.

\begin{figure}
  \caption[Visualisation of the Cantor space]{%
    A visualisation of the Cantor space. Nodes correspond to stages of information
    (i.e.~observed prefixes) about a sequence as there is a unique path from each node to
    the root. Dashed lines denote possible outcomes whereas plain lines denote actual
    ones. The encircled node is the stage of information at which the prefix $\emptylist{}
    \frown 0 \frown 0 \frown 1$ has been observed.
  }
  \label{fig:cantor}
  \centering
  \begin{forest}
    for tree={l sep=20pt}
    [$\emptylist{}$
      [1, edge=dashed
        [0, edge=dashed
          [0, edge=dashed
            [ $\vdots$, edge=dashed ] [ $\vdots$, edge=dashed ] ]
          [1, edge=dashed
            [ $\vdots$, edge=dashed ] [ $\vdots$, edge=dashed ] ] ]
        [1, edge=dashed
          [0, edge=dashed [ $\vdots$~, edge=dashed ] [ $\vdots$, edge=dashed ] ]
          [1, edge=dashed
            [ $\vdots$, edge=dashed ] [ $\vdots$~, edge=dashed ] ]
        ]
      ]
      [\textbf{0}, edge=thick,
        [\textbf{0}, edge=thick
          [0, edge=dashed
            [ $\vdots$, edge=dashed ] [ $\vdots$, edge=dashed ] ]
          [\textbf{1}, edge=thick, circle, draw={red, radius=0.3pt, thick}
            [ $\vdots$, edge=dashed ]
            [ $\vdots$, edge=dashed ] ] ]
        [1, edge=dashed
          [0, edge=dashed
            [ $\vdots$, edge=dashed ]
            [ $\vdots$, edge=dashed ] ]
          [ 1, edge=dashed
            [ $\vdots$, edge=dashed ]
            [ $\vdots$, edge=dashed ] ] ]
      ] 
    ]
  \end{forest}
\end{figure}

\section{The Cantor interaction system}

As we have explained in Section~\ref{sec:intr-sys}, we need to start defining the Cantor
space by giving its interaction structure. A visualisation of this can be found in
Fig.~\ref{fig:cantor}.
\begin{description}
  \item[Stages:] finite lists of booleans. A list of length $n$ is the stage of
    information at which first $n$ bits of the sequence have been found out.
  \item[Experiments:] at any stage, there is only one experiment that can be performed:
    asking for the next bit. Each sequence can be thought of as a black-box computing
    device with a button on it; whenever the button is pressed, it emits a new bit.
  \item[Outcomes:] the outcome of the experiment of asking for a new bit is simply the
    emitted bit itself
  \item[Revision:] when we receive some bit $\oftyI{b}{\bool{}}$ at stage $bs$, we proceed
    to the next stage by appending $b$ to $bs$. We will denote this $bs \frown b$.
\end{description}

To write this down formally, we first need to define the inductive type of (snoc) lists.

\begin{defn}[Boolean snoc lists]\label{defn:list}
  The type $\bitlist{}$ of lists of booleans is inductively defined by two constructors:
  \[
    \begin{prooftree}
      \infer0{\oftyI{\emptylist{}}{\bitlist{}}}
    \end{prooftree}
    \qquad
    \begin{prooftree}
      \hypo{\oftyI{b}{\bool{}}}
      \hypo{\oftyI{bs}{\bitlist{}}}
      \infer2[.]{\oftyI{\snoc{bs}{b}}{\bitlist{}}}
    \end{prooftree}
  \]
  We denote the singleton list consisting of $b$,
      $\singleton{b} \is \snoc{\emptylist}{b}$.
\end{defn}

\begin{defn}[Prepend for snoc lists]\label{defn:concat}
  Given two inhabitants $\oftyII{bs_0}{bs_1}{\bitlist{}}$, the operation of prepending of
  $bs_0$ to $bs_1$, denoted $\concat$, is defined by recursion as follows:
  \begin{alignat*}{3}
    \_&\concat\_                      \quad&&:\quad   && \bitlist{} \rightarrow \bitlist{} \rightarrow \bitlist{} \\
    bs_0 &\concat \emptylist          \quad&&\is\quad && bs_0                                 \\
    bs_0 &\concat (\snoc{bs_1'}{b_1}) \quad&&\is\quad && \snoc{(bs_0 \concat bs_1')}{b_1}     .
  \end{alignat*}
\end{defn}

The notation ``$\mathbb{B}$'' is mnemonic for \textbf{basic open} in that the inhabitants
of $\bitlist{}$ are basic opens of the Cantor space.

\begin{prop}\label{prop:list-set}
  $\bitlist{}$ is an h-set.
\end{prop}
\begin{proof}[Proof sketch.]
  It is easy to show that $\bitlist{}$ is discrete (Defn.~\ref{defn:discrete}) which is
  the case precisely because the type of \verbool{}s is discrete. It is therefore a set by
  Hedberg's theorem~(Theorem~\ref{thm:hedberg}).
\end{proof}

\begin{defn}[The Cantor poset]\label{defn:cantor-poset}
  The set $\bitlist{}$ forms a poset under the following order. Let
  $\oftyII{bs_0}{bs_1}{\bitlist{}}$;
  \begin{equation*}
    bs_1 \le bs_0 \quad\is\quad \sigmaty{bs_2}{\bitlist{}}{bs_1 = bs_0 \concat bs_2}.
  \end{equation*}
  We will refer to such a $bs_2$, that is the part of $bs_1$ after $bs_0$, as the
  \emph{difference} of $bs_1$ from $bs_0$. The fact that $\bitlist{}$ is a set is
  given in Proposition~\ref{prop:list-set}. Reflexivity is immediate by picking $bs_2 \is
  \emptylist{}$. Transitivity boils down to to the associativity of $\concat$. For
  antisymmetry, let $\oftyII{bs}{bs'}{\bitlist{}}$ such that $bs \le bs'$ and $bs' \le
  bs$. The result is immediate in the cases where either difference is $\emptylist$, and
  the case where it is not is impossible. This latter fact involves a non-trivial amount
  of bureaucracy to prove in a completely precise way but is intuitively apparent.

  Finally, note that the relation $\le$ is h-propositional (without the need for
  truncation!). This follows from the injectivity of $bs_0 \concat \_$.
\end{defn}

Now let us write down the interaction system that underlies the Cantor space.
\begin{defn}[The Cantor interaction system]\label{defn:cantor-is}
  We will call the Cantor interaction system $\cantoris{}$. It is defined as
  $\cantoris{} \is (A, B, C, d)$, where
  \begin{alignat*}{2}
    A                   \quad&\is\quad && \bitlist{}   \\
    B(bs)               \quad&\is\quad && \unitty{}    \\
    C(bs, \unittm{})    \quad&\is\quad && \bool{}      \\
    d(bs, \unittm{}, b) \quad&\is\quad && \snoc{bs}{b} .
  \end{alignat*}
\end{defn}

Definition~\ref{defn:cantor-is} provides the \emph{data} of the topology. Let us now prove
this actually forms a formal topology.
\begin{thm}[The Cantor formal topology]\label{thm:cantor-topo}
  The Cantor interaction system on the Cantor poset forms a formal topology, that is, it
  satisfies the monotonicity (Defn.~\ref{defn:mono}) and the simulation
  (Defn.~\ref{defn:sim}) properties.
\end{thm}
\begin{proof}
  Monotonicity is immediate since, given a boolean $b$, the operation $\snoc{\_}{b}$ of
  snocing it onto a list of booleans clearly gets us in a more refined stage, as witnessed
  by the difference of $\singleton{b}$.

  For the simulation property (Defn.~\ref{defn:sim}), let
  $\oftyII{bs_0}{bs_1}{\bitlist{}}$ and assume that $bs_1 \le bs_0$. Note that the
  experiment type is trivial so it suffices to show that for any outcome of
  $\oftyI{b_1}{C(bs_1, \unittm{})}$, there exists some $\oftyI{b_0}{C(bs_0, \unittm{})}$
  such that
  \begin{equation*}
    d(bs_1, \unittm{}, b_1) \sqsubseteq d(bs_0, \unittm{}, b_0).
  \end{equation*}
  Unfolding the definitional equalities in this statement, we can restate our goal as
  follows: for any boolean $b_1$ there exists some boolean $b_0$ such that
  \begin{equation*}
    \snoc{bs_1}{b_1} \le \snoc{bs_0}{b_0}.
  \end{equation*}
  Let $b_1$ be an arbitrary boolean. We shall now construct such a $b_0$. We proceed by
  case analysis on the difference of $bs_1$ from $bs_0$.
  \begin{itemize}
    \item Case: $\emptylist{}$. This means that $bs_0 = bs_1$. We choose $b_0 \is b_1$. It
      is clear that $\snoc{bs_1}{b_1} \le \snoc{bs_0}{b_1}$ by reflexivity.
    \item Case: $\snoc{bs}{b}$. In this case, $bs_1 = bs_0 \concat{} (\snoc{bs}{b})$ and
      we know that $\snoc{bs}{b}$ is nonempty. Call the first element (counting from left
      to right) of $\snoc{bs}{b}$, $b_h$, and the rest of it, $bs_t$ ($h$ for ``head'' and
      $t$ for ``tail''). We choose $b_0 \is b_h$. It remains to be shown that
      $\snoc{bs_1}{b_1} \le \snoc{bs_0}{b_h}$.
      Observe the following equality:
      \begin{align*}
        bs_1 \frown b_1 \quad&=\quad \snoc{(bs_0 \concat{} (\snoc{bs}{b}))}{b_1}       \\
                   \quad&=\quad \snoc{(bs_0 \concat{} (b_h \concat{} bs_t))}{b_1} \\
                   \quad&=\quad \snoc{((bs_0 \concat{} b_h) \concat{} bs_t)}{b_1} \\
                   \quad&\equiv\quad (bs_0 \concat{} b_h) \concat{} (\snoc{bs_t}{b_1}) .
      \end{align*}
      This means that $bs_1 \frown b_1 \le bs_0 \frown b_h$ is witnessed by the difference
      $bs_t \frown b_1$.
  \end{itemize}
\end{proof}

\section{The Cantor space is compact}

An important property of topological spaces in general topology is that of
\emph{compactness}. A space $X$ is called compact if
\begin{center}
  every open cover of $X$ has a finite subcover.
\end{center}
We remarked that an open cover of space $X$ is a collection $\{ V_i ~|~ i \in I \}$ of open
sets of $X$ such that $X = \bigcup_i V_i$. As we are viewing open sets as observable properties,
this means that an open cover $\setof{ V_i ~|~ i \in I }$ is a way of decomposing $X$ into
$\abs{I}$-many observable properties. If $X$ is compact, every time we can decompose $X$
into such a collection of open subsets, we can find a \textbf{finite subset} of this
collection that remains a decomposition of $X$ into observable properties. In other words,
even though $X$ might not be finite, its behaviour can somehow be
\emph{reduced to finitely many observable properties}.

As we have been representing topologies directly in terms of the notion of an open cover,
it will be not too hard for us to state this property for a formal topology. As we will
now proceed to do this, let us first clarify some notation. We will assume the usual list
type and denote the type of lists containing inhabitants of some type $A$, $\listty{A}$.
The cons operator and the empty list are denoted $\_\cons\_$ and $\emptyconslist{}$
respectively. Notice that we are denoting both empty lists of type $\bitlist{}$ and
$\listty{A}$ by $\emptyconslist{}$, as its type should be obvious in context. Similarly,
we will denote the singleton list and the append operation of both types using the same
notations: $\singletoncons{\_}$, $\textnormal{\texttt{++}}$, respectively. Again, the type
should be clear from the context. However, in the coloured version of this thesis, the
$\textnormal{\texttt{++}}$ operation on $\bitlist{}$ will be coloured
{\color{darkgreen}\emph{coloured green}} as it is a definiendum we have introduced by
means of a definitional equality (Defn.~\ref{defn:concat}).

One might wonder at this point why we are using two kinds of lists. We could certainly
modify $\bitlist{}$ to accommodate polymorphic lists, and use this instead of
$\listtynm{}$. We believe that the presentation is clearer when snoc lists are reserved
for the type $\bitlist{}$ to foreground its downwards-growing nature.

Let us now define an auxiliary function that we will need when stating compactness.
\begin{defn}\label{defn:fin-cover}
  Given a poset $\oftyI{P}{\poset{}_{n, n}}$ and $\oftyI{xs}{\listty{\abs{P}}}$, we define
  the following function expressing the subset of inhabitants of $\abs{P}$ that are below
  at least one element of $xs$.
  \begin{alignat*}{2}
    \fincovernm{}            \quad&:\quad   &&\listty{A} \rightarrow \pow{A}                          \\
    \fincover{\emptylist{}}  \quad&\is\quad &&\lambda \_.~\bot_n                                     \\
    \fincover{x \cons xs'}   \quad&\is\quad &&\lambda y.~\trunc{y \sqsubseteq_P x + y \inn \fincover{xs'}}  .
  \end{alignat*}
\end{defn}

We can now state\footnote{%
  This specific form of the compactness statement for a formal topology was communicated
  to the author by \thesupervisor{}~\cite{email-compactness}, the supervisor.
}
the compactness of a formal topology.
\begin{defn}[Compactness]\label{defn:compact}
  A given formal topology $\McF{}$ is compact if the following type is inhabited:
  \begin{align*}
    &\iscompact{\McF{}} \quad\is\quad \\
    &\hspace{2em}
    \pity{a}{A}{\pity{(U, \_)}{\dcsubset{A}}{\\
        &\hspace{4em}
        \covers{a}{U} \rightarrow
          \trunc{%
            \sigmaty{as}{\listty{A}}{%
              \covers{a}{\fincover{as}} \times \mempow{\fincover{as}}{U}
            }
          }.
      }
    }
  \end{align*}
\end{defn}

Notice that the result type must be truncated due to the existence of the $\rulesquash{}$
constructor. One could argue that it would have to be truncated \emph{anyway}, since
compactness is a property and should therefore be propositional. However, one could
consider refining this property so that it uniquely characterises the finite subcover.
This would render the property naturally propositional, hence freeing us from the
obligation to truncate it.

We will now prove that the Cantor space is compact. For this, we first prove three small
lemmas for the sake of clarity.

\begin{lemma}\label{lem:comp1}
  Given downwards-closed subsets $U$ and $V$ of the Cantor poset, if $\subsetof{U}{V}$
  then $\subsetof{\covers{\_}{U}}{\covers{\_}{V}}$.
\end{lemma}
\begin{proof}
  Corollary of Proposition~\ref{prop:lem4}.
\end{proof}

\begin{lemma}\label{lem:comp2}
  Given two lists $\oftyII{bss_0}{bss_1}{\listty{\bitlist{}}}$,
  $\subsetof{\fincover{bss_0}}{\fincover{bss_0 \append bss_1}}$ and
  $\subsetof{\fincover{bss_1}}{\fincover{bss_0 \append bss_1}}$.
\end{lemma}
\begin{proof}[Proof sketch]
  Straightforward induction. Notice, however that the recursion principle of propositional
  truncation has to be invoked.
\end{proof}

\begin{lemma}\label{lem:comp3}
  Given any $\oftyI{bs}{\bitlist{}}$ and $\oftyII{bss_0}{bss_1}{\listty{\bitlist{}}}$, if
  $bs \in \fincover{bss_0 \append bss_1}$ then either $bs \in \fincover{bss_0}$ or
  $bs \in \fincover{bss_1}$.
\end{lemma}
\begin{proof}[Proof sketch]
  Result follows by induction on $bss_0$.
\end{proof}

We are now ready to prove the compactness of the Cantor space. Note that we denote the two
inhabitants of $\bool{}$ by $\bitI{}$ and $\bitO{}$.

\begin{thm}
  $\cantoris{}$ is compact.
\end{thm}
\begin{proof}
  Let $\oftyI{bs}{\bitlist{}}$, and $U$, a downwards-closed subset of the Cantor poset
  such that $\covers{bs}{U}$. We need to show that $U$ has a finite subcover i.e., there
  exists some $\oftyI{bss}{\listty{\bitlist{}}}$ such that $\covers{bs}{\fincover{bss}}$
  and $\fincover{bs} \subseteq U$. We proceed by induction on the proof of $\covers{bs}{U}$.

  Case: $\ruledir{}$. Choose $bss \is \singletoncons{bs}$, namely, the singleton list
  consisting of $bs$. The first property follows by the $\ruledir{}$ rule and reflexivity
  whilst the second one holds by the downwards-closure of $U$.

  Case: $\rulebranch{}$. We know that $\covers{\snoc{bs}{b}}{U}$ for \emph{any} $b$, by
  the right premise of the $\rulebranch{}$ rule. Now, notice that we can appeal to the
  inductive hypothesis with both of $\snoc{bs}{\bitO{}}$ and $\snoc{bs}{\bitI{}}$, as we
  know $\covers{\snoc{bs}{\bitI{}}}{U}$ and $\covers{\snoc{bs}{\bitO{}}}{U}$. This gives
  us some $bss_0$ and $bss_1$ such that not only
    \begin{equation}\label{eqn:comp-ass-1}
      \covers{\snoc{bs}{\bitO}}{\fincover{bss_0}}
      \quad \text{and} \quad
      \covers{\snoc{bs}{\bitI{}}}{\fincover{bss_1}},
      \tag{$\dagger$}
    \end{equation}
    but also
    \begin{equation}\label{eqn:comp-ass-2}
      \subsetof{\fincover{bss₀}}{U}
      \quad \text{and} \quad
      \subsetof{\fincover{bss₁}}{U}.
      \tag{$\ddagger$}
    \end{equation}
    We now pick $bss \is bss_0 \append bss_1$. It remains to be shown that
    \begin{equation*}
      \covers{bs}{\fincover{bss_0 \append bss_1}}
      \quad\text{and}\quad
      \subsetof{\fincover{bss_0 \append bss_1}}{U}.
    \end{equation*}
    The second conjunct is easy to verify. Let $bss' \in \fincover{bss_0 \append bss_1}$. By
    Lemma~\ref{lem:comp3}, we have two cases: either $bss' \in \fincover{bss_0}$ or $bss' \in
    \fincover{bss_1}$. In either case, we have that $bss' \in U$ by (\ref{eqn:comp-ass-2}).

    For the left conjunct, notice that it suffices, by the $\rulebranch{}$ rule, to show
    \begin{equation*}
      \covers{\snoc{bs}{b}}{\fincover{bss_0 \append bss_1}}
    \end{equation*}
    for any $\oftyI{b}{\bool{}}$. Let $b$ be an arbitrary boolean. There are two cases for
    $b$.
    \begin{itemize}
      \item Case $b \equiv \bitO{}$. By Lemma~\ref{lem:comp1}, and the fact that
        \begin{equation*}
          \subsetof{\fincover{bss_0}}{\fincover{bss_0 \append bss_1}}
          \qquad\text{(by Lemma~\ref{lem:comp2})},
        \end{equation*}
        it suffices to show $\covers{\snoc{bs}{\bitO{}}}{\fincover{bss_0}}$ which we know
        by (\ref{eqn:comp-ass-1}).
      \item Case $b \equiv \bitI{}$. Exactly the same reasoning as in the $b \equiv \bitO{}$ case
        but with $bss_1$ instead $bss_o$.
    \end{itemize}

    Case $\rulesquash{}$. We are done by the $\rulesquash{}$ rule.
\end{proof}
