\newcommand\encircle[1]{%
  {\large {\color{red} \textcircled{{\color{black} {\normalsize \hspace{0.15mm}#1}}}}}
}

%% General

% Unary type declaration.
\newcommand{\oftyI}[2]{#1\hspace{0.1mm}:\hspace{0.1mm}#2}

% Binary type declaration.
\newcommand{\oftyII}[3]{#1, #2:#3}

% Introduction of a definitional equality.
\newcommand{\is}{\vcentcolon\equiv}

\newcommand{\idnm}{~=~}

% The `refl` constructor.
\newcommand{\refl}{\mathsf{refl}}

% The `zero` constructor.
\newcommand{\zero}{\mathsf{zero}}

% The `suc` constructor of natural numbers.
\newcommand{\sucnm}{\mathsf{suc}}
\newcommand{\suc}[1]{\sucnm{}\left(#1\right)}

% The type of natural numbers.
\newcommand{\nats}{\mathbb{N}}

% The type of booleans.
\newcommand{\bool}{\mathsf{Bool}}

% Pi type.
\newcommand{\pity}[3]{\prod_{(#1~:~#2)} #3}

% Sigma type.
\newcommand{\sigmaty}[3]{\sum_{(#1~:~#2)} #3}

% Universe.
\newcommand{\univ}{\mathcal{U}}

% Unit type.
\newcommand{\unitty}{\mathsf{Unit}}

% The only constructor of the Unit type.
\newcommand{\unittm}{\rulename{\star}}

\newcommand{\setof}[1]{\left\{ #1 \right\}}

\newcommand{\rulename}[1]{{\color{darkred} \mathsf{#1}}}

%% Foundations

\newcommand{\fiber}[2]{\hyperref[defn:fiber]{\mathsf{fiber}}\left(#1, #2\right)}

\newcommand{\isequiv}[1]{\hyperref[defn:equiv]{\mathsf{isEquiv}}\left(#1\right)}

\newcommand{\invequiv}[1]{\hyperref[prop:iso-equiv-equiv]{\mathsf{inv}}\left(#1\right)}

\newcommand{\idtoeqvnm}{\hyperref[defn:id-to-equiv]{\mathsf{idToEquiv}}}

\newcommand{\idtoeqv}[2]{\idtoeqvnm{}\left(#1, #2\right)}

\newcommand{\idequivnm}{\hyperref[defn:id-equiv]{\mathsf{idEqv}}}

\newcommand{\idequiv}[1]{\idequivnm{}\left(#1\right)}

\newcommand{\isdec}[1]{\hyperref[defn:decidable]{\mathsf{isDecidable}}\left(#1\right)}

\newcommand{\isdisc}[1]{\hyperref[defn:discrete]{\mathsf{isDiscrete}}\left(#1\right)}

\DeclareMathOperator{\typequivnm}{\hyperref[defn:equiv]{\simeq}}
\newcommand{\typequiv}[2]{#1 \typequivnm #2}

\newcommand{\istypiso}[1]{\hyperref[defn:type-iso]{\mathsf{isIso}}\left(#1\right)}

\newcommand{\typiso}[2]{#1 \hyperref[defn:type-iso]{\cong} #2}

\newcommand{\entire}[1]{\hyperref[defn:entire-subset]{\top}_{#1}}

% Logical equivalence.
\DeclareMathOperator{\logequivnm}{\hyperref[defn:iff]{\leftrightarrow}}
\newcommand{\logequiv}[2]{#1 \logequivnm{} #2}

% Extensional equality.
\newcommand{\exteq}[2]{#1 \hyperref[defn:exteq]{\sim} #2}

% Identity function.
\newcommand{\idfn}[1]{\mathsf{id}_{#1}}

% Homotopy levels.
\newcommand{\iscontrnm}{\hyperref[defn:contr]{\mathsf{isContr}}}
\newcommand{\iscontr}[1]{\iscontrnm{}\left(#1\right)}
\newcommand{\isprop}[1]{\hyperref[defn:hprop]{\mathsf{isProp}}\left(#1\right)}
\newcommand{\isset}[1]{\hyperref[defn:hset]{\mathsf{isSet}}\left(#1\right)}
\newcommand{\isofhlevel}[2]{%
  \hyperref[defn:hlevel]{\mathsf{isOfHLevel}}\left(#1, #2\right)
}

\newcommand{\capp}{\hyperref[defn:intersection]{\cap}}
\newcommand{\intersect}[2]{#1 \capp{} #2}

\newcommand{\formssns}[1]{\mathsf{SNS}\left(#1\right)}

\newcommand{\hprop}{\hyperref[defn:omega]{Ω}}

\newcommand{\abs}[1]{\left| #1 \right|}
\newcommand{\trunc}[1]{\left\| #1 \right\|}

% Powerset.
\newcommand{\pownm}{\hyperref[defn:pow]{\mathcal{P}}}
\newcommand{\pow}[1]{\pownm{}\left(#1\right)}

% Subset as a family.
\newcommand{\sub}[2]{\hyperref[defn:fam]{\mathsf{Fam}}_{#1}\left(#2\right)}

\newcommand{\subsetofnm}{\hyperref[defn:inclusion]{\subseteq}}
\newcommand{\subsetof}[2]{#1 \subsetofnm{} #2}

% Downwards-closure.
\newcommand{\isdcnm}{\hyperref[defn:dc-subset]{\mathsf{isDownwardsClosed}}}
\newcommand{\isdc}[1]{\isdcnm{}\left(#1\right)}
\newcommand{\dcsubsetnm}{\hyperref[defn:dc-subset]{\mathsf{DCSubset}}}
\newcommand{\dcsubset}[1]{\dcsubsetnm{}\left(#1\right)}
\newcommand{\dcframe}[1]{#1\hyperref[thm:down-set-frame]{\downarrow}}

\newcommand{\ordernm}{\hyperref[defn:poset]{\mathsf{Order}}}
\newcommand{\order}[2]{\ordernm{}_{#1}\left(#2\right)}

\newcommand{\posetstrnm}{\hyperref[defn:poset]{\mathsf{PosetStr}}}
\newcommand{\posetstr}[2]{\posetstrnm{}_{#1}\left(#2\right)}

% Membership in a family.
\DeclareMathOperator{\memfamnm}{\hyperref[defn:fam-mem]{\in}}
\newcommand{\memfam}[2]{#1 \memfamnm #2}

% Membership in a powerset.
\DeclareMathOperator{\inn}{\hyperref[defn:pow]{\in}}
\newcommand{\mempow}[2]{#1 \inn #2}

% Agda module name.
\newcommand{\modname}[1]{\texorpdfstring{{\color{AgdaModule} \texttt{#1}}}{#1}}

% Agda function name.
\newcommand{\fnname}[1]{{\color{AgdaFunction} \texttt{#1}}}

%% Posets.

\newcommand{\posetaxnm}{\hyperref[defn:poset]{\mathsf{PosetAx}}}
\newcommand{\posetax}[1]{\posetaxnm{}\left(#1\right)}

\newcommand{\poset}{\hyperref[defn:poset]{\mathsf{Poset}}}

\DeclareMathOperator{\posetisonm}{\hyperref[defn:poset-iso]{\cong_p}}
\newcommand{\posetiso}[2]{#1 \posetisonm{} #2}

\DeclareMathOperator{\posetequivnm}{\hyperref[defn:poset-equiv]{\simeq_p}}
\newcommand{\posetequiv}[2]{#1 \posetequivnm{}  #2}


\newcommand{\posof}[1]{\mathsf{pos}\left(#1\right)}

\newcommand{\ismonotonicnm}{\hyperref[defn:mono]{\mathsf{isMonotonic}}}
\newcommand{\ismonotonic}[1]{\ismonotonicnm{}\left(#1\right)}
\DeclareMathOperator{\mononm}{\hyperref[defn:mono-map]{\rightarrow_m}}
\newcommand{\mono}[2]{#1 \mononm{} #2}

%% Frames

\newcommand{\isframehomonm}{\hyperref[defn:frame-homo]{\mathsf{isFrameHomo}}}
\newcommand{\isframehomo}[1]{\isframehomonm{}\left(#1\right)}

\newcommand{\isframehomoeqvnm}{\hyperref[defn:frame-equiv]{\mathsf{isFrameHomoEqv}}}
\newcommand{\isframehomoeqv}[1]{%
  \isframehomoeqvnm\left(#1\right)
}

\DeclareMathOperator{\framehomonm}{\hyperref[defn:frame-homo]{\rightarrow_f}}
\newcommand{\framehomo}[2]{#1 \framehomonm{} #2}

\DeclareMathOperator{\frameequivnm}{\hyperref[defn:frame-equiv]{\simeq_f}}
\newcommand{\frameequiv}[2]{#1 \frameequivnm{} #2}

\DeclareMathOperator{\frameisonm}{\hyperref[defn:frame-iso]{\cong_f}}
\newcommand{\frameiso}[2]{#1 \frameisonm{} #2}

\newcommand{\verbool}{boolean}

\newcommand{\fix}[2]{\hyperref[thm:fixed-point-frame]{\mathsf{fix}}\left(#1, #2\right)}

\newcommand{\framestrnm}{\hyperref[defn:frame]{\mathsf{FrameStr}}}
\newcommand{\framestr}[1]{\framestrnm{}\left(#1\right)}
\newcommand{\frameaxnm}{\hyperref[defn:frame]{\mathsf{FrameAx}}}
\newcommand{\frameax}[1]{\frameaxnm{}\left(#1\right)}
\newcommand{\framenm}{\hyperref[defn:frame]{\mathsf{Frame}}}
\newcommand{\framety}[3]{\framenm{}_{#1, #2, #3}}

\newcommand{\rawframestrnm}{\hyperref[defn:frame]{\mathsf{RawFrameStr}}}
\newcommand{\rawframestr}[3]{\rawframestrnm{}_{#1, #2}\left(#3\right)}

\newcommand{\meet}[2]{#1 \wedge #2}
\newcommand{\joinnm}{\bigvee}
\newcommand{\join}[3]{\joinnm{}_{#1~:~#2} #3}

%% Nuclei.

\newcommand{\isnuclearnm}{\hyperref[defn:nucleus]{\mathsf{isNuclear}}}
\newcommand{\isnuclear}[1]{\isnuclearnm{}\left(#1\right)}
\newcommand{\nucleus}{\hyperref[defn:nucleus]{\mathsf{Nucleus}}}

\newcommand{\img}[2]{\setof{ #1\left( a \right) ~|~ \mempow{a}{#2} }}

%% Formal topology.

\newcommand{\treestrnm}{\hyperref[defn:intr-sys]{\mathsf{IntrStr}}}
\newcommand{\treestr}[1]{\treestrnm{}\left(#1\right)}
\newcommand{\intrsys}{\hyperref[defn:intr-sys]{\mathsf{IntrSys}}}

\newcommand{\hasmononm}{\hyperref[defn:mono]{\mathsf{hasMono}}}
\newcommand{\hasmono}[1]{\hasmononm{}\left(#1\right)}
\newcommand{\hassim}[1]{\hyperref[defn:sim]{\mathsf{hasSim}}\left(#1\right)}

\newcommand{\formaltopo}{\hyperref[defn:formal-topo]{\mathsf{FT}}}

\newcommand{\RHD}{\scalebox{1.2}{{\tt ▷}}}

\DeclareMathOperator{\covernm}{\scalebox{1.2}{{\tt ◁}}}
\newcommand{\covers}[2]{#1 \covernm{} #2}
\newcommand{\representsnm}{\hyperref[defn:rep]{\mathsf{represents}}}
\newcommand{\represents}[3]{\representsnm{}\left(#1, #2, #3\right)}

\newcommand{\ruledir}{{\color{darkred} \mathsf{dir}}}
\newcommand{\rulebranch}{{\color{darkred} \mathsf{branch}}}
\newcommand{\rulesquash}{{\color{darkred} \mathsf{squash}}}

\newcommand{\lifta}[1]{#1{\scriptstyle \downarrow}}

\newcommand{\isflat}[1]{\hyperref[defn:flat]{\mathsf{isFlat}}\left(#1\right)}

%% Cantor space.

\newcommand{\fincovernm}{\hyperref[defn:fin-cover]{\mathsf{down}}}
\newcommand{\fincover}[1]{\fincovernm{}\left(#1\right)}
\newcommand{\iscompact}[1]{\hyperref[defn:compact]{\mathsf{isCompact}}\left(#1\right)}

\newcommand{\listtynm}{\mathsf{List}}
\newcommand{\listty}[1]{\listtynm{}\left(#1\right)}
\DeclareMathOperator{\cons}{\text{\texttt{∷}}}
\newcommand{\bitlist}{\mathbb{B}}
\newcommand{\singleton}[1]{%
  \hyperref[defn:list]{\text{\texttt{[}}}~#1~\hyperref[defn:list]{\text{\texttt{]}}}
}
\newcommand{\singletoncons}[1]{%
  \text{\texttt{[}}~#1~\text{\texttt{]}}
}
\newcommand{\bitI}{\mathsf{1}}
\newcommand{\bitO}{\mathsf{0}}
\newcommand{\emptylist}{\rulename{\text{\texttt{[]}}}}
\newcommand{\emptyconslist}{\textnormal{\texttt{[]}}}

\DeclareMathOperator{\frownn}{\rulename{\frown}}
\newcommand{\snocnm}{\frownn}
\newcommand{\snoc}[2]{#1 \snocnm{} #2}
\newcommand{\cantoris}{\hyperref[defn:cantor-is]{\mathcal{C}}}

\DeclareMathOperator{\concat}{\hyperref[defn:concat]{\textnormal{\texttt{++}}}}
\DeclareMathOperator{\append}{\textnormal{\texttt{++}}}

%% Vernacular.

% Shorthand for referring to the `agda/cubical` library.
\newcommand{\libcub}{\texttt{cubical}}

\newcommand{\vermono}{monotonicity}
\newcommand{\versim}{simulation}

\newcommand{\veragda}{\textsc{Agda}}

%% Symbol abbreviations

\newcommand{\bF}{\mathbf{F}}
\newcommand{\bG}{\mathbf{G}}
\newcommand{\McF}{\mathcal{F}}
\newcommand{\MfU}{\mathfrak{U}}
