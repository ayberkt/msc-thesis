\chapter{Agda Formalisation}\label{app:agda-form}

\section{Discussion of some notational differences}

We first discuss some of the ways in which the notation of the \veragda{} formalisation
differs than the notation we use in the thesis. This is not intended to be a comprehensive
list of how each name maps to the \veragda{} formalisation. We discuss only salient
differences that we think might be confusing to the reader.

In Defn.~\ref{defn:fam}, we mentioned that we denote application of join operators by $\bigvee_i
U_i$. We use a very similar syntactic sugaring in the \veragda{} formalisation. The syntax
declaration for this can be found in the \modname{JoinSyntax} submodule of the
\modname{Frame} module.
\begin{code}%
\>[0]\AgdaKeyword{module}\AgdaSpace{}%
\AgdaModule{JoinSyntax}\AgdaSpace{}%
\AgdaSymbol{(}\AgdaBound{A}\AgdaSpace{}%
\AgdaSymbol{:}\AgdaSpace{}%
\AgdaFunction{Type}\AgdaSpace{}%
\AgdaGeneralizable{ℓ₀}\AgdaSymbol{)}\AgdaSpace{}%
\AgdaSymbol{\{}\AgdaBound{ℓ₂}\AgdaSpace{}%
\AgdaSymbol{:}\AgdaSpace{}%
\AgdaPostulate{Level}\AgdaSymbol{\}}\AgdaSpace{}%
\AgdaSymbol{(}\AgdaBound{join}\AgdaSpace{}%
\AgdaSymbol{:}\AgdaSpace{}%
\AgdaFunction{Fam}\AgdaSpace{}%
\AgdaBound{ℓ₂}\AgdaSpace{}%
\AgdaBound{A}\AgdaSpace{}%
\AgdaSymbol{→}\AgdaSpace{}%
\AgdaBound{A}\AgdaSymbol{)}\AgdaSpace{}%
\AgdaKeyword{where}\<%
\\
%
\\[\AgdaEmptyExtraSkip]%
\>[0][@{}l@{\AgdaIndent{0}}]%
\>[2]\AgdaFunction{join-of}\AgdaSpace{}%
\AgdaSymbol{:}\AgdaSpace{}%
\AgdaSymbol{\{}\AgdaBound{I}\AgdaSpace{}%
\AgdaSymbol{:}\AgdaSpace{}%
\AgdaFunction{Type}\AgdaSpace{}%
\AgdaBound{ℓ₂}\AgdaSymbol{\}}\AgdaSpace{}%
\AgdaSymbol{→}\AgdaSpace{}%
\AgdaSymbol{(}\AgdaBound{I}\AgdaSpace{}%
\AgdaSymbol{→}\AgdaSpace{}%
\AgdaBound{A}\AgdaSymbol{)}\AgdaSpace{}%
\AgdaSymbol{→}\AgdaSpace{}%
\AgdaBound{A}\<%
\\
%
\>[2]\AgdaFunction{join-of}\AgdaSpace{}%
\AgdaSymbol{\{}\AgdaArgument{I}\AgdaSpace{}%
\AgdaSymbol{=}\AgdaSpace{}%
\AgdaBound{I}\AgdaSymbol{\}}\AgdaSpace{}%
\AgdaBound{f}\AgdaSpace{}%
\AgdaSymbol{=}\AgdaSpace{}%
\AgdaBound{join}\AgdaSpace{}%
\AgdaSymbol{(}\AgdaBound{I}\AgdaSpace{}%
\AgdaOperator{\AgdaInductiveConstructor{,}}\AgdaSpace{}%
\AgdaBound{f}\AgdaSymbol{)}\<%
\\
%
\\[\AgdaEmptyExtraSkip]%
%
\>[2]\AgdaKeyword{syntax}\AgdaSpace{}%
\AgdaFunction{join-of}\AgdaSpace{}%
\AgdaSymbol{(λ}\AgdaSpace{}%
\AgdaBound{i}\AgdaSpace{}%
\AgdaSymbol{→}\AgdaSpace{}%
\AgdaBound{e}\AgdaSymbol{)}\AgdaSpace{}%
\AgdaSymbol{=}\AgdaSpace{}%
\AgdaFunction{⋁⟨}\AgdaSpace{}%
\AgdaBound{i}\AgdaSpace{}%
\AgdaFunction{⟩}\AgdaSpace{}%
\AgdaBound{e}\<%
\end{code}

The bracketed index \fnname{⟨}$i$\fnname{⟩} here is intended to be a plain text
approximation of subscripting.

Similarly, we use syntactic sugaring for the family comprehension notation ($\{ \cdots \}$) in
the \veragda{} formalisation as well. As the curly bracket symbols, \fnname{\{} and
\fnname{\}}, are not available in \veragda{}, we approximate these as \fnname{\textlquill} and
\fnname{\textrquill}. One instance of such a syntax declaration can be found in the \modname{Family}
module.

\begin{center}
\begin{code}%
\>[0]\AgdaFunction{img}\AgdaSpace{}%
\AgdaSymbol{:}\AgdaSpace{}%
\AgdaSymbol{\{}\AgdaBound{X}\AgdaSpace{}%
\AgdaSymbol{:}\AgdaSpace{}%
\AgdaFunction{Type}\AgdaSpace{}%
\AgdaGeneralizable{ℓ₀}\AgdaSymbol{\}}\AgdaSpace{}%
\AgdaSymbol{\{}\AgdaBound{Y}\AgdaSpace{}%
\AgdaSymbol{:}\AgdaSpace{}%
\AgdaFunction{Type}\AgdaSpace{}%
\AgdaGeneralizable{ℓ₁}\AgdaSymbol{\}}\AgdaSpace{}%
\AgdaSymbol{→}\AgdaSpace{}%
\AgdaSymbol{(}\AgdaBound{g}\AgdaSpace{}%
\AgdaSymbol{:}\AgdaSpace{}%
\AgdaBound{X}\AgdaSpace{}%
\AgdaSymbol{→}\AgdaSpace{}%
\AgdaBound{Y}\AgdaSymbol{)}\AgdaSpace{}%
\AgdaSymbol{→}\AgdaSpace{}%
\AgdaSymbol{(}\AgdaBound{U}\AgdaSpace{}%
\AgdaSymbol{:}\AgdaSpace{}%
\AgdaFunction{Fam}\AgdaSpace{}%
\AgdaGeneralizable{ℓ₂}\AgdaSpace{}%
\AgdaBound{X}\AgdaSymbol{)}\AgdaSpace{}%
\AgdaSymbol{→}\AgdaSpace{}%
\AgdaFunction{Fam}\AgdaSpace{}%
\AgdaGeneralizable{ℓ₂}\AgdaSpace{}%
\AgdaBound{Y}\<%
\\
\>[0]\AgdaFunction{img}\AgdaSpace{}%
\AgdaBound{g}\AgdaSpace{}%
\AgdaBound{(I , f)}\AgdaSpace{}%
\AgdaSymbol{=}\AgdaSpace{}%
\AgdaBound{I}\AgdaSpace{}%
\AgdaOperator{\AgdaInductiveConstructor{,}}\AgdaSpace{}%
\AgdaBound{g}\AgdaSpace{}%
\AgdaOperator{\AgdaFunction{∘}}\AgdaSpace{}%
\AgdaBound{f}\AgdaSpace{}\<%
\\
%
\\[\AgdaEmptyExtraSkip]%
\>[0]\AgdaKeyword{syntax}\AgdaSpace{}%
\AgdaFunction{img}\AgdaSpace{}%
\AgdaSymbol{(λ}\AgdaSpace{}%
\AgdaBound{x}\AgdaSpace{}%
\AgdaSymbol{→}\AgdaSpace{}%
\AgdaBound{e}\AgdaSymbol{)}\AgdaSpace{}%
\AgdaBound{U}\AgdaSpace{}%
\AgdaSymbol{=}\AgdaSpace{}%
\AgdaFunction{⁅}\AgdaSpace{}%
\AgdaBound{e}\AgdaSpace{}%
\AgdaFunction{∣}\AgdaSpace{}%
\AgdaBound{x}\AgdaSpace{}%
\AgdaFunction{ε}\AgdaSpace{}%
\AgdaBound{U}\AgdaSpace{}%
\AgdaFunction{⁆}\<%
\end{code}
\end{center}



\section{The \modname{Basis} module}

{
  \footnotesize
  \setmathfont{PragmataPro Mono Liga}
  \input{literate-latex/Basis}
}

\section{The \modname{Poset} module}

{
  \footnotesize
  \setmathfont{PragmataPro Mono Liga}
  \input{literate-latex/Poset}
}

\section{The \modname{Frame} module}

\footnotesize
\input{literate-latex/Frame}

\section{The \modname{Nucleus} module}

{
  \footnotesize
  \input{literate-latex/Nucleus}
}

\section{The \modname{Cover} module}

{
  \footnotesize
  \input{literate-latex/Cover}
}

\section{The \modname{CoverFormsNucleus} module}

{
  \footnotesize
  \input{literate-latex/CoverFormsNucleus}
}

\section{The \modname{FormalTopology} module}

{
  \footnotesize
  \input{literate-latex/FormalTopology}
}

\section{The \modname{UniversalProperty} module}

{
  \footnotesize
  \input{literate-latex/UniversalProperty}
}

\section{The \modname{CantorSpace} module}

{
  \footnotesize
  \input{literate-latex/CantorSpace}
}
