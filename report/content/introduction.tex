\chapter{Introduction}

This thesis is about topology, the branch of mathematics that studies \emph{continuous}
functions. The notion of a continuous function pervades practically all of mathematics, as
pointed out by JJ Sylvester: ``if I were asked to name, in one word, the pole star round
which the mathematical firmament revolves, the central idea which pervades the whole
corpus of mathematical doctrine, I should point to Continuity as contained in our notions
of space, and say, it is this, it is this!''~\cite[pg. 27]{armstrong_basic_2011}. Let us
then start by considering the question of what continuity is.

A continuous function is defined in elementary courses as a function for which ``small
changes to the input result in small changes to the output''. Usually right after this,
comes the $\epsilon-\delta$ definition of continuity. A function $f : \reals{} \rightarrow \reals{}$ is
\emph{continuous} if

\begin{equation*}
  \forall x, y \in \reals{}.\ \forall \epsilon > 0.\ \exists \delta > 0.\ 0 < | x - y | < \delta \rightarrow | f(x) - f(y) | < \epsilon.
\end{equation*}
