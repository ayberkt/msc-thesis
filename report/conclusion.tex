\chapter{Conclusion}\label{chap:conc}

The aim of formal topology is to make type-theoretical sense out of topology. The recent
formulation of \emph{univalent} type theories present interesting novelties for the future
of type-theoretical mathematics. The meaning of these novelties for existing bodies of
mathematical knowledge is an active area of research and it is fair to say that they have
not yet been fully explored. Considering this, we have set out to investigate a
natural question: what will formal topology look like in the context of univalent type
theory?

The first outcome of this investigation was a negative result: we have come to the
conclusion that an \emph{as is} development of formal topology in univalent type theory is
problematic as a consequence of the distinction between \emph{property} and
\emph{structure}, rendered visible through the homotopical view of types. In particular,
our development led to a situation in which the proof structure of the covering relation
had to be collapsed for it to behave as a property, and at the same time, kept intact for
it to be used in succeeding proofs.

We have then found, somewhat unexpectedly, that this problem can be circumvented by using
HITs. Using HITs, we were able to reconcile, in our specific case, the imbalance between
property and structure. This allowed us to successfully carry out a rudimentary
development of formal topology in a univalent setting: we have been able to formulate a
notion of formal topology and then prove its universal property. We hence conclude that
formal topology is not only possible in univalent type theory but is possible in a way
that makes non-trivial use of its conceptual novelties.

HITs have been previously used for similar purposes, and have allowed the circumvention of
of choice principles. The relevant work known to the author includes the higher
inductive-inductive definition of the Cauchy reals in the HoTT
book~\cite[Defn.~11.3.2]{hottbook} and the reconstruction of the partiality monad by
Altenkirch, Danielsson, and Kraus~\cite{partiality-revisited}. The latter seems to be
especially relevant to our use of HITs, as the view of topology as a theory of observable
properties is closely related to partial computations (see, for instance,
\cite{synthetic-topology, shulman-logic-of-space}).

In addition to providing a preliminary answer to the question of doing formal topology in
univalent type theory, we have carried out Coquand's idea of doing topology with
interaction systems on posets~\cite{coq-posets} to a further extent. We have found this to
work well in the context of univalent type theory. Apart from addressing issues of
univalence, this has been the secondary contribution of our development. To the author's
knowledge, this is the first formalised development of formal topologies as interaction
systems.

As a proof of concept of our approach to formal topology, we have implemented a
fundamental example: the Cantor space. Furthermore, we have proven that the formal Cantor
topology is compact. This exemplifies what we mean by ``making type-theoretical sense of
topology'': we have stated and then proven a non-trivial theorem of topology in a way that
is completely constructive and predicative.

After all, our development remains but an \emph{esquisse}. There remains much more work to
be done and it will be the undertaking of this that will put this approach to the test. We
now discuss some of these.

First and foremost, there are many more topological and locale-theoretic results to be
reconstructed in the setting of our development. For instance, one of the fundamental
theorems of topology is the Tychonoff theorem, formal versions of which have been
constructed~\cite{coq-tychonoff, vickers-tychonoff}. It is an interesting question if
these will fit well within our approach. Furthermore, a crucial class of spaces in
pointless topology is the \emph{Stone spaces}~\cite{stone-spaces}; indeed, the notion of a
Stone space is what brought the field into existence in the first place. Developing the
theory of Stone spaces in the context our development is an important direction for
further work.

Another limitation of our development is that we have not been able to explore the notion
of a \emph{formal point}, which can be introduced as a notion derived from the opens in
formal topology~\cite[pg.~94]{coq-sambin}. A prime example of this is the space of real
numbers, defined as the formal points of the formal topology of real numbers. It would be
interesting to explore such a definition of real numbers in univalent type theory.

Furthermore, formal topology has important connections to the theory of domains. As
pointed out by Sambin, domain theory can be viewed as a ``branch of formal
topology''~\cite{sambin-domains}. It was within our plans to provide a formal-topological
reconstruction of domain theory in univalent type theory, as an application of this
development. We have not been able to accomplish this due to time constraints. This would
be a very interesting example as it would likely make non-trivial use of the constructive
properties of our development. Domain theory in a univalent setting has recently been
investigated by de Jong~\cite{de-jong-domains}. A comparison of the two approaches might
prove beneficial.

Let us also discuss some of the lower-level shortcomings of our development that could be
more readily addressed.

As remarked in Chapter~\ref{chap:formal-topo}, our method for generating a frame from a
formal topology works only on formal topologies whose carrier set level and order level
are equal---this is a restriction we have imposed for the sake of presentational simplicity.
To reach full generality, these levels have to be generalised.

Furthermore, our statement of the universal property has not been modular. In other words,
we have not explained what it means, in general, for a frame to be the frame presented by
a formal topology. Instead, we have presented the universal property in an ad hoc way, for
the specific case of our generated frame. Ideally, this should be stated for \emph{any}
frame with a lifting map, our current statement then being one instance of it. Most
importantly, this would allow us to prove that two frames satisfying the universal
property are isomorphic, and thanks to univalence, equal.

A further direction to take for the modularisation of the universal property would be to
state it using categorical terms. Conceptually, the universal property ought to amount to
the existence of an adjoint pair of a free and a forgetful functor. However, it is not
clear what the category of formal topologies is, as the notion of a morphism between two
interaction systems is not clear. Therefore, the resolution of this issue is likely not
straightforward.

We believe that the type-theoretical distillation of topology enables a conceptual
clarification if it. Thanks to the invention of univalent type theory, type-theoretical
mathematics has been invigorated, and is now rapidly progressing. We hope the discipline
of formal topology can benefit from this progress, ideally through a renewed interest by
the type theory community.
