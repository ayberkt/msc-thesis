\chapter{Conclusion}\label{chap:conc}


The aim of formal topology is to make type-theoretical sense out of topology. The
homotopical insight into the structure of types that underlies \UF{} gives us a higher
resolution image of type theory, allowing us to have a finer-grained understanding of the
mathematical body of knowledge that is expressed in the language of types. Taking this
into account, we have set out to investigate a natural question: what is the meaning of
formal topology in \UF{}?

The first outcome of this investigation was a negative result: we have come to the
conclusion that an \emph{as is} development of formal topology in \UF{} is problematic, as
a consequence of the distinction between \emph{property} and \emph{structure}, rendered
visible through the homotopical view of types. In particular, our development led to a
situation in which the proof structure of a type has to be collapsed for it to behave as a
property, and at the same time, kept intact for it to be used in succeeding proofs.

We have then found, somewhat unexpectedly, that \UF{} in fact \emph{do} contain a remedy
for this situation, namely, HITs. Using HITs, we were able to reconcile, in our specific
case, the imbalance between property and structure. This allowed us to successfully carry
out a rudimentary development of formal topology in a univalent setting: we have been able
to prove the universal property of formal topologies. We hence conclude that formal
topology is not only possible in \UF{}, but possible precisely thanks to the conceptual
novelties furnished by \UF{}.

In addition to providing an answer to this question, we have carried out Coquand's idea of
doing topology with interaction systems on posets~\cite{coq-posets} to a further extent.
We have found this to work well, and enable a presentation more eloquent than that of
traditional formal topology. Apart from addressing issues of univalence, this has been the
secondary contribution of our development.

As a proof of concept of the efficacy of the univalent approach to formal topology, we
have implemented a fundamental example: the Cantor space. Furthermore, we have proven that
the Cantor space, as represented as a formal topology, is a compact space. This
exemplifies what we previously meant by ``making type-theoretical sense of topology'': we
have stated and then proven a non-trivial theorem of topology in a way that is completely
constructive and predicative.

After all, our development remains but an \emph{esquisse}. There remains much more work to
be done and it will be these that will put this approach to the test.

Most saliently, there are many more topological and locale-theoretic results to be
reconstructed in this setting. For instance, one of the fundamental theorems of topology
is the Tychonoff theorem, formal versions of which have been constructed. It must be
investigated if these will fit well within our approach. Furthermore, a very important of
a class of space in pointless topology is the class of \emph{Stone spaces}. Such important
examples of spaces must be developed within our approach.

We have not been able to explore the notion of \emph{formal points}. A prime example of
this is the space of real numbers, defined as the formal points of the formal topology of
the real numbers.

Furthermore, formal topology has important connections to the theory of domains. As
pointed out by Giovanni Sambin, it can be viewed as a ``branch of formal
topology''~\cite{sambin-domains}. It was within our plans to provide a formal topological
reconstruction of domain theory in \UF{} in the context of this development, although we
have not been able to accomplish this due to time constraints. This would be a very
interesting example as it would be likely make non-trivial use of the constructive
properties of our development.

Finally, let us also discuss some of the lower-level shortcomings of our development, and
certain ways it could be improved.

As remarked in Chapter~\ref{chap:formal-topo}, our method for generating a frame from a
formal topology works only on formal topologies whose carrier set level and order level
are equal. To reach full generality, these levels have to be generalised. We have not
carried out this generalisation for the presentational simplicity of the powerset operator
whose result is allowed to live at the same level as the type it's given.

Furthermore, our statement of the universal property has not been modular. We have
presented the universal property \emph{ad hoc}, for the specific case of our method of
generating a frame. Ideally, this should be stated for \emph{any} frame with our current
statement being one instance of it. Most importantly, this would allow us to prove state
that two frames satisfying the universal property are isomorphic, and thanks to
univalence, equal.

A further direction to take for the modularisation of the universal property would be to
state it using categorical terms. Conceptually, the universal property ought to amount to
the existence of an adjoint pair of a free and forgetful functor. However, it is not clear
what the category of formal topologies is, as the notion of a morphism between two
interaction systems is not clear. Therefore, the resolution of this issue is likely not
straightforward.

It is our belief that the type-theoretical distillation of topology, or rather, that which
is the subject matter of topology, enables a conceptual clarification if it. Thanks to the
invention of \UF{}, type-theoretical mathematics has been invigorated, and now progressing
at a fast pace. We hope that our rudimentary sketch of the beginnings of formal topology
in \UF{} will give rise to a renewed interest in formal topology, allowing it to benefit
from these recent advancements.
